\documentclass[italian,a4paper]{article}
\usepackage[tight,nice]{units}
\usepackage{babel,amsmath,amssymb,amsthm,graphicx,url}
\usepackage[text={5.5in,9in},centering]{geometry}
\usepackage[utf8x]{inputenc}
%\usepackage[T1]{fontenc}
\usepackage{ae,aecompl}
\usepackage[footnotesize,bf]{caption}
\usepackage[usenames]{color}
\usepackage{textcomp}
\usepackage{gensymb}
\include{pstricks}
\frenchspacing
\pagestyle{plain}
%------------- eliminare prime e ultime linee isolate
\clubpenalty=9999%
\widowpenalty=9999
%--- definizione numerazioni
\renewcommand{\theequation}{\thesection.\arabic{equation}}
\renewcommand{\thefigure}{\arabic{figure}}
\renewcommand{\thetable}{\arabic{table}}
\addto\captionsitalian{%
  \renewcommand{\figurename}%
{Figura}%
}
%
%------------- ridefinizione simbolo per elenchi puntati: en dash
%\renewcommand{\labelitemi}{\textbf{--}}
\renewcommand{\labelenumi}{\textbf{\arabic{enumi}.}}
\setlength{\abovecaptionskip}{\baselineskip}   % 0.5cm as an example
\setlength{\floatsep}{2\baselineskip}
\setlength{\belowcaptionskip}{\baselineskip}   % 0.5cm as an example
%--------- comandi insiemi numeri complessi, naturali, reali e altre abbreviazioni
\renewcommand{\leq}{\leqslant}
%--------- porzione dedicata ai float in una pagina:
\renewcommand{\textfraction}{0.05}
\renewcommand{\topfraction}{0.95}
\renewcommand{\bottomfraction}{0.95}
\renewcommand{\floatpagefraction}{0.35}
\setcounter{totalnumber}{5}
%---------
%
%---------
\begin{document}
\title{Relazione di laboratorio: partitore di tensione}
\author{\normalsize Ilaria Brivio (582116)\\%
\normalsize \url{brivio.ilaria@tiscali.it}%
\and %
\normalsize Matteo Abis (584206)\\ %
\normalsize \url{webmaster@latinblog.org}}
\date{\today}
\maketitle
%------------------
\section{Obiettivo dell'esperienza}
Obiettivo dell'esperienza è la realizzazione di un partitore di tensione in corrente continua, con particolare attenzione agli effetti dovuti alle resistenze interne degli strumenti impiegati e alla correlazione tra misure effettuate con lo stesso strumento. Con questo circuito è anche possibile verificare le leggi di Kirchhoff.

\section{Descrizione dell'apparato strumentale}
\begin{figure}[h]\caption{Schema del partitore di tensione.}\label{part}
\centering
 
\psset{unit=1in,cornersize=absolute,dimen=middle}%
\begin{pspicture}(-0.159722,-0.054722)(2.326389,1.554722)%
% dpic version 29.Oct.08 for PSTricks 0.93a or later
\psset{linewidth=0.8pt}%
\psset{linewidth=0.8pt}%
\psset{arrowsize=1.1pt 4,arrowlength=1.64,arrowinset=0}%
\psline(0,0)(0,0.25)
\pscircle(0,0.375){0.125}
\psline(0,0.25)(0,0.5)
\psline(0,0.5)(0,0.75)
\uput{0.501875ex}[dl](0,0.25){\llap{$ -$}}
\uput{0.501875ex}[l](-0.125,0.375){\llap{$ V_0$}}
\uput{0.501875ex}[ul](0,0.5){\llap{$ +$}}
\psline(0,0.75)(0,1.041667)
(0,1.041667)(-0.027778,1.055556)
(-0.027778,1.055556)(0.027778,1.083333)
(0.027778,1.083333)(-0.027778,1.111111)
(-0.027778,1.111111)(0.027778,1.138889)
(0.027778,1.138889)(-0.027778,1.166667)
(-0.027778,1.166667)(0.027778,1.194444)
(0.027778,1.194444)(0,1.208333)
(0,1.208333)(0,1.5)
\uput{0.501875ex}[l](-0.027778,1.125){\llap{$ R_G$}}
\psline(0,1.5)(1.125,1.5)
\psline[arrowsize=0.05in 0,arrowlength=2,arrowinset=0]{<-}(0.264815,1.5)(0.164815,1.5)
\uput{0.501875ex}[u](0.214815,1.5){$ I_1$}
\pscircle[fillstyle=solid,fillcolor=black](1.125,1.5){0.02}
\uput{0.501875ex}[u](1.125,1.52){$ A$}
\psline(1.125,1.5)(1.125,1.25)
(1.125,1.25)(1.166667,1.229167)
(1.166667,1.229167)(1.083333,1.1875)
(1.083333,1.1875)(1.166667,1.145833)
(1.166667,1.145833)(1.083333,1.104167)
(1.083333,1.104167)(1.166667,1.0625)
(1.166667,1.0625)(1.083333,1.020833)
(1.083333,1.020833)(1.125,1)
(1.125,1)(1.125,0.75)
\uput{0.501875ex}[l](1.083333,1.125){\llap{$ R_5$}}
\psline[arrowsize=0.05in 0,arrowlength=2,arrowinset=0]{<-}(1.125,1.3)(1.125,1.4)
\uput{0.501875ex}[r](1.125,1.35){\rlap{$ I_2$}}
\psline(1.125,0.75)(1.125,0.5)
(1.125,0.5)(1.166667,0.479167)
(1.166667,0.479167)(1.083333,0.4375)
(1.083333,0.4375)(1.166667,0.395833)
(1.166667,0.395833)(1.083333,0.354167)
(1.083333,0.354167)(1.166667,0.3125)
(1.166667,0.3125)(1.083333,0.270833)
(1.083333,0.270833)(1.125,0.25)
(1.125,0.25)(1.125,0)
\uput{0.501875ex}[l](1.083333,0.375){\llap{$ R_6$}}
\pscircle[fillstyle=solid,fillcolor=black](1.125,0){0.02}
\uput{0.501875ex}[d](1.125,-0.02){$ B$}
\psline(1.125,1.5)(2.25,1.5)
\psline[arrowsize=0.05in 0,arrowlength=2,arrowinset=0]{<-}(1.395,1.5)(1.295,1.5)
\uput{0.501875ex}[u](1.345,1.5){$ I_3$}
\psline(2.25,1.5)(2.25,1.4375)
(2.25,1.4375)(2.291667,1.416667)
(2.291667,1.416667)(2.208333,1.375)
(2.208333,1.375)(2.291667,1.333333)
(2.291667,1.333333)(2.208333,1.291667)
(2.208333,1.291667)(2.291667,1.25)
(2.291667,1.25)(2.208333,1.208333)
(2.208333,1.208333)(2.25,1.1875)
(2.25,1.1875)(2.25,1.125)
\uput{0.501875ex}[r](2.291667,1.3125){\rlap{$ R_1$}}
\psline(2.25,1.125)(2.25,1.0625)
(2.25,1.0625)(2.291667,1.041667)
(2.291667,1.041667)(2.208333,1)
(2.208333,1)(2.291667,0.958333)
(2.291667,0.958333)(2.208333,0.916667)
(2.208333,0.916667)(2.291667,0.875)
(2.291667,0.875)(2.208333,0.833333)
(2.208333,0.833333)(2.25,0.8125)
(2.25,0.8125)(2.25,0.75)
\uput{0.501875ex}[r](2.291667,0.9375){\rlap{$ R_2$}}
\psline(2.25,0.75)(2.25,0.6875)
(2.25,0.6875)(2.291667,0.666667)
(2.291667,0.666667)(2.208333,0.625)
(2.208333,0.625)(2.291667,0.583333)
(2.291667,0.583333)(2.208333,0.541667)
(2.208333,0.541667)(2.291667,0.5)
(2.291667,0.5)(2.208333,0.458333)
(2.208333,0.458333)(2.25,0.4375)
(2.25,0.4375)(2.25,0.375)
\uput{0.501875ex}[r](2.291667,0.5625){\rlap{$ R_3$}}
\psline(2.25,0.375)(2.25,0.3125)
(2.25,0.3125)(2.291667,0.291667)
(2.291667,0.291667)(2.208333,0.25)
(2.208333,0.25)(2.291667,0.208333)
(2.291667,0.208333)(2.208333,0.166667)
(2.208333,0.166667)(2.291667,0.125)
(2.291667,0.125)(2.208333,0.083333)
(2.208333,0.083333)(2.25,0.0625)
(2.25,0.0625)(2.25,0)
\uput{0.501875ex}[r](2.291667,0.1875){\rlap{$ R_4$}}
\psline(2.25,0)(0,0)
\end{pspicture}%

\end{figure}
\noindent Per realizzare questa esperienza abbiamo impiegato il generatore di corrente continua della postazione n.1 con $V_0 \sim \unit[5]{V}$, quattro resistenze $R_1,R_2, R_3, R_4 \sim \unit[100]{\ohm}$ e altre due resistenze $R_5, R_6 \sim \unit[550]{\ohm}$. Le misure sono state effettuate con i multimetri Fluke 111 e T110B.

\section{Descrizione della metodologia di misura}
Per prima cosa sono stati misurati i valori delle singole resistenze e della differenza di potenziale $V_0$ in modo diretto con il multimetro Fluke 111. Poi, con lo stesso multimetro il valore della resistenza complessiva equivalente del partitore e le differenze di potenziale ai capi di ciascuna resistenza. Infine, con il multimetro T110B, sono state misurate le correnti $I_1, I_2, I_3$ come indicate in figura~\ref{part}.

Per valutare la resistenza interna del generatore abbiamo misurato la caduta di potenziale in un circuito con una corrente di $\unit[100]{mA}$. Le resistenze del voltmetro del Fluke 111 e dell'amperometro del T110B sono state effettuate sia direttamente che con misure voltamperometriche.

\section{Risultati sperimentali ed elaborazione dati}
\subsection{Misure di resistenze interne degli strumenti}
\subsubsection{Resistenza interna del generatore}
\`E stata misurata la differenza di potenziale $V_0 = \unit[5.027\pm0.020]{V}$ ai capi del generatore, a vuoto, con il Fluke 111, fondo scala~\unit[6]{V}. Poi è stato collegato il circuito in figura con una corrente di~\unit[100.5$\pm$0.4]{mA} calibrata con la resistenza variabile e misurata con il T110B (FS~\unit[200]{mA}), ed è stata misurata con il Fluke una differenza di potenziale $V = \unit[5.018\pm0.020]{V}$. Per cui si trova che la resistenza interna del generatore è, come ci si aspettava, dell'ordine di \unit[0.1]{\ohm}:
\begin{equation*}
 R_G = \dfrac{V_0-V}{I} = \unit[0.089\pm0.016]{\ohm} \quad (18\%)
\end{equation*}
\begin{figure}[h]\caption{Circuito realizzato per la misura della resistenza interna del generatore $R_G$}\label{RG}
\centering
 
\psset{unit=1in,cornersize=absolute,dimen=middle}%
\begin{pspicture}(-0.159722,-0.02)(2.141421,1.625)%
% dpic version 29.Oct.08 for PSTricks 0.93a or later
\psset{linewidth=0.8pt}%
\psset{linewidth=0.8pt}%
\psset{arrowsize=1.1pt 4,arrowlength=1.64,arrowinset=0}%
\psline(0,0)(0,0.25)
\pscircle(0,0.375){0.125}
\psline(0,0.25)(0,0.5)
\psline(0,0.5)(0,0.75)
\uput{0.501875ex}[dl](0,0.25){\llap{$ -$}}
\uput{0.501875ex}[l](-0.125,0.375){\llap{$ V_0$}}
\uput{0.501875ex}[ul](0,0.5){\llap{$ +$}}
\psline(0,0.75)(0,1.0375)
(0,1.0375)(-0.029167,1.052083)
(-0.029167,1.052083)(0.029167,1.08125)
(0.029167,1.08125)(-0.029167,1.110417)
(-0.029167,1.110417)(0.029167,1.139583)
(0.029167,1.139583)(-0.029167,1.16875)
(-0.029167,1.16875)(0.029167,1.197917)
(0.029167,1.197917)(0,1.2125)
(0,1.2125)(0,1.5)
\uput{0.501875ex}[l](-0.029167,1.125){\llap{$ R_G$}}
\psline(0,1.5)(1,1.5)
\pscircle[fillstyle=solid,fillcolor=black](1,1.5){0.02}
\psline(1,1.5)(1,0.875)
\pscircle(1,0.75){0.125}
\rput(1,0.75){V}
\psline(1,0.625)(1,0)
\pscircle[fillstyle=solid,fillcolor=black](1,0){0.02}
\psline(1,1.5)(1.375,1.5)
\pscircle(1.5,1.5){0.125}
\rput(1.5,1.5){A}
\psline(1.625,1.5)(2,1.5)
\psline(2,1.5)(2,0.875)
(2,0.875)(2.041667,0.854167)
(2.041667,0.854167)(1.958333,0.8125)
(1.958333,0.8125)(2.041667,0.770833)
(2.041667,0.770833)(1.958333,0.729167)
(1.958333,0.729167)(2.041667,0.6875)
(2.041667,0.6875)(1.958333,0.645833)
(1.958333,0.645833)(2,0.625)
(2,0.625)(2,0)
\psline[arrowsize=0.05in 0,arrowlength=2,arrowinset=0]{->}(1.858579,0.608579)(2.141421,0.891421)
\psline(2,0)(0,0)
\end{pspicture}%

\end{figure}

\subsubsection{Resistenza interna del voltmetro del Fluke 111}
La resistenza del voltmetro è stata misurata in modo diretto con il multimetro T110B, $R_V^{\text{mis}} = \unit[11.15\pm0.10]{M\ohm}$, fondo scala \unit[20]{M\ohm}. Si noti che nel nostro circuito, $R_C = \unit[293.2\pm0.2]{\ohm}$, quindi l'alterazione delle correnti è dello 0.002\% quando si introduce il voltmetro, e quindi del tutto trascurabile. Inoltre, è stata misurata la differenza nella corrente che scorre nel circuito con una resistenza di carico $R_C \sim \unit[1]{M\ohm}$ senza voltmetro e poi con il voltmetro inserito. Tale differenza risulta $\Delta I = \unit[5.4-5.0]{\micro A} = \unit[0.40\pm0.08]{\micro A}$. Da cui:
\begin{equation*}
  R_V = \dfrac{V_0}{\Delta I} = \unit[12.6\pm2.6]{M\ohm} \quad (20\%)
\end{equation*}
\begin{figure}[h]\caption{Circuito realizzato per la misura della resistenza interna del voltmetro $R_V$}\label{RV}
\centering
 
\psset{unit=1in,cornersize=absolute,dimen=middle}%
\begin{pspicture}(-0.159722,-0.02)(1.951389,1.25)%
% dpic version 29.Oct.08 for PSTricks 0.93a or later
\psset{linewidth=0.8pt}%
\psset{linewidth=0.8pt}%
\psset{arrowsize=1.1pt 4,arrowlength=1.64,arrowinset=0}%
\psline(0,0)(0,0.4375)
\pscircle(0,0.5625){0.125}
\psline(0,0.4375)(0,0.6875)
\psline(0,0.6875)(0,1.125)
\uput{0.501875ex}[dl](0,0.4375){\llap{$ -$}}
\uput{0.501875ex}[l](-0.125,0.5625){\llap{$ V_0$}}
\uput{0.501875ex}[ul](0,0.6875){\llap{$ +$}}
\psline(0,1.125)(0.4375,1.125)
\pscircle(0.5625,1.125){0.125}
\rput(0.5625,1.125){A}
\psline(0.6875,1.125)(1.125,1.125)
\pscircle[fillstyle=solid,fillcolor=black](1.125,1.125){0.02}
\psline(1.125,1.125)(1.125,0.96875)
\pscircle(1.125,0.84375){0.125}
\rput(1.125,0.84375){V}
\psline(1.125,0.71875)(1.125,0.5625)
\psline(1.125,0.5625)(1.125,0.364583)
(1.125,0.364583)(1.152778,0.350694)
(1.152778,0.350694)(1.097222,0.322917)
(1.097222,0.322917)(1.152778,0.295139)
(1.152778,0.295139)(1.097222,0.267361)
(1.097222,0.267361)(1.152778,0.239583)
(1.152778,0.239583)(1.097222,0.211806)
(1.097222,0.211806)(1.125,0.197917)
(1.125,0.197917)(1.125,0)
\uput{0.501875ex}[l](1.097222,0.28125){\llap{$ R_V$}}
\pscircle[fillstyle=solid,fillcolor=black](1.125,0){0.02}
\psline(1.125,1.125)(1.875,1.125)
\psline(1.875,1.125)(1.875,0.6875)
(1.875,0.6875)(1.916667,0.666667)
(1.916667,0.666667)(1.833333,0.625)
(1.833333,0.625)(1.916667,0.583333)
(1.916667,0.583333)(1.833333,0.541667)
(1.833333,0.541667)(1.916667,0.5)
(1.916667,0.5)(1.833333,0.458333)
(1.833333,0.458333)(1.875,0.4375)
(1.875,0.4375)(1.875,0)
\uput{0.501875ex}[r](1.916667,0.5625){\rlap{$ R_C$}}
\psline(1.875,0)(0,0)
\end{pspicture}%

\end{figure}

\subsubsection{Resistenza interna dell'amperometro del T110B}
La misura diretta con il Fluke 111 della resistenza interna dell'amperometro del multimetro T110B (FS~\unit[600]{\ohm}) porge $R_A^{\text{mis}} = \unit[10.5\pm0.1]{\ohm}$. Nel circuito in figura l'amperometro misura una corrente $I = \unit[15.99\pm0.07]{mA}$ e il voltmetro (FS~\unit[6]{V}) una differenza di potenziale $V=\unit[0.166\pm0.001]{V}$. Quindi la resistenza interna dell'amperometro deve valere, per la legge di Ohm:
\begin{equation*}
 R_A = \dfrac V I =  \unit[10.38\pm0.08]{\ohm} \quad (0.8\%)
\end{equation*}
Poiché questi due risultati sono compatibili, nei calcoli che seguono impieghiamo la media pesata $R_A = \unit[10.43\pm0.06]{\ohm}$.
\begin{figure}[h]\caption{Circuito realizzato per la misura della resistenza interna $R_A$ dell'amperometro del T110B}\label{RA}
\centering
 
\psset{unit=1in,cornersize=absolute,dimen=middle}%
\begin{pspicture}(-0.159722,-0.02)(2,1.266421)%
% dpic version 29.Oct.08 for PSTricks 0.93a or later
\psset{linewidth=0.8pt}%
\psset{linewidth=0.8pt}%
\psset{arrowsize=1.1pt 4,arrowlength=1.64,arrowinset=0}%
\psline(0,0)(0,0.4375)
\pscircle(0,0.5625){0.125}
\psline(0,0.4375)(0,0.6875)
\psline(0,0.6875)(0,1.125)
\uput{0.501875ex}[dl](0,0.4375){\llap{$ -$}}
\uput{0.501875ex}[l](-0.125,0.5625){\llap{$ V_0$}}
\uput{0.501875ex}[ul](0,0.6875){\llap{$ +$}}
\psline(0,1.125)(0.4375,1.125)
(0.4375,1.125)(0.458333,1.166667)
(0.458333,1.166667)(0.5,1.083333)
(0.5,1.083333)(0.541667,1.166667)
(0.541667,1.166667)(0.583333,1.083333)
(0.583333,1.083333)(0.625,1.166667)
(0.625,1.166667)(0.666667,1.083333)
(0.666667,1.083333)(0.6875,1.125)
(0.6875,1.125)(1.125,1.125)
\psline[arrowsize=0.05in 0,arrowlength=2,arrowinset=0]{->}(0.421079,0.983579)(0.703921,1.266421)
\pscircle[fillstyle=solid,fillcolor=black](1.125,1.125){0.02}
\psline(1.125,1.125)(1.125,0.96875)
\pscircle(1.125,0.84375){0.125}
\rput(1.125,0.84375){A}
\psline(1.125,0.71875)(1.125,0.5625)
\psline(1.125,0.5625)(1.125,0.364583)
(1.125,0.364583)(1.152778,0.350694)
(1.152778,0.350694)(1.097222,0.322917)
(1.097222,0.322917)(1.152778,0.295139)
(1.152778,0.295139)(1.097222,0.267361)
(1.097222,0.267361)(1.152778,0.239583)
(1.152778,0.239583)(1.097222,0.211806)
(1.097222,0.211806)(1.125,0.197917)
(1.125,0.197917)(1.125,0)
\uput{0.501875ex}[l](1.097222,0.28125){\llap{$ R_A$}}
\pscircle[fillstyle=solid,fillcolor=black](1.125,0){0.02}
\psline(1.125,1.125)(1.875,1.125)
\psline(1.875,1.125)(1.875,0.6875)
\pscircle(1.875,0.5625){0.125}
\rput(1.875,0.5625){V}
\psline(1.875,0.4375)(1.875,0)
\psline(1.875,0)(0,0)
\end{pspicture}%

\end{figure}

\subsection{Misure sul partitore di tensione}
Sul partitore di tensione sono state misurate le singole resistenze, la resistenza equivalente tra i punti $A$ e $B$, le differenze di potenziale negli stessi punti.
Le misure sono state effettuate con il multimetro Fluke 111, fondo scala~\unit[600]{\ohm} per le resistenze, \unit[6]{V} per le differenze di potenziale. La seguente tabella riassume i risultati ottenuti:
\begin{table}[h]
\centering
 \begin{tabular}{c r@{ $\pm$ }l r@{ $\pm$ }l}
 &\multicolumn{2}{c}{R (\unit{\ohm})} &\multicolumn{2}{c}{$\Delta$V (\unit{V})}\\\hline
$R_1$ &99.7 &0.5 &1.258 &0.005\\
$R_2$ &99.7 &0.5 &1.258 &0.005\\
$R_3$ &99.5 &0.5 &1.254 &0.005\\
$R_4$ &99.7 &0.5 &1.257 &0.005\\
$R_5$ &556.2 &2.9 &2.516 &0.010\\
$R_6$ &555.0 &2.9 &2.509 &0.010\\
$AB$ &293.2 &1.5 &5.025 &0.020\\
 \end{tabular}
\end{table}\\
Per misurare le correnti è stato inserito in serie l'amperometro T110B (FS~\unit[20]{mA}) tra le resistenze 2 e 3, poi tra la 5 e la 6. Abbiamo ottenuto:
\begin{table}[h]
\centering
 \begin{tabular}{c r@{ $\pm$ }l r@{ $\pm$ }l c}
 &\multicolumn{2}{c}{$I^{\text{mis}}$ (\unit{mA})} &\multicolumn{2}{c}{$I^{\text{prev}}$ (\unit{mA})} &C\\\hline
$I_1$ &16.58 &0.07 & 16.55 & 0.11 & 0.23\\
$I_2$ &4.49 &0.02 & 4.48 & 0.03 & 0.28\\
$I_3$ &12.31&0.05 & 12.28 & 0.08 & 0.31\\\hline
$I_2+I_3$ &16.78&0.05 & 16.76 & 0.08& 0.22\\
 \end{tabular}
\end{table}\\
I valori previsti sono ricavati con la legge di Ohm, a partire dai valori misurati delle differenze di potenziale e delle resistenze nei rispettivi rami del circuito.
\section{Discussione dei risultati}
La misura della resistenza equivalente è compatibile ($C = 0.71$) con il valore teorico previsto di~\unit[293.4$\pm$0.2]{\ohm}. I valori delle differenze di potenziale previste e misurate, con la rispettiva compatibilità sono riportati nella seguente tabella:
\begin{table}[h]
\centering
 \begin{tabular}{c r@{ $\pm$ }l r@{ $\pm$ }l c}
 &\multicolumn{2}{c}{$\Delta V^{\text{prev}}$ (\unit{V})} &\multicolumn{2}{c}{$\Delta V^{\text{mis}}$ (\unit{V})}& C\\\hline
$V_1$ &1.257 &0.015 &1.258 &0.005 & 0.69\\
$V_2$ &1.257 &0.015 &1.258 &0.005 & 0.69\\
$V_3$ &1.254 &0.015 &1.254 &0.005 & 0.22\\
$V_4$ &1.257 &0.015 &1.257 &0.005 & 0.07\\
$V_5$ &2.515 &0.025 &2.516 &0.010 & 0.48\\
$V_6$ &2.510 &0.025 &2.509 &0.010 & 0.48\\
$V_{5+6}$ &5.025 &0.021 &5.025 &0.020 & 0.00\\
$V_{1+2+3+4}$ &5.027 &0.021 &5.025 &0.020 & 0.77\\
\end{tabular}
\end{table}\\
Come si vede, i dati si accordano molto bene con i valori previsti, che sono stati calcolati come segue. Diciamo $R_s$ la somma delle resistenze in serie con la resistenza considerata:
\begin{equation*}
 V_i = \dfrac{R_i}{R_s}V_0
\end{equation*}
Dai da presentati nel precedente paragrafo, la legge di Kirchhoff non è sembra perfettamente rispettata per le correnti che entrano nel nodo $A$ della figura~\ref{part}. Ciò è dovuto alla resistenza interna dell'amperometro con cui sono state effettuate le misure. Infatti, correggendo le correnti misurate in modo che:
\begin{equation*}
 I_i' =  I_i^{\text{mis}}\left(1+\dfrac{V_0}{R_A+R_s}\right)
\end{equation*}
Dove al solito $R_s$ indica la resistenza equivalente nel ramo considerato, si ha:
\begin{table}[h]
\centering
 \begin{tabular}{c r@{ $\pm$ }l}
 &\multicolumn{2}{c}{I (\unit{mA})}\\\hline
$I_1'$ &17.17 &0.07\\
$I_2'$ &4.51 &0.02\\
$I_3'$ &12.63&0.05\\\hline
$I_2'+I_3'$ &17.14&0.07 \\
 \end{tabular}
\end{table}\\
Per il calcolo delle correnti è stato usato il valore di $R_A$ misurato con il Fluke 111, per eliminare l'errore di scala, visto che le altre resistenze sono state anch'esse misurate con lo stesso strumento. Con questa correzione, la compatibilità risulta $C = 0.28$.
\section{Conclusioni}
Tutte le misure effettuate coincidono con i valori previsti, negli errori sperimentali. Ciò è particolarmente vero quando si tiene conto dell'effetto della resistenza interna degli strumenti, anche se quella del voltmetro e del generatore sono a tutti gli effetti trascurabili, essendo più piccole della sensibilità delle altre misure. Infine le leggi di Kirchhoff sulle differenze di potenziale e sulla somma delle correnti sono qui verificate.
\section{Appendice}
\subsection{Calcolo degli errori sperimentali}
Per il calcolo delle incertezze si è tenuto conto sia degli errori casuali che di quelli sistematici dovuti agli strumenti impiegati.

In particolare, come errore sulle singole misure si è assunta la deviazione standard $\sigma$ calcolata come $$\sigma_x^2=\sigma^2_\%+\sigma^2_r$$
Il primo termine è legato all'errore di scala dello strumento e vale $\sigma_\%=0.58\Delta_\%$, essendo $\Delta_\%=kx $ l'errore massimo; il secondo termine rappresenta l'errore casuale, che per gli strumenti digitali usati in quest'esperienza è calcolabile come $\sigma_r=0.58\Delta_{\text{digit}}$. I valori del fattore di scala $k$ e di $\Delta_{\text{digit}}$ dipendono dal fondoscala usato e sono forniti direttamente dal costruttore dello strumento.
Per esempio, per le resistenze nel partitore di tensione, misurate con il multimetro Fluke 111, fondo scala \unit[600]{\ohm}, si ha $k=0.9 \%$ e $\Delta_{\text{digit}}=2$, con ultima cifra significativa sull'ordine di \unit[0.1]{\ohm}; dunque $\sigma^2_R=(0.58\cdot R\cdot k)^2+(0.58\cdot 0.1 \cdot \Delta_{\text{digit}} )^2$.

Per le grandezze derivate l'incertezza è stata calcolata per propagazione e si è tenuto conto degli errori di scala ponendo ogni misura $x=kx^*$, pari al prodotto delle variabili indipendenti $k=1\pm \sigma_\%$, fattore di scala calcolato come sopra, e $x^*=x^*\pm \sigma_r$ affetta dal solo errore casuale.
Per esempio, per la resistenza interna dell'amperometro $R_A=\nicefrac{V}{I}$, posto $V=k_V V^*$ e $I=k_I I^*$
si ottiene $\sigma^2_{R_A}=R^2_A\cdot\left(\sigma^2_{V,\%}+\sigma^2_{\%,I}+I^{*-2}\sigma^2_{I^*,r}+V^{*-2}\sigma^2_{V^*1,r}\right)$.

\subsection{Calcolo delle compatibilità}
Le compatibiltà tra misure affette dal medesimo errore di scala sono state calcolate tenendo conto della correlazione tra di esse:
$$C=\dfrac{|x-y|}{\sqrt{\sigma^2_{x,r}+\sigma^2_{y,r}+(x-y)^2\sigma^2_\%}}$$
con $\sigma_{x,r}$, $\sigma_{y,r}$ errori casuali di $x$ e $y$ e $\sigma_\%$ errore di scala.
Per esempio, per le differenze di potenziale ai capi delle singole resistenze nel partitore, essendo $V^{\text{prev}}_i=\nicefrac{R_i}{R_S}V_0$, con $R_S$ somma delle resistenze in serie con $R_i$, i valori $V^{\text{prev}}_i$ risultano affetti dal medesimo errore di scala di $V_0$ e dunque anche di $V^{\text{mis}}_i$.
La compatibilità tra i valori è allora  $$C=\dfrac{|V^m-V^p|}{\sqrt{\sigma^2_{V^m,r}+\sigma^2_{V^p,r}+(V^m-V^p)^2\sigma^2_{V,\%}}}$$
Per le grandezze non correlate, invece, si è usata la formula 
$$C=\dfrac{|x-y|}{\sqrt{\sigma^2_x+\sigma^2_y}}$$
con $\sigma_x$ e $\sigma_y$ deviazioni standard associate alle misure.
\`{E} questo il caso, per esempio, del confronto tra la corrente $I'_1$ e la somma $I'_2+I'_3$, con i valori corretti tenendo conto della resistenza interna dell'amperometro.
\end{document}
