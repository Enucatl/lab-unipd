\documentclass[italian,a4paper]{article}
\usepackage[tight,nice]{units}
\usepackage{babel,amsmath,amssymb,amsthm,graphicx,url,gensymb}
\usepackage[text={5.5in,9in},centering]{geometry}
\usepackage[utf8x]{inputenc}
%\usepackage[T1]{fontenc}
\usepackage{ae,aecompl}
\usepackage[footnotesize,bf]{caption}
\usepackage[usenames]{color}
\include{pstricks}
\frenchspacing
\pagestyle{plain}
%------------- eliminare prime e ultime linee isolate
\clubpenalty=9999%
\widowpenalty=9999
%--- definizione numerazioni
\renewcommand{\theequation}{\thesection.\arabic{equation}}
\renewcommand{\thefigure}{\arabic{figure}}
\renewcommand{\thetable}{\arabic{table}}
\addto\captionsitalian{%
  \renewcommand{\figurename}%
{Grafico}%
}
%
%------------- ridefinizione simbolo per elenchi puntati: en dash
%\renewcommand{\labelitemi}{\textbf{--}}
\renewcommand{\labelenumi}{\textbf{\arabic{enumi}.}}
\setlength{\abovecaptionskip}{\baselineskip}   % 0.5cm as an example
\setlength{\floatsep}{2\baselineskip}
\setlength{\belowcaptionskip}{\baselineskip}   % 0.5cm as an example
%--------- comandi insiemi numeri complessi, naturali, reali e altre abbreviazioni
\renewcommand{\leq}{\leqslant}
%--------- porzione dedicata ai float in una pagina:
\renewcommand{\textfraction}{0.05}
\renewcommand{\topfraction}{0.95}
\renewcommand{\bottomfraction}{0.95}
\renewcommand{\floatpagefraction}{0.35}
\setcounter{totalnumber}{5}
%---------
%
%---------
\begin{document}
\title{Grafici sulle misure relative alla determinazione della costante di tempo di un circuito RC}
\author{\normalsize Ilaria Brivio (582116)\\%
\normalsize \url{brivio.ilaria@tiscali.it}%
\and %
\normalsize Matteo Abis (584206)\\ %
\normalsize \url{webmaster@latinblog.org}}
\date{\today}
\maketitle
%------------------
\begin{figure}[p]\caption{Tempo in ascissa, logaritmo del potenziale del condensatore in ordinata.}
\centering
% GNUPLOT: LaTeX picture using PSTRICKS macros
% Define new PST objects, if not already defined
\ifx\PSTloaded\undefined
\def\PSTloaded{t}
\psset{arrowsize=.01 3.2 1.4 .3}
\psset{dotsize=.01}
\catcode`@=11

\newpsobject{PST@Border}{psline}{linewidth=.0015,linestyle=solid}
\newpsobject{PST@Axes}{psline}{linewidth=.0015,linestyle=dotted,dotsep=.004}
\newpsobject{PST@Solid}{psline}{linewidth=.0015,linestyle=solid}
\newpsobject{PST@Dashed}{psline}{linewidth=.0015,linestyle=dashed,dash=.01 .01}
\newpsobject{PST@Dotted}{psline}{linewidth=.0025,linestyle=dotted,dotsep=.008}
\newpsobject{PST@LongDash}{psline}{linewidth=.0015,linestyle=dashed,dash=.02 .01}
\newpsobject{PST@Diamond}{psdots}{linewidth=.001,linestyle=solid,dotstyle=square,dotangle=45}
\newpsobject{PST@Filldiamond}{psdots}{linewidth=.001,linestyle=solid,dotstyle=square*,dotangle=45}
\newpsobject{PST@Cross}{psdots}{linewidth=.001,linestyle=solid,dotstyle=+,dotangle=45}
\newpsobject{PST@Plus}{psdots}{linewidth=.001,linestyle=solid,dotstyle=+}
\newpsobject{PST@Square}{psdots}{linewidth=.001,linestyle=solid,dotstyle=square}
\newpsobject{PST@Circle}{psdots}{linewidth=.001,linestyle=solid,dotstyle=o}
\newpsobject{PST@Triangle}{psdots}{linewidth=.001,linestyle=solid,dotstyle=triangle}
\newpsobject{PST@Pentagon}{psdots}{linewidth=.001,linestyle=solid,dotstyle=pentagon}
\newpsobject{PST@Fillsquare}{psdots}{linewidth=.001,linestyle=solid,dotstyle=square*}
\newpsobject{PST@Fillcircle}{psdots}{linewidth=.001,linestyle=solid,dotstyle=*}
\newpsobject{PST@Filltriangle}{psdots}{linewidth=.001,linestyle=solid,dotstyle=triangle*}
\newpsobject{PST@Fillpentagon}{psdots}{linewidth=.001,linestyle=solid,dotstyle=pentagon*}
\newpsobject{PST@Arrow}{psline}{linewidth=.001,linestyle=solid}
\catcode`@=12

\fi
\psset{unit=5.0in,xunit=5.0in,yunit=3.0in}
\pspicture(0.000000,0.000000)(1.000000,1.000000)
\ifx\nofigs\undefined
\catcode`@=11

\PST@Border(0.1590,0.1260)
(0.1740,0.1260)

\rput[r](0.1430,0.1260){-2.0}
\PST@Border(0.1590,0.2463)
(0.1740,0.2463)

\rput[r](0.1430,0.2463){-1.5}
\PST@Border(0.1590,0.3666)
(0.1740,0.3666)

\rput[r](0.1430,0.3666){-1.0}
\PST@Border(0.1590,0.4869)
(0.1740,0.4869)

\rput[r](0.1430,0.4869){-0.5}
\PST@Border(0.1590,0.6071)
(0.1740,0.6071)

\rput[r](0.1430,0.6071){0.0}
\PST@Border(0.1590,0.7274)
(0.1740,0.7274)

\rput[r](0.1430,0.7274){0.5}
\PST@Border(0.1590,0.8477)
(0.1740,0.8477)

\rput[r](0.1430,0.8477){1.0}
\PST@Border(0.1590,0.9680)
(0.1740,0.9680)

\rput[r](0.1430,0.9680){1.5}
\PST@Border(0.1590,0.1260)
(0.1590,0.1460)

\rput(0.1590,0.0840){0}
\PST@Border(0.2575,0.1260)
(0.2575,0.1460)

\rput(0.2575,0.0840){200}
\PST@Border(0.3560,0.1260)
(0.3560,0.1460)

\rput(0.3560,0.0840){400}
\PST@Border(0.4545,0.1260)
(0.4545,0.1460)

\rput(0.4545,0.0840){600}
\PST@Border(0.5530,0.1260)
(0.5530,0.1460)

\rput(0.5530,0.0840){800}
\PST@Border(0.6515,0.1260)
(0.6515,0.1460)

\rput(0.6515,0.0840){1000}
\PST@Border(0.7500,0.1260)
(0.7500,0.1460)

\rput(0.7500,0.0840){1200}
\PST@Border(0.8485,0.1260)
(0.8485,0.1460)

\rput(0.8485,0.0840){1400}
\PST@Border(0.9470,0.1260)
(0.9470,0.1460)

\rput(0.9470,0.0840){1600}
\PST@Border(0.1590,0.9680)
(0.1590,0.1260)
(0.9470,0.1260)
(0.9470,0.9680)
(0.1590,0.9680)

\rput{L}(0.0420,0.5470){$\log(V_0-V)$}
\rput(0.5530,0.0210){$t (\unit{\mu s})$}
\PST@Diamond(0.1984,0.8514)
\PST@Diamond(0.2378,0.8157)
\PST@Diamond(0.2772,0.7810)
\PST@Diamond(0.3166,0.7431)
\PST@Diamond(0.3560,0.7079)
\PST@Diamond(0.3954,0.6703)
\PST@Diamond(0.4348,0.6344)
\PST@Diamond(0.4742,0.5973)
\PST@Diamond(0.5136,0.5594)
\PST@Diamond(0.5530,0.5268)
\PST@Diamond(0.5924,0.4937)
\PST@Diamond(0.6318,0.4571)
\PST@Diamond(0.6712,0.4224)
\PST@Diamond(0.7106,0.3843)
\PST@Diamond(0.7500,0.3476)
\PST@Diamond(0.7894,0.3110)
\PST@Diamond(0.8288,0.2756)
\PST@Diamond(0.8682,0.2385)
\PST@Diamond(0.9076,0.2051)
\PST@Diamond(0.9470,0.1663)
\PST@Dashed(0.1984,0.8512)
(0.1984,0.8512)
(0.2060,0.8443)
(0.2135,0.8374)
(0.2211,0.8305)
(0.2286,0.8236)
(0.2362,0.8167)
(0.2438,0.8098)
(0.2513,0.8029)
(0.2589,0.7960)
(0.2665,0.7891)
(0.2740,0.7822)
(0.2816,0.7753)
(0.2891,0.7684)
(0.2967,0.7615)
(0.3043,0.7546)
(0.3118,0.7477)
(0.3194,0.7407)
(0.3269,0.7338)
(0.3345,0.7269)
(0.3421,0.7200)
(0.3496,0.7131)
(0.3572,0.7062)
(0.3648,0.6993)
(0.3723,0.6924)
(0.3799,0.6855)
(0.3874,0.6786)
(0.3950,0.6717)
(0.4026,0.6648)
(0.4101,0.6579)
(0.4177,0.6510)
(0.4252,0.6441)
(0.4328,0.6372)
(0.4404,0.6303)
(0.4479,0.6234)
(0.4555,0.6165)
(0.4631,0.6096)
(0.4706,0.6027)
(0.4782,0.5957)
(0.4857,0.5888)
(0.4933,0.5819)
(0.5009,0.5750)
(0.5084,0.5681)
(0.5160,0.5612)
(0.5235,0.5543)
(0.5311,0.5474)
(0.5387,0.5405)
(0.5462,0.5336)
(0.5538,0.5267)
(0.5614,0.5198)
(0.5689,0.5129)
(0.5765,0.5060)
(0.5840,0.4991)
(0.5916,0.4922)
(0.5992,0.4853)
(0.6067,0.4784)
(0.6143,0.4715)
(0.6219,0.4646)
(0.6294,0.4577)
(0.6370,0.4508)
(0.6445,0.4438)
(0.6521,0.4369)
(0.6597,0.4300)
(0.6672,0.4231)
(0.6748,0.4162)
(0.6823,0.4093)
(0.6899,0.4024)
(0.6975,0.3955)
(0.7050,0.3886)
(0.7126,0.3817)
(0.7202,0.3748)
(0.7277,0.3679)
(0.7353,0.3610)
(0.7428,0.3541)
(0.7504,0.3472)
(0.7580,0.3403)
(0.7655,0.3334)
(0.7731,0.3265)
(0.7806,0.3196)
(0.7882,0.3127)
(0.7958,0.3058)
(0.8033,0.2988)
(0.8109,0.2919)
(0.8185,0.2850)
(0.8260,0.2781)
(0.8336,0.2712)
(0.8411,0.2643)
(0.8487,0.2574)
(0.8563,0.2505)
(0.8638,0.2436)
(0.8714,0.2367)
(0.8789,0.2298)
(0.8865,0.2229)
(0.8941,0.2160)
(0.9016,0.2091)
(0.9092,0.2022)
(0.9168,0.1953)
(0.9243,0.1884)
(0.9319,0.1815)
(0.9394,0.1746)
(0.9470,0.1677)

\PST@Border(0.1590,0.9680)
(0.1590,0.1260)
(0.9470,0.1260)
(0.9470,0.9680)
(0.1590,0.9680)

\catcode`@=12
\fi
\endpspicture

\end{figure}

\begin{figure}[p]\caption{Tempo in ascissa, logaritmo del potenziale del condensatore in ordinata. Grafico dei residui.}
\centering
% GNUPLOT: LaTeX picture using PSTRICKS macros
% Define new PST objects, if not already defined
\ifx\PSTloaded\undefined
\def\PSTloaded{t}
\psset{arrowsize=.01 3.2 1.4 .3}
\psset{dotsize=.01}
\catcode`@=11

\newpsobject{PST@Border}{psline}{linewidth=.0015,linestyle=solid}
\newpsobject{PST@Axes}{psline}{linewidth=.0015,linestyle=dotted,dotsep=.004}
\newpsobject{PST@Solid}{psline}{linewidth=.0015,linestyle=solid}
\newpsobject{PST@Dashed}{psline}{linewidth=.0015,linestyle=dashed,dash=.01 .01}
\newpsobject{PST@Dotted}{psline}{linewidth=.0025,linestyle=dotted,dotsep=.008}
\newpsobject{PST@LongDash}{psline}{linewidth=.0015,linestyle=dashed,dash=.02 .01}
\newpsobject{PST@Diamond}{psdots}{linewidth=.001,linestyle=solid,dotstyle=square,dotangle=45}
\newpsobject{PST@Filldiamond}{psdots}{linewidth=.001,linestyle=solid,dotstyle=square*,dotangle=45}
\newpsobject{PST@Cross}{psdots}{linewidth=.001,linestyle=solid,dotstyle=+,dotangle=45}
\newpsobject{PST@Plus}{psdots}{linewidth=.001,linestyle=solid,dotstyle=+}
\newpsobject{PST@Square}{psdots}{linewidth=.001,linestyle=solid,dotstyle=square}
\newpsobject{PST@Circle}{psdots}{linewidth=.001,linestyle=solid,dotstyle=o}
\newpsobject{PST@Triangle}{psdots}{linewidth=.001,linestyle=solid,dotstyle=triangle}
\newpsobject{PST@Pentagon}{psdots}{linewidth=.001,linestyle=solid,dotstyle=pentagon}
\newpsobject{PST@Fillsquare}{psdots}{linewidth=.001,linestyle=solid,dotstyle=square*}
\newpsobject{PST@Fillcircle}{psdots}{linewidth=.001,linestyle=solid,dotstyle=*}
\newpsobject{PST@Filltriangle}{psdots}{linewidth=.001,linestyle=solid,dotstyle=triangle*}
\newpsobject{PST@Fillpentagon}{psdots}{linewidth=.001,linestyle=solid,dotstyle=pentagon*}
\newpsobject{PST@Arrow}{psline}{linewidth=.001,linestyle=solid}
\catcode`@=12

\fi
\psset{unit=5.0in,xunit=5.0in,yunit=3.0in}
\pspicture(0.000000,0.000000)(1.000000,1.000000)
\ifx\nofigs\undefined
\catcode`@=11

\PST@Border(0.1910,0.1260)
(0.2060,0.1260)

\rput[r](0.1750,0.1260){-0.030}
\PST@Border(0.1910,0.2025)
(0.2060,0.2025)

\rput[r](0.1750,0.2025){-0.025}
\PST@Border(0.1910,0.2791)
(0.2060,0.2791)

\rput[r](0.1750,0.2791){-0.020}
\PST@Border(0.1910,0.3556)
(0.2060,0.3556)

\rput[r](0.1750,0.3556){-0.015}
\PST@Border(0.1910,0.4322)
(0.2060,0.4322)

\rput[r](0.1750,0.4322){-0.010}
\PST@Border(0.1910,0.5087)
(0.2060,0.5087)

\rput[r](0.1750,0.5087){-0.005}
\PST@Border(0.1910,0.5853)
(0.2060,0.5853)

\rput[r](0.1750,0.5853){0.000}
\PST@Border(0.1910,0.6618)
(0.2060,0.6618)

\rput[r](0.1750,0.6618){0.005}
\PST@Border(0.1910,0.7384)
(0.2060,0.7384)

\rput[r](0.1750,0.7384){0.010}
\PST@Border(0.1910,0.8149)
(0.2060,0.8149)

\rput[r](0.1750,0.8149){0.015}
\PST@Border(0.1910,0.8915)
(0.2060,0.8915)

\rput[r](0.1750,0.8915){0.020}
\PST@Border(0.1910,0.9680)
(0.2060,0.9680)

\rput[r](0.1750,0.9680){0.025}
\PST@Border(0.1910,0.1260)
(0.1910,0.1460)

\rput(0.1910,0.0840){0}
\PST@Border(0.2855,0.1260)
(0.2855,0.1460)

\rput(0.2855,0.0840){200}
\PST@Border(0.3800,0.1260)
(0.3800,0.1460)

\rput(0.3800,0.0840){400}
\PST@Border(0.4745,0.1260)
(0.4745,0.1460)

\rput(0.4745,0.0840){600}
\PST@Border(0.5690,0.1260)
(0.5690,0.1460)

\rput(0.5690,0.0840){800}
\PST@Border(0.6635,0.1260)
(0.6635,0.1460)

\rput(0.6635,0.0840){1000}
\PST@Border(0.7580,0.1260)
(0.7580,0.1460)

\rput(0.7580,0.0840){1200}
\PST@Border(0.8525,0.1260)
(0.8525,0.1460)

\rput(0.8525,0.0840){1400}
\PST@Border(0.9470,0.1260)
(0.9470,0.1460)

\rput(0.9470,0.0840){1600}
\PST@Border(0.1910,0.9680)
(0.1910,0.1260)
(0.9470,0.1260)
(0.9470,0.9680)
(0.1910,0.9680)

\rput{L}(0.0420,0.5470){$\log(V_0-V)$}
\rput(0.5690,0.0210){$t (\unit{\mu s})$}
\PST@Solid(0.2288,0.4644)
(0.2288,0.7258)

\PST@Solid(0.2213,0.4644)
(0.2363,0.4644)

\PST@Solid(0.2213,0.7258)
(0.2363,0.7258)

\PST@Solid(0.2666,0.4862)
(0.2666,0.7475)

\PST@Solid(0.2591,0.4862)
(0.2741,0.4862)

\PST@Solid(0.2591,0.7475)
(0.2741,0.7475)

\PST@Solid(0.3044,0.5651)
(0.3044,0.8265)

\PST@Solid(0.2969,0.5651)
(0.3119,0.5651)

\PST@Solid(0.2969,0.8265)
(0.3119,0.8265)

\PST@Solid(0.3422,0.4450)
(0.3422,0.7064)

\PST@Solid(0.3347,0.4450)
(0.3497,0.4450)

\PST@Solid(0.3347,0.7064)
(0.3497,0.7064)

\PST@Solid(0.3800,0.4901)
(0.3800,0.7515)

\PST@Solid(0.3725,0.4901)
(0.3875,0.4901)

\PST@Solid(0.3725,0.7515)
(0.3875,0.7515)

\PST@Solid(0.4178,0.3860)
(0.4178,0.6474)

\PST@Solid(0.4103,0.3860)
(0.4253,0.3860)

\PST@Solid(0.4103,0.6474)
(0.4253,0.6474)

\PST@Solid(0.4556,0.3939)
(0.4556,0.6553)

\PST@Solid(0.4481,0.3939)
(0.4631,0.3939)

\PST@Solid(0.4481,0.6553)
(0.4631,0.6553)

\PST@Solid(0.4934,0.3234)
(0.4934,0.5848)

\PST@Solid(0.4859,0.3234)
(0.5009,0.3234)

\PST@Solid(0.4859,0.5848)
(0.5009,0.5848)

\PST@Solid(0.5312,0.1997)
(0.5312,0.4611)

\PST@Solid(0.5237,0.1997)
(0.5387,0.1997)

\PST@Solid(0.5237,0.4611)
(0.5387,0.4611)

\PST@Solid(0.5690,0.4129)
(0.5690,0.6743)

\PST@Solid(0.5615,0.4129)
(0.5765,0.4129)

\PST@Solid(0.5615,0.6743)
(0.5765,0.6743)

\PST@Solid(0.6068,0.5969)
(0.6068,0.8583)

\PST@Solid(0.5993,0.5969)
(0.6143,0.5969)

\PST@Solid(0.5993,0.8583)
(0.6143,0.8583)

\PST@Solid(0.6446,0.5591)
(0.6446,0.8205)

\PST@Solid(0.6371,0.5591)
(0.6521,0.5591)

\PST@Solid(0.6371,0.8205)
(0.6521,0.8205)

\PST@Solid(0.6824,0.6403)
(0.6824,0.9016)

\PST@Solid(0.6749,0.6403)
(0.6899,0.6403)

\PST@Solid(0.6749,0.9016)
(0.6899,0.9016)

\PST@Solid(0.7202,0.5037)
(0.7202,0.7651)

\PST@Solid(0.7127,0.5037)
(0.7277,0.5037)

\PST@Solid(0.7127,0.7651)
(0.7277,0.7651)

\PST@Solid(0.7580,0.4590)
(0.7580,0.7204)

\PST@Solid(0.7505,0.4590)
(0.7655,0.4590)

\PST@Solid(0.7505,0.7204)
(0.7655,0.7204)

\PST@Solid(0.7958,0.4185)
(0.7958,0.6799)

\PST@Solid(0.7883,0.4185)
(0.8033,0.4185)

\PST@Solid(0.7883,0.6799)
(0.8033,0.6799)

\PST@Solid(0.8336,0.4526)
(0.8336,0.7139)

\PST@Solid(0.8261,0.4526)
(0.8411,0.4526)

\PST@Solid(0.8261,0.7139)
(0.8411,0.7139)

\PST@Solid(0.8714,0.3821)
(0.8714,0.6435)

\PST@Solid(0.8639,0.3821)
(0.8789,0.3821)

\PST@Solid(0.8639,0.6435)
(0.8789,0.6435)

\PST@Solid(0.9092,0.5461)
(0.9092,0.8075)

\PST@Solid(0.9017,0.5461)
(0.9167,0.5461)

\PST@Solid(0.9017,0.8075)
(0.9167,0.8075)

\PST@Solid(0.9470,0.3667)
(0.9470,0.6281)

\PST@Solid(0.9395,0.3667)
(0.9545,0.3667)

\PST@Solid(0.9395,0.6281)
(0.9545,0.6281)

\PST@Diamond(0.2288,0.5951)
\PST@Diamond(0.2666,0.6169)
\PST@Diamond(0.3044,0.6958)
\PST@Diamond(0.3422,0.5757)
\PST@Diamond(0.3800,0.6208)
\PST@Diamond(0.4178,0.5167)
\PST@Diamond(0.4556,0.5246)
\PST@Diamond(0.4934,0.4541)
\PST@Diamond(0.5312,0.3304)
\PST@Diamond(0.5690,0.5436)
\PST@Diamond(0.6068,0.7276)
\PST@Diamond(0.6446,0.6898)
\PST@Diamond(0.6824,0.7709)
\PST@Diamond(0.7202,0.6344)
\PST@Diamond(0.7580,0.5897)
\PST@Diamond(0.7958,0.5492)
\PST@Diamond(0.8336,0.5833)
\PST@Diamond(0.8714,0.5128)
\PST@Diamond(0.9092,0.6768)
\PST@Diamond(0.9470,0.4974)
\PST@Border(0.1910,0.9680)
(0.1910,0.1260)
(0.9470,0.1260)
(0.9470,0.9680)
(0.1910,0.9680)

\catcode`@=12
\fi
\endpspicture

\end{figure}

\begin{figure}[p]\caption{Tempo in ascissa, logaritmo del potenziale del condensatore in ordinata.}
\centering
% GNUPLOT: LaTeX picture using PSTRICKS macros
% Define new PST objects, if not already defined
\ifx\PSTloaded\undefined
\def\PSTloaded{t}
\psset{arrowsize=.01 3.2 1.4 .3}
\psset{dotsize=.01}
\catcode`@=11

\newpsobject{PST@Border}{psline}{linewidth=.0015,linestyle=solid}
\newpsobject{PST@Axes}{psline}{linewidth=.0015,linestyle=dotted,dotsep=.004}
\newpsobject{PST@Solid}{psline}{linewidth=.0015,linestyle=solid}
\newpsobject{PST@Dashed}{psline}{linewidth=.0015,linestyle=dashed,dash=.01 .01}
\newpsobject{PST@Dotted}{psline}{linewidth=.0025,linestyle=dotted,dotsep=.008}
\newpsobject{PST@LongDash}{psline}{linewidth=.0015,linestyle=dashed,dash=.02 .01}
\newpsobject{PST@Diamond}{psdots}{linewidth=.001,linestyle=solid,dotstyle=square,dotangle=45}
\newpsobject{PST@Filldiamond}{psdots}{linewidth=.001,linestyle=solid,dotstyle=square*,dotangle=45}
\newpsobject{PST@Cross}{psdots}{linewidth=.001,linestyle=solid,dotstyle=+,dotangle=45}
\newpsobject{PST@Plus}{psdots}{linewidth=.001,linestyle=solid,dotstyle=+}
\newpsobject{PST@Square}{psdots}{linewidth=.001,linestyle=solid,dotstyle=square}
\newpsobject{PST@Circle}{psdots}{linewidth=.001,linestyle=solid,dotstyle=o}
\newpsobject{PST@Triangle}{psdots}{linewidth=.001,linestyle=solid,dotstyle=triangle}
\newpsobject{PST@Pentagon}{psdots}{linewidth=.001,linestyle=solid,dotstyle=pentagon}
\newpsobject{PST@Fillsquare}{psdots}{linewidth=.001,linestyle=solid,dotstyle=square*}
\newpsobject{PST@Fillcircle}{psdots}{linewidth=.001,linestyle=solid,dotstyle=*}
\newpsobject{PST@Filltriangle}{psdots}{linewidth=.001,linestyle=solid,dotstyle=triangle*}
\newpsobject{PST@Fillpentagon}{psdots}{linewidth=.001,linestyle=solid,dotstyle=pentagon*}
\newpsobject{PST@Arrow}{psline}{linewidth=.001,linestyle=solid}
\catcode`@=12

\fi
\psset{unit=5.0in,xunit=5.0in,yunit=3.0in}
\pspicture(0.000000,0.000000)(1.000000,1.000000)
\ifx\nofigs\undefined
\catcode`@=11

\PST@Border(0.1590,0.1260)
(0.1740,0.1260)

\rput[r](0.1430,0.1260){-2.0}
\PST@Border(0.1590,0.2463)
(0.1740,0.2463)

\rput[r](0.1430,0.2463){-1.5}
\PST@Border(0.1590,0.3666)
(0.1740,0.3666)

\rput[r](0.1430,0.3666){-1.0}
\PST@Border(0.1590,0.4869)
(0.1740,0.4869)

\rput[r](0.1430,0.4869){-0.5}
\PST@Border(0.1590,0.6071)
(0.1740,0.6071)

\rput[r](0.1430,0.6071){0.0}
\PST@Border(0.1590,0.7274)
(0.1740,0.7274)

\rput[r](0.1430,0.7274){0.5}
\PST@Border(0.1590,0.8477)
(0.1740,0.8477)

\rput[r](0.1430,0.8477){1.0}
\PST@Border(0.1590,0.9680)
(0.1740,0.9680)

\rput[r](0.1430,0.9680){1.5}
\PST@Border(0.1590,0.1260)
(0.1590,0.1460)

\rput(0.1590,0.0840){0}
\PST@Border(0.2575,0.1260)
(0.2575,0.1460)

\rput(0.2575,0.0840){200}
\PST@Border(0.3560,0.1260)
(0.3560,0.1460)

\rput(0.3560,0.0840){400}
\PST@Border(0.4545,0.1260)
(0.4545,0.1460)

\rput(0.4545,0.0840){600}
\PST@Border(0.5530,0.1260)
(0.5530,0.1460)

\rput(0.5530,0.0840){800}
\PST@Border(0.6515,0.1260)
(0.6515,0.1460)

\rput(0.6515,0.0840){1000}
\PST@Border(0.7500,0.1260)
(0.7500,0.1460)

\rput(0.7500,0.0840){1200}
\PST@Border(0.8485,0.1260)
(0.8485,0.1460)

\rput(0.8485,0.0840){1400}
\PST@Border(0.9470,0.1260)
(0.9470,0.1460)

\rput(0.9470,0.0840){1600}
\PST@Border(0.1590,0.9680)
(0.1590,0.1260)
(0.9470,0.1260)
(0.9470,0.9680)
(0.1590,0.9680)

\rput{L}(0.0420,0.5470){$\log(V)$}
\rput(0.5530,0.0210){$t (\unit{\mu s})$}
\PST@Diamond(0.1984,0.8566)
\PST@Diamond(0.2378,0.8217)
\PST@Diamond(0.2772,0.7856)
\PST@Diamond(0.3166,0.7512)
\PST@Diamond(0.3560,0.7141)
\PST@Diamond(0.3954,0.6776)
\PST@Diamond(0.4348,0.6428)
\PST@Diamond(0.4742,0.6071)
\PST@Diamond(0.5136,0.5764)
\PST@Diamond(0.5530,0.5308)
\PST@Diamond(0.5924,0.4952)
\PST@Diamond(0.6318,0.4607)
\PST@Diamond(0.6712,0.4245)
\PST@Diamond(0.7106,0.3891)
\PST@Diamond(0.7500,0.3532)
\PST@Diamond(0.7894,0.3175)
\PST@Diamond(0.8288,0.2793)
\PST@Diamond(0.8682,0.2429)
\PST@Diamond(0.9076,0.2101)
\PST@Diamond(0.9470,0.1722)
\PST@Dashed(0.1984,0.8585)
(0.1984,0.8585)
(0.2060,0.8516)
(0.2135,0.8446)
(0.2211,0.8377)
(0.2286,0.8308)
(0.2362,0.8238)
(0.2438,0.8169)
(0.2513,0.8100)
(0.2589,0.8031)
(0.2665,0.7961)
(0.2740,0.7892)
(0.2816,0.7823)
(0.2891,0.7753)
(0.2967,0.7684)
(0.3043,0.7615)
(0.3118,0.7545)
(0.3194,0.7476)
(0.3269,0.7407)
(0.3345,0.7338)
(0.3421,0.7268)
(0.3496,0.7199)
(0.3572,0.7130)
(0.3648,0.7060)
(0.3723,0.6991)
(0.3799,0.6922)
(0.3874,0.6852)
(0.3950,0.6783)
(0.4026,0.6714)
(0.4101,0.6644)
(0.4177,0.6575)
(0.4252,0.6506)
(0.4328,0.6437)
(0.4404,0.6367)
(0.4479,0.6298)
(0.4555,0.6229)
(0.4631,0.6159)
(0.4706,0.6090)
(0.4782,0.6021)
(0.4857,0.5951)
(0.4933,0.5882)
(0.5009,0.5813)
(0.5084,0.5743)
(0.5160,0.5674)
(0.5235,0.5605)
(0.5311,0.5536)
(0.5387,0.5466)
(0.5462,0.5397)
(0.5538,0.5328)
(0.5614,0.5258)
(0.5689,0.5189)
(0.5765,0.5120)
(0.5840,0.5050)
(0.5916,0.4981)
(0.5992,0.4912)
(0.6067,0.4842)
(0.6143,0.4773)
(0.6219,0.4704)
(0.6294,0.4635)
(0.6370,0.4565)
(0.6445,0.4496)
(0.6521,0.4427)
(0.6597,0.4357)
(0.6672,0.4288)
(0.6748,0.4219)
(0.6823,0.4149)
(0.6899,0.4080)
(0.6975,0.4011)
(0.7050,0.3942)
(0.7126,0.3872)
(0.7202,0.3803)
(0.7277,0.3734)
(0.7353,0.3664)
(0.7428,0.3595)
(0.7504,0.3526)
(0.7580,0.3456)
(0.7655,0.3387)
(0.7731,0.3318)
(0.7806,0.3248)
(0.7882,0.3179)
(0.7958,0.3110)
(0.8033,0.3041)
(0.8109,0.2971)
(0.8185,0.2902)
(0.8260,0.2833)
(0.8336,0.2763)
(0.8411,0.2694)
(0.8487,0.2625)
(0.8563,0.2555)
(0.8638,0.2486)
(0.8714,0.2417)
(0.8789,0.2347)
(0.8865,0.2278)
(0.8941,0.2209)
(0.9016,0.2140)
(0.9092,0.2070)
(0.9168,0.2001)
(0.9243,0.1932)
(0.9319,0.1862)
(0.9394,0.1793)
(0.9470,0.1724)

\PST@Border(0.1590,0.9680)
(0.1590,0.1260)
(0.9470,0.1260)
(0.9470,0.9680)
(0.1590,0.9680)

\catcode`@=12
\fi
\endpspicture

\end{figure}

\begin{figure}[p]\caption{Tempo in ascissa, logaritmo del potenziale del condensatore in ordinata. Grafico dei residui.}
\centering
% GNUPLOT: LaTeX picture using PSTRICKS macros
% Define new PST objects, if not already defined
\ifx\PSTloaded\undefined
\def\PSTloaded{t}
\psset{arrowsize=.01 3.2 1.4 .3}
\psset{dotsize=.01}
\catcode`@=11

\newpsobject{PST@Border}{psline}{linewidth=.0015,linestyle=solid}
\newpsobject{PST@Axes}{psline}{linewidth=.0015,linestyle=dotted,dotsep=.004}
\newpsobject{PST@Solid}{psline}{linewidth=.0015,linestyle=solid}
\newpsobject{PST@Dashed}{psline}{linewidth=.0015,linestyle=dashed,dash=.01 .01}
\newpsobject{PST@Dotted}{psline}{linewidth=.0025,linestyle=dotted,dotsep=.008}
\newpsobject{PST@LongDash}{psline}{linewidth=.0015,linestyle=dashed,dash=.02 .01}
\newpsobject{PST@Diamond}{psdots}{linewidth=.001,linestyle=solid,dotstyle=square,dotangle=45}
\newpsobject{PST@Filldiamond}{psdots}{linewidth=.001,linestyle=solid,dotstyle=square*,dotangle=45}
\newpsobject{PST@Cross}{psdots}{linewidth=.001,linestyle=solid,dotstyle=+,dotangle=45}
\newpsobject{PST@Plus}{psdots}{linewidth=.001,linestyle=solid,dotstyle=+}
\newpsobject{PST@Square}{psdots}{linewidth=.001,linestyle=solid,dotstyle=square}
\newpsobject{PST@Circle}{psdots}{linewidth=.001,linestyle=solid,dotstyle=o}
\newpsobject{PST@Triangle}{psdots}{linewidth=.001,linestyle=solid,dotstyle=triangle}
\newpsobject{PST@Pentagon}{psdots}{linewidth=.001,linestyle=solid,dotstyle=pentagon}
\newpsobject{PST@Fillsquare}{psdots}{linewidth=.001,linestyle=solid,dotstyle=square*}
\newpsobject{PST@Fillcircle}{psdots}{linewidth=.001,linestyle=solid,dotstyle=*}
\newpsobject{PST@Filltriangle}{psdots}{linewidth=.001,linestyle=solid,dotstyle=triangle*}
\newpsobject{PST@Fillpentagon}{psdots}{linewidth=.001,linestyle=solid,dotstyle=pentagon*}
\newpsobject{PST@Arrow}{psline}{linewidth=.001,linestyle=solid}
\catcode`@=12

\fi
\psset{unit=5.0in,xunit=5.0in,yunit=3.0in}
\pspicture(0.000000,0.000000)(1.000000,1.000000)
\ifx\nofigs\undefined
\catcode`@=11

\PST@Border(0.1750,0.1260)
(0.1900,0.1260)

\rput[r](0.1590,0.1260){-0.02}
\PST@Border(0.1750,0.2663)
(0.1900,0.2663)

\rput[r](0.1590,0.2663){-0.01}
\PST@Border(0.1750,0.4067)
(0.1900,0.4067)

\rput[r](0.1590,0.4067){0.00}
\PST@Border(0.1750,0.5470)
(0.1900,0.5470)

\rput[r](0.1590,0.5470){0.01}
\PST@Border(0.1750,0.6873)
(0.1900,0.6873)

\rput[r](0.1590,0.6873){0.02}
\PST@Border(0.1750,0.8277)
(0.1900,0.8277)

\rput[r](0.1590,0.8277){0.03}
\PST@Border(0.1750,0.9680)
(0.1900,0.9680)

\rput[r](0.1590,0.9680){0.04}
\PST@Border(0.1750,0.1260)
(0.1750,0.1460)

\rput(0.1750,0.0840){0}
\PST@Border(0.2715,0.1260)
(0.2715,0.1460)

\rput(0.2715,0.0840){200}
\PST@Border(0.3680,0.1260)
(0.3680,0.1460)

\rput(0.3680,0.0840){400}
\PST@Border(0.4645,0.1260)
(0.4645,0.1460)

\rput(0.4645,0.0840){600}
\PST@Border(0.5610,0.1260)
(0.5610,0.1460)

\rput(0.5610,0.0840){800}
\PST@Border(0.6575,0.1260)
(0.6575,0.1460)

\rput(0.6575,0.0840){1000}
\PST@Border(0.7540,0.1260)
(0.7540,0.1460)

\rput(0.7540,0.0840){1200}
\PST@Border(0.8505,0.1260)
(0.8505,0.1460)

\rput(0.8505,0.0840){1400}
\PST@Border(0.9470,0.1260)
(0.9470,0.1460)

\rput(0.9470,0.0840){1600}
\PST@Border(0.1750,0.9680)
(0.1750,0.1260)
(0.9470,0.1260)
(0.9470,0.9680)
(0.1750,0.9680)

\rput{L}(0.0420,0.5470){$\log(V)$}
\rput(0.5610,0.0210){$t (\unit{\mu s})$}
\PST@Solid(0.2136,0.1732)
(0.2136,0.4128)

\PST@Solid(0.2061,0.1732)
(0.2211,0.1732)

\PST@Solid(0.2061,0.4128)
(0.2211,0.4128)

\PST@Solid(0.2522,0.2486)
(0.2522,0.4882)

\PST@Solid(0.2447,0.2486)
(0.2597,0.2486)

\PST@Solid(0.2447,0.4882)
(0.2597,0.4882)

\PST@Solid(0.2908,0.2492)
(0.2908,0.4889)

\PST@Solid(0.2833,0.2492)
(0.2983,0.2492)

\PST@Solid(0.2833,0.4889)
(0.2983,0.4889)

\PST@Solid(0.3294,0.3476)
(0.3294,0.5872)

\PST@Solid(0.3219,0.3476)
(0.3369,0.3476)

\PST@Solid(0.3219,0.5872)
(0.3369,0.5872)

\PST@Solid(0.3680,0.2909)
(0.3680,0.5305)

\PST@Solid(0.3605,0.2909)
(0.3755,0.2909)

\PST@Solid(0.3605,0.5305)
(0.3755,0.5305)

\PST@Solid(0.4066,0.2641)
(0.4066,0.5037)

\PST@Solid(0.3991,0.2641)
(0.4141,0.2641)

\PST@Solid(0.3991,0.5037)
(0.4141,0.5037)

\PST@Solid(0.4452,0.3464)
(0.4452,0.5860)

\PST@Solid(0.4377,0.3464)
(0.4527,0.3464)

\PST@Solid(0.4377,0.5860)
(0.4527,0.5860)

\PST@Solid(0.4838,0.3701)
(0.4838,0.6097)

\PST@Solid(0.4763,0.3701)
(0.4913,0.3701)

\PST@Solid(0.4763,0.6097)
(0.4913,0.6097)

\PST@Solid(0.5224,0.6827)
(0.5224,0.9223)

\PST@Solid(0.5149,0.6827)
(0.5299,0.6827)

\PST@Solid(0.5149,0.9223)
(0.5299,0.9223)

\PST@Solid(0.5610,0.1282)
(0.5610,0.3678)

\PST@Solid(0.5535,0.1282)
(0.5685,0.1282)

\PST@Solid(0.5535,0.3678)
(0.5685,0.3678)

\PST@Solid(0.5996,0.1611)
(0.5996,0.4007)

\PST@Solid(0.5921,0.1611)
(0.6071,0.1611)

\PST@Solid(0.5921,0.4007)
(0.6071,0.4007)

\PST@Solid(0.6382,0.2526)
(0.6382,0.4922)

\PST@Solid(0.6307,0.2526)
(0.6457,0.2526)

\PST@Solid(0.6307,0.4922)
(0.6457,0.4922)

\PST@Solid(0.6768,0.2474)
(0.6768,0.4870)

\PST@Solid(0.6693,0.2474)
(0.6843,0.2474)

\PST@Solid(0.6693,0.4870)
(0.6843,0.4870)

\PST@Solid(0.7154,0.2903)
(0.7154,0.5299)

\PST@Solid(0.7079,0.2903)
(0.7229,0.2903)

\PST@Solid(0.7079,0.5299)
(0.7229,0.5299)

\PST@Solid(0.7540,0.3029)
(0.7540,0.5425)

\PST@Solid(0.7465,0.3029)
(0.7615,0.3029)

\PST@Solid(0.7465,0.5425)
(0.7615,0.5425)

\PST@Solid(0.7926,0.3266)
(0.7926,0.5662)

\PST@Solid(0.7851,0.3266)
(0.8001,0.3266)

\PST@Solid(0.7851,0.5662)
(0.8001,0.5662)

\PST@Solid(0.8312,0.2074)
(0.8312,0.4470)

\PST@Solid(0.8237,0.2074)
(0.8387,0.2074)

\PST@Solid(0.8237,0.4470)
(0.8387,0.4470)

\PST@Solid(0.8698,0.1871)
(0.8698,0.4267)

\PST@Solid(0.8623,0.1871)
(0.8773,0.1871)

\PST@Solid(0.8623,0.4267)
(0.8773,0.4267)

\PST@Solid(0.9084,0.3833)
(0.9084,0.6229)

\PST@Solid(0.9009,0.3833)
(0.9159,0.3833)

\PST@Solid(0.9009,0.6229)
(0.9159,0.6229)

\PST@Solid(0.9470,0.2778)
(0.9470,0.5174)

\PST@Solid(0.9395,0.2778)
(0.9545,0.2778)

\PST@Solid(0.9395,0.5174)
(0.9545,0.5174)

\PST@Diamond(0.2136,0.2930)
\PST@Diamond(0.2522,0.3684)
\PST@Diamond(0.2908,0.3690)
\PST@Diamond(0.3294,0.4674)
\PST@Diamond(0.3680,0.4107)
\PST@Diamond(0.4066,0.3839)
\PST@Diamond(0.4452,0.4662)
\PST@Diamond(0.4838,0.4899)
\PST@Diamond(0.5224,0.8025)
\PST@Diamond(0.5610,0.2480)
\PST@Diamond(0.5996,0.2809)
\PST@Diamond(0.6382,0.3724)
\PST@Diamond(0.6768,0.3672)
\PST@Diamond(0.7154,0.4101)
\PST@Diamond(0.7540,0.4227)
\PST@Diamond(0.7926,0.4464)
\PST@Diamond(0.8312,0.3272)
\PST@Diamond(0.8698,0.3069)
\PST@Diamond(0.9084,0.5031)
\PST@Diamond(0.9470,0.3976)
\PST@Border(0.1750,0.9680)
(0.1750,0.1260)
(0.9470,0.1260)
(0.9470,0.9680)
(0.1750,0.9680)

\catcode`@=12
\fi
\endpspicture

\end{figure}
\end{document}
