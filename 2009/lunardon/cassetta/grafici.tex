\documentclass[italian,a4paper]{article}
\usepackage[tight,nice]{units}
\usepackage{babel,amsmath,amssymb,amsthm,graphicx,url,gensymb}
\usepackage[text={5.5in,9in},centering]{geometry}
\usepackage[utf8x]{inputenc}
%\usepackage[T1]{fontenc}
\usepackage{ae,aecompl}
\usepackage[footnotesize,bf]{caption}
\usepackage[usenames]{color}
\include{pstricks}
\frenchspacing
\pagestyle{plain}
%------------- eliminare prime e ultime linee isolate
\clubpenalty=9999%
\widowpenalty=9999
%--- definizione numerazioni
\renewcommand{\theequation}{\thesection.\arabic{equation}}
\renewcommand{\thefigure}{\arabic{figure}}
\renewcommand{\thetable}{\arabic{table}}
\addto\captionsitalian{%
  \renewcommand{\figurename}%
{Grafico}%
}
%
%------------- ridefinizione simbolo per elenchi puntati: en dash
%\renewcommand{\labelitemi}{\textbf{--}}
\renewcommand{\labelenumi}{\textbf{\arabic{enumi}.}}
\setlength{\abovecaptionskip}{\baselineskip}   % 0.5cm as an example
\setlength{\floatsep}{2\baselineskip}
\setlength{\belowcaptionskip}{\baselineskip}   % 0.5cm as an example
%--------- comandi insiemi numeri complessi, naturali, reali e altre abbreviazioni
\renewcommand{\leq}{\leqslant}
%--------- porzione dedicata ai float in una pagina:
\renewcommand{\textfraction}{0.05}
\renewcommand{\topfraction}{0.95}
\renewcommand{\bottomfraction}{0.95}
\renewcommand{\floatpagefraction}{0.35}
\setcounter{totalnumber}{5}
%---------
%
%---------
\begin{document}
\title{Grafici sulle misure relative ai componenti della cassetta multifunzionale}
\author{\normalsize Ilaria Brivio (582116)\\%
\normalsize \url{brivio.ilaria@tiscali.it}%
\and %
\normalsize Matteo Abis (584206)\\ %
\normalsize \url{webmaster@latinblog.org}}
\date{\today}
\maketitle
%------------------
\begin{figure}[p]\caption{Resistenza in Ohm misurata rispetto ai giri sul potenziometro.}
\centering
% GNUPLOT: LaTeX picture using PSTRICKS macros
% Define new PST objects, if not already defined
\ifx\PSTloaded\undefined
\def\PSTloaded{t}
\psset{arrowsize=.01 3.2 1.4 .3}
\psset{dotsize=.125}
\catcode`@=11

\newpsobject{PST@Border}{psline}{linewidth=.0015,linestyle=solid}
\newpsobject{PST@Axes}{psline}{linewidth=.0015,linestyle=dotted,dotsep=.004}
\newpsobject{PST@Solid}{psline}{linewidth=.0015,linestyle=solid}
\newpsobject{PST@Dashed}{psline}{linewidth=.0015,linestyle=dashed,dash=.01 .01}
\newpsobject{PST@Dotted}{psline}{linewidth=.0025,linestyle=dotted,dotsep=.008}
\newpsobject{PST@LongDash}{psline}{linewidth=.0015,linestyle=dashed,dash=.02 .01}
\newpsobject{PST@Diamond}{psdots}{linewidth=.001,linestyle=solid,dotstyle=square,dotangle=45}
\newpsobject{PST@Filldiamond}{psdots}{linewidth=.001,linestyle=solid,dotstyle=square*,dotangle=45}
\newpsobject{PST@Cross}{psdots}{linewidth=.001,linestyle=solid,dotstyle=+,dotangle=45}
\newpsobject{PST@Plus}{psdots}{linewidth=.001,linestyle=solid,dotstyle=+}
\newpsobject{PST@Square}{psdots}{linewidth=.001,linestyle=solid,dotstyle=square}
\newpsobject{PST@Circle}{psdots}{linewidth=.001,linestyle=solid,dotstyle=o}
\newpsobject{PST@Triangle}{psdots}{linewidth=.001,linestyle=solid,dotstyle=triangle}
\newpsobject{PST@Pentagon}{psdots}{linewidth=.001,linestyle=solid,dotstyle=pentagon}
\newpsobject{PST@Fillsquare}{psdots}{linewidth=.001,linestyle=solid,dotstyle=square*}
\newpsobject{PST@Fillcircle}{psdots}{linewidth=.001,linestyle=solid,dotstyle=*}
\newpsobject{PST@Filltriangle}{psdots}{linewidth=.001,linestyle=solid,dotstyle=triangle*}
\newpsobject{PST@Fillpentagon}{psdots}{linewidth=.001,linestyle=solid,dotstyle=pentagon*}
\newpsobject{PST@Arrow}{psline}{linewidth=.001,linestyle=solid}
\catcode`@=12

\fi
\psset{unit=5.0in,xunit=5.0in,yunit=3.0in}
\pspicture(0.000000,0.000000)(1.000000,1.000000)
\ifx\nofigs\undefined
\catcode`@=11

\PST@Border(0.2070,0.1260)
(0.2220,0.1260)

\rput[r](0.1910,0.1260){-0.0002}
\PST@Border(0.2070,0.2025)
(0.2220,0.2025)

\rput[r](0.1910,0.2025){0.0000}
\PST@Border(0.2070,0.2791)
(0.2220,0.2791)

\rput[r](0.1910,0.2791){0.0002}
\PST@Border(0.2070,0.3556)
(0.2220,0.3556)

\rput[r](0.1910,0.3556){0.0004}
\PST@Border(0.2070,0.4322)
(0.2220,0.4322)

\rput[r](0.1910,0.4322){0.0006}
\PST@Border(0.2070,0.5087)
(0.2220,0.5087)

\rput[r](0.1910,0.5087){0.0008}
\PST@Border(0.2070,0.5853)
(0.2220,0.5853)

\rput[r](0.1910,0.5853){0.0010}
\PST@Border(0.2070,0.6618)
(0.2220,0.6618)

\rput[r](0.1910,0.6618){0.0012}
\PST@Border(0.2070,0.7384)
(0.2220,0.7384)

\rput[r](0.1910,0.7384){0.0014}
\PST@Border(0.2070,0.8149)
(0.2220,0.8149)

\rput[r](0.1910,0.8149){0.0016}
\PST@Border(0.2070,0.8915)
(0.2220,0.8915)

\rput[r](0.1910,0.8915){0.0018}
\PST@Border(0.2070,0.9680)
(0.2220,0.9680)

\rput[r](0.1910,0.9680){0.0020}
\PST@Border(0.2070,0.1260)
(0.2070,0.1460)

\rput(0.2070,0.0840){ 0}
\PST@Border(0.3009,0.1260)
(0.3009,0.1460)

\rput(0.3009,0.0840){ 5}
\PST@Border(0.3948,0.1260)
(0.3948,0.1460)

\rput(0.3948,0.0840){ 10}
\PST@Border(0.4886,0.1260)
(0.4886,0.1460)

\rput(0.4886,0.0840){ 15}
\PST@Border(0.5825,0.1260)
(0.5825,0.1460)

\rput(0.5825,0.0840){ 20}
\PST@Border(0.6764,0.1260)
(0.6764,0.1460)

\rput(0.6764,0.0840){ 25}
\PST@Border(0.7703,0.1260)
(0.7703,0.1460)

\rput(0.7703,0.0840){ 30}
\PST@Border(0.8641,0.1260)
(0.8641,0.1460)

\rput(0.8641,0.0840){ 35}
\PST@Border(0.9580,0.1260)
(0.9580,0.1460)

\rput(0.9580,0.0840){ 40}
\PST@Border(0.2070,0.9680)
(0.2070,0.1260)
(0.9580,0.1260)
(0.9580,0.9680)
(0.2070,0.9680)

\rput{L}(0.0420,0.5470){$\Delta x$}
\rput(0.5825,0.0210){$\Delta F$}
\rput[r](0.6710,0.9270){punti allungamento}
\PST@Circle(0.2070,0.2025)
\PST@Circle(0.2806,0.2867)
\PST@Circle(0.3542,0.3633)
\PST@Circle(0.4280,0.4437)
\PST@Circle(0.5016,0.5317)
\PST@Circle(0.5752,0.5929)
\PST@Circle(0.6488,0.6924)
\PST@Circle(0.7226,0.7690)
\PST@Circle(0.7962,0.8532)
\PST@Circle(0.8698,0.9336)
\PST@Circle(0.7265,0.9270)
\rput[r](0.6710,0.8850){punti accorciamento}
\PST@Cross(0.2070,0.2025)
\PST@Cross(0.2806,0.2867)
\PST@Cross(0.3542,0.3709)
\PST@Cross(0.4280,0.4551)
\PST@Cross(0.5016,0.5355)
\PST@Cross(0.5752,0.6159)
\PST@Cross(0.6488,0.6963)
\PST@Cross(0.7226,0.7728)
\PST@Cross(0.7962,0.8608)
\PST@Cross(0.8698,0.9336)
\PST@Cross(0.7265,0.8850)
\rput[r](0.6710,0.8430){interpolazione allungamento}
\PST@Dashed(0.6870,0.8430)
(0.7660,0.8430)

\PST@Dashed(0.2070,0.2020)
(0.2070,0.2020)
(0.2137,0.2094)
(0.2204,0.2168)
(0.2271,0.2241)
(0.2338,0.2315)
(0.2405,0.2389)
(0.2472,0.2462)
(0.2539,0.2536)
(0.2606,0.2610)
(0.2673,0.2684)
(0.2739,0.2757)
(0.2806,0.2831)
(0.2873,0.2905)
(0.2940,0.2978)
(0.3007,0.3052)
(0.3074,0.3126)
(0.3141,0.3200)
(0.3208,0.3273)
(0.3275,0.3347)
(0.3342,0.3421)
(0.3409,0.3494)
(0.3476,0.3568)
(0.3543,0.3642)
(0.3610,0.3716)
(0.3677,0.3789)
(0.3744,0.3863)
(0.3811,0.3937)
(0.3878,0.4010)
(0.3944,0.4084)
(0.4011,0.4158)
(0.4078,0.4232)
(0.4145,0.4305)
(0.4212,0.4379)
(0.4279,0.4453)
(0.4346,0.4526)
(0.4413,0.4600)
(0.4480,0.4674)
(0.4547,0.4748)
(0.4614,0.4821)
(0.4681,0.4895)
(0.4748,0.4969)
(0.4815,0.5042)
(0.4882,0.5116)
(0.4949,0.5190)
(0.5016,0.5264)
(0.5083,0.5337)
(0.5149,0.5411)
(0.5216,0.5485)
(0.5283,0.5558)
(0.5350,0.5632)
(0.5417,0.5706)
(0.5484,0.5780)
(0.5551,0.5853)
(0.5618,0.5927)
(0.5685,0.6001)
(0.5752,0.6074)
(0.5819,0.6148)
(0.5886,0.6222)
(0.5953,0.6296)
(0.6020,0.6369)
(0.6087,0.6443)
(0.6154,0.6517)
(0.6221,0.6590)
(0.6288,0.6664)
(0.6354,0.6738)
(0.6421,0.6812)
(0.6488,0.6885)
(0.6555,0.6959)
(0.6622,0.7033)
(0.6689,0.7106)
(0.6756,0.7180)
(0.6823,0.7254)
(0.6890,0.7328)
(0.6957,0.7401)
(0.7024,0.7475)
(0.7091,0.7549)
(0.7158,0.7622)
(0.7225,0.7696)
(0.7292,0.7770)
(0.7359,0.7844)
(0.7426,0.7917)
(0.7493,0.7991)
(0.7560,0.8065)
(0.7626,0.8138)
(0.7693,0.8212)
(0.7760,0.8286)
(0.7827,0.8360)
(0.7894,0.8433)
(0.7961,0.8507)
(0.8028,0.8581)
(0.8095,0.8654)
(0.8162,0.8728)
(0.8229,0.8802)
(0.8296,0.8876)
(0.8363,0.8949)
(0.8430,0.9023)
(0.8497,0.9097)
(0.8564,0.9170)
(0.8631,0.9244)
(0.8698,0.9318)

\rput[r](0.6710,0.8010){interpolazione accorciamento}
\PST@Dotted(0.6870,0.8010)
(0.7660,0.8010)

\PST@Dotted(0.2070,0.2073)
(0.2070,0.2073)
(0.2137,0.2147)
(0.2204,0.2221)
(0.2271,0.2295)
(0.2338,0.2369)
(0.2405,0.2442)
(0.2472,0.2516)
(0.2539,0.2590)
(0.2606,0.2664)
(0.2673,0.2738)
(0.2739,0.2812)
(0.2806,0.2886)
(0.2873,0.2960)
(0.2940,0.3034)
(0.3007,0.3107)
(0.3074,0.3181)
(0.3141,0.3255)
(0.3208,0.3329)
(0.3275,0.3403)
(0.3342,0.3477)
(0.3409,0.3551)
(0.3476,0.3625)
(0.3543,0.3698)
(0.3610,0.3772)
(0.3677,0.3846)
(0.3744,0.3920)
(0.3811,0.3994)
(0.3878,0.4068)
(0.3944,0.4142)
(0.4011,0.4216)
(0.4078,0.4290)
(0.4145,0.4363)
(0.4212,0.4437)
(0.4279,0.4511)
(0.4346,0.4585)
(0.4413,0.4659)
(0.4480,0.4733)
(0.4547,0.4807)
(0.4614,0.4881)
(0.4681,0.4954)
(0.4748,0.5028)
(0.4815,0.5102)
(0.4882,0.5176)
(0.4949,0.5250)
(0.5016,0.5324)
(0.5083,0.5398)
(0.5149,0.5472)
(0.5216,0.5546)
(0.5283,0.5619)
(0.5350,0.5693)
(0.5417,0.5767)
(0.5484,0.5841)
(0.5551,0.5915)
(0.5618,0.5989)
(0.5685,0.6063)
(0.5752,0.6137)
(0.5819,0.6210)
(0.5886,0.6284)
(0.5953,0.6358)
(0.6020,0.6432)
(0.6087,0.6506)
(0.6154,0.6580)
(0.6221,0.6654)
(0.6288,0.6728)
(0.6354,0.6802)
(0.6421,0.6875)
(0.6488,0.6949)
(0.6555,0.7023)
(0.6622,0.7097)
(0.6689,0.7171)
(0.6756,0.7245)
(0.6823,0.7319)
(0.6890,0.7393)
(0.6957,0.7466)
(0.7024,0.7540)
(0.7091,0.7614)
(0.7158,0.7688)
(0.7225,0.7762)
(0.7292,0.7836)
(0.7359,0.7910)
(0.7426,0.7984)
(0.7493,0.8058)
(0.7560,0.8131)
(0.7626,0.8205)
(0.7693,0.8279)
(0.7760,0.8353)
(0.7827,0.8427)
(0.7894,0.8501)
(0.7961,0.8575)
(0.8028,0.8649)
(0.8095,0.8722)
(0.8162,0.8796)
(0.8229,0.8870)
(0.8296,0.8944)
(0.8363,0.9018)
(0.8430,0.9092)
(0.8497,0.9166)
(0.8564,0.9240)
(0.8631,0.9314)
(0.8698,0.9387)

\PST@Border(0.2070,0.9680)
(0.2070,0.1260)
(0.9580,0.1260)
(0.9580,0.9680)
(0.2070,0.9680)

\catcode`@=12
\fi
\endpspicture

\end{figure}
\begin{figure}[p]\caption{Grafico dei residui, con errore strumentale.}
\centering
% GNUPLOT: LaTeX picture using PSTRICKS macros
% Define new PST objects, if not already defined
\ifx\PSTloaded\undefined
\def\PSTloaded{t}
\psset{arrowsize=.01 3.2 1.4 .3}
\psset{dotsize=.08}
\catcode`@=11

\newpsobject{PST@Border}{psline}{linewidth=.0015,linestyle=solid}
\newpsobject{PST@Axes}{psline}{linewidth=.0015,linestyle=dotted,dotsep=.004}
\newpsobject{PST@Solid}{psline}{linewidth=.0015,linestyle=solid}
\newpsobject{PST@Dashed}{psline}{linewidth=.0015,linestyle=dashed,dash=.01 .01}
\newpsobject{PST@Dotted}{psline}{linewidth=.0025,linestyle=dotted,dotsep=.008}
\newpsobject{PST@LongDash}{psline}{linewidth=.0015,linestyle=dashed,dash=.02 .01}
\newpsobject{PST@Diamond}{psdots}{linewidth=.001,linestyle=solid,dotstyle=square,dotangle=45}
\newpsobject{PST@Filldiamond}{psdots}{linewidth=.001,linestyle=solid,dotstyle=square*,dotangle=45}
\newpsobject{PST@Cross}{psdots}{linewidth=.001,linestyle=solid,dotstyle=+,dotangle=45}
\newpsobject{PST@Plus}{psdots}{linewidth=.001,linestyle=solid,dotstyle=+}
\newpsobject{PST@Square}{psdots}{linewidth=.001,linestyle=solid,dotstyle=square}
\newpsobject{PST@Circle}{psdots}{linewidth=.001,linestyle=solid,dotstyle=o}
\newpsobject{PST@Triangle}{psdots}{linewidth=.001,linestyle=solid,dotstyle=triangle}
\newpsobject{PST@Pentagon}{psdots}{linewidth=.001,linestyle=solid,dotstyle=pentagon}
\newpsobject{PST@Fillsquare}{psdots}{linewidth=.001,linestyle=solid,dotstyle=square*}
\newpsobject{PST@Fillcircle}{psdots}{linewidth=.001,linestyle=solid,dotstyle=*}
\newpsobject{PST@Filltriangle}{psdots}{linewidth=.001,linestyle=solid,dotstyle=triangle*}
\newpsobject{PST@Fillpentagon}{psdots}{linewidth=.001,linestyle=solid,dotstyle=pentagon*}
\newpsobject{PST@Arrow}{psline}{linewidth=.001,linestyle=solid}
\catcode`@=12

\fi
\psset{unit=5.0in,xunit=5.0in,yunit=3.0in}
\pspicture(0.000000,0.000000)(1.000000,1.000000)
\ifx\nofigs\undefined
\catcode`@=11

\PST@Border(0.1170,0.0840)
(0.1320,0.0840)

\rput[r](0.1010,0.0840){-1.5}
\PST@Border(0.1170,0.2313)
(0.1320,0.2313)

\rput[r](0.1010,0.2313){-1.0}
\PST@Border(0.1170,0.3787)
(0.1320,0.3787)

\rput[r](0.1010,0.3787){-0.5}
\PST@Border(0.1170,0.5260)
(0.1320,0.5260)

\rput[r](0.1010,0.5260){0.0}
\PST@Border(0.1170,0.6733)
(0.1320,0.6733)

\rput[r](0.1010,0.6733){0.5}
\PST@Border(0.1170,0.8207)
(0.1320,0.8207)

\rput[r](0.1010,0.8207){1.0}
\PST@Border(0.1170,0.9680)
(0.1320,0.9680)

\rput[r](0.1010,0.9680){1.5}
\PST@Border(0.1170,0.0840)
(0.1170,0.1040)

\rput(0.1170,0.0420){0.0}
\PST@Border(0.2699,0.0840)
(0.2699,0.1040)

\rput(0.2699,0.0420){1.0}
\PST@Border(0.4228,0.0840)
(0.4228,0.1040)

\rput(0.4228,0.0420){2.0}
\PST@Border(0.5757,0.0840)
(0.5757,0.1040)

\rput(0.5757,0.0420){3.0}
\PST@Border(0.7286,0.0840)
(0.7286,0.1040)

\rput(0.7286,0.0420){4.0}
\PST@Border(0.8815,0.0840)
(0.8815,0.1040)

\rput(0.8815,0.0420){5.0}
\PST@Border(0.1170,0.9680)
(0.1170,0.0840)
(0.9580,0.0840)
(0.9580,0.9680)
(0.1170,0.9680)

\PST@Solid(0.1935,0.5304)
(0.1935,0.6483)

\PST@Solid(0.1860,0.5304)
(0.2010,0.5304)

\PST@Solid(0.1860,0.6483)
(0.2010,0.6483)

\PST@Solid(0.2699,0.5348)
(0.2699,0.7116)

\PST@Solid(0.2624,0.5348)
(0.2774,0.5348)

\PST@Solid(0.2624,0.7116)
(0.2774,0.7116)

\PST@Solid(0.3464,0.2741)
(0.3464,0.5098)

\PST@Solid(0.3389,0.2741)
(0.3539,0.2741)

\PST@Solid(0.3389,0.5098)
(0.3539,0.5098)

\PST@Solid(0.4228,0.3374)
(0.4228,0.6321)

\PST@Solid(0.4153,0.3374)
(0.4303,0.3374)

\PST@Solid(0.4153,0.6321)
(0.4303,0.6321)

\PST@Solid(0.4993,0.3713)
(0.4993,0.7838)

\PST@Solid(0.4918,0.3713)
(0.5068,0.3713)

\PST@Solid(0.4918,0.7838)
(0.5068,0.7838)

\PST@Solid(0.5757,0.2873)
(0.5757,0.7588)

\PST@Solid(0.5682,0.2873)
(0.5832,0.2873)

\PST@Solid(0.5682,0.7588)
(0.5832,0.7588)

\PST@Solid(0.6522,0.2917)
(0.6522,0.8221)

\PST@Solid(0.6447,0.2917)
(0.6597,0.2917)

\PST@Solid(0.6447,0.8221)
(0.6597,0.8221)

\PST@Solid(0.7286,0.2078)
(0.7286,0.8560)

\PST@Solid(0.7211,0.2078)
(0.7361,0.2078)

\PST@Solid(0.7211,0.8560)
(0.7361,0.8560)

\PST@Solid(0.8051,0.2122)
(0.8051,0.9194)

\PST@Solid(0.7976,0.2122)
(0.8126,0.2122)

\PST@Solid(0.7976,0.9194)
(0.8126,0.9194)

\PST@Solid(0.8815,0.1577)
(0.8815,0.9238)

\PST@Solid(0.8740,0.1577)
(0.8890,0.1577)

\PST@Solid(0.8740,0.9238)
(0.8890,0.9238)

\PST@Diamond(0.1935,0.5894)
\PST@Diamond(0.2699,0.6232)
\PST@Diamond(0.3464,0.3919)
\PST@Diamond(0.4228,0.4847)
\PST@Diamond(0.4993,0.5776)
\PST@Diamond(0.5757,0.5231)
\PST@Diamond(0.6522,0.5569)
\PST@Diamond(0.7286,0.5319)
\PST@Diamond(0.8051,0.5658)
\PST@Diamond(0.8815,0.5407)
\PST@Border(0.1170,0.9680)
(0.1170,0.0840)
(0.9580,0.0840)
(0.9580,0.9680)
(0.1170,0.9680)

\catcode`@=12
\fi
\endpspicture

\end{figure}
\begin{figure}[p]\caption{Grafico dei residui, con errore a posteriori.}
\centering
% GNUPLOT: LaTeX picture using PSTRICKS macros
% Define new PST objects, if not already defined
\ifx\PSTloaded\undefined
\def\PSTloaded{t}
\psset{arrowsize=.01 3.2 1.4 .3}
\psset{dotsize=.08}
\catcode`@=11

\newpsobject{PST@Border}{psline}{linewidth=.0015,linestyle=solid}
\newpsobject{PST@Axes}{psline}{linewidth=.0015,linestyle=dotted,dotsep=.004}
\newpsobject{PST@Solid}{psline}{linewidth=.0015,linestyle=solid}
\newpsobject{PST@Dashed}{psline}{linewidth=.0015,linestyle=dashed,dash=.01 .01}
\newpsobject{PST@Dotted}{psline}{linewidth=.0025,linestyle=dotted,dotsep=.008}
\newpsobject{PST@LongDash}{psline}{linewidth=.0015,linestyle=dashed,dash=.02 .01}
\newpsobject{PST@Diamond}{psdots}{linewidth=.001,linestyle=solid,dotstyle=square,dotangle=45}
\newpsobject{PST@Filldiamond}{psdots}{linewidth=.001,linestyle=solid,dotstyle=square*,dotangle=45}
\newpsobject{PST@Cross}{psdots}{linewidth=.001,linestyle=solid,dotstyle=+,dotangle=45}
\newpsobject{PST@Plus}{psdots}{linewidth=.001,linestyle=solid,dotstyle=+}
\newpsobject{PST@Square}{psdots}{linewidth=.001,linestyle=solid,dotstyle=square}
\newpsobject{PST@Circle}{psdots}{linewidth=.001,linestyle=solid,dotstyle=o}
\newpsobject{PST@Triangle}{psdots}{linewidth=.001,linestyle=solid,dotstyle=triangle}
\newpsobject{PST@Pentagon}{psdots}{linewidth=.001,linestyle=solid,dotstyle=pentagon}
\newpsobject{PST@Fillsquare}{psdots}{linewidth=.001,linestyle=solid,dotstyle=square*}
\newpsobject{PST@Fillcircle}{psdots}{linewidth=.001,linestyle=solid,dotstyle=*}
\newpsobject{PST@Filltriangle}{psdots}{linewidth=.001,linestyle=solid,dotstyle=triangle*}
\newpsobject{PST@Fillpentagon}{psdots}{linewidth=.001,linestyle=solid,dotstyle=pentagon*}
\newpsobject{PST@Arrow}{psline}{linewidth=.001,linestyle=solid}
\catcode`@=12

\fi
\psset{unit=5.0in,xunit=5.0in,yunit=3.0in}
\pspicture(0.000000,0.000000)(1.000000,1.000000)
\ifx\nofigs\undefined
\catcode`@=11

\PST@Border(0.1170,0.0840)
(0.1320,0.0840)

\rput[r](0.1010,0.0840){-0.8}
\PST@Border(0.1170,0.2103)
(0.1320,0.2103)

\rput[r](0.1010,0.2103){-0.6}
\PST@Border(0.1170,0.3366)
(0.1320,0.3366)

\rput[r](0.1010,0.3366){-0.4}
\PST@Border(0.1170,0.4629)
(0.1320,0.4629)

\rput[r](0.1010,0.4629){-0.2}
\PST@Border(0.1170,0.5891)
(0.1320,0.5891)

\rput[r](0.1010,0.5891){0.0}
\PST@Border(0.1170,0.7154)
(0.1320,0.7154)

\rput[r](0.1010,0.7154){0.2}
\PST@Border(0.1170,0.8417)
(0.1320,0.8417)

\rput[r](0.1010,0.8417){0.4}
\PST@Border(0.1170,0.9680)
(0.1320,0.9680)

\rput[r](0.1010,0.9680){0.6}
\PST@Border(0.1170,0.0840)
(0.1170,0.1040)

\rput(0.1170,0.0420){0.0}
\PST@Border(0.2699,0.0840)
(0.2699,0.1040)

\rput(0.2699,0.0420){1.0}
\PST@Border(0.4228,0.0840)
(0.4228,0.1040)

\rput(0.4228,0.0420){2.0}
\PST@Border(0.5757,0.0840)
(0.5757,0.1040)

\rput(0.5757,0.0420){3.0}
\PST@Border(0.7286,0.0840)
(0.7286,0.1040)

\rput(0.7286,0.0420){4.0}
\PST@Border(0.8815,0.0840)
(0.8815,0.1040)

\rput(0.8815,0.0420){5.0}
\PST@Border(0.1170,0.9680)
(0.1170,0.0840)
(0.9580,0.0840)
(0.9580,0.9680)
(0.1170,0.9680)

\PST@Solid(0.1935,0.5760)
(0.1935,0.8738)

\PST@Solid(0.1860,0.5760)
(0.2010,0.5760)

\PST@Solid(0.1860,0.8738)
(0.2010,0.8738)

\PST@Solid(0.2699,0.6487)
(0.2699,0.9464)

\PST@Solid(0.2624,0.6487)
(0.2774,0.6487)

\PST@Solid(0.2624,0.9464)
(0.2774,0.9464)

\PST@Solid(0.3464,0.1530)
(0.3464,0.4507)

\PST@Solid(0.3389,0.1530)
(0.3539,0.1530)

\PST@Solid(0.3389,0.4507)
(0.3539,0.4507)

\PST@Solid(0.4228,0.3519)
(0.4228,0.6496)

\PST@Solid(0.4153,0.3519)
(0.4303,0.3519)

\PST@Solid(0.4153,0.6496)
(0.4303,0.6496)

\PST@Solid(0.4993,0.5508)
(0.4993,0.8485)

\PST@Solid(0.4918,0.5508)
(0.5068,0.5508)

\PST@Solid(0.4918,0.8485)
(0.5068,0.8485)

\PST@Solid(0.5757,0.4340)
(0.5757,0.7317)

\PST@Solid(0.5682,0.4340)
(0.5832,0.4340)

\PST@Solid(0.5682,0.7317)
(0.5832,0.7317)

\PST@Solid(0.6522,0.5066)
(0.6522,0.8043)

\PST@Solid(0.6447,0.5066)
(0.6597,0.5066)

\PST@Solid(0.6447,0.8043)
(0.6597,0.8043)

\PST@Solid(0.7286,0.4529)
(0.7286,0.7506)

\PST@Solid(0.7211,0.4529)
(0.7361,0.4529)

\PST@Solid(0.7211,0.7506)
(0.7361,0.7506)

\PST@Solid(0.8051,0.5255)
(0.8051,0.8232)

\PST@Solid(0.7976,0.5255)
(0.8126,0.5255)

\PST@Solid(0.7976,0.8232)
(0.8126,0.8232)

\PST@Solid(0.8815,0.4719)
(0.8815,0.7696)

\PST@Solid(0.8740,0.4719)
(0.8890,0.4719)

\PST@Solid(0.8740,0.7696)
(0.8890,0.7696)

\PST@Diamond(0.1935,0.7249)
\PST@Diamond(0.2699,0.7975)
\PST@Diamond(0.3464,0.3018)
\PST@Diamond(0.4228,0.5007)
\PST@Diamond(0.4993,0.6996)
\PST@Diamond(0.5757,0.5828)
\PST@Diamond(0.6522,0.6554)
\PST@Diamond(0.7286,0.6018)
\PST@Diamond(0.8051,0.6744)
\PST@Diamond(0.8815,0.6207)
\PST@Border(0.1170,0.9680)
(0.1170,0.0840)
(0.9580,0.0840)
(0.9580,0.9680)
(0.1170,0.9680)

\catcode`@=12
\fi
\endpspicture

\end{figure}
\begin{figure}[p]\caption{Corrente in mA in ascissa, differenza di potenziale in ordinata (V) misurata ai capi della resistenza R1. Le barre di errore sono più piccole delle dimensioni dei punti in questa scala.}
\centering
% GNUPLOT: LaTeX picture using PSTRICKS macros
% Define new PST objects, if not already defined
\ifx\PSTloaded\undefined
\def\PSTloaded{t}
\psset{arrowsize=.01 3.2 1.4 .3}
\psset{dotsize=.08}
\catcode`@=11

\newpsobject{PST@Border}{psline}{linewidth=.0015,linestyle=solid}
\newpsobject{PST@Axes}{psline}{linewidth=.0015,linestyle=dotted,dotsep=.004}
\newpsobject{PST@Solid}{psline}{linewidth=.0015,linestyle=solid}
\newpsobject{PST@Dashed}{psline}{linewidth=.0015,linestyle=dashed,dash=.01 .01}
\newpsobject{PST@Dotted}{psline}{linewidth=.0025,linestyle=dotted,dotsep=.008}
\newpsobject{PST@LongDash}{psline}{linewidth=.0015,linestyle=dashed,dash=.02 .01}
\newpsobject{PST@Diamond}{psdots}{linewidth=.001,linestyle=solid,dotstyle=square,dotangle=45}
\newpsobject{PST@Filldiamond}{psdots}{linewidth=.001,linestyle=solid,dotstyle=square*,dotangle=45}
\newpsobject{PST@Cross}{psdots}{linewidth=.001,linestyle=solid,dotstyle=+,dotangle=45}
\newpsobject{PST@Plus}{psdots}{linewidth=.001,linestyle=solid,dotstyle=+}
\newpsobject{PST@Square}{psdots}{linewidth=.001,linestyle=solid,dotstyle=square}
\newpsobject{PST@Circle}{psdots}{linewidth=.001,linestyle=solid,dotstyle=o}
\newpsobject{PST@Triangle}{psdots}{linewidth=.001,linestyle=solid,dotstyle=triangle}
\newpsobject{PST@Pentagon}{psdots}{linewidth=.001,linestyle=solid,dotstyle=pentagon}
\newpsobject{PST@Fillsquare}{psdots}{linewidth=.001,linestyle=solid,dotstyle=square*}
\newpsobject{PST@Fillcircle}{psdots}{linewidth=.001,linestyle=solid,dotstyle=*}
\newpsobject{PST@Filltriangle}{psdots}{linewidth=.001,linestyle=solid,dotstyle=triangle*}
\newpsobject{PST@Fillpentagon}{psdots}{linewidth=.001,linestyle=solid,dotstyle=pentagon*}
\newpsobject{PST@Arrow}{psline}{linewidth=.001,linestyle=solid}
\catcode`@=12

\fi
\psset{unit=5.0in,xunit=5.0in,yunit=3.0in}
\pspicture(0.000000,0.000000)(1.000000,1.000000)
\ifx\nofigs\undefined
\catcode`@=11

\PST@Border(0.1010,0.0840)
(0.1160,0.0840)

\rput[r](0.0850,0.0840){1.2}
\PST@Border(0.1010,0.2103)
(0.1160,0.2103)

\rput[r](0.0850,0.2103){1.4}
\PST@Border(0.1010,0.3366)
(0.1160,0.3366)

\rput[r](0.0850,0.3366){1.6}
\PST@Border(0.1010,0.4629)
(0.1160,0.4629)

\rput[r](0.0850,0.4629){1.8}
\PST@Border(0.1010,0.5891)
(0.1160,0.5891)

\rput[r](0.0850,0.5891){2.0}
\PST@Border(0.1010,0.7154)
(0.1160,0.7154)

\rput[r](0.0850,0.7154){2.2}
\PST@Border(0.1010,0.8417)
(0.1160,0.8417)

\rput[r](0.0850,0.8417){2.4}
\PST@Border(0.1010,0.9680)
(0.1160,0.9680)

\rput[r](0.0850,0.9680){2.6}
\PST@Border(0.1010,0.0840)
(0.1010,0.1040)

\rput(0.1010,0.0420){ 20}
\PST@Border(0.1962,0.0840)
(0.1962,0.1040)

\rput(0.1962,0.0420){ 22}
\PST@Border(0.2914,0.0840)
(0.2914,0.1040)

\rput(0.2914,0.0420){ 24}
\PST@Border(0.3867,0.0840)
(0.3867,0.1040)

\rput(0.3867,0.0420){ 26}
\PST@Border(0.4819,0.0840)
(0.4819,0.1040)

\rput(0.4819,0.0420){ 28}
\PST@Border(0.5771,0.0840)
(0.5771,0.1040)

\rput(0.5771,0.0420){ 30}
\PST@Border(0.6723,0.0840)
(0.6723,0.1040)

\rput(0.6723,0.0420){ 32}
\PST@Border(0.7676,0.0840)
(0.7676,0.1040)

\rput(0.7676,0.0420){ 34}
\PST@Border(0.8628,0.0840)
(0.8628,0.1040)

\rput(0.8628,0.0420){ 36}
\PST@Border(0.9580,0.0840)
(0.9580,0.1040)

\rput(0.9580,0.0420){ 38}
\PST@Border(0.1010,0.9680)
(0.1010,0.0840)
(0.9580,0.0840)
(0.9580,0.9680)
(0.1010,0.9680)

\PST@Solid(0.1010,0.1837)
(0.1010,0.1837)
(0.1097,0.1915)
(0.1183,0.1993)
(0.1270,0.2072)
(0.1356,0.2150)
(0.1443,0.2228)
(0.1529,0.2307)
(0.1616,0.2385)
(0.1703,0.2463)
(0.1789,0.2542)
(0.1876,0.2620)
(0.1962,0.2698)
(0.2049,0.2776)
(0.2135,0.2855)
(0.2222,0.2933)
(0.2308,0.3011)
(0.2395,0.3090)
(0.2482,0.3168)
(0.2568,0.3246)
(0.2655,0.3325)
(0.2741,0.3403)
(0.2828,0.3481)
(0.2914,0.3559)
(0.3001,0.3638)
(0.3088,0.3716)
(0.3174,0.3794)
(0.3261,0.3873)
(0.3347,0.3951)
(0.3434,0.4029)
(0.3520,0.4108)
(0.3607,0.4186)
(0.3694,0.4264)
(0.3780,0.4342)
(0.3867,0.4421)
(0.3953,0.4499)
(0.4040,0.4577)
(0.4126,0.4656)
(0.4213,0.4734)
(0.4299,0.4812)
(0.4386,0.4891)
(0.4473,0.4969)
(0.4559,0.5047)
(0.4646,0.5125)
(0.4732,0.5204)
(0.4819,0.5282)
(0.4905,0.5360)
(0.4992,0.5439)
(0.5079,0.5517)
(0.5165,0.5595)
(0.5252,0.5674)
(0.5338,0.5752)
(0.5425,0.5830)
(0.5511,0.5908)
(0.5598,0.5987)
(0.5685,0.6065)
(0.5771,0.6143)
(0.5858,0.6222)
(0.5944,0.6300)
(0.6031,0.6378)
(0.6117,0.6457)
(0.6204,0.6535)
(0.6291,0.6613)
(0.6377,0.6691)
(0.6464,0.6770)
(0.6550,0.6848)
(0.6637,0.6926)
(0.6723,0.7005)
(0.6810,0.7083)
(0.6896,0.7161)
(0.6983,0.7240)
(0.7070,0.7318)
(0.7156,0.7396)
(0.7243,0.7474)
(0.7329,0.7553)
(0.7416,0.7631)
(0.7502,0.7709)
(0.7589,0.7788)
(0.7676,0.7866)
(0.7762,0.7944)
(0.7849,0.8023)
(0.7935,0.8101)
(0.8022,0.8179)
(0.8108,0.8258)
(0.8195,0.8336)
(0.8282,0.8414)
(0.8368,0.8492)
(0.8455,0.8571)
(0.8541,0.8649)
(0.8628,0.8727)
(0.8714,0.8806)
(0.8801,0.8884)
(0.8887,0.8962)
(0.8974,0.9041)
(0.9061,0.9119)
(0.9147,0.9197)
(0.9234,0.9275)
(0.9320,0.9354)
(0.9407,0.9432)
(0.9493,0.9510)
(0.9580,0.9589)

\PST@Diamond(0.9580,0.9617)
\PST@Diamond(0.8628,0.8689)
\PST@Diamond(0.7676,0.7830)
\PST@Diamond(0.6723,0.6952)
\PST@Diamond(0.5771,0.6125)
\PST@Diamond(0.4819,0.5456)
\PST@Diamond(0.3867,0.4477)
\PST@Diamond(0.2914,0.3486)
\PST@Diamond(0.1962,0.2696)
\PST@Diamond(0.1010,0.1800)
\PST@Border(0.1010,0.9680)
(0.1010,0.0840)
(0.9580,0.0840)
(0.9580,0.9680)
(0.1010,0.9680)

\catcode`@=12
\fi
\endpspicture

\end{figure}
\begin{figure}[p]\caption{Grafico dei residui, con errore strumentale. Il punto più in alto è stato scartato.}
\centering
% GNUPLOT: LaTeX picture using PSTRICKS macros
% Define new PST objects, if not already defined
\ifx\PSTloaded\undefined
\def\PSTloaded{t}
\psset{arrowsize=.01 3.2 1.4 .3}
\psset{dotsize=.08}
\catcode`@=11

\newpsobject{PST@Border}{psline}{linewidth=.0015,linestyle=solid}
\newpsobject{PST@Axes}{psline}{linewidth=.0015,linestyle=dotted,dotsep=.004}
\newpsobject{PST@Solid}{psline}{linewidth=.0015,linestyle=solid}
\newpsobject{PST@Dashed}{psline}{linewidth=.0015,linestyle=dashed,dash=.01 .01}
\newpsobject{PST@Dotted}{psline}{linewidth=.0025,linestyle=dotted,dotsep=.008}
\newpsobject{PST@LongDash}{psline}{linewidth=.0015,linestyle=dashed,dash=.02 .01}
\newpsobject{PST@Diamond}{psdots}{linewidth=.001,linestyle=solid,dotstyle=square,dotangle=45}
\newpsobject{PST@Filldiamond}{psdots}{linewidth=.001,linestyle=solid,dotstyle=square*,dotangle=45}
\newpsobject{PST@Cross}{psdots}{linewidth=.001,linestyle=solid,dotstyle=+,dotangle=45}
\newpsobject{PST@Plus}{psdots}{linewidth=.001,linestyle=solid,dotstyle=+}
\newpsobject{PST@Square}{psdots}{linewidth=.001,linestyle=solid,dotstyle=square}
\newpsobject{PST@Circle}{psdots}{linewidth=.001,linestyle=solid,dotstyle=o}
\newpsobject{PST@Triangle}{psdots}{linewidth=.001,linestyle=solid,dotstyle=triangle}
\newpsobject{PST@Pentagon}{psdots}{linewidth=.001,linestyle=solid,dotstyle=pentagon}
\newpsobject{PST@Fillsquare}{psdots}{linewidth=.001,linestyle=solid,dotstyle=square*}
\newpsobject{PST@Fillcircle}{psdots}{linewidth=.001,linestyle=solid,dotstyle=*}
\newpsobject{PST@Filltriangle}{psdots}{linewidth=.001,linestyle=solid,dotstyle=triangle*}
\newpsobject{PST@Fillpentagon}{psdots}{linewidth=.001,linestyle=solid,dotstyle=pentagon*}
\newpsobject{PST@Arrow}{psline}{linewidth=.001,linestyle=solid}
\catcode`@=12

\fi
\psset{unit=5.0in,xunit=5.0in,yunit=3.0in}
\pspicture(0.000000,0.000000)(1.000000,1.000000)
\ifx\nofigs\undefined
\catcode`@=11

\PST@Border(0.1330,0.0840)
(0.1480,0.0840)

\rput[r](0.1170,0.0840){-0.03}
\PST@Border(0.1330,0.2313)
(0.1480,0.2313)

\rput[r](0.1170,0.2313){-0.02}
\PST@Border(0.1330,0.3787)
(0.1480,0.3787)

\rput[r](0.1170,0.3787){-0.01}
\PST@Border(0.1330,0.5260)
(0.1480,0.5260)

\rput[r](0.1170,0.5260){0.00}
\PST@Border(0.1330,0.6733)
(0.1480,0.6733)

\rput[r](0.1170,0.6733){0.01}
\PST@Border(0.1330,0.8207)
(0.1480,0.8207)

\rput[r](0.1170,0.8207){0.02}
\PST@Border(0.1330,0.9680)
(0.1480,0.9680)

\rput[r](0.1170,0.9680){0.03}
\PST@Border(0.2080,0.0840)
(0.2080,0.1040)

\rput(0.2080,0.0420){20.0}
\PST@Border(0.3955,0.0840)
(0.3955,0.1040)

\rput(0.3955,0.0420){25.0}
\PST@Border(0.5830,0.0840)
(0.5830,0.1040)

\rput(0.5830,0.0420){30.0}
\PST@Border(0.7705,0.0840)
(0.7705,0.1040)

\rput(0.7705,0.0420){35.0}
\PST@Border(0.9580,0.0840)
(0.9580,0.1040)

\rput(0.9580,0.0420){40.0}
\PST@Border(0.1330,0.9680)
(0.1330,0.0840)
(0.9580,0.0840)
(0.9580,0.9680)
(0.1330,0.9680)

\PST@Solid(0.8830,0.3404)
(0.8830,0.6646)

\PST@Solid(0.8755,0.3404)
(0.8905,0.3404)

\PST@Solid(0.8755,0.6646)
(0.8905,0.6646)

\PST@Solid(0.8080,0.1990)
(0.8080,0.4937)

\PST@Solid(0.8005,0.1990)
(0.8155,0.1990)

\PST@Solid(0.8005,0.4937)
(0.8155,0.4937)

\PST@Solid(0.7330,0.2196)
(0.7330,0.4848)

\PST@Solid(0.7255,0.2196)
(0.7405,0.2196)

\PST@Solid(0.7255,0.4848)
(0.7405,0.4848)

\PST@Solid(0.6580,0.1813)
(0.6580,0.4465)

\PST@Solid(0.6505,0.1813)
(0.6655,0.1813)

\PST@Solid(0.6505,0.4465)
(0.6655,0.4465)

\PST@Solid(0.5830,0.2756)
(0.5830,0.5114)

\PST@Solid(0.5755,0.2756)
(0.5905,0.2756)

\PST@Solid(0.5755,0.5114)
(0.5905,0.5114)

\PST@Solid(0.5080,0.7235)
(0.5080,0.9592)

\PST@Solid(0.5005,0.7235)
(0.5155,0.7235)

\PST@Solid(0.5005,0.9592)
(0.5155,0.9592)

\PST@Solid(0.4330,0.4642)
(0.4330,0.6705)

\PST@Solid(0.4255,0.4642)
(0.4405,0.4642)

\PST@Solid(0.4255,0.6705)
(0.4405,0.6705)

\PST@Solid(0.3580,0.1607)
(0.3580,0.3670)

\PST@Solid(0.3505,0.1607)
(0.3655,0.1607)

\PST@Solid(0.3505,0.3670)
(0.3655,0.3670)

\PST@Solid(0.2830,0.3434)
(0.2830,0.5202)

\PST@Solid(0.2755,0.3434)
(0.2905,0.3434)

\PST@Solid(0.2755,0.5202)
(0.2905,0.5202)

\PST@Solid(0.2080,0.2609)
(0.2080,0.4377)

\PST@Solid(0.2005,0.2609)
(0.2155,0.2609)

\PST@Solid(0.2005,0.4377)
(0.2155,0.4377)

\PST@Diamond(0.8830,0.5025)
\PST@Diamond(0.8080,0.3463)
\PST@Diamond(0.7330,0.3522)
\PST@Diamond(0.6580,0.3139)
\PST@Diamond(0.5830,0.3935)
\PST@Diamond(0.5080,0.8414)
\PST@Diamond(0.4330,0.5673)
\PST@Diamond(0.3580,0.2638)
\PST@Diamond(0.2830,0.4318)
\PST@Diamond(0.2080,0.3493)
\PST@Border(0.1330,0.9680)
(0.1330,0.0840)
(0.9580,0.0840)
(0.9580,0.9680)
(0.1330,0.9680)

\catcode`@=12
\fi
\endpspicture

\end{figure}
\begin{figure}[p]\caption{Grafico dei residui, con errore a posteriori.  Il punto più in alto è stato scartato.}
\centering
% GNUPLOT: LaTeX picture using PSTRICKS macros
% Define new PST objects, if not already defined
\ifx\PSTloaded\undefined
\def\PSTloaded{t}
\psset{arrowsize=.01 3.2 1.4 .3}
\psset{dotsize=.08}
\catcode`@=11

\newpsobject{PST@Border}{psline}{linewidth=.0015,linestyle=solid}
\newpsobject{PST@Axes}{psline}{linewidth=.0015,linestyle=dotted,dotsep=.004}
\newpsobject{PST@Solid}{psline}{linewidth=.0015,linestyle=solid}
\newpsobject{PST@Dashed}{psline}{linewidth=.0015,linestyle=dashed,dash=.01 .01}
\newpsobject{PST@Dotted}{psline}{linewidth=.0025,linestyle=dotted,dotsep=.008}
\newpsobject{PST@LongDash}{psline}{linewidth=.0015,linestyle=dashed,dash=.02 .01}
\newpsobject{PST@Diamond}{psdots}{linewidth=.001,linestyle=solid,dotstyle=square,dotangle=45}
\newpsobject{PST@Filldiamond}{psdots}{linewidth=.001,linestyle=solid,dotstyle=square*,dotangle=45}
\newpsobject{PST@Cross}{psdots}{linewidth=.001,linestyle=solid,dotstyle=+,dotangle=45}
\newpsobject{PST@Plus}{psdots}{linewidth=.001,linestyle=solid,dotstyle=+}
\newpsobject{PST@Square}{psdots}{linewidth=.001,linestyle=solid,dotstyle=square}
\newpsobject{PST@Circle}{psdots}{linewidth=.001,linestyle=solid,dotstyle=o}
\newpsobject{PST@Triangle}{psdots}{linewidth=.001,linestyle=solid,dotstyle=triangle}
\newpsobject{PST@Pentagon}{psdots}{linewidth=.001,linestyle=solid,dotstyle=pentagon}
\newpsobject{PST@Fillsquare}{psdots}{linewidth=.001,linestyle=solid,dotstyle=square*}
\newpsobject{PST@Fillcircle}{psdots}{linewidth=.001,linestyle=solid,dotstyle=*}
\newpsobject{PST@Filltriangle}{psdots}{linewidth=.001,linestyle=solid,dotstyle=triangle*}
\newpsobject{PST@Fillpentagon}{psdots}{linewidth=.001,linestyle=solid,dotstyle=pentagon*}
\newpsobject{PST@Arrow}{psline}{linewidth=.001,linestyle=solid}
\catcode`@=12

\fi
\psset{unit=5.0in,xunit=5.0in,yunit=3.0in}
\pspicture(0.000000,0.000000)(1.000000,1.000000)
\ifx\nofigs\undefined
\catcode`@=11

\PST@Border(0.1330,0.0840)
(0.1480,0.0840)

\rput[r](0.1170,0.0840){-0.04}
\PST@Border(0.1330,0.1945)
(0.1480,0.1945)

\rput[r](0.1170,0.1945){-0.03}
\PST@Border(0.1330,0.3050)
(0.1480,0.3050)

\rput[r](0.1170,0.3050){-0.02}
\PST@Border(0.1330,0.4155)
(0.1480,0.4155)

\rput[r](0.1170,0.4155){-0.01}
\PST@Border(0.1330,0.5260)
(0.1480,0.5260)

\rput[r](0.1170,0.5260){0.00}
\PST@Border(0.1330,0.6365)
(0.1480,0.6365)

\rput[r](0.1170,0.6365){0.01}
\PST@Border(0.1330,0.7470)
(0.1480,0.7470)

\rput[r](0.1170,0.7470){0.02}
\PST@Border(0.1330,0.8575)
(0.1480,0.8575)

\rput[r](0.1170,0.8575){0.03}
\PST@Border(0.1330,0.9680)
(0.1480,0.9680)

\rput[r](0.1170,0.9680){0.04}
\PST@Border(0.2080,0.0840)
(0.2080,0.1040)

\rput(0.2080,0.0420){20.0}
\PST@Border(0.3955,0.0840)
(0.3955,0.1040)

\rput(0.3955,0.0420){25.0}
\PST@Border(0.5830,0.0840)
(0.5830,0.1040)

\rput(0.5830,0.0420){30.0}
\PST@Border(0.7705,0.0840)
(0.7705,0.1040)

\rput(0.7705,0.0420){35.0}
\PST@Border(0.9580,0.0840)
(0.9580,0.1040)

\rput(0.9580,0.0420){40.0}
\PST@Border(0.1330,0.9680)
(0.1330,0.0840)
(0.9580,0.0840)
(0.9580,0.9680)
(0.1330,0.9680)

\PST@Solid(0.8830,0.3548)
(0.8830,0.6620)

\PST@Solid(0.8755,0.3548)
(0.8905,0.3548)

\PST@Solid(0.8755,0.6620)
(0.8905,0.6620)

\PST@Solid(0.8080,0.2377)
(0.8080,0.5449)

\PST@Solid(0.8005,0.2377)
(0.8155,0.2377)

\PST@Solid(0.8005,0.5449)
(0.8155,0.5449)

\PST@Solid(0.7330,0.2421)
(0.7330,0.5493)

\PST@Solid(0.7255,0.2421)
(0.7405,0.2421)

\PST@Solid(0.7255,0.5493)
(0.7405,0.5493)

\PST@Solid(0.6580,0.2133)
(0.6580,0.5206)

\PST@Solid(0.6505,0.2133)
(0.6655,0.2133)

\PST@Solid(0.6505,0.5206)
(0.6655,0.5206)

\PST@Solid(0.5830,0.2730)
(0.5830,0.5802)

\PST@Solid(0.5755,0.2730)
(0.5905,0.2730)

\PST@Solid(0.5755,0.5802)
(0.5905,0.5802)

\PST@Solid(0.5080,0.6089)
(0.5080,0.9161)

\PST@Solid(0.5005,0.6089)
(0.5155,0.6089)

\PST@Solid(0.5005,0.9161)
(0.5155,0.9161)

\PST@Solid(0.4330,0.4034)
(0.4330,0.7106)

\PST@Solid(0.4255,0.4034)
(0.4405,0.4034)

\PST@Solid(0.4255,0.7106)
(0.4405,0.7106)

\PST@Solid(0.3580,0.1758)
(0.3580,0.4830)

\PST@Solid(0.3505,0.1758)
(0.3655,0.1758)

\PST@Solid(0.3505,0.4830)
(0.3655,0.4830)

\PST@Solid(0.2830,0.3017)
(0.2830,0.6090)

\PST@Solid(0.2755,0.3017)
(0.2905,0.3017)

\PST@Solid(0.2755,0.6090)
(0.2905,0.6090)

\PST@Solid(0.2080,0.2399)
(0.2080,0.5471)

\PST@Solid(0.2005,0.2399)
(0.2155,0.2399)

\PST@Solid(0.2005,0.5471)
(0.2155,0.5471)

\PST@Diamond(0.8830,0.5084)
\PST@Diamond(0.8080,0.3913)
\PST@Diamond(0.7330,0.3957)
\PST@Diamond(0.6580,0.3669)
\PST@Diamond(0.5830,0.4266)
\PST@Diamond(0.5080,0.7625)
\PST@Diamond(0.4330,0.5570)
\PST@Diamond(0.3580,0.3294)
\PST@Diamond(0.2830,0.4553)
\PST@Diamond(0.2080,0.3935)
\PST@Border(0.1330,0.9680)
(0.1330,0.0840)
(0.9580,0.0840)
(0.9580,0.9680)
(0.1330,0.9680)

\catcode`@=12
\fi
\endpspicture

\end{figure}
\begin{figure}[p]\caption{Grafico dei residui, con errore a posteriori. Parametri ricalcolati eliminando il quinto punto dai grafici precedenti.}
\centering
% GNUPLOT: LaTeX picture using PSTRICKS macros
% Define new PST objects, if not already defined
\ifx\PSTloaded\undefined
\def\PSTloaded{t}
\psset{arrowsize=.01 3.2 1.4 .3}
\psset{dotsize=.08}
\catcode`@=11

\newpsobject{PST@Border}{psline}{linewidth=.0015,linestyle=solid}
\newpsobject{PST@Axes}{psline}{linewidth=.0015,linestyle=dotted,dotsep=.004}
\newpsobject{PST@Solid}{psline}{linewidth=.0015,linestyle=solid}
\newpsobject{PST@Dashed}{psline}{linewidth=.0015,linestyle=dashed,dash=.01 .01}
\newpsobject{PST@Dotted}{psline}{linewidth=.0025,linestyle=dotted,dotsep=.008}
\newpsobject{PST@LongDash}{psline}{linewidth=.0015,linestyle=dashed,dash=.02 .01}
\newpsobject{PST@Diamond}{psdots}{linewidth=.001,linestyle=solid,dotstyle=square,dotangle=45}
\newpsobject{PST@Filldiamond}{psdots}{linewidth=.001,linestyle=solid,dotstyle=square*,dotangle=45}
\newpsobject{PST@Cross}{psdots}{linewidth=.001,linestyle=solid,dotstyle=+,dotangle=45}
\newpsobject{PST@Plus}{psdots}{linewidth=.001,linestyle=solid,dotstyle=+}
\newpsobject{PST@Square}{psdots}{linewidth=.001,linestyle=solid,dotstyle=square}
\newpsobject{PST@Circle}{psdots}{linewidth=.001,linestyle=solid,dotstyle=o}
\newpsobject{PST@Triangle}{psdots}{linewidth=.001,linestyle=solid,dotstyle=triangle}
\newpsobject{PST@Pentagon}{psdots}{linewidth=.001,linestyle=solid,dotstyle=pentagon}
\newpsobject{PST@Fillsquare}{psdots}{linewidth=.001,linestyle=solid,dotstyle=square*}
\newpsobject{PST@Fillcircle}{psdots}{linewidth=.001,linestyle=solid,dotstyle=*}
\newpsobject{PST@Filltriangle}{psdots}{linewidth=.001,linestyle=solid,dotstyle=triangle*}
\newpsobject{PST@Fillpentagon}{psdots}{linewidth=.001,linestyle=solid,dotstyle=pentagon*}
\newpsobject{PST@Arrow}{psline}{linewidth=.001,linestyle=solid}
\catcode`@=12

\fi
\psset{unit=5.0in,xunit=5.0in,yunit=3.0in}
\pspicture(0.000000,0.000000)(1.000000,1.000000)
\ifx\nofigs\undefined
\catcode`@=11

\PST@Border(0.1330,0.0840)
(0.1480,0.0840)

\rput[r](0.1170,0.0840){-0.02}
\PST@Border(0.1330,0.1945)
(0.1480,0.1945)

\rput[r](0.1170,0.1945){-0.01}
\PST@Border(0.1330,0.3050)
(0.1480,0.3050)

\rput[r](0.1170,0.3050){-0.01}
\PST@Border(0.1330,0.4155)
(0.1480,0.4155)

\rput[r](0.1170,0.4155){-0.00}
\PST@Border(0.1330,0.5260)
(0.1480,0.5260)

\rput[r](0.1170,0.5260){0.00}
\PST@Border(0.1330,0.6365)
(0.1480,0.6365)

\rput[r](0.1170,0.6365){0.01}
\PST@Border(0.1330,0.7470)
(0.1480,0.7470)

\rput[r](0.1170,0.7470){0.01}
\PST@Border(0.1330,0.8575)
(0.1480,0.8575)

\rput[r](0.1170,0.8575){0.02}
\PST@Border(0.1330,0.9680)
(0.1480,0.9680)

\rput[r](0.1170,0.9680){0.02}
\PST@Border(0.2080,0.0840)
(0.2080,0.1040)

\rput(0.2080,0.0420){20.0}
\PST@Border(0.3955,0.0840)
(0.3955,0.1040)

\rput(0.3955,0.0420){25.0}
\PST@Border(0.5830,0.0840)
(0.5830,0.1040)

\rput(0.5830,0.0420){30.0}
\PST@Border(0.7705,0.0840)
(0.7705,0.1040)

\rput(0.7705,0.0420){35.0}
\PST@Border(0.9580,0.0840)
(0.9580,0.1040)

\rput(0.9580,0.0420){40.0}
\PST@Border(0.1330,0.9680)
(0.1330,0.0840)
(0.9580,0.0840)
(0.9580,0.9680)
(0.1330,0.9680)

\PST@Solid(0.8830,0.4777)
(0.8830,0.7945)

\PST@Solid(0.8755,0.4777)
(0.8905,0.4777)

\PST@Solid(0.8755,0.7945)
(0.8905,0.7945)

\PST@Solid(0.8080,0.2474)
(0.8080,0.5642)

\PST@Solid(0.8005,0.2474)
(0.8155,0.2474)

\PST@Solid(0.8005,0.5642)
(0.8155,0.5642)

\PST@Solid(0.7330,0.2602)
(0.7330,0.5770)

\PST@Solid(0.7255,0.2602)
(0.7405,0.2602)

\PST@Solid(0.7255,0.5770)
(0.7405,0.5770)

\PST@Solid(0.6580,0.2067)
(0.6580,0.5235)

\PST@Solid(0.6505,0.2067)
(0.6655,0.2067)

\PST@Solid(0.6505,0.5235)
(0.6655,0.5235)

\PST@Solid(0.5830,0.3300)
(0.5830,0.6468)

\PST@Solid(0.5755,0.3300)
(0.5905,0.3300)

\PST@Solid(0.5755,0.6468)
(0.5905,0.6468)

\PST@Solid(0.4330,0.5988)
(0.4330,0.9156)

\PST@Solid(0.4255,0.5988)
(0.4405,0.5988)

\PST@Solid(0.4255,0.9156)
(0.4405,0.9156)

\PST@Solid(0.3580,0.1475)
(0.3580,0.4643)

\PST@Solid(0.3505,0.1475)
(0.3655,0.1475)

\PST@Solid(0.3505,0.4643)
(0.3655,0.4643)

\PST@Solid(0.2830,0.4034)
(0.2830,0.7202)

\PST@Solid(0.2755,0.4034)
(0.2905,0.4034)

\PST@Solid(0.2755,0.7202)
(0.2905,0.7202)

\PST@Solid(0.2080,0.2836)
(0.2080,0.6004)

\PST@Solid(0.2005,0.2836)
(0.2155,0.2836)

\PST@Solid(0.2005,0.6004)
(0.2155,0.6004)

\PST@Diamond(0.8830,0.6361)
\PST@Diamond(0.8080,0.4058)
\PST@Diamond(0.7330,0.4186)
\PST@Diamond(0.6580,0.3651)
\PST@Diamond(0.5830,0.4884)
\PST@Diamond(0.4330,0.7572)
\PST@Diamond(0.3580,0.3059)
\PST@Diamond(0.2830,0.5618)
\PST@Diamond(0.2080,0.4420)
\PST@Border(0.1330,0.9680)
(0.1330,0.0840)
(0.9580,0.0840)
(0.9580,0.9680)
(0.1330,0.9680)

\catcode`@=12
\fi
\endpspicture

\end{figure}
\begin{figure}[p]\caption{Corrente in mA in ascissa, differenza di potenziale in ordinata (V) misurata ai capi della resistenza R2. Le barre di errore sono più piccole delle dimensioni dei punti in questa scala.}
\centering
% GNUPLOT: LaTeX picture using PSTRICKS macros
% Define new PST objects, if not already defined
\ifx\PSTloaded\undefined
\def\PSTloaded{t}
\psset{arrowsize=.01 3.2 1.4 .3}
\psset{dotsize=.08}
\catcode`@=11

\newpsobject{PST@Border}{psline}{linewidth=.0015,linestyle=solid}
\newpsobject{PST@Axes}{psline}{linewidth=.0015,linestyle=dotted,dotsep=.004}
\newpsobject{PST@Solid}{psline}{linewidth=.0015,linestyle=solid}
\newpsobject{PST@Dashed}{psline}{linewidth=.0015,linestyle=dashed,dash=.01 .01}
\newpsobject{PST@Dotted}{psline}{linewidth=.0025,linestyle=dotted,dotsep=.008}
\newpsobject{PST@LongDash}{psline}{linewidth=.0015,linestyle=dashed,dash=.02 .01}
\newpsobject{PST@Diamond}{psdots}{linewidth=.001,linestyle=solid,dotstyle=square,dotangle=45}
\newpsobject{PST@Filldiamond}{psdots}{linewidth=.001,linestyle=solid,dotstyle=square*,dotangle=45}
\newpsobject{PST@Cross}{psdots}{linewidth=.001,linestyle=solid,dotstyle=+,dotangle=45}
\newpsobject{PST@Plus}{psdots}{linewidth=.001,linestyle=solid,dotstyle=+}
\newpsobject{PST@Square}{psdots}{linewidth=.001,linestyle=solid,dotstyle=square}
\newpsobject{PST@Circle}{psdots}{linewidth=.001,linestyle=solid,dotstyle=o}
\newpsobject{PST@Triangle}{psdots}{linewidth=.001,linestyle=solid,dotstyle=triangle}
\newpsobject{PST@Pentagon}{psdots}{linewidth=.001,linestyle=solid,dotstyle=pentagon}
\newpsobject{PST@Fillsquare}{psdots}{linewidth=.001,linestyle=solid,dotstyle=square*}
\newpsobject{PST@Fillcircle}{psdots}{linewidth=.001,linestyle=solid,dotstyle=*}
\newpsobject{PST@Filltriangle}{psdots}{linewidth=.001,linestyle=solid,dotstyle=triangle*}
\newpsobject{PST@Fillpentagon}{psdots}{linewidth=.001,linestyle=solid,dotstyle=pentagon*}
\newpsobject{PST@Arrow}{psline}{linewidth=.001,linestyle=solid}
\catcode`@=12

\fi
\psset{unit=5.0in,xunit=5.0in,yunit=3.0in}
\pspicture(0.000000,0.000000)(1.000000,1.000000)
\ifx\nofigs\undefined
\catcode`@=11

\PST@Border(0.1010,0.0840)
(0.1160,0.0840)

\rput[r](0.0850,0.0840){1.4}
\PST@Border(0.1010,0.2103)
(0.1160,0.2103)

\rput[r](0.0850,0.2103){1.6}
\PST@Border(0.1010,0.3366)
(0.1160,0.3366)

\rput[r](0.0850,0.3366){1.8}
\PST@Border(0.1010,0.4629)
(0.1160,0.4629)

\rput[r](0.0850,0.4629){2.0}
\PST@Border(0.1010,0.5891)
(0.1160,0.5891)

\rput[r](0.0850,0.5891){2.2}
\PST@Border(0.1010,0.7154)
(0.1160,0.7154)

\rput[r](0.0850,0.7154){2.4}
\PST@Border(0.1010,0.8417)
(0.1160,0.8417)

\rput[r](0.0850,0.8417){2.6}
\PST@Border(0.1010,0.9680)
(0.1160,0.9680)

\rput[r](0.0850,0.9680){2.8}
\PST@Border(0.1010,0.0840)
(0.1010,0.1040)

\rput(0.1010,0.0420){ 22}
\PST@Border(0.1962,0.0840)
(0.1962,0.1040)

\rput(0.1962,0.0420){ 24}
\PST@Border(0.2914,0.0840)
(0.2914,0.1040)

\rput(0.2914,0.0420){ 26}
\PST@Border(0.3867,0.0840)
(0.3867,0.1040)

\rput(0.3867,0.0420){ 28}
\PST@Border(0.4819,0.0840)
(0.4819,0.1040)

\rput(0.4819,0.0420){ 30}
\PST@Border(0.5771,0.0840)
(0.5771,0.1040)

\rput(0.5771,0.0420){ 32}
\PST@Border(0.6723,0.0840)
(0.6723,0.1040)

\rput(0.6723,0.0420){ 34}
\PST@Border(0.7676,0.0840)
(0.7676,0.1040)

\rput(0.7676,0.0420){ 36}
\PST@Border(0.8628,0.0840)
(0.8628,0.1040)

\rput(0.8628,0.0420){ 38}
\PST@Border(0.9580,0.0840)
(0.9580,0.1040)

\rput(0.9580,0.0420){ 40}
\PST@Border(0.1010,0.9680)
(0.1010,0.0840)
(0.9580,0.0840)
(0.9580,0.9680)
(0.1010,0.9680)

\PST@Solid(0.1010,0.1459)
(0.1010,0.1459)
(0.1097,0.1536)
(0.1183,0.1614)
(0.1270,0.1692)
(0.1356,0.1770)
(0.1443,0.1847)
(0.1529,0.1925)
(0.1616,0.2003)
(0.1703,0.2081)
(0.1789,0.2159)
(0.1876,0.2236)
(0.1962,0.2314)
(0.2049,0.2392)
(0.2135,0.2470)
(0.2222,0.2548)
(0.2308,0.2625)
(0.2395,0.2703)
(0.2482,0.2781)
(0.2568,0.2859)
(0.2655,0.2937)
(0.2741,0.3014)
(0.2828,0.3092)
(0.2914,0.3170)
(0.3001,0.3248)
(0.3088,0.3326)
(0.3174,0.3403)
(0.3261,0.3481)
(0.3347,0.3559)
(0.3434,0.3637)
(0.3520,0.3714)
(0.3607,0.3792)
(0.3694,0.3870)
(0.3780,0.3948)
(0.3867,0.4026)
(0.3953,0.4103)
(0.4040,0.4181)
(0.4126,0.4259)
(0.4213,0.4337)
(0.4299,0.4415)
(0.4386,0.4492)
(0.4473,0.4570)
(0.4559,0.4648)
(0.4646,0.4726)
(0.4732,0.4804)
(0.4819,0.4881)
(0.4905,0.4959)
(0.4992,0.5037)
(0.5079,0.5115)
(0.5165,0.5192)
(0.5252,0.5270)
(0.5338,0.5348)
(0.5425,0.5426)
(0.5511,0.5504)
(0.5598,0.5581)
(0.5685,0.5659)
(0.5771,0.5737)
(0.5858,0.5815)
(0.5944,0.5893)
(0.6031,0.5970)
(0.6117,0.6048)
(0.6204,0.6126)
(0.6291,0.6204)
(0.6377,0.6282)
(0.6464,0.6359)
(0.6550,0.6437)
(0.6637,0.6515)
(0.6723,0.6593)
(0.6810,0.6671)
(0.6896,0.6748)
(0.6983,0.6826)
(0.7070,0.6904)
(0.7156,0.6982)
(0.7243,0.7059)
(0.7329,0.7137)
(0.7416,0.7215)
(0.7502,0.7293)
(0.7589,0.7371)
(0.7676,0.7448)
(0.7762,0.7526)
(0.7849,0.7604)
(0.7935,0.7682)
(0.8022,0.7760)
(0.8108,0.7837)
(0.8195,0.7915)
(0.8282,0.7993)
(0.8368,0.8071)
(0.8455,0.8149)
(0.8541,0.8226)
(0.8628,0.8304)
(0.8714,0.8382)
(0.8801,0.8460)
(0.8887,0.8538)
(0.8974,0.8615)
(0.9061,0.8693)
(0.9147,0.8771)
(0.9234,0.8849)
(0.9320,0.8926)
(0.9407,0.9004)
(0.9493,0.9082)
(0.9580,0.9160)

\PST@Diamond(0.9580,0.9156)
\PST@Diamond(0.8771,0.8436)
\PST@Diamond(0.7676,0.7464)
\PST@Diamond(0.6723,0.6580)
\PST@Diamond(0.5771,0.5734)
\PST@Diamond(0.4819,0.4875)
\PST@Diamond(0.3867,0.4035)
\PST@Diamond(0.2962,0.3202)
\PST@Diamond(0.1962,0.2330)
\PST@Diamond(0.1010,0.1452)
\PST@Border(0.1010,0.9680)
(0.1010,0.0840)
(0.9580,0.0840)
(0.9580,0.9680)
(0.1010,0.9680)

\catcode`@=12
\fi
\endpspicture

\end{figure}
\begin{figure}[p]\caption{Grafico dei residui, con errore strumentale.}
\centering
% GNUPLOT: LaTeX picture using PSTRICKS macros
% Define new PST objects, if not already defined
\ifx\PSTloaded\undefined
\def\PSTloaded{t}
\psset{arrowsize=.01 3.2 1.4 .3}
\psset{dotsize=.08}
\catcode`@=11

\newpsobject{PST@Border}{psline}{linewidth=.0015,linestyle=solid}
\newpsobject{PST@Axes}{psline}{linewidth=.0015,linestyle=dotted,dotsep=.004}
\newpsobject{PST@Solid}{psline}{linewidth=.0015,linestyle=solid}
\newpsobject{PST@Dashed}{psline}{linewidth=.0015,linestyle=dashed,dash=.01 .01}
\newpsobject{PST@Dotted}{psline}{linewidth=.0025,linestyle=dotted,dotsep=.008}
\newpsobject{PST@LongDash}{psline}{linewidth=.0015,linestyle=dashed,dash=.02 .01}
\newpsobject{PST@Diamond}{psdots}{linewidth=.001,linestyle=solid,dotstyle=square,dotangle=45}
\newpsobject{PST@Filldiamond}{psdots}{linewidth=.001,linestyle=solid,dotstyle=square*,dotangle=45}
\newpsobject{PST@Cross}{psdots}{linewidth=.001,linestyle=solid,dotstyle=+,dotangle=45}
\newpsobject{PST@Plus}{psdots}{linewidth=.001,linestyle=solid,dotstyle=+}
\newpsobject{PST@Square}{psdots}{linewidth=.001,linestyle=solid,dotstyle=square}
\newpsobject{PST@Circle}{psdots}{linewidth=.001,linestyle=solid,dotstyle=o}
\newpsobject{PST@Triangle}{psdots}{linewidth=.001,linestyle=solid,dotstyle=triangle}
\newpsobject{PST@Pentagon}{psdots}{linewidth=.001,linestyle=solid,dotstyle=pentagon}
\newpsobject{PST@Fillsquare}{psdots}{linewidth=.001,linestyle=solid,dotstyle=square*}
\newpsobject{PST@Fillcircle}{psdots}{linewidth=.001,linestyle=solid,dotstyle=*}
\newpsobject{PST@Filltriangle}{psdots}{linewidth=.001,linestyle=solid,dotstyle=triangle*}
\newpsobject{PST@Fillpentagon}{psdots}{linewidth=.001,linestyle=solid,dotstyle=pentagon*}
\newpsobject{PST@Arrow}{psline}{linewidth=.001,linestyle=solid}
\catcode`@=12

\fi
\psset{unit=5.0in,xunit=5.0in,yunit=3.0in}
\pspicture(0.000000,0.000000)(1.000000,1.000000)
\ifx\nofigs\undefined
\catcode`@=11

\PST@Border(0.1490,0.0840)
(0.1640,0.0840)

\rput[r](0.1330,0.0840){-0.015}
\PST@Border(0.1490,0.2313)
(0.1640,0.2313)

\rput[r](0.1330,0.2313){-0.010}
\PST@Border(0.1490,0.3787)
(0.1640,0.3787)

\rput[r](0.1330,0.3787){-0.005}
\PST@Border(0.1490,0.5260)
(0.1640,0.5260)

\rput[r](0.1330,0.5260){0.000}
\PST@Border(0.1490,0.6733)
(0.1640,0.6733)

\rput[r](0.1330,0.6733){0.005}
\PST@Border(0.1490,0.8207)
(0.1640,0.8207)

\rput[r](0.1330,0.8207){0.010}
\PST@Border(0.1490,0.9680)
(0.1640,0.9680)

\rput[r](0.1330,0.9680){0.015}
\PST@Border(0.1490,0.0840)
(0.1490,0.1040)

\rput(0.1490,0.0420){ 20}
\PST@Border(0.3329,0.0840)
(0.3329,0.1040)

\rput(0.3329,0.0420){ 25}
\PST@Border(0.5167,0.0840)
(0.5167,0.1040)

\rput(0.5167,0.0420){ 30}
\PST@Border(0.7006,0.0840)
(0.7006,0.1040)

\rput(0.7006,0.0420){ 35}
\PST@Border(0.8845,0.0840)
(0.8845,0.1040)

\rput(0.8845,0.0420){ 40}
\PST@Border(0.1490,0.9680)
(0.1490,0.0840)
(0.9580,0.0840)
(0.9580,0.9680)
(0.1490,0.9680)

\PST@Solid(0.8845,0.1835)
(0.8845,0.8318)

\PST@Solid(0.8770,0.1835)
(0.8920,0.1835)

\PST@Solid(0.8770,0.8318)
(0.8920,0.8318)

\PST@Solid(0.8219,0.2186)
(0.8219,0.8669)

\PST@Solid(0.8144,0.2186)
(0.8294,0.2186)

\PST@Solid(0.8144,0.8669)
(0.8294,0.8669)

\PST@Solid(0.7374,0.3025)
(0.7374,0.8918)

\PST@Solid(0.7299,0.3025)
(0.7449,0.3025)

\PST@Solid(0.7299,0.8918)
(0.7449,0.8918)

\PST@Solid(0.6638,0.1999)
(0.6638,0.7303)

\PST@Solid(0.6563,0.1999)
(0.6713,0.1999)

\PST@Solid(0.6563,0.7303)
(0.6713,0.7303)

\PST@Solid(0.5903,0.2447)
(0.5903,0.7751)

\PST@Solid(0.5828,0.2447)
(0.5978,0.2447)

\PST@Solid(0.5828,0.7751)
(0.5978,0.7751)

\PST@Solid(0.5167,0.2599)
(0.5167,0.7314)

\PST@Solid(0.5092,0.2599)
(0.5242,0.2599)

\PST@Solid(0.5092,0.7314)
(0.5242,0.7314)

\PST@Solid(0.4432,0.3341)
(0.4432,0.8056)

\PST@Solid(0.4357,0.3341)
(0.4507,0.3341)

\PST@Solid(0.4357,0.8056)
(0.4507,0.8056)

\PST@Solid(0.3733,0.2676)
(0.3733,0.6802)

\PST@Solid(0.3658,0.2676)
(0.3808,0.2676)

\PST@Solid(0.3658,0.6802)
(0.3808,0.6802)

\PST@Solid(0.2961,0.3942)
(0.2961,0.8067)

\PST@Solid(0.2886,0.3942)
(0.3036,0.3942)

\PST@Solid(0.2886,0.8067)
(0.3036,0.8067)

\PST@Solid(0.2225,0.3211)
(0.2225,0.6747)

\PST@Solid(0.2150,0.3211)
(0.2300,0.3211)

\PST@Solid(0.2150,0.6747)
(0.2300,0.6747)

\PST@Diamond(0.8845,0.5077)
\PST@Diamond(0.8219,0.5427)
\PST@Diamond(0.7374,0.5972)
\PST@Diamond(0.6638,0.4651)
\PST@Diamond(0.5903,0.5099)
\PST@Diamond(0.5167,0.4957)
\PST@Diamond(0.4432,0.5699)
\PST@Diamond(0.3733,0.4739)
\PST@Diamond(0.2961,0.6004)
\PST@Diamond(0.2225,0.4979)
\PST@Border(0.1490,0.9680)
(0.1490,0.0840)
(0.9580,0.0840)
(0.9580,0.9680)
(0.1490,0.9680)

\catcode`@=12
\fi
\endpspicture

\end{figure}
\begin{figure}[p]\caption{Grafico dei residui, con errore a posteriori.}
\centering
% GNUPLOT: LaTeX picture using PSTRICKS macros
% Define new PST objects, if not already defined
\ifx\PSTloaded\undefined
\def\PSTloaded{t}
\psset{arrowsize=.01 3.2 1.4 .3}
\psset{dotsize=.08}
\catcode`@=11

\newpsobject{PST@Border}{psline}{linewidth=.0015,linestyle=solid}
\newpsobject{PST@Axes}{psline}{linewidth=.0015,linestyle=dotted,dotsep=.004}
\newpsobject{PST@Solid}{psline}{linewidth=.0015,linestyle=solid}
\newpsobject{PST@Dashed}{psline}{linewidth=.0015,linestyle=dashed,dash=.01 .01}
\newpsobject{PST@Dotted}{psline}{linewidth=.0025,linestyle=dotted,dotsep=.008}
\newpsobject{PST@LongDash}{psline}{linewidth=.0015,linestyle=dashed,dash=.02 .01}
\newpsobject{PST@Diamond}{psdots}{linewidth=.001,linestyle=solid,dotstyle=square,dotangle=45}
\newpsobject{PST@Filldiamond}{psdots}{linewidth=.001,linestyle=solid,dotstyle=square*,dotangle=45}
\newpsobject{PST@Cross}{psdots}{linewidth=.001,linestyle=solid,dotstyle=+,dotangle=45}
\newpsobject{PST@Plus}{psdots}{linewidth=.001,linestyle=solid,dotstyle=+}
\newpsobject{PST@Square}{psdots}{linewidth=.001,linestyle=solid,dotstyle=square}
\newpsobject{PST@Circle}{psdots}{linewidth=.001,linestyle=solid,dotstyle=o}
\newpsobject{PST@Triangle}{psdots}{linewidth=.001,linestyle=solid,dotstyle=triangle}
\newpsobject{PST@Pentagon}{psdots}{linewidth=.001,linestyle=solid,dotstyle=pentagon}
\newpsobject{PST@Fillsquare}{psdots}{linewidth=.001,linestyle=solid,dotstyle=square*}
\newpsobject{PST@Fillcircle}{psdots}{linewidth=.001,linestyle=solid,dotstyle=*}
\newpsobject{PST@Filltriangle}{psdots}{linewidth=.001,linestyle=solid,dotstyle=triangle*}
\newpsobject{PST@Fillpentagon}{psdots}{linewidth=.001,linestyle=solid,dotstyle=pentagon*}
\newpsobject{PST@Arrow}{psline}{linewidth=.001,linestyle=solid}
\catcode`@=12

\fi
\psset{unit=5.0in,xunit=5.0in,yunit=3.0in}
\pspicture(0.000000,0.000000)(1.000000,1.000000)
\ifx\nofigs\undefined
\catcode`@=11

\PST@Border(0.1490,0.0840)
(0.1640,0.0840)

\rput[r](0.1330,0.0840){-0.004}
\PST@Border(0.1490,0.1822)
(0.1640,0.1822)

\rput[r](0.1330,0.1822){-0.003}
\PST@Border(0.1490,0.2804)
(0.1640,0.2804)

\rput[r](0.1330,0.2804){-0.002}
\PST@Border(0.1490,0.3787)
(0.1640,0.3787)

\rput[r](0.1330,0.3787){-0.001}
\PST@Border(0.1490,0.4769)
(0.1640,0.4769)

\rput[r](0.1330,0.4769){0.000}
\PST@Border(0.1490,0.5751)
(0.1640,0.5751)

\rput[r](0.1330,0.5751){0.001}
\PST@Border(0.1490,0.6733)
(0.1640,0.6733)

\rput[r](0.1330,0.6733){0.002}
\PST@Border(0.1490,0.7716)
(0.1640,0.7716)

\rput[r](0.1330,0.7716){0.003}
\PST@Border(0.1490,0.8698)
(0.1640,0.8698)

\rput[r](0.1330,0.8698){0.004}
\PST@Border(0.1490,0.9680)
(0.1640,0.9680)

\rput[r](0.1330,0.9680){0.005}
\PST@Border(0.1490,0.0840)
(0.1490,0.1040)

\rput(0.1490,0.0420){ 20}
\PST@Border(0.3329,0.0840)
(0.3329,0.1040)

\rput(0.3329,0.0420){ 25}
\PST@Border(0.5167,0.0840)
(0.5167,0.1040)

\rput(0.5167,0.0420){ 30}
\PST@Border(0.7006,0.0840)
(0.7006,0.1040)

\rput(0.7006,0.0420){ 35}
\PST@Border(0.8845,0.0840)
(0.8845,0.1040)

\rput(0.8845,0.0420){ 40}
\PST@Border(0.1490,0.9680)
(0.1490,0.0840)
(0.9580,0.0840)
(0.9580,0.9680)
(0.1490,0.9680)

\PST@Solid(0.8845,0.2428)
(0.8845,0.5887)

\PST@Solid(0.8770,0.2428)
(0.8920,0.2428)

\PST@Solid(0.8770,0.5887)
(0.8920,0.5887)

\PST@Solid(0.8219,0.3597)
(0.8219,0.7057)

\PST@Solid(0.8144,0.3597)
(0.8294,0.3597)

\PST@Solid(0.8144,0.7057)
(0.8294,0.7057)

\PST@Solid(0.7374,0.5411)
(0.7374,0.8870)

\PST@Solid(0.7299,0.5411)
(0.7449,0.5411)

\PST@Solid(0.7299,0.8870)
(0.7449,0.8870)

\PST@Solid(0.6638,0.1009)
(0.6638,0.4469)

\PST@Solid(0.6563,0.1009)
(0.6713,0.1009)

\PST@Solid(0.6563,0.4469)
(0.6713,0.4469)

\PST@Solid(0.5903,0.2501)
(0.5903,0.5960)

\PST@Solid(0.5828,0.2501)
(0.5978,0.2501)

\PST@Solid(0.5828,0.5960)
(0.5978,0.5960)

\PST@Solid(0.5167,0.2028)
(0.5167,0.5488)

\PST@Solid(0.5092,0.2028)
(0.5242,0.2028)

\PST@Solid(0.5092,0.5488)
(0.5242,0.5488)

\PST@Solid(0.4432,0.4502)
(0.4432,0.7961)

\PST@Solid(0.4357,0.4502)
(0.4507,0.4502)

\PST@Solid(0.4357,0.7961)
(0.4507,0.7961)

\PST@Solid(0.3733,0.1302)
(0.3733,0.4762)

\PST@Solid(0.3658,0.1302)
(0.3808,0.1302)

\PST@Solid(0.3658,0.4762)
(0.3808,0.4762)

\PST@Solid(0.2961,0.5521)
(0.2961,0.8980)

\PST@Solid(0.2886,0.5521)
(0.3036,0.5521)

\PST@Solid(0.2886,0.8980)
(0.3036,0.8980)

\PST@Solid(0.2225,0.2101)
(0.2225,0.5561)

\PST@Solid(0.2150,0.2101)
(0.2300,0.2101)

\PST@Solid(0.2150,0.5561)
(0.2300,0.5561)

\PST@Diamond(0.8845,0.4158)
\PST@Diamond(0.8219,0.5327)
\PST@Diamond(0.7374,0.7141)
\PST@Diamond(0.6638,0.2739)
\PST@Diamond(0.5903,0.4231)
\PST@Diamond(0.5167,0.3758)
\PST@Diamond(0.4432,0.6232)
\PST@Diamond(0.3733,0.3032)
\PST@Diamond(0.2961,0.7250)
\PST@Diamond(0.2225,0.3831)
\PST@Border(0.1490,0.9680)
(0.1490,0.0840)
(0.9580,0.0840)
(0.9580,0.9680)
(0.1490,0.9680)

\catcode`@=12
\fi
\endpspicture

\end{figure}
\end{document}
