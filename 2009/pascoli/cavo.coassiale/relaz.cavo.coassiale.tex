\documentclass[italian,a4paper]{article}
\usepackage[tight,nice]{units} %unità di misura
\usepackage{babel,amsmath,amssymb,amsthm,graphicx,url}
\usepackage[text={5.5in,9in},centering]{geometry}
\usepackage[utf8x]{inputenc}
\usepackage[T1]{fontenc}
\usepackage{ae,aecompl}
\usepackage[footnotesize,bf]{caption}
\usepackage[usenames]{color}
\usepackage{textcomp}
\usepackage{gensymb}
\usepackage{pstricks}
\frenchspacing
\pagestyle{plain}
%------------- eliminare prime e ultime linee isolate
\clubpenalty=9999%
\widowpenalty=9999
%--- definizione numerazioni
\renewcommand{\theequation}{\thesection.\arabic{equation}}
\renewcommand{\thefigure}{\arabic{figure}}
\renewcommand{\thetable}{\arabic{table}}
\addto\captionsitalian{%
\renewcommand{\figurename}%
{Grafico}%
}
%
%------------- ridefinizione simbolo per elenchi puntati: en dash
%\renewcommand{\labelitemi}{\textbf{--}}
\renewcommand{\labelenumi}{\textbf{\arabic{enumi}.}}
\setlength{\abovecaptionskip}{\baselineskip}   % 0.5cm as an example
\setlength{\floatsep}{2\baselineskip}
\setlength{\belowcaptionskip}{\baselineskip}   % 0.5cm as an example
%--------- comandi insiemi numeri complessi, naturali, reali e altre abbreviazioni
\renewcommand{\leq}{\leqslant}
%--------- porzione dedicata ai float in una pagina:
\renewcommand{\textfraction}{0.05}
\renewcommand{\topfraction}{0.95}
\renewcommand{\bottomfraction}{0.95}
\renewcommand{\floatpagefraction}{0.35}
\setcounter{totalnumber}{5}
%---------
%
%---------
\begin{document}
\title{Relazione di laboratorio: cavo coassiale}
\author{\normalsize Ilaria Brivio (582116)\\%
\normalsize \url{brivio.ilaria@tiscali.it}%
\and %
\normalsize Matteo Abis (584206)\\ %
\normalsize \url{webmaster@latinblog.org}
\and %
\normalsize Lorenzo Rossato (579393)\\ %
\normalsize \url{supergiovane05@hotmail.com}}
\date{\today}
\maketitle
%------------------
\noindent
Abbiamo utilizzato un cavo coassiale di lunghezza $\ell=\unit[49.95 \pm 0.02]{m}$.
%
%Misura della resistenza
Dopo aver cortocircuitato un estremo del cavo ne abbiamo misurato la
resistenza complessiva connettendo all'altro capo il multimetro T110B. La
misura, realizzata con fondoscala \unit[200]{\ohm}, è di \unit[3.2]{\ohm},
mentre la resistenza del cortocircuito, misurata con la stessa scala, è di
\unit[0.2]{\ohm}. Quest'ultima è stata sottratta, e la resistenza per metro
risulta
\begin{align*}
R &= \dfrac{3.0}{\ell}=\unit[60 \pm 8]{\ohm/km} \\
\dfrac{\Delta R}{R} &= \dfrac{\Delta R_{\text{tot}}}{R_{\text{tot}}} +
\dfrac{\Delta \ell}{\ell} = 14\% \end{align*}
L'errore sulla misura della resistenza è infatti lo 0.5\% + 4 digit.
%
%{Misura dell'impedenza caratteristica
Successivamente abbiamo connesso un potenziometro a fine linea, mentre all'altra estremità abbiamo collegato, con uno sdoppiatore a T, il generatore di funzioni e l'oscilloscopio in accoppiamento DC.
Con il generatore abbiamo immesso nella linea un'onda con le seguenti
caratteristiche:
\begin{table}[h]
    \centering
    \begin{tabular}{rl}
        forma: & onda quadra\\
        frequenza: & \unit[18.51]{kHz}\\
        ampiezza pp: & \unit[1.2]{V}\\
        valor medio: & \unit[0]{V}
    \end{tabular}
\end{table}\\
La connessione all'oscilloscopio è stata realizzata applicando a un capo della T la terminazione che separa i conduttori del cavo e collegando a questa una sonda precedentemente compensata.

Quando la resistenza del potenziometro vale circa \unit[30]{\ohm} il segnale sull'oscilloscopio ha l'andamento riportato in figura~\ref{30ohm}. A partire da questo valore abbiamo aumentato gradualmente la resistenza fino a osservare la scomparsa dell'onda riflessa: in queste condizioni il segnale ha assunto l'andamento in figura~\ref{50ohm}. Abbiamo quindi misurato il valore del potenziometro, pari all'impedenza caratteristica del cavo, con il multimetro T110B, fondoscala \unit[200]{\ohm}:
\begin{equation*}
Z_0=\unit[50.0 \pm 0.6]{\ohm}
\end{equation*}
%Misura dell'attenuazione e della velocità di percorrenza
Infine abbiamo sostituito il potenziometro a fine linea con un cortocircuito, mantenendo inalterata la configurazione a inizio linea, con lo stesso segnale in ingresso. L'andamento del segnale sull'oscilloscopio è riportato in figura~\ref{corto}.
Il tempo $t$ impiegato dal segnale per percorrere due volte la lunghezza
$\ell$ del cavo è dato dall'intervallo tra gli istanti $A$ e $B$. In quanto
segue chiameremo dunque:
\begin{align*}
    t &= t_B - t_A = \unit[504.0]{ns}\\
    V_0 &= V_B - V_A = \unit[608]{mV}\\
    V_r &= V_B - V_C = \unit[536]{mV}
\end{align*}
  \begin{figure}[h]
     \caption{Andamento qualitativo del segnale visto dall'oscilloscopio quando il segnale in ingresso è un gradino di tensione (onda quadra) e l'estremità del cavo è cortocircuitata. il risultato è dato dalla sovrapposizione dell'onda in ingresso e di quella riflessa. $t = t_B - t_A = \unit[504.0]{ns}$, $V_0 = V_B - V_A =
\unit[608]{mV}$, $V_r = V_B - V_C = \unit[536]{mV}$.}
     \begin{center}
         % GNUPLOT: LaTeX picture using PSTRICKS macros
% Define new PST objects, if not already defined
\ifx\PSTloaded\undefined
\def\PSTloaded{t}
\psset{arrowsize=.01 3.2 1.4 .3}
\psset{dotsize=.01}
\catcode`@=11

\newpsobject{PST@Border}{psline}{linewidth=.0015,linestyle=solid}
\newpsobject{PST@Axes}{psline}{linewidth=.0015,linestyle=dotted,dotsep=.004}
\newpsobject{PST@Solid}{psline}{linewidth=.0015,linestyle=solid}
\newpsobject{PST@Dashed}{psline}{linewidth=.0015,linestyle=dashed,dash=.01 .01}
\newpsobject{PST@Dotted}{psline}{linewidth=.0025,linestyle=dotted,dotsep=.008}
\newpsobject{PST@LongDash}{psline}{linewidth=.0015,linestyle=dashed,dash=.02 .01}
\newpsobject{PST@Diamond}{psdots}{linewidth=.001,linestyle=solid,dotstyle=square,dotangle=45}
\newpsobject{PST@Filldiamond}{psdots}{linewidth=.001,linestyle=solid,dotstyle=square*,dotangle=45}
\newpsobject{PST@Cross}{psdots}{linewidth=.001,linestyle=solid,dotstyle=+,dotangle=45}
\newpsobject{PST@Plus}{psdots}{linewidth=.001,linestyle=solid,dotstyle=+}
\newpsobject{PST@Square}{psdots}{linewidth=.001,linestyle=solid,dotstyle=square}
\newpsobject{PST@Circle}{psdots}{linewidth=.001,linestyle=solid,dotstyle=o}
\newpsobject{PST@Triangle}{psdots}{linewidth=.001,linestyle=solid,dotstyle=triangle}
\newpsobject{PST@Pentagon}{psdots}{linewidth=.001,linestyle=solid,dotstyle=pentagon}
\newpsobject{PST@Fillsquare}{psdots}{linewidth=.001,linestyle=solid,dotstyle=square*}
\newpsobject{PST@Fillcircle}{psdots}{linewidth=.001,linestyle=solid,dotstyle=*}
\newpsobject{PST@Filltriangle}{psdots}{linewidth=.001,linestyle=solid,dotstyle=triangle*}
\newpsobject{PST@Fillpentagon}{psdots}{linewidth=.001,linestyle=solid,dotstyle=pentagon*}
\newpsobject{PST@Arrow}{psline}{linewidth=.001,linestyle=solid}
\catcode`@=12

\fi
\psset{unit=5.0in,xunit=5.0in,yunit=3.0in}
\pspicture(0.000000,0.000000)(1.000000,1.000000)
\ifx\nofigs\undefined
\catcode`@=11

\PST@Border(0.1910,0.1344)
(0.2060,0.1344)

\rput[r](0.1750,0.1344){-200}
\PST@Border(0.1910,0.3011)
(0.2060,0.3011)

\rput[r](0.1750,0.3011){ 0}
\PST@Border(0.1910,0.4678)
(0.2060,0.4678)

\rput[r](0.1750,0.4678){ 200}
\PST@Border(0.1910,0.6346)
(0.2060,0.6346)

\rput[r](0.1750,0.6346){ 400}
\PST@Border(0.1910,0.8013)
(0.2060,0.8013)

\rput[r](0.1750,0.8013){ 600}
\PST@Border(0.1910,0.9680)
(0.2060,0.9680)

\rput[r](0.1750,0.9680){ 800}
\PST@Border(0.1910,0.1344)
(0.1910,0.1544)

\rput(0.1910,0.0924){ 0}
\PST@Border(0.2875,0.1344)
(0.2875,0.1544)

\rput(0.2875,0.0924){ 100}
\PST@Border(0.3840,0.1344)
(0.3840,0.1544)

\rput(0.3840,0.0924){ 200}
\PST@Border(0.4805,0.1344)
(0.4805,0.1544)

\rput(0.4805,0.0924){ 300}
\PST@Border(0.5770,0.1344)
(0.5770,0.1544)

\rput(0.5770,0.0924){ 400}
\PST@Border(0.6735,0.1344)
(0.6735,0.1544)

\rput(0.6735,0.0924){ 500}
\PST@Border(0.7700,0.1344)
(0.7700,0.1544)

\rput(0.7700,0.0924){ 600}
\PST@Border(0.8665,0.1344)
(0.8665,0.1544)

\rput(0.8665,0.0924){ 700}
\PST@Border(0.9630,0.1344)
(0.9630,0.1544)

\rput(0.9630,0.0924){ 800}
\PST@Border(0.1910,0.9680)
(0.1910,0.1344)
(0.9630,0.1344)
(0.9630,0.9680)
(0.1910,0.9680)

\rput{L}(0.0740,0.5512){potenziale (mV)}
\rput(0.5770,0.0294){tempo (ns)}
\rput[l](0.2950,0.2911){$A$}
\PST@Diamond(0.2875,0.2811)
\rput[l](0.7814,0.7979){$B$}
\PST@Diamond(0.7739,0.7879)
\rput[r](0.8103,0.3511){$C$}
\PST@Diamond(0.8028,0.3411)
\PST@Solid(0.1910,0.2811)
(0.1910,0.2811)
(0.2875,0.2811)
(0.3840,0.7879)
(0.7739,0.7879)
(0.8221,0.3411)
(0.9186,0.3411)

\PST@Border(0.1910,0.9680)
(0.1910,0.1344)
(0.9630,0.1344)
(0.9630,0.9680)
(0.1910,0.9680)

\catcode`@=12
\fi
\endpspicture

     \end{center}
     \label{corto}
 \end{figure}
La misura è effettuata con una scala di \unit[100]{ns}/div e
\unit[100]{mV/div}. Dunque la velocità del segnale lungo la linea, con
tolleranza  propagata linearmente da quella sul cavo e quella sulle
misure di tempo dell'oscilloscopio (dal manuale) è:
\begin{align*}
    v &=\dfrac{2\ell}{t}=\unit[(1.982 \pm 0.005) \cdot 10^8]{m/s} \approx
    \dfrac{2}{3} \text{c}    \\
    \dfrac{\Delta v}{v} &= \dfrac{\Delta \ell}{\ell} +
    \dfrac{\Delta t}{t} = 0.04\% + 0.2\%\\
    \Delta t &= 100\text{ppm} + \dfrac{\text{div}}{100} +
    \unit[0.6]{ns}
\end{align*}
Inoltre, si può ricavare l'attenuazione $\alpha$ dalla relazione:
\begin{equation*}
V_r=V_0 e^{-2\alpha \ell}
\end{equation*}
Gli errori di scala sulle
differenze di potenziale si eliminano perch\'e sono state misurate con
la stessa divisione:
\begin{align*}
\alpha &= \dfrac{1}{2l}\log{\dfrac{V_0}{V_r}} =
\unit[(1261.7 \pm 0.5) \cdot 10^{-6}]{m^{-1}}\\
\dfrac{\Delta \alpha}{\alpha} &= \dfrac{\Delta \ell}{\ell} = 0.04 \%
\end{align*}
Dalle misure effettuate abbiamo ricavato i valori di conduttanza $G$, induttanza $L$ e capacità $C$ del cavo per metro:
\begin{align*}
G &= \dfrac{R}{Z_0^2} = \unit[24 \pm 4]{\micro S/m}\\
L &= \dfrac{Z_0}{v} = \unit[0.252 \pm 0.004]{\micro H/m}\\
C &= \dfrac{1}{Z_0v} = \unit[100.9 \pm 1.4]{pF/m}\\
\end{align*}
 Le tolleranze sono associate allo stesso modo delle misure precedenti,
 ovvero con propagazione lineare:
\begin{align*}
    \dfrac{\Delta G}{G} &= 2\dfrac{\Delta Z_0}{Z_0} + \dfrac{\Delta R}{R} =
    16.4\%\\
    \dfrac{\Delta L}{L} &= \dfrac{\Delta C}{C} = \dfrac{\Delta Z_0}{Z_0} +
    \dfrac{\Delta v}{v} = 1.4\%\\
\end{align*}
 \begin{figure}[h]
     \caption{Andamento qualitativo del segnale visto dall'oscilloscopio quando la resistenza del potenziometro connesso a fine linea vale circa \unit[30]{\ohm}}
     \begin{center}
         % GNUPLOT: LaTeX picture using PSTRICKS macros
% Define new PST objects, if not already defined
\ifx\PSTloaded\undefined
\def\PSTloaded{t}
\psset{arrowsize=.01 3.2 1.4 .3}
\psset{dotsize=.01}
\catcode`@=11

\newpsobject{PST@Border}{psline}{linewidth=.0015,linestyle=solid}
\newpsobject{PST@Axes}{psline}{linewidth=.0015,linestyle=dotted,dotsep=.004}
\newpsobject{PST@Solid}{psline}{linewidth=.0015,linestyle=solid}
\newpsobject{PST@Dashed}{psline}{linewidth=.0015,linestyle=dashed,dash=.01 .01}
\newpsobject{PST@Dotted}{psline}{linewidth=.0025,linestyle=dotted,dotsep=.008}
\newpsobject{PST@LongDash}{psline}{linewidth=.0015,linestyle=dashed,dash=.02 .01}
\newpsobject{PST@Diamond}{psdots}{linewidth=.001,linestyle=solid,dotstyle=square,dotangle=45}
\newpsobject{PST@Filldiamond}{psdots}{linewidth=.001,linestyle=solid,dotstyle=square*,dotangle=45}
\newpsobject{PST@Cross}{psdots}{linewidth=.001,linestyle=solid,dotstyle=+,dotangle=45}
\newpsobject{PST@Plus}{psdots}{linewidth=.001,linestyle=solid,dotstyle=+}
\newpsobject{PST@Square}{psdots}{linewidth=.001,linestyle=solid,dotstyle=square}
\newpsobject{PST@Circle}{psdots}{linewidth=.001,linestyle=solid,dotstyle=o}
\newpsobject{PST@Triangle}{psdots}{linewidth=.001,linestyle=solid,dotstyle=triangle}
\newpsobject{PST@Pentagon}{psdots}{linewidth=.001,linestyle=solid,dotstyle=pentagon}
\newpsobject{PST@Fillsquare}{psdots}{linewidth=.001,linestyle=solid,dotstyle=square*}
\newpsobject{PST@Fillcircle}{psdots}{linewidth=.001,linestyle=solid,dotstyle=*}
\newpsobject{PST@Filltriangle}{psdots}{linewidth=.001,linestyle=solid,dotstyle=triangle*}
\newpsobject{PST@Fillpentagon}{psdots}{linewidth=.001,linestyle=solid,dotstyle=pentagon*}
\newpsobject{PST@Arrow}{psline}{linewidth=.001,linestyle=solid}
\catcode`@=12

\fi
\psset{unit=5.0in,xunit=5.0in,yunit=3.0in}
\pspicture(0.000000,0.000000)(1.000000,1.000000)
\ifx\nofigs\undefined
\catcode`@=11

\PST@Border(0.1910,0.2178)
(0.2060,0.2178)

\rput[r](0.1750,0.2178){-400}
\PST@Border(0.1910,0.3845)
(0.2060,0.3845)

\rput[r](0.1750,0.3845){-200}
\PST@Border(0.1910,0.5512)
(0.2060,0.5512)

\rput[r](0.1750,0.5512){ 0}
\PST@Border(0.1910,0.7179)
(0.2060,0.7179)

\rput[r](0.1750,0.7179){ 200}
\PST@Border(0.1910,0.8846)
(0.2060,0.8846)

\rput[r](0.1750,0.8846){ 400}
\PST@Border(0.1910,0.1344)
(0.1910,0.1544)

\rput(0.1910,0.0924){ 0}
\PST@Border(0.2682,0.1344)
(0.2682,0.1544)

\rput(0.2682,0.0924){ 5}
\PST@Border(0.3454,0.1344)
(0.3454,0.1544)

\rput(0.3454,0.0924){ 10}
\PST@Border(0.4226,0.1344)
(0.4226,0.1544)

\rput(0.4226,0.0924){ 15}
\PST@Border(0.4998,0.1344)
(0.4998,0.1544)

\rput(0.4998,0.0924){ 20}
\PST@Border(0.5770,0.1344)
(0.5770,0.1544)

\rput(0.5770,0.0924){ 25}
\PST@Border(0.6542,0.1344)
(0.6542,0.1544)

\rput(0.6542,0.0924){ 30}
\PST@Border(0.7314,0.1344)
(0.7314,0.1544)

\rput(0.7314,0.0924){ 35}
\PST@Border(0.8086,0.1344)
(0.8086,0.1544)

\rput(0.8086,0.0924){ 40}
\PST@Border(0.8858,0.1344)
(0.8858,0.1544)

\rput(0.8858,0.0924){ 45}
\PST@Border(0.9630,0.1344)
(0.9630,0.1544)

\rput(0.9630,0.0924){ 50}
\PST@Border(0.1910,0.9680)
(0.1910,0.1344)
(0.9630,0.1344)
(0.9630,0.9680)
(0.1910,0.9680)

\rput{L}(0.0740,0.5512){potenziale (mV)}
\rput(0.5770,0.0294){tempo (\micro s)}
\PST@Solid(0.1910,0.7446)
(0.1910,0.7446)
(0.3454,0.7446)
(0.3454,0.2244)
(0.3531,0.2244)
(0.3531,0.3511)
(0.7592,0.3511)
(0.7592,0.8780)
(0.7669,0.8780)
(0.7669,0.7446)
(0.9476,0.7446)

\PST@Border(0.1910,0.9680)
(0.1910,0.1344)
(0.9630,0.1344)
(0.9630,0.9680)
(0.1910,0.9680)

\catcode`@=12
\fi
\endpspicture

     \end{center}
     \label{30ohm}
 \end{figure}
 
\begin{figure}[h]
     \caption{Andamento qualitativo del segnale visto dall'oscilloscopio quando si osserva l'assenza di distorsioni dell'onda quadra in ingresso. In queste condizioni il cavo coassiale è terminato sulla propria impedenza caratteristica, che vale \unit[50.0]{\ohm}}
     \begin{center}
         % GNUPLOT: LaTeX picture using PSTRICKS macros
% Define new PST objects, if not already defined
\ifx\PSTloaded\undefined
\def\PSTloaded{t}
\psset{arrowsize=.01 3.2 1.4 .3}
\psset{dotsize=.01}
\catcode`@=11

\newpsobject{PST@Border}{psline}{linewidth=.0015,linestyle=solid}
\newpsobject{PST@Axes}{psline}{linewidth=.0015,linestyle=dotted,dotsep=.004}
\newpsobject{PST@Solid}{psline}{linewidth=.0015,linestyle=solid}
\newpsobject{PST@Dashed}{psline}{linewidth=.0015,linestyle=dashed,dash=.01 .01}
\newpsobject{PST@Dotted}{psline}{linewidth=.0025,linestyle=dotted,dotsep=.008}
\newpsobject{PST@LongDash}{psline}{linewidth=.0015,linestyle=dashed,dash=.02 .01}
\newpsobject{PST@Diamond}{psdots}{linewidth=.001,linestyle=solid,dotstyle=square,dotangle=45}
\newpsobject{PST@Filldiamond}{psdots}{linewidth=.001,linestyle=solid,dotstyle=square*,dotangle=45}
\newpsobject{PST@Cross}{psdots}{linewidth=.001,linestyle=solid,dotstyle=+,dotangle=45}
\newpsobject{PST@Plus}{psdots}{linewidth=.001,linestyle=solid,dotstyle=+}
\newpsobject{PST@Square}{psdots}{linewidth=.001,linestyle=solid,dotstyle=square}
\newpsobject{PST@Circle}{psdots}{linewidth=.001,linestyle=solid,dotstyle=o}
\newpsobject{PST@Triangle}{psdots}{linewidth=.001,linestyle=solid,dotstyle=triangle}
\newpsobject{PST@Pentagon}{psdots}{linewidth=.001,linestyle=solid,dotstyle=pentagon}
\newpsobject{PST@Fillsquare}{psdots}{linewidth=.001,linestyle=solid,dotstyle=square*}
\newpsobject{PST@Fillcircle}{psdots}{linewidth=.001,linestyle=solid,dotstyle=*}
\newpsobject{PST@Filltriangle}{psdots}{linewidth=.001,linestyle=solid,dotstyle=triangle*}
\newpsobject{PST@Fillpentagon}{psdots}{linewidth=.001,linestyle=solid,dotstyle=pentagon*}
\newpsobject{PST@Arrow}{psline}{linewidth=.001,linestyle=solid}
\catcode`@=12

\fi
\psset{unit=5.0in,xunit=5.0in,yunit=3.0in}
\pspicture(0.000000,0.000000)(1.000000,1.000000)
\ifx\nofigs\undefined
\catcode`@=11

\PST@Border(0.1910,0.2178)
(0.2060,0.2178)

\rput[r](0.1750,0.2178){-400}
\PST@Border(0.1910,0.3845)
(0.2060,0.3845)

\rput[r](0.1750,0.3845){-200}
\PST@Border(0.1910,0.5512)
(0.2060,0.5512)

\rput[r](0.1750,0.5512){ 0}
\PST@Border(0.1910,0.7179)
(0.2060,0.7179)

\rput[r](0.1750,0.7179){ 200}
\PST@Border(0.1910,0.8846)
(0.2060,0.8846)

\rput[r](0.1750,0.8846){ 400}
\PST@Border(0.1910,0.1344)
(0.1910,0.1544)

\rput(0.1910,0.0924){ 0}
\PST@Border(0.3840,0.1344)
(0.3840,0.1544)

\rput(0.3840,0.0924){ 5}
\PST@Border(0.5770,0.1344)
(0.5770,0.1544)

\rput(0.5770,0.0924){ 10}
\PST@Border(0.7700,0.1344)
(0.7700,0.1544)

\rput(0.7700,0.0924){ 15}
\PST@Border(0.9630,0.1344)
(0.9630,0.1544)

\rput(0.9630,0.0924){ 20}
\PST@Border(0.1910,0.9680)
(0.1910,0.1344)
(0.9630,0.1344)
(0.9630,0.9680)
(0.1910,0.9680)

\rput{L}(0.0740,0.5512){potenziale (mV)}
\rput(0.5770,0.0294){tempo (\micro s)}
\PST@Solid(0.1910,0.2878)
(0.1910,0.2878)
(0.5770,0.2878)
(0.5770,0.7979)
(0.9630,0.7979)

\PST@Border(0.1910,0.9680)
(0.1910,0.1344)
(0.9630,0.1344)
(0.9630,0.9680)
(0.1910,0.9680)

\catcode`@=12
\fi
\endpspicture

     \end{center}
     \label{50ohm}
 \end{figure}
\end{document}
