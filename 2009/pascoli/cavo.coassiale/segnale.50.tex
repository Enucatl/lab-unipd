% GNUPLOT: LaTeX picture using PSTRICKS macros
% Define new PST objects, if not already defined
\ifx\PSTloaded\undefined
\def\PSTloaded{t}
\psset{arrowsize=.01 3.2 1.4 .3}
\psset{dotsize=.01}
\catcode`@=11

\newpsobject{PST@Border}{psline}{linewidth=.0015,linestyle=solid}
\newpsobject{PST@Axes}{psline}{linewidth=.0015,linestyle=dotted,dotsep=.004}
\newpsobject{PST@Solid}{psline}{linewidth=.0015,linestyle=solid}
\newpsobject{PST@Dashed}{psline}{linewidth=.0015,linestyle=dashed,dash=.01 .01}
\newpsobject{PST@Dotted}{psline}{linewidth=.0025,linestyle=dotted,dotsep=.008}
\newpsobject{PST@LongDash}{psline}{linewidth=.0015,linestyle=dashed,dash=.02 .01}
\newpsobject{PST@Diamond}{psdots}{linewidth=.001,linestyle=solid,dotstyle=square,dotangle=45}
\newpsobject{PST@Filldiamond}{psdots}{linewidth=.001,linestyle=solid,dotstyle=square*,dotangle=45}
\newpsobject{PST@Cross}{psdots}{linewidth=.001,linestyle=solid,dotstyle=+,dotangle=45}
\newpsobject{PST@Plus}{psdots}{linewidth=.001,linestyle=solid,dotstyle=+}
\newpsobject{PST@Square}{psdots}{linewidth=.001,linestyle=solid,dotstyle=square}
\newpsobject{PST@Circle}{psdots}{linewidth=.001,linestyle=solid,dotstyle=o}
\newpsobject{PST@Triangle}{psdots}{linewidth=.001,linestyle=solid,dotstyle=triangle}
\newpsobject{PST@Pentagon}{psdots}{linewidth=.001,linestyle=solid,dotstyle=pentagon}
\newpsobject{PST@Fillsquare}{psdots}{linewidth=.001,linestyle=solid,dotstyle=square*}
\newpsobject{PST@Fillcircle}{psdots}{linewidth=.001,linestyle=solid,dotstyle=*}
\newpsobject{PST@Filltriangle}{psdots}{linewidth=.001,linestyle=solid,dotstyle=triangle*}
\newpsobject{PST@Fillpentagon}{psdots}{linewidth=.001,linestyle=solid,dotstyle=pentagon*}
\newpsobject{PST@Arrow}{psline}{linewidth=.001,linestyle=solid}
\catcode`@=12

\fi
\psset{unit=5.0in,xunit=5.0in,yunit=3.0in}
\pspicture(0.000000,0.000000)(1.000000,1.000000)
\ifx\nofigs\undefined
\catcode`@=11

\PST@Border(0.1910,0.2178)
(0.2060,0.2178)

\rput[r](0.1750,0.2178){-400}
\PST@Border(0.1910,0.3845)
(0.2060,0.3845)

\rput[r](0.1750,0.3845){-200}
\PST@Border(0.1910,0.5512)
(0.2060,0.5512)

\rput[r](0.1750,0.5512){ 0}
\PST@Border(0.1910,0.7179)
(0.2060,0.7179)

\rput[r](0.1750,0.7179){ 200}
\PST@Border(0.1910,0.8846)
(0.2060,0.8846)

\rput[r](0.1750,0.8846){ 400}
\PST@Border(0.1910,0.1344)
(0.1910,0.1544)

\rput(0.1910,0.0924){ 0}
\PST@Border(0.3840,0.1344)
(0.3840,0.1544)

\rput(0.3840,0.0924){ 5}
\PST@Border(0.5770,0.1344)
(0.5770,0.1544)

\rput(0.5770,0.0924){ 10}
\PST@Border(0.7700,0.1344)
(0.7700,0.1544)

\rput(0.7700,0.0924){ 15}
\PST@Border(0.9630,0.1344)
(0.9630,0.1544)

\rput(0.9630,0.0924){ 20}
\PST@Border(0.1910,0.9680)
(0.1910,0.1344)
(0.9630,0.1344)
(0.9630,0.9680)
(0.1910,0.9680)

\rput{L}(0.0740,0.5512){potenziale (mV)}
\rput(0.5770,0.0294){tempo (\micro s)}
\PST@Solid(0.1910,0.2878)
(0.1910,0.2878)
(0.5770,0.2878)
(0.5770,0.7979)
(0.9630,0.7979)

\PST@Border(0.1910,0.9680)
(0.1910,0.1344)
(0.9630,0.1344)
(0.9630,0.9680)
(0.1910,0.9680)

\catcode`@=12
\fi
\endpspicture
