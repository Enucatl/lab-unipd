% !TEX encoding = UTF-8 Unicode
\documentclass[italian,a4paper]{article}
\usepackage[tight,nice]{units} %unità di misura
\usepackage{babel,amsmath,amssymb,amsthm,graphicx,url,textcomp,gensymb, ctable}
\graphicspath{{./matteo/}}
\usepackage[text={5.5in,9in},centering]{geometry}
\usepackage[utf8x]{inputenc}
\usepackage[T1]{fontenc}
\usepackage{ae,aecompl}
\usepackage[small,bf]{caption}
\frenchspacing
\pagestyle{plain}
%------------- eliminare prime e ultime linee isolate
\clubpenalty=9999%
\widowpenalty=9999
%--- definizione numerazioni
\renewcommand{\theequation}{\thesection.\arabic{equation}}
\renewcommand{\thefigure}{\thesection.\arabic{figure}}
\renewcommand{\thetable}{\arabic{table}}
\addto\captionsitalian{\renewcommand{\figurename}{Fgura}}
%------------- ridefinizione simbolo per elenchi puntati: en dash
\renewcommand{\labelenumi}{\textbf{\arabic{enumi}.}}
\renewcommand{\labelitemi}{\textbf{- }}
\setlength{\abovecaptionskip}{\baselineskip}   % 0.5cm as an example
\setlength{\floatsep}{2\baselineskip}
\setlength{\belowcaptionskip}{\baselineskip}   % 0.5cm as an example
%--------- comandi insiemi numeri complessi, naturali, reali e altre abbreviazioni
\renewcommand{\leq}{\leqslant}
\renewcommand{\geq}{\geqslant}
%--------- porzione dedicata ai float in una pagina:
\renewcommand{\textfraction}{0.05}
\renewcommand{\topfraction}{0.95}
\renewcommand{\bottomfraction}{0.95}
\renewcommand{\floatpagefraction}{0.35}
\setcounter{totalnumber}{5}
%-------------------------ambiente codice
\usepackage{listings}
\lstset{language=C++,inputencoding=utf8,breaklines=true,extendedchars=true, basicstyle=\small, columns=flexible, linewidth=\textwidth, lineskip=1pt, breakatwhitespace=true, lastline=99999, showstringspaces=false}
\lstloadlanguages{C++}
%-----------------------elenco compatto
\newenvironment{packed_item}{
\begin{itemize}
  \setlength{\itemsep}{1pt}
  \setlength{\parskip}{0pt}
  \setlength{\parsep}{0pt}
}{\end{itemize}}
\newenvironment{packed_enum}{
\begin{enumerate}
  \setlength{\itemsep}{1pt}
  \setlength{\parskip}{0pt}
  \setlength{\parsep}{0pt}
}{\end{enumerate}}
%------------------------- pacchetto tikz
\usepackage{tikz}
\usetikzlibrary{shapes.geometric,shapes.arrows,decorations.pathmorphing}
\usetikzlibrary{matrix,chains,scopes,positioning,arrows,fit,shapes.multipart}%, shapes.gates.logic.IEC}

%------------shortcuts
\newcommand{\g}{\gamma}
\renewcommand{\a}{$_{\alpha}$}
\renewcommand{\b}{$_{\beta}$}
\newcommand{\s}{\sigma}
\renewcommand{\d}{\delta}
\newcommand{\D}{\Delta}
\newcommand{\Na}{$^{22}$Na}
\newcommand{\Ne}{$^{22}$Ne}
\newcommand{\Am}{$^{241}$Am}
\newcommand{\Np}{$^{237}$Np}

%---------
%
%---------
\begin{document}
\title{Fluorescenza X II (Fisica Ambientale)}
\author{\normalsize Ilaria Brivio (1014190)\\%
\normalsize \url{brivio.ilaria@tiscali.it}%
\and %
\normalsize Matteo Abis (584206)\\ %		%%% Matricola Matteo da modificare
\normalsize \url{webmaster@latinblog.org}
}
\date{\today}
\date{\today}
\maketitle
%------------------
\section{Obiettivi dell'esperimento}
\begin{itemize}
\item Verifica della legge di attenuazione della radiazione elettromagnetica attraverso un materiale
$$ I = I_{0}e^{-\mu x}$$
 e stima del coefficiente di attenuazione $\mu$ nel caso dell'assorbimento in alluminio delle emissioni di una sorgente di \Am
\item Verifica della legge di Mooseley che lega il numero atomico del materiale all'energia delle sue emissioni X 
$$ E = k (Z - 1)^{2}$$
attraverso l'analisi dello spettro di emissione di sei campioni mono-elementali
\item Analisi della composizione di un campione di particolato atmosferico e di altri campioni multi-elementali
\end{itemize}

\section{Apparato strumentale}
\subsection*{Descrizione dell'apparato}
L'apparato sperimentale è costituito da un contenitore al cui interno sono montati:
\begin{packed_item}
\item un generatore di raggi X a cristallo piroelettrico COOL X
\item un rivelatore al silicio XR100-CR
\item un portacampioni estraibile che permette il corretto posizionamento dei target.
\end{packed_item}
Il rivelatore è connesso a un PC, da cui è possibile gestire l'acquisizione degli spettri mediante un apposito software.

% campioni e sorgente Am

\subsection*{Calibrazione dell'apparato e analisi dello spettro di \Am}
Per la calibrazione in energia del sistema di acquisizione è stata posizionata una sorgente di \Am dinanzi al rivelatore ed è stato acquisito uno spettro di prova per un tempo di \unit[314.28]{s}.
Sono stati identificati su questo spettro tre righe di riferimento: quelle delle transizioni X nel \Np, di energia 13.95 e \unit[17.74]{keV}, e quella del decadimento $\g$ del \Am, di energia \unit[59.54]{keV}. Sono stati stimati centroide $x$, ampiezza a mezza altezza $\s$ ed integrale $N$ di ciascun picco: % modalità di calcolo (programma o rifatto con fit) 
\begin{table}[h!]\centering
\begin{tabular}{*4c}
$E$ (keV)&		$x$ (canale)&		$\s$(can.)&	$N$\\\hline
13.95&			199.95&			0.261&		3097\\
17.74&			255.39&			0.293&		2370\\
59.54&			863.50&			0.378&		269
\end{tabular}
\caption{Centroide, ampiezza a mezza altezza e integrale dei tre picchi di riferimento per la calibrazione del sistema con una sorgente di \Am.}\label{calib}
\end{table}\\
A partire da questi valori sono stati deteminati i parametri $a = \unit[0.0685]{keV / can.}$ e $b=\unit[0.2944]{can.}$ per la formula di calibrazione
\begin{equation}
E = a x + b
\end{equation}.

% individuazione altri picchi

\section{Verifica della legge di assorbimento e stima di $\mu$}
Sono state prese in considerazione le radiazioni di energia 13.95 e \unit[17.74]{keV} emesse dalla sorgente di \Am. Sono stati acquisiti sei diversi spettri, frapponendo tra sorgente e rivelatore un numero crescente di fogli di alluminio.

Per stimare lo spessore $d$ di un singolo foglio è stato pesato un campione di $S = \unit[25]{cm^{2}}$. Nota la densità $\rho = \unit[2.7]{g / cm^{3}}$:
\begin{equation}
d = \frac{\rho}{m}\frac{1}{S} = \frac{\unit[2.7]{g / cm^{3}}}{\unit[1.55]{g}} \frac{1}{ \unit[25]{cm^{2}}}= \unit[0.23]{mm}
\end{equation}

\begin{table}[h!]\centering
\begin{tabular}{*4c}
spessore Al (mm)&		tempo (s)&	$N_{13.95}$&		$N_{17.74}$\\\hline
0&					326.06&		3151&			2367\\
0.23&				341.95&		1845&			1974\\
0.46&				589.05&		1972&			2466\\
0.69&				1088.86&		2026&			3535\\
0.92&				2013.59&		2021&			4750\\
1.15&				2027.53&		1317&			3732
\end{tabular}
\caption{Tempi di acquisizione e integrali dei due picchi di interesse per i sei spettri acquisiti con spessori crescenti di Al.}\label{assorb}
\end{table}

\section{Verifica della legge di Mooseley}
Per la verifica della legge di Mooseley sono stati analizzati dodici campioni mono-elementali, fissati su un apposito supporto in alluminio.
Per ciascuno di essi sono stati individuati i picchi corrispondenti alle emissioni K\a, K\b, L\a, L\b del materiale considerato.


\section{Analisi di campioni multi-elementali}
\subsection*{Particolato atmosferico}
Il campione di particolato atmosferico è stato ottenuto con un sistema che preleva l'aria mediante una pompa facendola passare attraverso un filtro antipolvere.

Sono stati analizzati due diversi campioni:
%campioni part.


\subsection*{Altri campioni}

\section{Conclusioni}

\clearpage
%\section{Appendice}
%\begin{figure}[h!]\centering
%\includegraphics[width=.8\textwidth]{prova_coinc23.r4.eps}
%\caption{Spettro del rivelatore R4 acquisito nella prova per le coincidenze tra R2 ed R3.}\label{gr:prova23.r4}
%\end{figure}
\end{document}

 grafici disponibili:
am calib
spessori : 0 1 2 3 4 5 x .22mm Al
monoelemento: fe cu ni y nb in
multielemento: banconota, particolato



