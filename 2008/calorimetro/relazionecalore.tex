\documentclass[italian,a4paper]{article}
\usepackage[tight,nice]{units}
\usepackage{babel,amsmath,amssymb,amsthm,graphicx,url,gensymb}
\usepackage[text={5.5in,9in},centering]{geometry}
\usepackage[utf8x]{inputenc}
\usepackage[T1]{fontenc}
\usepackage{ae,aecompl}
\usepackage[footnotesize,bf]{caption}
\usepackage[usenames]{color}
\include{pstricks}
\frenchspacing
\pagestyle{plain}
%------------- eliminare prime e ultime linee isolate
\clubpenalty=9999%
\widowpenalty=9999
%--- definizione numerazioni
\renewcommand{\theequation}{\thesection.\arabic{equation}}
\renewcommand{\thefigure}{\arabic{figure}}
\renewcommand{\thetable}{\thesection.\arabic{table}}
\addto\captionsitalian{%
  \renewcommand{\figurename}%
{Grafico}%
}
%
%------------- ridefinizione simbolo per elenchi puntati: en dash
%\renewcommand{\labelitemi}{\textbf{--}}
\renewcommand{\labelenumi}{\textbf{\arabic{enumi}.}}
\setlength{\abovecaptionskip}{\baselineskip}   % 0.5cm as an example
\setlength{\floatsep}{2\baselineskip}
\setlength{\belowcaptionskip}{\baselineskip}   % 0.5cm as an example
%--------- comandi insiemi numeri complessi, naturali, reali e altre abbreviazioni
\renewcommand{\leq}{\leqslant}
%--------- porzione dedicata ai float in una pagina:
\renewcommand{\textfraction}{0.05}
\renewcommand{\topfraction}{0.95}
\renewcommand{\bottomfraction}{0.95}
\renewcommand{\floatpagefraction}{0.35}
\setcounter{totalnumber}{5}
%---------
%
%---------
\begin{document}
\title{Relazione di laboratorio: il calorimetro}
\author{\normalsize Ilaria Brivio (582116)\\%
\normalsize \url{brivio.ilaria@tiscali.it}%
\and %
\normalsize Matteo Abis (584206)\\ %
\normalsize \url{webmaster@latinblog.org}}
\date{\today}
\maketitle
%------------------
\section{Obiettivo dell'esperienza}
Obiettivo dell'esperienza è misurare il calore specifico di un corpo di materiale sconosciuto.
\section{Descrizione dell'apparato strumentale}
Lo strumento impiegato è un calorimetro delle mescolanze di Regnault. Il contenitore esterno isola un cilindro interno in rame stagnato (calore specifico $c_r = \unitfrac[0.093\pm0.001]{cal}{g\celsius}$) che appoggia sul fondo sostenuto da punte di plastica. Nel cilindro si versa poi acqua distillata ($c_a = \unitfrac[1]{cal}{g\celsius}$ per definizione). Attraverso il coperchio si possono introdurre un termometro con sensibilità $\unit[10]{\celsius^{-1}}$ e un agitatore di rame stagnato che permette di mescolare l'acqua all'interno per rendere il più possibile omogenea la temperatura. Nel calorimetro si introduce infine un cilindro di metallo precedentemente riscaldato a una temperatura  di $\unit[100.0\pm0.1]{\celsius}$ con una resistenza. Lo scambio termico avviene tra il sistema campione e l'ambiente, dove per ambiente si intende l'acqua, il contenitore dell'acqua, l'agitatore e il termometro. Il termometro ha capacità termica $k=\unitfrac[2.20\pm0.05]{cal}{\celsius}$. Per pesare il corpo e l'acqua introdotta nel contenitore si usa una bilancia elettronica di sensibilità $\unit[100]{g^{-1}}$, come errore statistico si usa un terzo di questo valore, così come per il termometro.
\section{Descrizione della metodologia di misura}
Si stimano innanzitutto le masse degli strumenti impiegati con una pesata semplice di contenitore e agitatore prima ($m_r$), poi del contenitore e agitatore con acqua ($m_t$) e infine del cilindro campione ($m_x$). La massa dell'acqua $m_a$ risulta per differenza e l'errore è associato per propagazione. Dalle misure, in grammi, risultano:
\begin{table}[h]\centering
\begin{tabular}{r@{$=$}r@{$\pm$}ll}
 $m_r$ & 60.490 &0.003\\
 $m_t$ & 190.350&0.003\\
 $m_x$ & 81.500&0.003\\
 $m_a$ & 129.860&0.005
\end{tabular}
\end{table}\\
Il campione è stato riscaldato fino a una temperatura di $\unit[100.0\pm0.1]{\celsius}$ e poi trasferito rapidamente nel calorimetro. Tuttavia si stima che nel tragitto la sua temperatura sia diminuita di $\unit[1.5\pm1.0]{\celsius}$. Come temperatura iniziale si considera perciò $T_{c,i} = \unit[98.5\pm1.0]{\celsius}$. Prima di immergere il cilindro nell'acqua, si è misurata la temperatura del sistema $T_A = \unit[20.40\pm0.03]{\celsius}$.
Dall'inserimento del campione, è stata registrata la temperatura dell'acqua a intervalli di quindici secondi fino al raggiungimento di un massimo e ogni trenta secondi dopo il massimo fino a una differenza di temperatura di \unit[0.2]{\celsius}.
\section{Risultati sperimentali ed elaborazione dati}
\begin{table}[t]\centering
\begin{tabular}{rlrl }
15 &20.60 &30 &21.45\\
45 &22.45 &60 &23.40\\
75 &24.90 &90 &25.30\\
105 &25.40 &120 &25.50\\
135 &25.50 &150 &25.50\\
165 &25.48 &180 &25.45\\
195 &25.44 &225 &25.40\\
255 &25.40 &285 &25.40\\
315 &25.38 &345 &25.35\\
375 &25.32 &405 &25.30\\
435 &25.30 &465 &25.29\\
495 &25.28 &525 &25.26\\
555 &25.24 &585 &25.22
\end{tabular}
\end{table}
Disponendo la temperatura letta dal termometro in funzione del tempo (grafico~\ref{temp}) si vede che questa ha un transiente iniziale di rapida crescita e poi una diminuzione pressoché lineare dovuta alla dispersione di calore verso l'esterno. Interpolando i dati dopo il massimo con una retta si può ricavare un'intercetta che sarà la miglior stima della temperatura di equilibrio $T_f$. Tale intercetta risulta $T_f=\unit[25.564\pm0.009]{\celsius}$.Come temperatura di equilibrio $T_f$ si è usata la stima fornita dall'interpolazione piuttosto che il massimo valore osservato perché quest'ultimo è affetto da un errore maggiore a causa della dispersione del calore e delle disomogeneità nella temperatura del liquido.
\begin{figure}[t]\caption{Temperatura rilevata in ordinata, tempo in ascissa.}\label{temp}
\centering 
% GNUPLOT: LaTeX picture using PSTRICKS macros
% Define new PST objects, if not already defined
\ifx\PSTloaded\undefined
\def\PSTloaded{t}
\psset{arrowsize=.01 3.2 1.4 .3}
\psset{dotsize=.08}
\catcode`@=11

\newpsobject{PST@Border}{psline}{linewidth=.0015,linestyle=solid}
\newpsobject{PST@Axes}{psline}{linewidth=.0015,linestyle=dotted,dotsep=.004}
\newpsobject{PST@Solid}{psline}{linewidth=.0015,linestyle=solid}
\newpsobject{PST@Dashed}{psline}{linewidth=.0015,linestyle=dashed,dash=.01 .01}
\newpsobject{PST@Dotted}{psline}{linewidth=.0025,linestyle=dotted,dotsep=.008}
\newpsobject{PST@LongDash}{psline}{linewidth=.0015,linestyle=dashed,dash=.02 .01}
\newpsobject{PST@Diamond}{psdots}{linewidth=.001,linestyle=solid,dotstyle=*}
\newpsobject{PST@Filldiamond}{psdots}{linewidth=.001,linestyle=solid,dotstyle=square*,dotangle=45}
\newpsobject{PST@Cross}{psdots}{linewidth=.001,linestyle=solid,dotstyle=+,dotangle=45}
\newpsobject{PST@Plus}{psdots}{linewidth=.001,linestyle=solid,dotstyle=+}
\newpsobject{PST@Square}{psdots}{linewidth=.001,linestyle=solid,dotstyle=square}
\newpsobject{PST@Circle}{psdots}{linewidth=.001,linestyle=solid,dotstyle=o}
\newpsobject{PST@Triangle}{psdots}{linewidth=.001,linestyle=solid,dotstyle=triangle}
\newpsobject{PST@Pentagon}{psdots}{linewidth=.001,linestyle=solid,dotstyle=pentagon}
\newpsobject{PST@Fillsquare}{psdots}{linewidth=.001,linestyle=solid,dotstyle=square*}
\newpsobject{PST@Fillcircle}{psdots}{linewidth=.001,linestyle=solid,dotstyle=*}
\newpsobject{PST@Filltriangle}{psdots}{linewidth=.001,linestyle=solid,dotstyle=triangle*}
\newpsobject{PST@Fillpentagon}{psdots}{linewidth=.001,linestyle=solid,dotstyle=pentagon*}
\newpsobject{PST@Arrow}{psline}{linewidth=.001,linestyle=solid}
\catcode`@=12

\fi
\psset{unit=5.0in,xunit=5.0in,yunit=3.0in}
\pspicture(0.000000,0.000000)(1.000000,1.000000)
\ifx\nofigs\undefined
\catcode`@=11

\PST@Border(0.1590,0.1260)
(0.1740,0.1260)

\rput[r](0.1430,0.1260){20.5}
\PST@Border(0.1590,0.2025)
(0.1740,0.2025)

\rput[r](0.1430,0.2025){21.0}
\PST@Border(0.1590,0.2791)
(0.1740,0.2791)

\rput[r](0.1430,0.2791){21.5}
\PST@Border(0.1590,0.3556)
(0.1740,0.3556)

\rput[r](0.1430,0.3556){22.0}
\PST@Border(0.1590,0.4322)
(0.1740,0.4322)

\rput[r](0.1430,0.4322){22.5}
\PST@Border(0.1590,0.5087)
(0.1740,0.5087)

\rput[r](0.1430,0.5087){23.0}
\PST@Border(0.1590,0.5853)
(0.1740,0.5853)

\rput[r](0.1430,0.5853){23.5}
\PST@Border(0.1590,0.6618)
(0.1740,0.6618)

\rput[r](0.1430,0.6618){24.0}
\PST@Border(0.1590,0.7384)
(0.1740,0.7384)

\rput[r](0.1430,0.7384){24.5}
\PST@Border(0.1590,0.8149)
(0.1740,0.8149)

\rput[r](0.1430,0.8149){25.0}
\PST@Border(0.1590,0.8915)
(0.1740,0.8915)

\rput[r](0.1430,0.8915){25.5}
\PST@Border(0.1590,0.9680)
(0.1740,0.9680)

\rput[r](0.1430,0.9680){26.0}
\PST@Border(0.1590,0.1260)
(0.1590,0.1460)

\rput(0.1590,0.0840){0}
\PST@Border(0.2903,0.1260)
(0.2903,0.1460)

\rput(0.2903,0.0840){100}
\PST@Border(0.4217,0.1260)
(0.4217,0.1460)

\rput(0.4217,0.0840){200}
\PST@Border(0.5530,0.1260)
(0.5530,0.1460)

\rput(0.5530,0.0840){300}
\PST@Border(0.6843,0.1260)
(0.6843,0.1460)

\rput(0.6843,0.0840){400}
\PST@Border(0.8157,0.1260)
(0.8157,0.1460)

\rput(0.8157,0.0840){500}
\PST@Border(0.9470,0.1260)
(0.9470,0.1460)

\rput(0.9470,0.0840){600}
\PST@Border(0.1590,0.9680)
(0.1590,0.1260)
(0.9470,0.1260)
(0.9470,0.9680)
(0.1590,0.9680)

\rput{L}(0.0420,0.5470){temperatura (\unit{\celsius})}
\rput(0.5530,0.0210){tempo (\unit{s})}
\PST@Diamond(0.1787,0.1413)
\PST@Diamond(0.1984,0.2714)
\PST@Diamond(0.2181,0.4245)
\PST@Diamond(0.2378,0.5700)
\PST@Diamond(0.2575,0.7996)
\PST@Diamond(0.2772,0.8608)
\PST@Diamond(0.2969,0.8761)
\PST@Diamond(0.3166,0.8915)
\PST@Diamond(0.3363,0.8915)
\PST@Diamond(0.3560,0.8915)
\PST@Diamond(0.3757,0.8884)
\PST@Diamond(0.3954,0.8838)
\PST@Diamond(0.4151,0.8823)
\PST@Diamond(0.4545,0.8761)
\PST@Diamond(0.4939,0.8761)
\PST@Diamond(0.5333,0.8761)
\PST@Diamond(0.5727,0.8731)
\PST@Diamond(0.6121,0.8685)
\PST@Diamond(0.6515,0.8639)
\PST@Diamond(0.6909,0.8608)
\PST@Diamond(0.7303,0.8608)
\PST@Diamond(0.7697,0.8593)
\PST@Diamond(0.8091,0.8578)
\PST@Diamond(0.8485,0.8547)
\PST@Diamond(0.8879,0.8517)
\PST@Diamond(0.9273,0.8486)
\PST@Solid(0.1787,0.8999)
(0.1787,0.8999)
(0.1863,0.8994)
(0.1938,0.8988)
(0.2014,0.8983)
(0.2089,0.8978)
(0.2165,0.8973)
(0.2241,0.8967)
(0.2316,0.8962)
(0.2392,0.8957)
(0.2468,0.8951)
(0.2543,0.8946)
(0.2619,0.8941)
(0.2694,0.8936)
(0.2770,0.8930)
(0.2846,0.8925)
(0.2921,0.8920)
(0.2997,0.8915)
(0.3072,0.8909)
(0.3148,0.8904)
(0.3224,0.8899)
(0.3299,0.8893)
(0.3375,0.8888)
(0.3451,0.8883)
(0.3526,0.8878)
(0.3602,0.8872)
(0.3677,0.8867)
(0.3753,0.8862)
(0.3829,0.8857)
(0.3904,0.8851)
(0.3980,0.8846)
(0.4055,0.8841)
(0.4131,0.8835)
(0.4207,0.8830)
(0.4282,0.8825)
(0.4358,0.8820)
(0.4434,0.8814)
(0.4509,0.8809)
(0.4585,0.8804)
(0.4660,0.8799)
(0.4736,0.8793)
(0.4812,0.8788)
(0.4887,0.8783)
(0.4963,0.8777)
(0.5038,0.8772)
(0.5114,0.8767)
(0.5190,0.8762)
(0.5265,0.8756)
(0.5341,0.8751)
(0.5417,0.8746)
(0.5492,0.8741)
(0.5568,0.8735)
(0.5643,0.8730)
(0.5719,0.8725)
(0.5795,0.8719)
(0.5870,0.8714)
(0.5946,0.8709)
(0.6022,0.8704)
(0.6097,0.8698)
(0.6173,0.8693)
(0.6248,0.8688)
(0.6324,0.8683)
(0.6400,0.8677)
(0.6475,0.8672)
(0.6551,0.8667)
(0.6626,0.8661)
(0.6702,0.8656)
(0.6778,0.8651)
(0.6853,0.8646)
(0.6929,0.8640)
(0.7005,0.8635)
(0.7080,0.8630)
(0.7156,0.8625)
(0.7231,0.8619)
(0.7307,0.8614)
(0.7383,0.8609)
(0.7458,0.8603)
(0.7534,0.8598)
(0.7609,0.8593)
(0.7685,0.8588)
(0.7761,0.8582)
(0.7836,0.8577)
(0.7912,0.8572)
(0.7988,0.8567)
(0.8063,0.8561)
(0.8139,0.8556)
(0.8214,0.8551)
(0.8290,0.8545)
(0.8366,0.8540)
(0.8441,0.8535)
(0.8517,0.8530)
(0.8592,0.8524)
(0.8668,0.8519)
(0.8744,0.8514)
(0.8819,0.8509)
(0.8895,0.8503)
(0.8971,0.8498)
(0.9046,0.8493)
(0.9122,0.8487)
(0.9197,0.8482)
(0.9273,0.8477)

\PST@Border(0.1590,0.9680)
(0.1590,0.1260)
(0.9470,0.1260)
(0.9470,0.9680)
(0.1590,0.9680)

\catcode`@=12
\fi
\endpspicture

\end{figure}
Si stima infine il calore specifico, con errore per propagazione, del campione:
\begin{equation*}
 c_x =\dfrac{(m_a c_a + m_r c_r +k)(T_f-T_A)}{m_x(T_{c,i}-T_f)} = \unitfrac[0.120\pm0.002]{cal}{g\celsius}
\end{equation*}
\section{Conclusioni}

\end{document}
