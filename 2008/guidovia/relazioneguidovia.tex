\documentclass[italian,a4paper]{article}
\usepackage[tight,nice]{units}
\usepackage{babel,amsmath,amssymb,amsthm,graphicx,url,wrapfig,multirow}
\usepackage[text={6in,9in},centering]{geometry}
\usepackage[utf8x]{inputenc}
\usepackage[T1]{fontenc}
\usepackage{ae,aecompl}
\usepackage[Euler]{upgreek}
\usepackage[footnotesize,bf]{caption}
\usepackage[usenames]{color}
\frenchspacing
\pagestyle{plain}
%------------- eliminare prime e ultime linee isolate
\clubpenalty=9999%
\widowpenalty=9999
%--- definizione numerazioni
\renewcommand{\theequation}{\thesection.\arabic{equation}}
\renewcommand{\thefigure}{\thesection.\arabic{figure}}
\renewcommand{\thetable}{\thesection.\arabic{table}}
\addto\captionsitalian{%
  \renewcommand{\figurename}%
{Grafico}%
}
%
%------------- ridefinizione simbolo per elenchi puntati: en dash
%\renewcommand{\labelitemi}{\textbf{--}}
\renewcommand{\labelenumi}{\textbf{(\roman{enumi})}}
\setlength{\abovecaptionskip}{\baselineskip}   % 0.5cm as an example
\setlength{\floatsep}{2\baselineskip}
\setlength{\belowcaptionskip}{\baselineskip}   % 0.5cm as an example
%------------- nuovi environment senza spazi
%\newenvironment{packed_item}{
%\begin{itemize}
%  \setlength{\itemsep}{1pt}
%  \setlength{\parskip}{0pt}
%  \setlength{\parsep}{0pt}
%}{\end{itemize}}
%\newenvironment{packed_enum}{
%\begin{enumerate}
%  \setlength{\itemsep}{1pt}
%  \setlength{\parskip}{0pt}
%  \setlength{\parsep}{0pt}
%}{\end{enumerate}}
%\newenvironment{packed_description}{
%\begin{enumerate}
%   \setlength{\itemsep}{1pt}
%   \setlength{\parskip}{0pt}
%   \setlength{\parsep}{0pt}
% }{\end{enumerate}}
%--------- comandi insiemi numeri complessi, naturali, reali e altre abbreviazioni
\newcommand{\e}{\mathrm{e}}
\newcommand{\ms}{(\unitfrac{m}{s})}
\newcommand{\epsi}{\varepsilon}
\newcommand{\eqnum}{\setcounter{equation}{0}}
\newcommand{\spazio}{\vspace{.5\baselineskip}\\}
\newcommand{\nessuno}{\emph{n}}
\newcommand{\sottile}{\emph{s}}
\newcommand{\grosso}{\emph{g}}
\newcommand{\ciccione}{\emph{sg}}
\newcommand{\di}{\mathrm{d}} %simbolo di differenziale
\renewcommand{\leq}{\leqslant}
\renewcommand{\pi}{\uppi} % costa
%--------- porzione dedicata ai float in una pagina:
\renewcommand{\textfraction}{0.05}
\renewcommand{\topfraction}{0.95}
\renewcommand{\bottomfraction}{0.95}
\renewcommand{\floatpagefraction}{0.35}
\setcounter{totalnumber}{5}
%---------
%
%---------
\begin{document}
\title{Relazione di Laboratorio: la guidovia}
\author{\normalsize Ilaria Brivio (582116)\\%
\normalsize \url{brivio.ilaria@tiscali.it}%
\and %
\normalsize Matteo Abis (584206)\\ %
\normalsize \url{webmaster@latinblog.org}}
\date{\today}
\maketitle
%------------------
\section{Obiettivo dell'esperienza}
L'obiettivo dell'esperienza è stimare l'accelerazione di gravità misurando l'accelerazione su un piano inclinato, tenendo anche in considerazione il ruolo dell'attrito.
\section{Descrizione dell'apparato strumentale}
Il piano inclinato è costituito da una guidovia di alluminio, la cui inclinazione è regolabile mediante una vite nella misura di $5'$~per ogni giro. Un compressore pompa aria all'interno della guida, che fuoriesce da fori sulla superficie, tenendo sospesa la slitta e riducendo quindi l'attrito. Alla partenza la slitta è tenuta ferma da un elettromagnete, che si disattiva con un pulsante. Il cronometro ha una sensibilità di $\unit[10^4]{s^{-1}}$ e si attiva automaticamente al passaggio della slitta su due traguardi fissati a una scala graduata con sensibilità $\unit[10^3]{m^{-1}}$.

Sono state utilizzate due slitte diverse per le due parti dell'esperimento. La prima ha le due estremità di materiale ferromagnetico, che viene quindi attratto dall'elettromagnete sulla guida. La seconda, invece, ha a un'estremità un magnete permanente, montato in modo da essere respinto, e all'altra estremità un cerchio di velcro. Sulle slitte può essere montato un carico supplementare di ottone. Nella seconda parte dell'esperienza sono stati inoltre impiegati due spessori metallici diversi per modificare l'effetto dell'elettromagnete sulla slitta.
\section{Descrizione della metodologia di misura}
Per la prima parte dell'esperienza, ovvero per la stima della costante $g$, è stato fissato il primo traguardo sulla misura di $\unit[40]{cm}$ sulla scala, misurando cinque tempi di percorrenza per ogni posizione del secondo traguardo a $50$, 60, 70, etc. fino a $\unit[130]{cm}$ e per ogni angolo $\alpha = 15', 30',45'$ e $45'$ con slitta carica. Da queste misure sono stati poi calcolati per differenza i tempi di percorrenza di ciascun tratto lungo $\unit[10]{cm}$.

Come errore sull'orizzontalità della guidovia è stata stimata la semidispersione tra le posizioni della vite effettuate con la slitta il più possibile ferma alle estremità e al centro del percorso. Tale errore risulta essere circa $1/9$ di giro della vite, ovvero $\delta_{\alpha} = \unit[1.62 \cdot 10^{-4}]{rad}$.

L'errore sulla posizione dei traguardi è stato determinato nel modo seguente: un componente del gruppo ha spostato per dieci volte in modo casuale un traguardo lungo la guida e lo ha riportato alla misura originaria, e l'altro ha stimato la più piccola sottosuddivisione rispetto alla quale poteva misurare una differenza di posizionamento. Tale errore è risultato quindi essere $\sigma_{s} = \unit[2\cdot 10^{-4}]{m}$.

Per stimare invece l'effetto dell'attrito nel moto, è stata posizionata la guida orizzontalmente con la stessa procedura impiegata nella stima dell'errore sull'angolo $\alpha$. Sono stati misurati cinque tempi di percorrenza di tratti lunghi $\unit[20]{cm}$ (40--60, 50--70, $\dots$, 110--130), ripetendo le misure prima senza spessori, poi con il solo spessore sottile, quindi con quello grosso e infine con entrambi, sia con la slitta carica che con la slitta scarica.
\section{Risultati sperimentali ed elaborazione dati}
\subsection{Stima dell'accelerazione di gravità}
Sono riportati nelle tabelle~\ref{15tab}, \ref{30tab}, \ref{45tab}, \ref{45ctab}, e i tempi misurati, di cui sono stati calcolati le medie, lo scarto quadratico medio, l'errore quadratico medio, l'errore sulla media, i tempi di percorrenza di ciascun tratto di $\unit[10]{cm}$, per differenza, e le relative velocità medie. Gli errori sulle misure indirette sono stati ricavati dalla formula di propagazione degli errori. Per esempio, gli errori sulle velocità medie sono:
\begin{equation*}
 \sigma_{f(x,y)}^2 = \left(\dfrac{\partial f}{\partial x}\sigma_x \right)^2 + \left(\dfrac{\partial f}{\partial y}\sigma_y \right)^2 \qquad
\sigma_{\bar v}^2 = \Big( -\dfrac{s}{t ^2}\sigma_{ t} \Big)^2 + \Big(\dfrac{1}{ t}\sigma_{ s} \Big)^2
\end{equation*}
Per ricavare l'accelerazione lungo il piano inclinato sono stati interpolati i dati delle velocità medie (grafici~\ref{15graf}, \ref{30graf}, \ref{45graf}, \ref{45cgraf}) con una relazione lineare ($v=at+b$). Dividendo l'accelerazione per il seno del corrispondente angolo sono stati ricavati i valori di $g$~\footnote{Come errore statistico associato ad $\alpha$, da inserire nella formula di propagazione degli errori relativa a $g$, è stato usato $\nicefrac{\delta_{\alpha}}{3} = \unit[2.02\cdot 10^{-5}]{rad}$.}. I risultati, con i loro errori, sono riassunti nella seguente tabella\footnote{Il pedice $_c$ indica le misure relative alla slitta carica.}:
\begin{table}[h]
\centering
 \begin{tabular}{l  *3{r@{.}l @{ $\pm$ }r@{.}l}}
  &\multicolumn{4}{c}{$a$ (\nicefrac{m}{s$^2$})}
  &\multicolumn{4}{c}{$b$ (\nicefrac{m}{s})}
  &\multicolumn{4}{c}{$g$ (\nicefrac{m}{s$^2$})}\\ \hline
  $15'$ & 0&038 &0&003 &0&035 &0&007 &8&70 & 0&62\\
  $30'$ & 0&083 &0&002 &0&197 &0&003 &9&52 & 0&21\\
  $45'$ & 0&126 &0&002 &0&243 &0&002 &9&62 & 0&12\\
  $45'_c$&0&125 &0&002 &0&245 &0&002 &9&54 & 0&13\\
 \end{tabular}
\end{table}\\
Queste misure sono state confrontate fra loro e con il valore atteso a Padova\footnote{Non è riportato nessun errore su questa misura perché lo si può ritenere irrilevante.} $g=\unitfrac[9.806]{m}{s^2}$:
\begin{table}[h]\centering
\begin{tabular}{r|*4{r}}
  $\lambda$	& $45'_c$ 	& $45'$ 	& $30'$ 	&$15'$	\\
 \hline
 $g$ 		& $2.05$	& $1.55$	&$1.36$		&1.78	\\
 $15'$  	& $1.33$	& $1.46$	&1.25		& \\
 $30'$  	& $0.08$	& 0.41		& 	&\\
 $45'$  	& $0.45$	& 		& 	&
\end{tabular}
\end{table}\\
In particolare la compatibilità è mediocre (tra 1 e 2) se non scarsa (tra 2 e 3) rispetto al valore atteso, anche se è in genere buona (minore di 1) tra le misure sperimentali, escluso il valore per l'inclinazione di $15'$.
\subsection{Effetti dell'attrito sul moto}
Come è evidente dai grafici e dalle tabelle in appendice, la seconda parte dell'esperimento si può considerare come non riuscita a causa dell'inadeguata scelta dell'orizzontalità lungo quasi tutta la guidovia. La slitta infatti accelera dopo il traguardo dei \unit[70]{cm} in qualsiasi condizione di carico e di schermatura dell'elettromagnete. A questo si aggiunge il fatto che, completate le misure dei tratti \unit[40--60]{cm} e \unit[50--70]{cm} si è notato che il compressore trasmetteva delle vibrazioni alla slitta, che procedeva quindi urtando contro le pareti laterali della guida. Spostando il compressore su un altro tavolo l'effetto è sparito, cosa che ha ulteriormente accorciato i tempi di percorrenza di $5$--$10$~centesimi di secondo.

Per quanto abbiano scarso valore sperimentale, possiamo accettare \emph{a posteriori} solo i dati elaborati a partire dai tratti~\unit[40--60]{cm} e~\unit[50--70]{cm}. Si vorrebbe correggere la stima di $g$ calcolando il coefficiente di attrito $c$ della slitta con l'aria:
\begin{equation}\label{g}
 g = \dfrac{1}{\sin\alpha}\left(a + \dfrac{c}{m}v \right)
\end{equation}
Sono stati ricavati i valori di $C : = \nicefrac{c}{m}$, e come approssimazione della velocità $v$ è stata impiegata la velocità \emph{complessiva}, ovvero la media delle velocità sui primi e sugli ultimi dieci centimetri della guida: $v = (v_{130}-v_{50})/2$.

Come spiegato sopra, il valore di $C$ considerato è stato ricavato dal coefficiente angolare della retta passante per i primi due punti dei grafici (vedi appendice). Nella tabella sottostante sono riportati i valori ottenuti, con gli errori calcolati per propagazione. L'unità di misura è $\unit[10^{-3}]{s^{-1}}$.
\begin{table}[!h]
\centering
 \begin{tabular}{*2{r} r@{ $\pm$ }l l @{\hspace{3\tabcolsep}}*2{r} r@{ $\pm$ }l}
 & & $C$ & $\sigma_C$ & & & &$C$ & $\sigma_C$\\\cline{2-4} \cline{7-9}
\multirow{4}{*}{scarico} &\nessuno &$-7.573$&$19.910$ & &\multirow{4}{*}{carico} &\nessuno &$-2.059$&$31.457$\\
&\sottile &$-11.123$&$24.097$ & & &\sottile &$-12.657$&$19.549$\\
&\grosso &$-8.933$&$16.820$ & & &\grosso &$2.666$&$41.631$\\
&\ciccione &$11.105$&$36.377$ & & &\ciccione &$-4.258$&$36.169$\\
 \end{tabular}
\end{table}\\
Si nota la presenza di due valori positivi, che quindi sono esclusi dal calcolo della media pesata dei~$C$:
\begin{equation}\label{mediapesata}
 \bar{C}=\left(\sum_i \dfrac{C_i}{\sigma_{C_i}^2} \right)\left(\sum_i \dfrac{1}{\sigma_{C_i}^2} \right)^{-1} \qquad \sigma_{\bar{C}} = \left(\sum_i \dfrac{1}{\sigma_{C_i}^2} \right)^{-\frac 1 2}
\end{equation}
Le medie così calcolate sono, rispettivamente per la slitta scarica e la slitta carica:
\begin{align*}
 \bar{C}_s &= \unit[(-8.977 \pm 0.128)\cdot 10^{-3}]{s^{-1}}\\
  \bar{C}_c &= \unit[(-8.756 \pm 0.228)\cdot 10^{-3}]{s^{-1}}
\end{align*}
Applicando infine questi valori nell'equazione~\eqref{g} si può correggere la stima dell'accelerazione di gravità:
\begin{table}[h]
\centering
 \begin{tabular}{l  r@{.}l @{ $\pm$ }r@{.}l}
  &\multicolumn{4}{c}{$g$ (\nicefrac{m}{s$^2$})}\\[2 pt]
  $15'$   &9&143 & 0&902\\
  $30'$   &9&849 & 0&292\\
  $45'$   &9&896 & 0&161\\
  $45'_c$ &9&805 & 0&168
 \end{tabular}
\end{table}\\
La compatibilità delle misure con il valore atteso di $g$ risulta, nonostante la scarsa rigorosità della procedura, molto buona:
\begin{table}[h]\centering
\begin{tabular}{r|*4{r}}
  $\lambda$	& $45'_c$ 	& $45'$ 	& $30'$ 	&$15'$	\\
 \hline
 $g$ 		& $0.006$	& $0.559$	&$0.147$	&$0.735$\\
 $15'$  	& $0.722$	& $0.822$	&$0.559$	& \\
 $30'$  	& $0.131$	& $0.141$	& 	&\\
 $45'$  	& $0.391$	& 		& 	&
\end{tabular}
\end{table}\\
\section{Discussione dei risultati}
I valori di $g$ ottenuti nella prima parte dell'esperienza sono tutti chiaramente in difetto e poco compatibili con il valore atteso. Trascurare l'attrito è quindi evidentemente un'approssimazione molto forte. In particolare, il valore dell'accelerazione con la guidovia inclinata di $15'$ è molto più basso e soffre di un errore molto maggiore rispetto agli altri. Questo può essere dovuto al fatto che la slitta più lenta e leggera subisce maggiormente gli effetti di piccole deformazioni della guida, improvvise vibrazioni trasmesse dal tavolo o bruschi movimenti d'aria. Dei cinque punti finali del grafico~\ref{15graf}, tre sono nettamente sotto la retta interpolante. Escludendo questi il valore di $g$ sarebbe stimato in $\unitfrac[9.72 \pm 0.32]{m}{s^{2}}$. Se tali errori fossero di natura sistematica e dovuti alla forma della guida, lo schema si ripresenterebbe anche nei grafici successivi, cosa che però non è affatto evidente. Un'altra fonte di errore sistematico per un gruppo di misure consecutive può essere dovuto all'elettromagnete. Se, infatti, l'operatore lascia il pulsante che lo disattiva prima che la slitta si sia allontanata a sufficienza, ci può essere un piccolo rallentamento dovuto all'attrazione magnetica. Questo effetto è stato osservato più volte durante l'esperimento e si può quantificare in un ritardo di 2--3 centesimi di secondo. Inoltre, a parità di tempo trascorso dalla partenza, la slitta che percorre la guida meno inclinata si è allontanata meno, e questo spiegherebbe perché tale fenomeno assuma proporzioni sensibili solo con la prima serie di misure. Un'altra ipotesi è il cattivo posizionamento della guida nell'inclinazione voluta di $15'$, ma sembra da escludere poiché un errore di questo tipo si ripercuoterebbe su tutte le misure con la stessa inclinazione.

\section{Conclusioni}
La prima parte dell'esperienza ha prodotto valori di $g$ sistematicamente in difetto, a causa dell'approssimazione della formula, che trascurava gli attriti. Inoltre, l'errore più grande si è presentato sulle misure relative all'angolo più piccolo, per i motivi indicati sopra. Non è stata rilevata invece nessuna differenza significativa tra le altre misure con slitta scarica e quelle con slitta carica.

La seconda parte, pur essendo scarsamente valida, ha comunque permesso di correggere in modo soddisfacente le stime di $g$. Dei valori di $g$ così ottenuti, risulta più compatibile con l'accelerazione attesa quello relativo alla slitta carica. 
\newpage
\section{Appendice}
\begin{table}[h]\caption{Rilevazioni dei tempi di percorrenza~(s) per l'inclinazione di $15'$,
con cinque misure e la media sui vari traguardi~(cm). Sono riportati anche i tempi sui singoli tratti di \unit[10]{cm}, ottenuti per differenza, e le velocità medie~(\nicefrac{m}{s}) su tali tratti. Gli errori riportati sono scarto quadratico medio $s$, errore quadratico medio $\sigma$ ed errore sulla media, tutti nella stessa unità di misura del valore cui fanno riferimento.}\label{15tab}
 \centering \small
 \begin{tabular}{r *9{r@{.}l}}
 &\multicolumn{2}{c}{50}
&\multicolumn{2}{c}{60}
&\multicolumn{2}{c}{70}
&\multicolumn{2}{c}{80}
&\multicolumn{2}{c}{90}
&\multicolumn{2}{c}{100}
&\multicolumn{2}{c}{110}
&\multicolumn{2}{c}{120}
&\multicolumn{2}{c}{130}\\[3 pt]
  &0&6330 &1&2300 &1&7445 &2&2010 &2&6261 &3&0141 &3&4015 &3&7527 &4&1305\\
  &0&6633 &1&2271 &1&7497 &2&1999 &2&6458 &3&0103 &3&4115 &3&7837 &4&0995\\
  &0&6579 &1&2137 &1&7358 &2&1892 &2&6303 &3&0161 &3&4081 &3&7695 &4&1135\\
  &0&6591 &1&2233 &1&7453 &2&1749 &2&6395 &3&0267 &3&4099 &3&7246 &4&1383\\
  &0&6634 &1&2339 &1&7407 &2&2063 &2&6153 &3&0198 &3&4101 &3&7536 &4&0861\\[3 pt]
$\bar{t}$&0&6553 &1&2256 &1&7432 &2&1943 &2&6314 &3&0174 &3&4082 &3&7568 &4&1136\\
$s$&0&0114 &0&0069 &0&0047 &0&0112 &0&0106 &0&0056 &0&0035 &0&0197 &0&0192\\
$\sigma$&0&0127 &0&0077 &0&0052 &0&0125 &0&0118 &0&0062 &0&0039 &0&0221 &0&0215\\
$\sigma_{\bar{t}}$&0&0057 &0&0034 &0&0023 &0&0056 &0&0053 &0&0028 &0&0018 &0&0099 &0&0096\\[3 pt]
$\Delta t$&0&6553 &0&5703 &0&5176 &0&4511 &0&4371 &0&3860 &0&3908 &0&3486 &0&3568\\
$\sigma_{\Delta t}$&0&0057 &0&0067 &0&0042 &0&0060 &0&0077 &0&0060 &0&0033 &0&0100 &0&0138\\
$\bar{v}$&0&1526 &0&1754 &0&1932 &0&2217 &0&2288 &0&2591 &0&2559 &0&2869 &0&2803\\
$\sigma_{\bar{v}}$ &0&0013 &0&0020 &0&0016 &0&0030 &0&0040 &0&0040 &0&0022 &0&0083 &0&0108
\end{tabular}
\end{table}
\begin{table}[h]\caption{Rilevazioni dei tempi di percorrenza~(s) per l'inclinazione di $30'$,
con cinque misure e la media sui vari traguardi~(cm). Sono riportati anche i tempi sui singoli tratti di \unit[10]{cm}, ottenuti per differenza, e le velocità medie~(\nicefrac{m}{s}) su tali tratti. Gli errori riportati sono scarto quadratico medio $s$, errore quadratico medio $\sigma$ ed errore sulla media, tutti nella stessa unità di misura del valore cui fanno riferimento.}\label{30tab}
 \centering \small
 \begin{tabular}{r *9{r@{.}l}}
 &\multicolumn{2}{c}{50}
&\multicolumn{2}{c}{60}
&\multicolumn{2}{c}{70}
&\multicolumn{2}{c}{80}
&\multicolumn{2}{c}{90}
&\multicolumn{2}{c}{100}
&\multicolumn{2}{c}{110}
&\multicolumn{2}{c}{120}
&\multicolumn{2}{c}{130}\\[3 pt]
  &0&4623 &0&8597 &1&2116 &1&5369 &1&8285 &2&0985 &2&3639 &2&6291 &2&8580\\
  &0&4640 &0&8594 &1&2089 &1&5267 &1&8290 &2&1013 &2&3697 &2&6073 &2&8571\\
  &0&4627 &0&8559 &1&2156 &1&5389 &1&8273 &2&1019 &2&3763 &2&6153 &2&8505\\
  &0&4625 &0&8582 &1&2087 &1&5377 &1&8308 &2&1089 &2&3710 &2&6261 &2&8465\\
  &0&4623 &0&8558 &1&2100 &1&5291 &1&8299 &2&1161 &2&3751 &2&6301 &2&8529\\[3 pt]
$\bar{t}$ &0&4628 &0&8578 &1&2110 &1&5339 &1&8291 &2&1053 &2&3712 &2&6216 &2&8530\\
$s$ &0&0006 &0&0017 &0&0025 &0&0050 &0&0012 &0&0064 &0&0044 &0&0089 &0&0043\\
$\sigma$ &0&0007 &0&0019 &0&0028 &0&0056 &0&0013 &0&0071 &0&0049 &0&0099 &0&0048\\
$\sigma_{\bar{t}}$ &0&0003 &0&0008 &0&0013 &0&0025 &0&0006 &0&0032 &0&0022 &0&0044 &0&0021\\[3 pt]
$\Delta t$ &0&4628 &0&3950 &0&3532 &0&3229 &0&2952 &0&2762 &0&2659 &0&2504 &0&2314\\
$\sigma_{\Delta t}$ &0&0003 &0&0009 &0&0015 &0&0028 &0&0026 &0&0032 &0&0039 &0&0049 &0&0049\\
$\bar{v}$ &0&2161 &0&2531 &0&2832 &0&3097 &0&3387 &0&3620 &0&3761 &0&3994 &0&4321\\
$\sigma_{\bar{v}}$ &0&0001 &0&0006 &0&0012 &0&0027 &0&0029 &0&0043 &0&0055 &0&0079 &0&0092
\end{tabular}
\end{table}
\begin{table}[p]\caption{Rilevazioni dei tempi di percorrenza~(s) per l'inclinazione di $45'$,
con cinque misure e la media sui vari traguardi~(cm). Sono riportati anche i tempi sui singoli tratti di \unit[10]{cm}, ottenuti per differenza, e le velocità medie~(\nicefrac{m}{s}) su tali tratti. Gli errori riportati sono scarto quadratico medio $s$, errore quadratico medio $\sigma$ ed errore sulla media, tutti nella stessa unità di misura del valore cui fanno riferimento.}\label{45tab}
 \centering \small
 \begin{tabular}{r *9{r@{.}l}}
 &\multicolumn{2}{c}{50}
&\multicolumn{2}{c}{60}
&\multicolumn{2}{c}{70}
&\multicolumn{2}{c}{80}
&\multicolumn{2}{c}{90}
&\multicolumn{2}{c}{100}
&\multicolumn{2}{c}{110}
&\multicolumn{2}{c}{120}
&\multicolumn{2}{c}{130}\\[3 pt]
  &0&3751 &0&6972 &0&9835 &1&2461 &1&4865 &1&7147 &1&9255 &2&1240 &2&3191\\
  &0&3753 &0&6969 &0&9843 &1&2455 &1&4832 &1&7099 &1&9275 &2&1307 &2&3164\\
  &0&3757 &0&6971 &0&9855 &1&2451 &1&4862 &1&7102 &1&9229 &2&1252 &2&3145\\
  &0&3763 &0&6971 &0&9864 &1&2415 &1&4833 &1&7117 &1&9267 &2&1257 &2&3163\\
  &0&3744 &0&6971 &0&9853 &1&2471 &1&4869 &1&7081 &1&9278 &2&1274 &2&3169\\[3 pt]
$\bar{t}$ &0&3754 &0&6970 &0&9850 &1&2451 &1&4852 &1&7109 &1&9261 &2&1266 &2&3166\\
$s$ &0&0006 &0&0001 &0&0010 &0&0019 &0&0016 &0&0022 &0&0018 &0&0023 &0&0015\\
$\sigma$ &0&0007 &0&0001 &0&0011 &0&0021 &0&0018 &0&0025 &0&0020 &0&0026 &0&0016\\
$\sigma_{\bar{t}}$ &0&0003 &0&0001 &0&0005 &0&0010 &0&0008 &0&0011 &0&0009 &0&0012 &0&0007\\[3 pt]
$\Delta t$ &0&3754 &0&3217 &0&2879 &0&2601 &0&2402 &0&2257 &0&2152 &0&2005 &0&1900\\
$\sigma_{\Delta t}$ &0&0003 &0&0003 &0&0005 &0&0011 &0&0013 &0&0014 &0&0014 &0&0015 &0&0014\\
$\bar{v}$ &0&2664 &0&3108 &0&3473 &0&3845 &0&4164 &0&4431 &0&4648 &0&4987 &0&5262\\
$\sigma_{\bar{v}}$ &0&0002 &0&0003 &0&0006 &0&0016 &0&0022 &0&0027 &0&0031 &0&0036 &0&0038
\end{tabular}
\end{table}
\begin{table}[p]\caption{Rilevazioni dei tempi di percorrenza~(s) per l'inclinazione di $45'$ con slitta carica,
con cinque misure e la media sui vari traguardi~(cm). Sono riportati anche i tempi sui singoli tratti di \unit[10]{cm}, ottenuti per differenza, e le velocità medie~(\nicefrac{m}{s}) su tali tratti. Gli errori riportati sono scarto quadratico medio $s$, errore quadratico medio $\sigma$ ed errore sulla media, tutti nella stessa unità di misura del valore cui fanno riferimento.}\label{45ctab}
 \centering \small
 \begin{tabular}{r *9{r@{.}l}}
 &\multicolumn{2}{c}{50}
&\multicolumn{2}{c}{60}
&\multicolumn{2}{c}{70}
&\multicolumn{2}{c}{80}
&\multicolumn{2}{c}{90}
&\multicolumn{2}{c}{100}
&\multicolumn{2}{c}{110}
&\multicolumn{2}{c}{120}
&\multicolumn{2}{c}{130}\\[3 pt]
   &0&3755 &0&6957 &0&9843 &1&2425 &1&4837 &1&7025 &1&9183 &2&1163 &2&3195\\
   &0&3749 &0&6955 &0&9827 &1&2413 &1&4825 &1&7071 &1&9185 &2&1177 &2&3095\\
   &0&3746 &0&6963 &0&9821 &1&2413 &1&4823 &1&7063 &1&9175 &2&1184 &2&3107\\
   &0&3751 &0&6959 &0&9827 &1&2415 &1&4833 &1&7083 &1&9183 &2&1175 &2&3104\\
   &0&3757 &0&6965 &0&9827 &1&2404 &1&4825 &1&7071 &1&9153 &2&1192 &2&3093\\[3 pt]
$\bar{t}$ &0&3752 &0&6960 &0&9829 &1&2414 &1&4829 &1&7063 &1&9176 &2&1178 &2&3119\\
$s$ &0&0004 &0&0004 &0&0007 &0&0007 &0&0005 &0&0020 &0&0012 &0&0010 &0&0038\\
$\sigma$ &0&0004 &0&0004 &0&0008 &0&0007 &0&0006 &0&0022 &0&0013 &0&0011 &0&0043\\
$\sigma_{\bar{t}}$ &0&0002 &0&0002 &0&0004 &0&0003 &0&0003 &0&0010 &0&0006 &0&0005 &0&0019\\[3 pt]
$\Delta t$ &0&3752 &0&3208 &0&2869 &0&2585 &0&2415 &0&2234 &0&2113 &0&2002 &0&1941\\
$\sigma_{\Delta t}$ &0&0002 &0&0003 &0&0004 &0&0005 &0&0004 &0&0010 &0&0012 &0&0008 &0&0020\\
$\bar{v}$ &0&2666 &0&3117 &0&3485 &0&3868 &0&4141 &0&4476 &0&4732 &0&4994 &0&5153\\
$\sigma_{\bar{v}}$ &0&0001 &0&0003 &0&0005 &0&0007 &0&0007 &0&0021 &0&0026 &0&0019 &0&0053
\end{tabular}
\end{table}
 \begin {figure}[p]\caption{Inclinazione di $15'$,
velocità in ordinata~(\nicefrac{m}{s}) e tempo in ascissa~(s).
I punti sperimentali sono raffigurati con le barre di errore su entrambe le misure. La linea continua è la retta interpolante, la linea tratteggiata quella teorica con il valore atteso di $g$, senza considerare gli attriti.}\label{15graf}
\centering
        % GNUPLOT: LaTeX picture
\setlength{\unitlength}{0.240900pt}
\ifx\plotpoint\undefined\newsavebox{\plotpoint}\fi
\begin{picture}(1500,900)(0,0)
\sbox{\plotpoint}{\rule[-0.200pt]{0.400pt}{0.400pt}}%
\put(161,123){\makebox(0,0)[r]{0.10}}
\put(181.0,123.0){\rule[-0.200pt]{4.818pt}{0.400pt}}
\put(161,270){\makebox(0,0)[r]{0.15}}
\put(181.0,270.0){\rule[-0.200pt]{4.818pt}{0.400pt}}
\put(161,418){\makebox(0,0)[r]{0.20}}
\put(181.0,418.0){\rule[-0.200pt]{4.818pt}{0.400pt}}
\put(161,565){\makebox(0,0)[r]{0.25}}
\put(181.0,565.0){\rule[-0.200pt]{4.818pt}{0.400pt}}
\put(161,713){\makebox(0,0)[r]{0.30}}
\put(181.0,713.0){\rule[-0.200pt]{4.818pt}{0.400pt}}
\put(161,860){\makebox(0,0)[r]{0.35}}
\put(181.0,860.0){\rule[-0.200pt]{4.818pt}{0.400pt}}
\put(181,82){\makebox(0,0){0.0}}
\put(181.0,123.0){\rule[-0.200pt]{0.400pt}{4.818pt}}
\put(341,82){\makebox(0,0){0.5}}
\put(341.0,123.0){\rule[-0.200pt]{0.400pt}{4.818pt}}
\put(501,82){\makebox(0,0){1.0}}
\put(501.0,123.0){\rule[-0.200pt]{0.400pt}{4.818pt}}
\put(660,82){\makebox(0,0){1.5}}
\put(660.0,123.0){\rule[-0.200pt]{0.400pt}{4.818pt}}
\put(820,82){\makebox(0,0){2.0}}
\put(820.0,123.0){\rule[-0.200pt]{0.400pt}{4.818pt}}
\put(980,82){\makebox(0,0){2.5}}
\put(980.0,123.0){\rule[-0.200pt]{0.400pt}{4.818pt}}
\put(1140,82){\makebox(0,0){3.0}}
\put(1140.0,123.0){\rule[-0.200pt]{0.400pt}{4.818pt}}
\put(1299,82){\makebox(0,0){3.5}}
\put(1299.0,123.0){\rule[-0.200pt]{0.400pt}{4.818pt}}
\put(1459,82){\makebox(0,0){4.0}}
\put(1459.0,123.0){\rule[-0.200pt]{0.400pt}{4.818pt}}
\put(181.0,123.0){\rule[-0.200pt]{0.400pt}{177.543pt}}
\put(181.0,123.0){\rule[-0.200pt]{307.870pt}{0.400pt}}
\put(1459.0,123.0){\rule[-0.200pt]{0.400pt}{177.543pt}}
\put(181.0,860.0){\rule[-0.200pt]{307.870pt}{0.400pt}}
\put(40,491){\makebox(0,0){\rotatebox{90}{velocità $\ms$}}}
\put(820,21){\makebox(0,0){tempo (s)}}
\put(286.0,274.0){\rule[-0.200pt]{0.400pt}{1.927pt}}
\put(276.0,274.0){\rule[-0.200pt]{4.818pt}{0.400pt}}
\put(276.0,282.0){\rule[-0.200pt]{4.818pt}{0.400pt}}
\put(481.0,339.0){\rule[-0.200pt]{0.400pt}{2.891pt}}
\put(471.0,339.0){\rule[-0.200pt]{4.818pt}{0.400pt}}
\put(471.0,351.0){\rule[-0.200pt]{4.818pt}{0.400pt}}
\put(655.0,393.0){\rule[-0.200pt]{0.400pt}{2.168pt}}
\put(645.0,393.0){\rule[-0.200pt]{4.818pt}{0.400pt}}
\put(645.0,402.0){\rule[-0.200pt]{4.818pt}{0.400pt}}
\put(810.0,473.0){\rule[-0.200pt]{0.400pt}{4.336pt}}
\put(800.0,473.0){\rule[-0.200pt]{4.818pt}{0.400pt}}
\put(800.0,491.0){\rule[-0.200pt]{4.818pt}{0.400pt}}
\put(952.0,491.0){\rule[-0.200pt]{0.400pt}{5.541pt}}
\put(942.0,491.0){\rule[-0.200pt]{4.818pt}{0.400pt}}
\put(942.0,514.0){\rule[-0.200pt]{4.818pt}{0.400pt}}
\put(1083.0,580.0){\rule[-0.200pt]{0.400pt}{5.782pt}}
\put(1073.0,580.0){\rule[-0.200pt]{4.818pt}{0.400pt}}
\put(1073.0,604.0){\rule[-0.200pt]{4.818pt}{0.400pt}}
\put(1207.0,576.0){\rule[-0.200pt]{0.400pt}{3.132pt}}
\put(1197.0,576.0){\rule[-0.200pt]{4.818pt}{0.400pt}}
\put(1197.0,589.0){\rule[-0.200pt]{4.818pt}{0.400pt}}
\put(1326.0,650.0){\rule[-0.200pt]{0.400pt}{11.563pt}}
\put(1316.0,650.0){\rule[-0.200pt]{4.818pt}{0.400pt}}
\put(1316.0,698.0){\rule[-0.200pt]{4.818pt}{0.400pt}}
\put(1438.0,623.0){\rule[-0.200pt]{0.400pt}{15.177pt}}
\put(1428.0,623.0){\rule[-0.200pt]{4.818pt}{0.400pt}}
\put(1428.0,686.0){\rule[-0.200pt]{4.818pt}{0.400pt}}
\put(285.0,278.0){\rule[-0.200pt]{0.482pt}{0.400pt}}
\put(285.0,268.0){\rule[-0.200pt]{0.400pt}{4.818pt}}
\put(287.0,268.0){\rule[-0.200pt]{0.400pt}{4.818pt}}
\put(480.0,345.0){\rule[-0.200pt]{0.723pt}{0.400pt}}
\put(480.0,335.0){\rule[-0.200pt]{0.400pt}{4.818pt}}
\put(483.0,335.0){\rule[-0.200pt]{0.400pt}{4.818pt}}
\put(655.0,398.0){\usebox{\plotpoint}}
\put(655.0,388.0){\rule[-0.200pt]{0.400pt}{4.818pt}}
\put(656.0,388.0){\rule[-0.200pt]{0.400pt}{4.818pt}}
\put(809.0,482.0){\rule[-0.200pt]{0.482pt}{0.400pt}}
\put(809.0,472.0){\rule[-0.200pt]{0.400pt}{4.818pt}}
\put(811.0,472.0){\rule[-0.200pt]{0.400pt}{4.818pt}}
\put(951.0,503.0){\rule[-0.200pt]{0.482pt}{0.400pt}}
\put(951.0,493.0){\rule[-0.200pt]{0.400pt}{4.818pt}}
\put(953.0,493.0){\rule[-0.200pt]{0.400pt}{4.818pt}}
\put(1082.0,592.0){\rule[-0.200pt]{0.482pt}{0.400pt}}
\put(1082.0,582.0){\rule[-0.200pt]{0.400pt}{4.818pt}}
\put(1084.0,582.0){\rule[-0.200pt]{0.400pt}{4.818pt}}
\put(1207.0,583.0){\usebox{\plotpoint}}
\put(1207.0,573.0){\rule[-0.200pt]{0.400pt}{4.818pt}}
\put(1208.0,573.0){\rule[-0.200pt]{0.400pt}{4.818pt}}
\put(1324.0,674.0){\rule[-0.200pt]{0.723pt}{0.400pt}}
\put(1324.0,664.0){\rule[-0.200pt]{0.400pt}{4.818pt}}
\put(1327.0,664.0){\rule[-0.200pt]{0.400pt}{4.818pt}}
\put(1436.0,655.0){\rule[-0.200pt]{1.204pt}{0.400pt}}
\put(1436.0,645.0){\rule[-0.200pt]{0.400pt}{4.818pt}}
\put(286,278){\circle*{12}}
\put(481,345){\circle*{12}}
\put(655,398){\circle*{12}}
\put(810,482){\circle*{12}}
\put(952,503){\circle*{12}}
\put(1083,592){\circle*{12}}
\put(1207,583){\circle*{12}}
\put(1326,674){\circle*{12}}
\put(1438,655){\circle*{12}}
\put(1441.0,645.0){\rule[-0.200pt]{0.400pt}{4.818pt}}
\put(285,280){\usebox{\plotpoint}}
\multiput(285.00,280.60)(1.505,0.468){5}{\rule{1.200pt}{0.113pt}}
\multiput(285.00,279.17)(8.509,4.000){2}{\rule{0.600pt}{0.400pt}}
\multiput(296.00,284.60)(1.651,0.468){5}{\rule{1.300pt}{0.113pt}}
\multiput(296.00,283.17)(9.302,4.000){2}{\rule{0.650pt}{0.400pt}}
\multiput(308.00,288.60)(1.651,0.468){5}{\rule{1.300pt}{0.113pt}}
\multiput(308.00,287.17)(9.302,4.000){2}{\rule{0.650pt}{0.400pt}}
\multiput(320.00,292.60)(1.651,0.468){5}{\rule{1.300pt}{0.113pt}}
\multiput(320.00,291.17)(9.302,4.000){2}{\rule{0.650pt}{0.400pt}}
\multiput(332.00,296.59)(1.155,0.477){7}{\rule{0.980pt}{0.115pt}}
\multiput(332.00,295.17)(8.966,5.000){2}{\rule{0.490pt}{0.400pt}}
\multiput(343.00,301.60)(1.651,0.468){5}{\rule{1.300pt}{0.113pt}}
\multiput(343.00,300.17)(9.302,4.000){2}{\rule{0.650pt}{0.400pt}}
\multiput(355.00,305.60)(1.651,0.468){5}{\rule{1.300pt}{0.113pt}}
\multiput(355.00,304.17)(9.302,4.000){2}{\rule{0.650pt}{0.400pt}}
\multiput(367.00,309.60)(1.505,0.468){5}{\rule{1.200pt}{0.113pt}}
\multiput(367.00,308.17)(8.509,4.000){2}{\rule{0.600pt}{0.400pt}}
\multiput(378.00,313.60)(1.651,0.468){5}{\rule{1.300pt}{0.113pt}}
\multiput(378.00,312.17)(9.302,4.000){2}{\rule{0.650pt}{0.400pt}}
\multiput(390.00,317.60)(1.651,0.468){5}{\rule{1.300pt}{0.113pt}}
\multiput(390.00,316.17)(9.302,4.000){2}{\rule{0.650pt}{0.400pt}}
\multiput(402.00,321.60)(1.505,0.468){5}{\rule{1.200pt}{0.113pt}}
\multiput(402.00,320.17)(8.509,4.000){2}{\rule{0.600pt}{0.400pt}}
\multiput(413.00,325.60)(1.651,0.468){5}{\rule{1.300pt}{0.113pt}}
\multiput(413.00,324.17)(9.302,4.000){2}{\rule{0.650pt}{0.400pt}}
\multiput(425.00,329.60)(1.651,0.468){5}{\rule{1.300pt}{0.113pt}}
\multiput(425.00,328.17)(9.302,4.000){2}{\rule{0.650pt}{0.400pt}}
\multiput(437.00,333.60)(1.505,0.468){5}{\rule{1.200pt}{0.113pt}}
\multiput(437.00,332.17)(8.509,4.000){2}{\rule{0.600pt}{0.400pt}}
\multiput(448.00,337.60)(1.651,0.468){5}{\rule{1.300pt}{0.113pt}}
\multiput(448.00,336.17)(9.302,4.000){2}{\rule{0.650pt}{0.400pt}}
\multiput(460.00,341.59)(1.267,0.477){7}{\rule{1.060pt}{0.115pt}}
\multiput(460.00,340.17)(9.800,5.000){2}{\rule{0.530pt}{0.400pt}}
\multiput(472.00,346.60)(1.505,0.468){5}{\rule{1.200pt}{0.113pt}}
\multiput(472.00,345.17)(8.509,4.000){2}{\rule{0.600pt}{0.400pt}}
\multiput(483.00,350.60)(1.651,0.468){5}{\rule{1.300pt}{0.113pt}}
\multiput(483.00,349.17)(9.302,4.000){2}{\rule{0.650pt}{0.400pt}}
\multiput(495.00,354.60)(1.651,0.468){5}{\rule{1.300pt}{0.113pt}}
\multiput(495.00,353.17)(9.302,4.000){2}{\rule{0.650pt}{0.400pt}}
\multiput(507.00,358.60)(1.505,0.468){5}{\rule{1.200pt}{0.113pt}}
\multiput(507.00,357.17)(8.509,4.000){2}{\rule{0.600pt}{0.400pt}}
\multiput(518.00,362.60)(1.651,0.468){5}{\rule{1.300pt}{0.113pt}}
\multiput(518.00,361.17)(9.302,4.000){2}{\rule{0.650pt}{0.400pt}}
\multiput(530.00,366.60)(1.651,0.468){5}{\rule{1.300pt}{0.113pt}}
\multiput(530.00,365.17)(9.302,4.000){2}{\rule{0.650pt}{0.400pt}}
\multiput(542.00,370.60)(1.505,0.468){5}{\rule{1.200pt}{0.113pt}}
\multiput(542.00,369.17)(8.509,4.000){2}{\rule{0.600pt}{0.400pt}}
\multiput(553.00,374.60)(1.651,0.468){5}{\rule{1.300pt}{0.113pt}}
\multiput(553.00,373.17)(9.302,4.000){2}{\rule{0.650pt}{0.400pt}}
\multiput(565.00,378.60)(1.651,0.468){5}{\rule{1.300pt}{0.113pt}}
\multiput(565.00,377.17)(9.302,4.000){2}{\rule{0.650pt}{0.400pt}}
\multiput(577.00,382.60)(1.505,0.468){5}{\rule{1.200pt}{0.113pt}}
\multiput(577.00,381.17)(8.509,4.000){2}{\rule{0.600pt}{0.400pt}}
\multiput(588.00,386.60)(1.651,0.468){5}{\rule{1.300pt}{0.113pt}}
\multiput(588.00,385.17)(9.302,4.000){2}{\rule{0.650pt}{0.400pt}}
\multiput(600.00,390.59)(1.267,0.477){7}{\rule{1.060pt}{0.115pt}}
\multiput(600.00,389.17)(9.800,5.000){2}{\rule{0.530pt}{0.400pt}}
\multiput(612.00,395.60)(1.505,0.468){5}{\rule{1.200pt}{0.113pt}}
\multiput(612.00,394.17)(8.509,4.000){2}{\rule{0.600pt}{0.400pt}}
\multiput(623.00,399.60)(1.651,0.468){5}{\rule{1.300pt}{0.113pt}}
\multiput(623.00,398.17)(9.302,4.000){2}{\rule{0.650pt}{0.400pt}}
\multiput(635.00,403.60)(1.651,0.468){5}{\rule{1.300pt}{0.113pt}}
\multiput(635.00,402.17)(9.302,4.000){2}{\rule{0.650pt}{0.400pt}}
\multiput(647.00,407.60)(1.505,0.468){5}{\rule{1.200pt}{0.113pt}}
\multiput(647.00,406.17)(8.509,4.000){2}{\rule{0.600pt}{0.400pt}}
\multiput(658.00,411.60)(1.651,0.468){5}{\rule{1.300pt}{0.113pt}}
\multiput(658.00,410.17)(9.302,4.000){2}{\rule{0.650pt}{0.400pt}}
\multiput(670.00,415.60)(1.651,0.468){5}{\rule{1.300pt}{0.113pt}}
\multiput(670.00,414.17)(9.302,4.000){2}{\rule{0.650pt}{0.400pt}}
\multiput(682.00,419.60)(1.505,0.468){5}{\rule{1.200pt}{0.113pt}}
\multiput(682.00,418.17)(8.509,4.000){2}{\rule{0.600pt}{0.400pt}}
\multiput(693.00,423.60)(1.651,0.468){5}{\rule{1.300pt}{0.113pt}}
\multiput(693.00,422.17)(9.302,4.000){2}{\rule{0.650pt}{0.400pt}}
\multiput(705.00,427.60)(1.651,0.468){5}{\rule{1.300pt}{0.113pt}}
\multiput(705.00,426.17)(9.302,4.000){2}{\rule{0.650pt}{0.400pt}}
\multiput(717.00,431.60)(1.505,0.468){5}{\rule{1.200pt}{0.113pt}}
\multiput(717.00,430.17)(8.509,4.000){2}{\rule{0.600pt}{0.400pt}}
\multiput(728.00,435.59)(1.267,0.477){7}{\rule{1.060pt}{0.115pt}}
\multiput(728.00,434.17)(9.800,5.000){2}{\rule{0.530pt}{0.400pt}}
\multiput(740.00,440.60)(1.651,0.468){5}{\rule{1.300pt}{0.113pt}}
\multiput(740.00,439.17)(9.302,4.000){2}{\rule{0.650pt}{0.400pt}}
\multiput(752.00,444.60)(1.505,0.468){5}{\rule{1.200pt}{0.113pt}}
\multiput(752.00,443.17)(8.509,4.000){2}{\rule{0.600pt}{0.400pt}}
\multiput(763.00,448.60)(1.651,0.468){5}{\rule{1.300pt}{0.113pt}}
\multiput(763.00,447.17)(9.302,4.000){2}{\rule{0.650pt}{0.400pt}}
\multiput(775.00,452.60)(1.651,0.468){5}{\rule{1.300pt}{0.113pt}}
\multiput(775.00,451.17)(9.302,4.000){2}{\rule{0.650pt}{0.400pt}}
\multiput(787.00,456.60)(1.505,0.468){5}{\rule{1.200pt}{0.113pt}}
\multiput(787.00,455.17)(8.509,4.000){2}{\rule{0.600pt}{0.400pt}}
\multiput(798.00,460.60)(1.651,0.468){5}{\rule{1.300pt}{0.113pt}}
\multiput(798.00,459.17)(9.302,4.000){2}{\rule{0.650pt}{0.400pt}}
\multiput(810.00,464.60)(1.651,0.468){5}{\rule{1.300pt}{0.113pt}}
\multiput(810.00,463.17)(9.302,4.000){2}{\rule{0.650pt}{0.400pt}}
\multiput(822.00,468.60)(1.505,0.468){5}{\rule{1.200pt}{0.113pt}}
\multiput(822.00,467.17)(8.509,4.000){2}{\rule{0.600pt}{0.400pt}}
\multiput(833.00,472.60)(1.651,0.468){5}{\rule{1.300pt}{0.113pt}}
\multiput(833.00,471.17)(9.302,4.000){2}{\rule{0.650pt}{0.400pt}}
\multiput(845.00,476.60)(1.651,0.468){5}{\rule{1.300pt}{0.113pt}}
\multiput(845.00,475.17)(9.302,4.000){2}{\rule{0.650pt}{0.400pt}}
\multiput(857.00,480.59)(1.155,0.477){7}{\rule{0.980pt}{0.115pt}}
\multiput(857.00,479.17)(8.966,5.000){2}{\rule{0.490pt}{0.400pt}}
\multiput(868.00,485.60)(1.651,0.468){5}{\rule{1.300pt}{0.113pt}}
\multiput(868.00,484.17)(9.302,4.000){2}{\rule{0.650pt}{0.400pt}}
\multiput(880.00,489.60)(1.651,0.468){5}{\rule{1.300pt}{0.113pt}}
\multiput(880.00,488.17)(9.302,4.000){2}{\rule{0.650pt}{0.400pt}}
\multiput(892.00,493.60)(1.651,0.468){5}{\rule{1.300pt}{0.113pt}}
\multiput(892.00,492.17)(9.302,4.000){2}{\rule{0.650pt}{0.400pt}}
\multiput(904.00,497.60)(1.505,0.468){5}{\rule{1.200pt}{0.113pt}}
\multiput(904.00,496.17)(8.509,4.000){2}{\rule{0.600pt}{0.400pt}}
\multiput(915.00,501.60)(1.651,0.468){5}{\rule{1.300pt}{0.113pt}}
\multiput(915.00,500.17)(9.302,4.000){2}{\rule{0.650pt}{0.400pt}}
\multiput(927.00,505.60)(1.651,0.468){5}{\rule{1.300pt}{0.113pt}}
\multiput(927.00,504.17)(9.302,4.000){2}{\rule{0.650pt}{0.400pt}}
\multiput(939.00,509.60)(1.505,0.468){5}{\rule{1.200pt}{0.113pt}}
\multiput(939.00,508.17)(8.509,4.000){2}{\rule{0.600pt}{0.400pt}}
\multiput(950.00,513.60)(1.651,0.468){5}{\rule{1.300pt}{0.113pt}}
\multiput(950.00,512.17)(9.302,4.000){2}{\rule{0.650pt}{0.400pt}}
\multiput(962.00,517.60)(1.651,0.468){5}{\rule{1.300pt}{0.113pt}}
\multiput(962.00,516.17)(9.302,4.000){2}{\rule{0.650pt}{0.400pt}}
\multiput(974.00,521.60)(1.505,0.468){5}{\rule{1.200pt}{0.113pt}}
\multiput(974.00,520.17)(8.509,4.000){2}{\rule{0.600pt}{0.400pt}}
\multiput(985.00,525.60)(1.651,0.468){5}{\rule{1.300pt}{0.113pt}}
\multiput(985.00,524.17)(9.302,4.000){2}{\rule{0.650pt}{0.400pt}}
\multiput(997.00,529.59)(1.267,0.477){7}{\rule{1.060pt}{0.115pt}}
\multiput(997.00,528.17)(9.800,5.000){2}{\rule{0.530pt}{0.400pt}}
\multiput(1009.00,534.60)(1.505,0.468){5}{\rule{1.200pt}{0.113pt}}
\multiput(1009.00,533.17)(8.509,4.000){2}{\rule{0.600pt}{0.400pt}}
\multiput(1020.00,538.60)(1.651,0.468){5}{\rule{1.300pt}{0.113pt}}
\multiput(1020.00,537.17)(9.302,4.000){2}{\rule{0.650pt}{0.400pt}}
\multiput(1032.00,542.60)(1.651,0.468){5}{\rule{1.300pt}{0.113pt}}
\multiput(1032.00,541.17)(9.302,4.000){2}{\rule{0.650pt}{0.400pt}}
\multiput(1044.00,546.60)(1.505,0.468){5}{\rule{1.200pt}{0.113pt}}
\multiput(1044.00,545.17)(8.509,4.000){2}{\rule{0.600pt}{0.400pt}}
\multiput(1055.00,550.60)(1.651,0.468){5}{\rule{1.300pt}{0.113pt}}
\multiput(1055.00,549.17)(9.302,4.000){2}{\rule{0.650pt}{0.400pt}}
\multiput(1067.00,554.60)(1.651,0.468){5}{\rule{1.300pt}{0.113pt}}
\multiput(1067.00,553.17)(9.302,4.000){2}{\rule{0.650pt}{0.400pt}}
\multiput(1079.00,558.60)(1.505,0.468){5}{\rule{1.200pt}{0.113pt}}
\multiput(1079.00,557.17)(8.509,4.000){2}{\rule{0.600pt}{0.400pt}}
\multiput(1090.00,562.60)(1.651,0.468){5}{\rule{1.300pt}{0.113pt}}
\multiput(1090.00,561.17)(9.302,4.000){2}{\rule{0.650pt}{0.400pt}}
\multiput(1102.00,566.60)(1.651,0.468){5}{\rule{1.300pt}{0.113pt}}
\multiput(1102.00,565.17)(9.302,4.000){2}{\rule{0.650pt}{0.400pt}}
\multiput(1114.00,570.60)(1.505,0.468){5}{\rule{1.200pt}{0.113pt}}
\multiput(1114.00,569.17)(8.509,4.000){2}{\rule{0.600pt}{0.400pt}}
\multiput(1125.00,574.59)(1.267,0.477){7}{\rule{1.060pt}{0.115pt}}
\multiput(1125.00,573.17)(9.800,5.000){2}{\rule{0.530pt}{0.400pt}}
\multiput(1137.00,579.60)(1.651,0.468){5}{\rule{1.300pt}{0.113pt}}
\multiput(1137.00,578.17)(9.302,4.000){2}{\rule{0.650pt}{0.400pt}}
\multiput(1149.00,583.60)(1.505,0.468){5}{\rule{1.200pt}{0.113pt}}
\multiput(1149.00,582.17)(8.509,4.000){2}{\rule{0.600pt}{0.400pt}}
\multiput(1160.00,587.60)(1.651,0.468){5}{\rule{1.300pt}{0.113pt}}
\multiput(1160.00,586.17)(9.302,4.000){2}{\rule{0.650pt}{0.400pt}}
\multiput(1172.00,591.60)(1.651,0.468){5}{\rule{1.300pt}{0.113pt}}
\multiput(1172.00,590.17)(9.302,4.000){2}{\rule{0.650pt}{0.400pt}}
\multiput(1184.00,595.60)(1.505,0.468){5}{\rule{1.200pt}{0.113pt}}
\multiput(1184.00,594.17)(8.509,4.000){2}{\rule{0.600pt}{0.400pt}}
\multiput(1195.00,599.60)(1.651,0.468){5}{\rule{1.300pt}{0.113pt}}
\multiput(1195.00,598.17)(9.302,4.000){2}{\rule{0.650pt}{0.400pt}}
\multiput(1207.00,603.60)(1.651,0.468){5}{\rule{1.300pt}{0.113pt}}
\multiput(1207.00,602.17)(9.302,4.000){2}{\rule{0.650pt}{0.400pt}}
\multiput(1219.00,607.60)(1.505,0.468){5}{\rule{1.200pt}{0.113pt}}
\multiput(1219.00,606.17)(8.509,4.000){2}{\rule{0.600pt}{0.400pt}}
\multiput(1230.00,611.60)(1.651,0.468){5}{\rule{1.300pt}{0.113pt}}
\multiput(1230.00,610.17)(9.302,4.000){2}{\rule{0.650pt}{0.400pt}}
\multiput(1242.00,615.60)(1.651,0.468){5}{\rule{1.300pt}{0.113pt}}
\multiput(1242.00,614.17)(9.302,4.000){2}{\rule{0.650pt}{0.400pt}}
\multiput(1254.00,619.59)(1.155,0.477){7}{\rule{0.980pt}{0.115pt}}
\multiput(1254.00,618.17)(8.966,5.000){2}{\rule{0.490pt}{0.400pt}}
\multiput(1265.00,624.60)(1.651,0.468){5}{\rule{1.300pt}{0.113pt}}
\multiput(1265.00,623.17)(9.302,4.000){2}{\rule{0.650pt}{0.400pt}}
\multiput(1277.00,628.60)(1.651,0.468){5}{\rule{1.300pt}{0.113pt}}
\multiput(1277.00,627.17)(9.302,4.000){2}{\rule{0.650pt}{0.400pt}}
\multiput(1289.00,632.60)(1.505,0.468){5}{\rule{1.200pt}{0.113pt}}
\multiput(1289.00,631.17)(8.509,4.000){2}{\rule{0.600pt}{0.400pt}}
\multiput(1300.00,636.60)(1.651,0.468){5}{\rule{1.300pt}{0.113pt}}
\multiput(1300.00,635.17)(9.302,4.000){2}{\rule{0.650pt}{0.400pt}}
\multiput(1312.00,640.60)(1.651,0.468){5}{\rule{1.300pt}{0.113pt}}
\multiput(1312.00,639.17)(9.302,4.000){2}{\rule{0.650pt}{0.400pt}}
\multiput(1324.00,644.60)(1.505,0.468){5}{\rule{1.200pt}{0.113pt}}
\multiput(1324.00,643.17)(8.509,4.000){2}{\rule{0.600pt}{0.400pt}}
\multiput(1335.00,648.60)(1.651,0.468){5}{\rule{1.300pt}{0.113pt}}
\multiput(1335.00,647.17)(9.302,4.000){2}{\rule{0.650pt}{0.400pt}}
\multiput(1347.00,652.60)(1.651,0.468){5}{\rule{1.300pt}{0.113pt}}
\multiput(1347.00,651.17)(9.302,4.000){2}{\rule{0.650pt}{0.400pt}}
\multiput(1359.00,656.60)(1.505,0.468){5}{\rule{1.200pt}{0.113pt}}
\multiput(1359.00,655.17)(8.509,4.000){2}{\rule{0.600pt}{0.400pt}}
\multiput(1370.00,660.60)(1.651,0.468){5}{\rule{1.300pt}{0.113pt}}
\multiput(1370.00,659.17)(9.302,4.000){2}{\rule{0.650pt}{0.400pt}}
\multiput(1382.00,664.60)(1.651,0.468){5}{\rule{1.300pt}{0.113pt}}
\multiput(1382.00,663.17)(9.302,4.000){2}{\rule{0.650pt}{0.400pt}}
\multiput(1394.00,668.59)(1.155,0.477){7}{\rule{0.980pt}{0.115pt}}
\multiput(1394.00,667.17)(8.966,5.000){2}{\rule{0.490pt}{0.400pt}}
\multiput(1405.00,673.60)(1.651,0.468){5}{\rule{1.300pt}{0.113pt}}
\multiput(1405.00,672.17)(9.302,4.000){2}{\rule{0.650pt}{0.400pt}}
\multiput(1417.00,677.60)(1.651,0.468){5}{\rule{1.300pt}{0.113pt}}
\multiput(1417.00,676.17)(9.302,4.000){2}{\rule{0.650pt}{0.400pt}}
\multiput(1429.00,681.60)(1.651,0.468){5}{\rule{1.300pt}{0.113pt}}
\multiput(1429.00,680.17)(9.302,4.000){2}{\rule{0.650pt}{0.400pt}}
\sbox{\plotpoint}{\rule[-0.500pt]{1.000pt}{1.000pt}}%
\put(285,285){\usebox{\plotpoint}}
\put(285.00,285.00){\usebox{\plotpoint}}
\put(304.35,292.48){\usebox{\plotpoint}}
\put(323.61,300.20){\usebox{\plotpoint}}
\put(342.84,307.93){\usebox{\plotpoint}}
\put(362.33,315.05){\usebox{\plotpoint}}
\put(381.42,323.14){\usebox{\plotpoint}}
\put(400.81,330.51){\usebox{\plotpoint}}
\put(420.17,337.99){\usebox{\plotpoint}}
\put(439.37,345.86){\usebox{\plotpoint}}
\put(458.68,353.45){\usebox{\plotpoint}}
\put(478.08,360.76){\usebox{\plotpoint}}
\put(497.23,368.74){\usebox{\plotpoint}}
\put(516.52,376.33){\usebox{\plotpoint}}
\put(535.82,383.94){\usebox{\plotpoint}}
\put(555.04,391.68){\usebox{\plotpoint}}
\put(574.47,398.95){\usebox{\plotpoint}}
\put(593.63,406.88){\usebox{\plotpoint}}
\put(612.98,414.35){\usebox{\plotpoint}}
\put(632.31,421.88){\usebox{\plotpoint}}
\put(651.55,429.66){\usebox{\plotpoint}}
\put(670.85,437.28){\usebox{\plotpoint}}
\put(690.19,444.72){\usebox{\plotpoint}}
\put(709.43,452.48){\usebox{\plotpoint}}
\put(728.66,460.22){\usebox{\plotpoint}}
\put(748.13,467.39){\usebox{\plotpoint}}
\put(767.25,475.42){\usebox{\plotpoint}}
\put(786.61,482.84){\usebox{\plotpoint}}
\put(805.83,490.61){\usebox{\plotpoint}}
\put(825.16,498.15){\usebox{\plotpoint}}
\put(844.46,505.77){\usebox{\plotpoint}}
\put(863.74,513.45){\usebox{\plotpoint}}
\put(883.05,521.02){\usebox{\plotpoint}}
\put(902.45,528.36){\usebox{\plotpoint}}
\put(921.64,536.21){\usebox{\plotpoint}}
\put(940.98,543.72){\usebox{\plotpoint}}
\put(960.30,551.29){\usebox{\plotpoint}}
\put(979.55,559.02){\usebox{\plotpoint}}
\put(998.86,566.62){\usebox{\plotpoint}}
\put(1018.17,574.17){\usebox{\plotpoint}}
\put(1037.45,581.82){\usebox{\plotpoint}}
\put(1056.63,589.68){\usebox{\plotpoint}}
\put(1076.03,597.01){\usebox{\plotpoint}}
\put(1095.26,604.75){\usebox{\plotpoint}}
\put(1114.59,612.27){\usebox{\plotpoint}}
\put(1133.84,619.95){\usebox{\plotpoint}}
\put(1153.16,627.51){\usebox{\plotpoint}}
\put(1172.44,635.18){\usebox{\plotpoint}}
\put(1191.74,642.81){\usebox{\plotpoint}}
\put(1211.06,650.35){\usebox{\plotpoint}}
\put(1230.28,658.12){\usebox{\plotpoint}}
\put(1249.65,665.55){\usebox{\plotpoint}}
\put(1268.88,673.29){\usebox{\plotpoint}}
\put(1288.26,680.69){\usebox{\plotpoint}}
\put(1307.46,688.49){\usebox{\plotpoint}}
\put(1326.71,696.23){\usebox{\plotpoint}}
\put(1346.05,703.68){\usebox{\plotpoint}}
\put(1365.34,711.31){\usebox{\plotpoint}}
\put(1384.59,719.08){\usebox{\plotpoint}}
\put(1403.92,726.61){\usebox{\plotpoint}}
\put(1423.27,734.09){\usebox{\plotpoint}}
\put(1441,741){\usebox{\plotpoint}}
\sbox{\plotpoint}{\rule[-0.200pt]{0.400pt}{0.400pt}}%
\put(181.0,123.0){\rule[-0.200pt]{0.400pt}{177.543pt}}
\put(181.0,123.0){\rule[-0.200pt]{307.870pt}{0.400pt}}
\put(1459.0,123.0){\rule[-0.200pt]{0.400pt}{177.543pt}}
\put(181.0,860.0){\rule[-0.200pt]{307.870pt}{0.400pt}}
\end{picture}

\end{figure}
 \begin {figure}[p]\caption{Inclinazione di $30'$,
velocità in ordinata~(\nicefrac{m}{s}) e tempo in ascissa~(s).
I punti sperimentali sono raffigurati con le barre di errore su entrambe le misure. La linea continua è la retta interpolante, la linea tratteggiata quella teorica con il valore atteso di $g$, senza considerare gli attriti.}\label{30graf}
\centering
        % GNUPLOT: LaTeX picture
\setlength{\unitlength}{0.240900pt}
\ifx\plotpoint\undefined\newsavebox{\plotpoint}\fi
\begin{picture}(1500,900)(0,0)
\sbox{\plotpoint}{\rule[-0.200pt]{0.400pt}{0.400pt}}%
\put(161,123){\makebox(0,0)[r]{0.20}}
\put(181.0,123.0){\rule[-0.200pt]{4.818pt}{0.400pt}}
\put(161,270){\makebox(0,0)[r]{0.25}}
\put(181.0,270.0){\rule[-0.200pt]{4.818pt}{0.400pt}}
\put(161,418){\makebox(0,0)[r]{0.30}}
\put(181.0,418.0){\rule[-0.200pt]{4.818pt}{0.400pt}}
\put(161,565){\makebox(0,0)[r]{0.35}}
\put(181.0,565.0){\rule[-0.200pt]{4.818pt}{0.400pt}}
\put(161,713){\makebox(0,0)[r]{0.40}}
\put(181.0,713.0){\rule[-0.200pt]{4.818pt}{0.400pt}}
\put(161,860){\makebox(0,0)[r]{0.45}}
\put(181.0,860.0){\rule[-0.200pt]{4.818pt}{0.400pt}}
\put(181,82){\makebox(0,0){0.0}}
\put(181.0,123.0){\rule[-0.200pt]{0.400pt}{4.818pt}}
\put(394,82){\makebox(0,0){0.5}}
\put(394.0,123.0){\rule[-0.200pt]{0.400pt}{4.818pt}}
\put(607,82){\makebox(0,0){1.0}}
\put(607.0,123.0){\rule[-0.200pt]{0.400pt}{4.818pt}}
\put(820,82){\makebox(0,0){1.5}}
\put(820.0,123.0){\rule[-0.200pt]{0.400pt}{4.818pt}}
\put(1033,82){\makebox(0,0){2.0}}
\put(1033.0,123.0){\rule[-0.200pt]{0.400pt}{4.818pt}}
\put(1246,82){\makebox(0,0){2.5}}
\put(1246.0,123.0){\rule[-0.200pt]{0.400pt}{4.818pt}}
\put(1459,82){\makebox(0,0){3.0}}
\put(1459.0,123.0){\rule[-0.200pt]{0.400pt}{4.818pt}}
\put(181.0,123.0){\rule[-0.200pt]{0.400pt}{177.543pt}}
\put(181.0,123.0){\rule[-0.200pt]{307.870pt}{0.400pt}}
\put(1459.0,123.0){\rule[-0.200pt]{0.400pt}{177.543pt}}
\put(181.0,860.0){\rule[-0.200pt]{307.870pt}{0.400pt}}
\put(40,491){\makebox(0,0){\rotatebox{90}{velocità $\ms$}}}
\put(820,21){\makebox(0,0){tempo (s)}}
\put(280.0,170.0){\usebox{\plotpoint}}
\put(270.0,170.0){\rule[-0.200pt]{4.818pt}{0.400pt}}
\put(270.0,171.0){\rule[-0.200pt]{4.818pt}{0.400pt}}
\put(462.0,278.0){\rule[-0.200pt]{0.400pt}{0.723pt}}
\put(452.0,278.0){\rule[-0.200pt]{4.818pt}{0.400pt}}
\put(452.0,281.0){\rule[-0.200pt]{4.818pt}{0.400pt}}
\put(622.0,365.0){\rule[-0.200pt]{0.400pt}{1.686pt}}
\put(612.0,365.0){\rule[-0.200pt]{4.818pt}{0.400pt}}
\put(612.0,372.0){\rule[-0.200pt]{4.818pt}{0.400pt}}
\put(766.0,438.0){\rule[-0.200pt]{0.400pt}{3.854pt}}
\put(756.0,438.0){\rule[-0.200pt]{4.818pt}{0.400pt}}
\put(756.0,454.0){\rule[-0.200pt]{4.818pt}{0.400pt}}
\put(897.0,523.0){\rule[-0.200pt]{0.400pt}{4.095pt}}
\put(887.0,523.0){\rule[-0.200pt]{4.818pt}{0.400pt}}
\put(887.0,540.0){\rule[-0.200pt]{4.818pt}{0.400pt}}
\put(1019.0,588.0){\rule[-0.200pt]{0.400pt}{6.022pt}}
\put(1009.0,588.0){\rule[-0.200pt]{4.818pt}{0.400pt}}
\put(1009.0,613.0){\rule[-0.200pt]{4.818pt}{0.400pt}}
\put(1135.0,626.0){\rule[-0.200pt]{0.400pt}{7.709pt}}
\put(1125.0,626.0){\rule[-0.200pt]{4.818pt}{0.400pt}}
\put(1125.0,658.0){\rule[-0.200pt]{4.818pt}{0.400pt}}
\put(1244.0,688.0){\rule[-0.200pt]{0.400pt}{11.081pt}}
\put(1234.0,688.0){\rule[-0.200pt]{4.818pt}{0.400pt}}
\put(1234.0,734.0){\rule[-0.200pt]{4.818pt}{0.400pt}}
\put(1347.0,780.0){\rule[-0.200pt]{0.400pt}{13.009pt}}
\put(1337.0,780.0){\rule[-0.200pt]{4.818pt}{0.400pt}}
\put(1337.0,834.0){\rule[-0.200pt]{4.818pt}{0.400pt}}
\put(279.0,170.0){\usebox{\plotpoint}}
\put(279.0,160.0){\rule[-0.200pt]{0.400pt}{4.818pt}}
\put(280.0,160.0){\rule[-0.200pt]{0.400pt}{4.818pt}}
\put(462,280){\usebox{\plotpoint}}
\put(462.0,270.0){\rule[-0.200pt]{0.400pt}{4.818pt}}
\put(462.0,270.0){\rule[-0.200pt]{0.400pt}{4.818pt}}
\put(621.0,368.0){\usebox{\plotpoint}}
\put(621.0,358.0){\rule[-0.200pt]{0.400pt}{4.818pt}}
\put(622.0,358.0){\rule[-0.200pt]{0.400pt}{4.818pt}}
\put(765.0,446.0){\usebox{\plotpoint}}
\put(765.0,436.0){\rule[-0.200pt]{0.400pt}{4.818pt}}
\put(766.0,436.0){\rule[-0.200pt]{0.400pt}{4.818pt}}
\put(897.0,532.0){\usebox{\plotpoint}}
\put(897.0,522.0){\rule[-0.200pt]{0.400pt}{4.818pt}}
\put(898.0,522.0){\rule[-0.200pt]{0.400pt}{4.818pt}}
\put(1018.0,601.0){\rule[-0.200pt]{0.482pt}{0.400pt}}
\put(1018.0,591.0){\rule[-0.200pt]{0.400pt}{4.818pt}}
\put(1020.0,591.0){\rule[-0.200pt]{0.400pt}{4.818pt}}
\put(1134.0,642.0){\usebox{\plotpoint}}
\put(1134.0,632.0){\rule[-0.200pt]{0.400pt}{4.818pt}}
\put(1135.0,632.0){\rule[-0.200pt]{0.400pt}{4.818pt}}
\put(1243.0,711.0){\rule[-0.200pt]{0.723pt}{0.400pt}}
\put(1243.0,701.0){\rule[-0.200pt]{0.400pt}{4.818pt}}
\put(1246.0,701.0){\rule[-0.200pt]{0.400pt}{4.818pt}}
\put(1346.0,807.0){\rule[-0.200pt]{0.482pt}{0.400pt}}
\put(1346.0,797.0){\rule[-0.200pt]{0.400pt}{4.818pt}}
\put(280,170){\circle*{12}}
\put(462,280){\circle*{12}}
\put(622,368){\circle*{12}}
\put(766,446){\circle*{12}}
\put(897,532){\circle*{12}}
\put(1019,601){\circle*{12}}
\put(1135,642){\circle*{12}}
\put(1244,711){\circle*{12}}
\put(1347,807){\circle*{12}}
\put(1348.0,797.0){\rule[-0.200pt]{0.400pt}{4.818pt}}
\put(279,171){\usebox{\plotpoint}}
\multiput(279.00,171.59)(0.943,0.482){9}{\rule{0.833pt}{0.116pt}}
\multiput(279.00,170.17)(9.270,6.000){2}{\rule{0.417pt}{0.400pt}}
\multiput(290.00,177.59)(0.943,0.482){9}{\rule{0.833pt}{0.116pt}}
\multiput(290.00,176.17)(9.270,6.000){2}{\rule{0.417pt}{0.400pt}}
\multiput(301.00,183.59)(0.943,0.482){9}{\rule{0.833pt}{0.116pt}}
\multiput(301.00,182.17)(9.270,6.000){2}{\rule{0.417pt}{0.400pt}}
\multiput(312.00,189.59)(0.798,0.485){11}{\rule{0.729pt}{0.117pt}}
\multiput(312.00,188.17)(9.488,7.000){2}{\rule{0.364pt}{0.400pt}}
\multiput(323.00,196.59)(0.852,0.482){9}{\rule{0.767pt}{0.116pt}}
\multiput(323.00,195.17)(8.409,6.000){2}{\rule{0.383pt}{0.400pt}}
\multiput(333.00,202.59)(0.943,0.482){9}{\rule{0.833pt}{0.116pt}}
\multiput(333.00,201.17)(9.270,6.000){2}{\rule{0.417pt}{0.400pt}}
\multiput(344.00,208.59)(0.943,0.482){9}{\rule{0.833pt}{0.116pt}}
\multiput(344.00,207.17)(9.270,6.000){2}{\rule{0.417pt}{0.400pt}}
\multiput(355.00,214.59)(0.943,0.482){9}{\rule{0.833pt}{0.116pt}}
\multiput(355.00,213.17)(9.270,6.000){2}{\rule{0.417pt}{0.400pt}}
\multiput(366.00,220.59)(0.798,0.485){11}{\rule{0.729pt}{0.117pt}}
\multiput(366.00,219.17)(9.488,7.000){2}{\rule{0.364pt}{0.400pt}}
\multiput(377.00,227.59)(0.852,0.482){9}{\rule{0.767pt}{0.116pt}}
\multiput(377.00,226.17)(8.409,6.000){2}{\rule{0.383pt}{0.400pt}}
\multiput(387.00,233.59)(0.943,0.482){9}{\rule{0.833pt}{0.116pt}}
\multiput(387.00,232.17)(9.270,6.000){2}{\rule{0.417pt}{0.400pt}}
\multiput(398.00,239.59)(0.943,0.482){9}{\rule{0.833pt}{0.116pt}}
\multiput(398.00,238.17)(9.270,6.000){2}{\rule{0.417pt}{0.400pt}}
\multiput(409.00,245.59)(0.943,0.482){9}{\rule{0.833pt}{0.116pt}}
\multiput(409.00,244.17)(9.270,6.000){2}{\rule{0.417pt}{0.400pt}}
\multiput(420.00,251.59)(0.798,0.485){11}{\rule{0.729pt}{0.117pt}}
\multiput(420.00,250.17)(9.488,7.000){2}{\rule{0.364pt}{0.400pt}}
\multiput(431.00,258.59)(0.852,0.482){9}{\rule{0.767pt}{0.116pt}}
\multiput(431.00,257.17)(8.409,6.000){2}{\rule{0.383pt}{0.400pt}}
\multiput(441.00,264.59)(0.943,0.482){9}{\rule{0.833pt}{0.116pt}}
\multiput(441.00,263.17)(9.270,6.000){2}{\rule{0.417pt}{0.400pt}}
\multiput(452.00,270.59)(0.943,0.482){9}{\rule{0.833pt}{0.116pt}}
\multiput(452.00,269.17)(9.270,6.000){2}{\rule{0.417pt}{0.400pt}}
\multiput(463.00,276.59)(0.943,0.482){9}{\rule{0.833pt}{0.116pt}}
\multiput(463.00,275.17)(9.270,6.000){2}{\rule{0.417pt}{0.400pt}}
\multiput(474.00,282.59)(0.798,0.485){11}{\rule{0.729pt}{0.117pt}}
\multiput(474.00,281.17)(9.488,7.000){2}{\rule{0.364pt}{0.400pt}}
\multiput(485.00,289.59)(0.852,0.482){9}{\rule{0.767pt}{0.116pt}}
\multiput(485.00,288.17)(8.409,6.000){2}{\rule{0.383pt}{0.400pt}}
\multiput(495.00,295.59)(0.943,0.482){9}{\rule{0.833pt}{0.116pt}}
\multiput(495.00,294.17)(9.270,6.000){2}{\rule{0.417pt}{0.400pt}}
\multiput(506.00,301.59)(0.943,0.482){9}{\rule{0.833pt}{0.116pt}}
\multiput(506.00,300.17)(9.270,6.000){2}{\rule{0.417pt}{0.400pt}}
\multiput(517.00,307.59)(0.943,0.482){9}{\rule{0.833pt}{0.116pt}}
\multiput(517.00,306.17)(9.270,6.000){2}{\rule{0.417pt}{0.400pt}}
\multiput(528.00,313.59)(0.798,0.485){11}{\rule{0.729pt}{0.117pt}}
\multiput(528.00,312.17)(9.488,7.000){2}{\rule{0.364pt}{0.400pt}}
\multiput(539.00,320.59)(0.852,0.482){9}{\rule{0.767pt}{0.116pt}}
\multiput(539.00,319.17)(8.409,6.000){2}{\rule{0.383pt}{0.400pt}}
\multiput(549.00,326.59)(0.943,0.482){9}{\rule{0.833pt}{0.116pt}}
\multiput(549.00,325.17)(9.270,6.000){2}{\rule{0.417pt}{0.400pt}}
\multiput(560.00,332.59)(0.943,0.482){9}{\rule{0.833pt}{0.116pt}}
\multiput(560.00,331.17)(9.270,6.000){2}{\rule{0.417pt}{0.400pt}}
\multiput(571.00,338.59)(0.943,0.482){9}{\rule{0.833pt}{0.116pt}}
\multiput(571.00,337.17)(9.270,6.000){2}{\rule{0.417pt}{0.400pt}}
\multiput(582.00,344.59)(0.798,0.485){11}{\rule{0.729pt}{0.117pt}}
\multiput(582.00,343.17)(9.488,7.000){2}{\rule{0.364pt}{0.400pt}}
\multiput(593.00,351.59)(0.852,0.482){9}{\rule{0.767pt}{0.116pt}}
\multiput(593.00,350.17)(8.409,6.000){2}{\rule{0.383pt}{0.400pt}}
\multiput(603.00,357.59)(0.943,0.482){9}{\rule{0.833pt}{0.116pt}}
\multiput(603.00,356.17)(9.270,6.000){2}{\rule{0.417pt}{0.400pt}}
\multiput(614.00,363.59)(0.943,0.482){9}{\rule{0.833pt}{0.116pt}}
\multiput(614.00,362.17)(9.270,6.000){2}{\rule{0.417pt}{0.400pt}}
\multiput(625.00,369.59)(0.943,0.482){9}{\rule{0.833pt}{0.116pt}}
\multiput(625.00,368.17)(9.270,6.000){2}{\rule{0.417pt}{0.400pt}}
\multiput(636.00,375.59)(0.798,0.485){11}{\rule{0.729pt}{0.117pt}}
\multiput(636.00,374.17)(9.488,7.000){2}{\rule{0.364pt}{0.400pt}}
\multiput(647.00,382.59)(0.852,0.482){9}{\rule{0.767pt}{0.116pt}}
\multiput(647.00,381.17)(8.409,6.000){2}{\rule{0.383pt}{0.400pt}}
\multiput(657.00,388.59)(0.943,0.482){9}{\rule{0.833pt}{0.116pt}}
\multiput(657.00,387.17)(9.270,6.000){2}{\rule{0.417pt}{0.400pt}}
\multiput(668.00,394.59)(0.943,0.482){9}{\rule{0.833pt}{0.116pt}}
\multiput(668.00,393.17)(9.270,6.000){2}{\rule{0.417pt}{0.400pt}}
\multiput(679.00,400.59)(0.943,0.482){9}{\rule{0.833pt}{0.116pt}}
\multiput(679.00,399.17)(9.270,6.000){2}{\rule{0.417pt}{0.400pt}}
\multiput(690.00,406.59)(0.721,0.485){11}{\rule{0.671pt}{0.117pt}}
\multiput(690.00,405.17)(8.606,7.000){2}{\rule{0.336pt}{0.400pt}}
\multiput(700.00,413.59)(0.943,0.482){9}{\rule{0.833pt}{0.116pt}}
\multiput(700.00,412.17)(9.270,6.000){2}{\rule{0.417pt}{0.400pt}}
\multiput(711.00,419.59)(0.943,0.482){9}{\rule{0.833pt}{0.116pt}}
\multiput(711.00,418.17)(9.270,6.000){2}{\rule{0.417pt}{0.400pt}}
\multiput(722.00,425.59)(0.943,0.482){9}{\rule{0.833pt}{0.116pt}}
\multiput(722.00,424.17)(9.270,6.000){2}{\rule{0.417pt}{0.400pt}}
\multiput(733.00,431.59)(0.943,0.482){9}{\rule{0.833pt}{0.116pt}}
\multiput(733.00,430.17)(9.270,6.000){2}{\rule{0.417pt}{0.400pt}}
\multiput(744.00,437.59)(0.721,0.485){11}{\rule{0.671pt}{0.117pt}}
\multiput(744.00,436.17)(8.606,7.000){2}{\rule{0.336pt}{0.400pt}}
\multiput(754.00,444.59)(0.943,0.482){9}{\rule{0.833pt}{0.116pt}}
\multiput(754.00,443.17)(9.270,6.000){2}{\rule{0.417pt}{0.400pt}}
\multiput(765.00,450.59)(0.943,0.482){9}{\rule{0.833pt}{0.116pt}}
\multiput(765.00,449.17)(9.270,6.000){2}{\rule{0.417pt}{0.400pt}}
\multiput(776.00,456.59)(0.943,0.482){9}{\rule{0.833pt}{0.116pt}}
\multiput(776.00,455.17)(9.270,6.000){2}{\rule{0.417pt}{0.400pt}}
\multiput(787.00,462.59)(0.798,0.485){11}{\rule{0.729pt}{0.117pt}}
\multiput(787.00,461.17)(9.488,7.000){2}{\rule{0.364pt}{0.400pt}}
\multiput(798.00,469.59)(0.852,0.482){9}{\rule{0.767pt}{0.116pt}}
\multiput(798.00,468.17)(8.409,6.000){2}{\rule{0.383pt}{0.400pt}}
\multiput(808.00,475.59)(0.943,0.482){9}{\rule{0.833pt}{0.116pt}}
\multiput(808.00,474.17)(9.270,6.000){2}{\rule{0.417pt}{0.400pt}}
\multiput(819.00,481.59)(0.943,0.482){9}{\rule{0.833pt}{0.116pt}}
\multiput(819.00,480.17)(9.270,6.000){2}{\rule{0.417pt}{0.400pt}}
\multiput(830.00,487.59)(0.943,0.482){9}{\rule{0.833pt}{0.116pt}}
\multiput(830.00,486.17)(9.270,6.000){2}{\rule{0.417pt}{0.400pt}}
\multiput(841.00,493.59)(0.798,0.485){11}{\rule{0.729pt}{0.117pt}}
\multiput(841.00,492.17)(9.488,7.000){2}{\rule{0.364pt}{0.400pt}}
\multiput(852.00,500.59)(0.852,0.482){9}{\rule{0.767pt}{0.116pt}}
\multiput(852.00,499.17)(8.409,6.000){2}{\rule{0.383pt}{0.400pt}}
\multiput(862.00,506.59)(0.943,0.482){9}{\rule{0.833pt}{0.116pt}}
\multiput(862.00,505.17)(9.270,6.000){2}{\rule{0.417pt}{0.400pt}}
\multiput(873.00,512.59)(0.943,0.482){9}{\rule{0.833pt}{0.116pt}}
\multiput(873.00,511.17)(9.270,6.000){2}{\rule{0.417pt}{0.400pt}}
\multiput(884.00,518.59)(0.943,0.482){9}{\rule{0.833pt}{0.116pt}}
\multiput(884.00,517.17)(9.270,6.000){2}{\rule{0.417pt}{0.400pt}}
\multiput(895.00,524.59)(0.798,0.485){11}{\rule{0.729pt}{0.117pt}}
\multiput(895.00,523.17)(9.488,7.000){2}{\rule{0.364pt}{0.400pt}}
\multiput(906.00,531.59)(0.852,0.482){9}{\rule{0.767pt}{0.116pt}}
\multiput(906.00,530.17)(8.409,6.000){2}{\rule{0.383pt}{0.400pt}}
\multiput(916.00,537.59)(0.943,0.482){9}{\rule{0.833pt}{0.116pt}}
\multiput(916.00,536.17)(9.270,6.000){2}{\rule{0.417pt}{0.400pt}}
\multiput(927.00,543.59)(0.943,0.482){9}{\rule{0.833pt}{0.116pt}}
\multiput(927.00,542.17)(9.270,6.000){2}{\rule{0.417pt}{0.400pt}}
\multiput(938.00,549.59)(0.943,0.482){9}{\rule{0.833pt}{0.116pt}}
\multiput(938.00,548.17)(9.270,6.000){2}{\rule{0.417pt}{0.400pt}}
\multiput(949.00,555.59)(0.798,0.485){11}{\rule{0.729pt}{0.117pt}}
\multiput(949.00,554.17)(9.488,7.000){2}{\rule{0.364pt}{0.400pt}}
\multiput(960.00,562.59)(0.852,0.482){9}{\rule{0.767pt}{0.116pt}}
\multiput(960.00,561.17)(8.409,6.000){2}{\rule{0.383pt}{0.400pt}}
\multiput(970.00,568.59)(0.943,0.482){9}{\rule{0.833pt}{0.116pt}}
\multiput(970.00,567.17)(9.270,6.000){2}{\rule{0.417pt}{0.400pt}}
\multiput(981.00,574.59)(0.943,0.482){9}{\rule{0.833pt}{0.116pt}}
\multiput(981.00,573.17)(9.270,6.000){2}{\rule{0.417pt}{0.400pt}}
\multiput(992.00,580.59)(0.943,0.482){9}{\rule{0.833pt}{0.116pt}}
\multiput(992.00,579.17)(9.270,6.000){2}{\rule{0.417pt}{0.400pt}}
\multiput(1003.00,586.59)(0.798,0.485){11}{\rule{0.729pt}{0.117pt}}
\multiput(1003.00,585.17)(9.488,7.000){2}{\rule{0.364pt}{0.400pt}}
\multiput(1014.00,593.59)(0.852,0.482){9}{\rule{0.767pt}{0.116pt}}
\multiput(1014.00,592.17)(8.409,6.000){2}{\rule{0.383pt}{0.400pt}}
\multiput(1024.00,599.59)(0.943,0.482){9}{\rule{0.833pt}{0.116pt}}
\multiput(1024.00,598.17)(9.270,6.000){2}{\rule{0.417pt}{0.400pt}}
\multiput(1035.00,605.59)(0.943,0.482){9}{\rule{0.833pt}{0.116pt}}
\multiput(1035.00,604.17)(9.270,6.000){2}{\rule{0.417pt}{0.400pt}}
\multiput(1046.00,611.59)(0.943,0.482){9}{\rule{0.833pt}{0.116pt}}
\multiput(1046.00,610.17)(9.270,6.000){2}{\rule{0.417pt}{0.400pt}}
\multiput(1057.00,617.59)(0.721,0.485){11}{\rule{0.671pt}{0.117pt}}
\multiput(1057.00,616.17)(8.606,7.000){2}{\rule{0.336pt}{0.400pt}}
\multiput(1067.00,624.59)(0.943,0.482){9}{\rule{0.833pt}{0.116pt}}
\multiput(1067.00,623.17)(9.270,6.000){2}{\rule{0.417pt}{0.400pt}}
\multiput(1078.00,630.59)(0.943,0.482){9}{\rule{0.833pt}{0.116pt}}
\multiput(1078.00,629.17)(9.270,6.000){2}{\rule{0.417pt}{0.400pt}}
\multiput(1089.00,636.59)(0.943,0.482){9}{\rule{0.833pt}{0.116pt}}
\multiput(1089.00,635.17)(9.270,6.000){2}{\rule{0.417pt}{0.400pt}}
\multiput(1100.00,642.59)(0.943,0.482){9}{\rule{0.833pt}{0.116pt}}
\multiput(1100.00,641.17)(9.270,6.000){2}{\rule{0.417pt}{0.400pt}}
\multiput(1111.00,648.59)(0.721,0.485){11}{\rule{0.671pt}{0.117pt}}
\multiput(1111.00,647.17)(8.606,7.000){2}{\rule{0.336pt}{0.400pt}}
\multiput(1121.00,655.59)(0.943,0.482){9}{\rule{0.833pt}{0.116pt}}
\multiput(1121.00,654.17)(9.270,6.000){2}{\rule{0.417pt}{0.400pt}}
\multiput(1132.00,661.59)(0.943,0.482){9}{\rule{0.833pt}{0.116pt}}
\multiput(1132.00,660.17)(9.270,6.000){2}{\rule{0.417pt}{0.400pt}}
\multiput(1143.00,667.59)(0.943,0.482){9}{\rule{0.833pt}{0.116pt}}
\multiput(1143.00,666.17)(9.270,6.000){2}{\rule{0.417pt}{0.400pt}}
\multiput(1154.00,673.59)(0.943,0.482){9}{\rule{0.833pt}{0.116pt}}
\multiput(1154.00,672.17)(9.270,6.000){2}{\rule{0.417pt}{0.400pt}}
\multiput(1165.00,679.59)(0.721,0.485){11}{\rule{0.671pt}{0.117pt}}
\multiput(1165.00,678.17)(8.606,7.000){2}{\rule{0.336pt}{0.400pt}}
\multiput(1175.00,686.59)(0.943,0.482){9}{\rule{0.833pt}{0.116pt}}
\multiput(1175.00,685.17)(9.270,6.000){2}{\rule{0.417pt}{0.400pt}}
\multiput(1186.00,692.59)(0.943,0.482){9}{\rule{0.833pt}{0.116pt}}
\multiput(1186.00,691.17)(9.270,6.000){2}{\rule{0.417pt}{0.400pt}}
\multiput(1197.00,698.59)(0.943,0.482){9}{\rule{0.833pt}{0.116pt}}
\multiput(1197.00,697.17)(9.270,6.000){2}{\rule{0.417pt}{0.400pt}}
\multiput(1208.00,704.59)(0.943,0.482){9}{\rule{0.833pt}{0.116pt}}
\multiput(1208.00,703.17)(9.270,6.000){2}{\rule{0.417pt}{0.400pt}}
\multiput(1219.00,710.59)(0.721,0.485){11}{\rule{0.671pt}{0.117pt}}
\multiput(1219.00,709.17)(8.606,7.000){2}{\rule{0.336pt}{0.400pt}}
\multiput(1229.00,717.59)(0.943,0.482){9}{\rule{0.833pt}{0.116pt}}
\multiput(1229.00,716.17)(9.270,6.000){2}{\rule{0.417pt}{0.400pt}}
\multiput(1240.00,723.59)(0.943,0.482){9}{\rule{0.833pt}{0.116pt}}
\multiput(1240.00,722.17)(9.270,6.000){2}{\rule{0.417pt}{0.400pt}}
\multiput(1251.00,729.59)(0.943,0.482){9}{\rule{0.833pt}{0.116pt}}
\multiput(1251.00,728.17)(9.270,6.000){2}{\rule{0.417pt}{0.400pt}}
\multiput(1262.00,735.59)(0.943,0.482){9}{\rule{0.833pt}{0.116pt}}
\multiput(1262.00,734.17)(9.270,6.000){2}{\rule{0.417pt}{0.400pt}}
\multiput(1273.00,741.59)(0.721,0.485){11}{\rule{0.671pt}{0.117pt}}
\multiput(1273.00,740.17)(8.606,7.000){2}{\rule{0.336pt}{0.400pt}}
\multiput(1283.00,748.59)(0.943,0.482){9}{\rule{0.833pt}{0.116pt}}
\multiput(1283.00,747.17)(9.270,6.000){2}{\rule{0.417pt}{0.400pt}}
\multiput(1294.00,754.59)(0.943,0.482){9}{\rule{0.833pt}{0.116pt}}
\multiput(1294.00,753.17)(9.270,6.000){2}{\rule{0.417pt}{0.400pt}}
\multiput(1305.00,760.59)(0.943,0.482){9}{\rule{0.833pt}{0.116pt}}
\multiput(1305.00,759.17)(9.270,6.000){2}{\rule{0.417pt}{0.400pt}}
\multiput(1316.00,766.59)(0.943,0.482){9}{\rule{0.833pt}{0.116pt}}
\multiput(1316.00,765.17)(9.270,6.000){2}{\rule{0.417pt}{0.400pt}}
\multiput(1327.00,772.59)(0.721,0.485){11}{\rule{0.671pt}{0.117pt}}
\multiput(1327.00,771.17)(8.606,7.000){2}{\rule{0.336pt}{0.400pt}}
\multiput(1337.00,779.59)(0.943,0.482){9}{\rule{0.833pt}{0.116pt}}
\multiput(1337.00,778.17)(9.270,6.000){2}{\rule{0.417pt}{0.400pt}}
\sbox{\plotpoint}{\rule[-0.500pt]{1.000pt}{1.000pt}}%
\put(279,172){\usebox{\plotpoint}}
\put(279.00,172.00){\usebox{\plotpoint}}
\put(296.77,182.70){\usebox{\plotpoint}}
\put(314.55,193.39){\usebox{\plotpoint}}
\put(332.54,203.73){\usebox{\plotpoint}}
\put(350.31,214.44){\usebox{\plotpoint}}
\put(368.08,225.14){\usebox{\plotpoint}}
\put(386.09,235.45){\usebox{\plotpoint}}
\put(403.84,246.19){\usebox{\plotpoint}}
\put(421.62,256.88){\usebox{\plotpoint}}
\put(439.63,267.18){\usebox{\plotpoint}}
\put(457.37,277.93){\usebox{\plotpoint}}
\put(475.15,288.63){\usebox{\plotpoint}}
\put(493.17,298.90){\usebox{\plotpoint}}
\put(510.91,309.68){\usebox{\plotpoint}}
\put(528.68,320.37){\usebox{\plotpoint}}
\put(546.72,330.63){\usebox{\plotpoint}}
\put(564.44,341.42){\usebox{\plotpoint}}
\put(582.63,351.40){\usebox{\plotpoint}}
\put(600.26,362.36){\usebox{\plotpoint}}
\put(617.97,373.17){\usebox{\plotpoint}}
\put(636.19,383.12){\usebox{\plotpoint}}
\put(653.81,394.08){\usebox{\plotpoint}}
\put(671.50,404.91){\usebox{\plotpoint}}
\put(689.73,414.85){\usebox{\plotpoint}}
\put(707.23,425.94){\usebox{\plotpoint}}
\put(725.01,436.64){\usebox{\plotpoint}}
\put(743.23,446.58){\usebox{\plotpoint}}
\put(760.73,457.67){\usebox{\plotpoint}}
\put(778.51,468.37){\usebox{\plotpoint}}
\put(796.73,478.31){\usebox{\plotpoint}}
\put(814.23,489.40){\usebox{\plotpoint}}
\put(832.36,499.50){\usebox{\plotpoint}}
\put(850.23,510.03){\usebox{\plotpoint}}
\put(867.73,521.13){\usebox{\plotpoint}}
\put(885.88,531.20){\usebox{\plotpoint}}
\put(903.73,541.76){\usebox{\plotpoint}}
\put(921.24,552.86){\usebox{\plotpoint}}
\put(939.40,562.89){\usebox{\plotpoint}}
\put(957.23,573.49){\usebox{\plotpoint}}
\put(974.74,584.58){\usebox{\plotpoint}}
\put(992.92,594.59){\usebox{\plotpoint}}
\put(1010.73,605.22){\usebox{\plotpoint}}
\put(1028.24,616.31){\usebox{\plotpoint}}
\put(1046.44,626.28){\usebox{\plotpoint}}
\put(1064.07,637.24){\usebox{\plotpoint}}
\put(1081.77,648.06){\usebox{\plotpoint}}
\put(1099.99,658.00){\usebox{\plotpoint}}
\put(1117.61,668.97){\usebox{\plotpoint}}
\put(1135.60,679.29){\usebox{\plotpoint}}
\put(1153.52,689.74){\usebox{\plotpoint}}
\put(1171.15,700.69){\usebox{\plotpoint}}
\put(1189.15,711.01){\usebox{\plotpoint}}
\put(1207.06,721.49){\usebox{\plotpoint}}
\put(1224.70,732.42){\usebox{\plotpoint}}
\put(1242.71,742.72){\usebox{\plotpoint}}
\put(1260.59,753.23){\usebox{\plotpoint}}
\put(1278.24,764.14){\usebox{\plotpoint}}
\put(1296.26,774.44){\usebox{\plotpoint}}
\put(1314.12,784.98){\usebox{\plotpoint}}
\put(1331.78,795.87){\usebox{\plotpoint}}
\put(1348,805){\usebox{\plotpoint}}
\sbox{\plotpoint}{\rule[-0.200pt]{0.400pt}{0.400pt}}%
\put(181.0,123.0){\rule[-0.200pt]{0.400pt}{177.543pt}}
\put(181.0,123.0){\rule[-0.200pt]{307.870pt}{0.400pt}}
\put(1459.0,123.0){\rule[-0.200pt]{0.400pt}{177.543pt}}
\put(181.0,860.0){\rule[-0.200pt]{307.870pt}{0.400pt}}
\end{picture}

\end{figure}
 \begin {figure}[p]\caption{Inclinazione di $45'$,
velocità in ordinata~(\nicefrac{m}{s}) e tempo in ascissa~(s).
I punti sperimentali sono raffigurati con le barre di errore su entrambe le misure. La linea continua è la retta interpolante, la linea tratteggiata quella teorica con il valore atteso di $g$, senza considerare gli attriti.}\label{45graf}
\centering
        % GNUPLOT: LaTeX picture
\setlength{\unitlength}{0.240900pt}
\ifx\plotpoint\undefined\newsavebox{\plotpoint}\fi
\begin{picture}(1500,900)(0,0)
\sbox{\plotpoint}{\rule[-0.200pt]{0.400pt}{0.400pt}}%
\put(161,123){\makebox(0,0)[r]{0.25}}
\put(181.0,123.0){\rule[-0.200pt]{4.818pt}{0.400pt}}
\put(161,246){\makebox(0,0)[r]{0.30}}
\put(181.0,246.0){\rule[-0.200pt]{4.818pt}{0.400pt}}
\put(161,369){\makebox(0,0)[r]{0.35}}
\put(181.0,369.0){\rule[-0.200pt]{4.818pt}{0.400pt}}
\put(161,491){\makebox(0,0)[r]{0.40}}
\put(181.0,491.0){\rule[-0.200pt]{4.818pt}{0.400pt}}
\put(161,614){\makebox(0,0)[r]{0.45}}
\put(181.0,614.0){\rule[-0.200pt]{4.818pt}{0.400pt}}
\put(161,737){\makebox(0,0)[r]{0.50}}
\put(181.0,737.0){\rule[-0.200pt]{4.818pt}{0.400pt}}
\put(161,860){\makebox(0,0)[r]{0.55}}
\put(181.0,860.0){\rule[-0.200pt]{4.818pt}{0.400pt}}
\put(181,82){\makebox(0,0){0.0}}
\put(181.0,123.0){\rule[-0.200pt]{0.400pt}{4.818pt}}
\put(437,82){\makebox(0,0){0.5}}
\put(437.0,123.0){\rule[-0.200pt]{0.400pt}{4.818pt}}
\put(692,82){\makebox(0,0){1.0}}
\put(692.0,123.0){\rule[-0.200pt]{0.400pt}{4.818pt}}
\put(948,82){\makebox(0,0){1.5}}
\put(948.0,123.0){\rule[-0.200pt]{0.400pt}{4.818pt}}
\put(1203,82){\makebox(0,0){2.0}}
\put(1203.0,123.0){\rule[-0.200pt]{0.400pt}{4.818pt}}
\put(1459,82){\makebox(0,0){2.5}}
\put(1459.0,123.0){\rule[-0.200pt]{0.400pt}{4.818pt}}
\put(181.0,123.0){\rule[-0.200pt]{0.400pt}{177.543pt}}
\put(181.0,123.0){\rule[-0.200pt]{307.870pt}{0.400pt}}
\put(1459.0,123.0){\rule[-0.200pt]{0.400pt}{177.543pt}}
\put(181.0,860.0){\rule[-0.200pt]{307.870pt}{0.400pt}}
\put(40,491){\makebox(0,0){\rotatebox{90}{velocità $\ms$}}}
\put(820,21){\makebox(0,0){tempo (s)}}
\put(277.0,163.0){\usebox{\plotpoint}}
\put(267.0,163.0){\rule[-0.200pt]{4.818pt}{0.400pt}}
\put(267.0,164.0){\rule[-0.200pt]{4.818pt}{0.400pt}}
\put(455.0,272.0){\usebox{\plotpoint}}
\put(445.0,272.0){\rule[-0.200pt]{4.818pt}{0.400pt}}
\put(445.0,273.0){\rule[-0.200pt]{4.818pt}{0.400pt}}
\put(611.0,361.0){\rule[-0.200pt]{0.400pt}{0.723pt}}
\put(601.0,361.0){\rule[-0.200pt]{4.818pt}{0.400pt}}
\put(601.0,364.0){\rule[-0.200pt]{4.818pt}{0.400pt}}
\put(751.0,449.0){\rule[-0.200pt]{0.400pt}{1.927pt}}
\put(741.0,449.0){\rule[-0.200pt]{4.818pt}{0.400pt}}
\put(741.0,457.0){\rule[-0.200pt]{4.818pt}{0.400pt}}
\put(879.0,526.0){\rule[-0.200pt]{0.400pt}{2.650pt}}
\put(869.0,526.0){\rule[-0.200pt]{4.818pt}{0.400pt}}
\put(869.0,537.0){\rule[-0.200pt]{4.818pt}{0.400pt}}
\put(998.0,591.0){\rule[-0.200pt]{0.400pt}{3.132pt}}
\put(988.0,591.0){\rule[-0.200pt]{4.818pt}{0.400pt}}
\put(988.0,604.0){\rule[-0.200pt]{4.818pt}{0.400pt}}
\put(1111.0,643.0){\rule[-0.200pt]{0.400pt}{3.613pt}}
\put(1101.0,643.0){\rule[-0.200pt]{4.818pt}{0.400pt}}
\put(1101.0,658.0){\rule[-0.200pt]{4.818pt}{0.400pt}}
\put(1217.0,725.0){\rule[-0.200pt]{0.400pt}{4.336pt}}
\put(1207.0,725.0){\rule[-0.200pt]{4.818pt}{0.400pt}}
\put(1207.0,743.0){\rule[-0.200pt]{4.818pt}{0.400pt}}
\put(1317.0,792.0){\rule[-0.200pt]{0.400pt}{4.577pt}}
\put(1307.0,792.0){\rule[-0.200pt]{4.818pt}{0.400pt}}
\put(1307.0,811.0){\rule[-0.200pt]{4.818pt}{0.400pt}}
\put(277,163){\usebox{\plotpoint}}
\put(277.0,153.0){\rule[-0.200pt]{0.400pt}{4.818pt}}
\put(277.0,153.0){\rule[-0.200pt]{0.400pt}{4.818pt}}
\put(455,272){\usebox{\plotpoint}}
\put(455.0,262.0){\rule[-0.200pt]{0.400pt}{4.818pt}}
\put(455.0,262.0){\rule[-0.200pt]{0.400pt}{4.818pt}}
\put(611,362){\usebox{\plotpoint}}
\put(611.0,352.0){\rule[-0.200pt]{0.400pt}{4.818pt}}
\put(611.0,352.0){\rule[-0.200pt]{0.400pt}{4.818pt}}
\put(751,453){\usebox{\plotpoint}}
\put(751.0,443.0){\rule[-0.200pt]{0.400pt}{4.818pt}}
\put(751.0,443.0){\rule[-0.200pt]{0.400pt}{4.818pt}}
\put(879,532){\usebox{\plotpoint}}
\put(879.0,522.0){\rule[-0.200pt]{0.400pt}{4.818pt}}
\put(879.0,522.0){\rule[-0.200pt]{0.400pt}{4.818pt}}
\put(998,597){\usebox{\plotpoint}}
\put(998.0,587.0){\rule[-0.200pt]{0.400pt}{4.818pt}}
\put(998.0,587.0){\rule[-0.200pt]{0.400pt}{4.818pt}}
\put(1110.0,651.0){\usebox{\plotpoint}}
\put(1110.0,641.0){\rule[-0.200pt]{0.400pt}{4.818pt}}
\put(1111.0,641.0){\rule[-0.200pt]{0.400pt}{4.818pt}}
\put(1216.0,734.0){\usebox{\plotpoint}}
\put(1216.0,724.0){\rule[-0.200pt]{0.400pt}{4.818pt}}
\put(1217.0,724.0){\rule[-0.200pt]{0.400pt}{4.818pt}}
\put(1316.0,802.0){\usebox{\plotpoint}}
\put(1316.0,792.0){\rule[-0.200pt]{0.400pt}{4.818pt}}
\put(277,163){\circle*{12}}
\put(455,272){\circle*{12}}
\put(611,362){\circle*{12}}
\put(751,453){\circle*{12}}
\put(879,532){\circle*{12}}
\put(998,597){\circle*{12}}
\put(1111,651){\circle*{12}}
\put(1217,734){\circle*{12}}
\put(1317,802){\circle*{12}}
\put(1317.0,792.0){\rule[-0.200pt]{0.400pt}{4.818pt}}
\put(277,163){\usebox{\plotpoint}}
\multiput(277.00,163.59)(0.852,0.482){9}{\rule{0.767pt}{0.116pt}}
\multiput(277.00,162.17)(8.409,6.000){2}{\rule{0.383pt}{0.400pt}}
\multiput(287.00,169.59)(0.943,0.482){9}{\rule{0.833pt}{0.116pt}}
\multiput(287.00,168.17)(9.270,6.000){2}{\rule{0.417pt}{0.400pt}}
\multiput(298.00,175.59)(0.721,0.485){11}{\rule{0.671pt}{0.117pt}}
\multiput(298.00,174.17)(8.606,7.000){2}{\rule{0.336pt}{0.400pt}}
\multiput(308.00,182.59)(0.943,0.482){9}{\rule{0.833pt}{0.116pt}}
\multiput(308.00,181.17)(9.270,6.000){2}{\rule{0.417pt}{0.400pt}}
\multiput(319.00,188.59)(0.721,0.485){11}{\rule{0.671pt}{0.117pt}}
\multiput(319.00,187.17)(8.606,7.000){2}{\rule{0.336pt}{0.400pt}}
\multiput(329.00,195.59)(0.943,0.482){9}{\rule{0.833pt}{0.116pt}}
\multiput(329.00,194.17)(9.270,6.000){2}{\rule{0.417pt}{0.400pt}}
\multiput(340.00,201.59)(0.852,0.482){9}{\rule{0.767pt}{0.116pt}}
\multiput(340.00,200.17)(8.409,6.000){2}{\rule{0.383pt}{0.400pt}}
\multiput(350.00,207.59)(0.798,0.485){11}{\rule{0.729pt}{0.117pt}}
\multiput(350.00,206.17)(9.488,7.000){2}{\rule{0.364pt}{0.400pt}}
\multiput(361.00,214.59)(0.852,0.482){9}{\rule{0.767pt}{0.116pt}}
\multiput(361.00,213.17)(8.409,6.000){2}{\rule{0.383pt}{0.400pt}}
\multiput(371.00,220.59)(0.943,0.482){9}{\rule{0.833pt}{0.116pt}}
\multiput(371.00,219.17)(9.270,6.000){2}{\rule{0.417pt}{0.400pt}}
\multiput(382.00,226.59)(0.721,0.485){11}{\rule{0.671pt}{0.117pt}}
\multiput(382.00,225.17)(8.606,7.000){2}{\rule{0.336pt}{0.400pt}}
\multiput(392.00,233.59)(0.943,0.482){9}{\rule{0.833pt}{0.116pt}}
\multiput(392.00,232.17)(9.270,6.000){2}{\rule{0.417pt}{0.400pt}}
\multiput(403.00,239.59)(0.852,0.482){9}{\rule{0.767pt}{0.116pt}}
\multiput(403.00,238.17)(8.409,6.000){2}{\rule{0.383pt}{0.400pt}}
\multiput(413.00,245.59)(0.798,0.485){11}{\rule{0.729pt}{0.117pt}}
\multiput(413.00,244.17)(9.488,7.000){2}{\rule{0.364pt}{0.400pt}}
\multiput(424.00,252.59)(0.852,0.482){9}{\rule{0.767pt}{0.116pt}}
\multiput(424.00,251.17)(8.409,6.000){2}{\rule{0.383pt}{0.400pt}}
\multiput(434.00,258.59)(0.798,0.485){11}{\rule{0.729pt}{0.117pt}}
\multiput(434.00,257.17)(9.488,7.000){2}{\rule{0.364pt}{0.400pt}}
\multiput(445.00,265.59)(0.852,0.482){9}{\rule{0.767pt}{0.116pt}}
\multiput(445.00,264.17)(8.409,6.000){2}{\rule{0.383pt}{0.400pt}}
\multiput(455.00,271.59)(0.943,0.482){9}{\rule{0.833pt}{0.116pt}}
\multiput(455.00,270.17)(9.270,6.000){2}{\rule{0.417pt}{0.400pt}}
\multiput(466.00,277.59)(0.721,0.485){11}{\rule{0.671pt}{0.117pt}}
\multiput(466.00,276.17)(8.606,7.000){2}{\rule{0.336pt}{0.400pt}}
\multiput(476.00,284.59)(0.943,0.482){9}{\rule{0.833pt}{0.116pt}}
\multiput(476.00,283.17)(9.270,6.000){2}{\rule{0.417pt}{0.400pt}}
\multiput(487.00,290.59)(0.852,0.482){9}{\rule{0.767pt}{0.116pt}}
\multiput(487.00,289.17)(8.409,6.000){2}{\rule{0.383pt}{0.400pt}}
\multiput(497.00,296.59)(0.798,0.485){11}{\rule{0.729pt}{0.117pt}}
\multiput(497.00,295.17)(9.488,7.000){2}{\rule{0.364pt}{0.400pt}}
\multiput(508.00,303.59)(0.943,0.482){9}{\rule{0.833pt}{0.116pt}}
\multiput(508.00,302.17)(9.270,6.000){2}{\rule{0.417pt}{0.400pt}}
\multiput(519.00,309.59)(0.852,0.482){9}{\rule{0.767pt}{0.116pt}}
\multiput(519.00,308.17)(8.409,6.000){2}{\rule{0.383pt}{0.400pt}}
\multiput(529.00,315.59)(0.798,0.485){11}{\rule{0.729pt}{0.117pt}}
\multiput(529.00,314.17)(9.488,7.000){2}{\rule{0.364pt}{0.400pt}}
\multiput(540.00,322.59)(0.852,0.482){9}{\rule{0.767pt}{0.116pt}}
\multiput(540.00,321.17)(8.409,6.000){2}{\rule{0.383pt}{0.400pt}}
\multiput(550.00,328.59)(0.798,0.485){11}{\rule{0.729pt}{0.117pt}}
\multiput(550.00,327.17)(9.488,7.000){2}{\rule{0.364pt}{0.400pt}}
\multiput(561.00,335.59)(0.852,0.482){9}{\rule{0.767pt}{0.116pt}}
\multiput(561.00,334.17)(8.409,6.000){2}{\rule{0.383pt}{0.400pt}}
\multiput(571.00,341.59)(0.943,0.482){9}{\rule{0.833pt}{0.116pt}}
\multiput(571.00,340.17)(9.270,6.000){2}{\rule{0.417pt}{0.400pt}}
\multiput(582.00,347.59)(0.721,0.485){11}{\rule{0.671pt}{0.117pt}}
\multiput(582.00,346.17)(8.606,7.000){2}{\rule{0.336pt}{0.400pt}}
\multiput(592.00,354.59)(0.943,0.482){9}{\rule{0.833pt}{0.116pt}}
\multiput(592.00,353.17)(9.270,6.000){2}{\rule{0.417pt}{0.400pt}}
\multiput(603.00,360.59)(0.852,0.482){9}{\rule{0.767pt}{0.116pt}}
\multiput(603.00,359.17)(8.409,6.000){2}{\rule{0.383pt}{0.400pt}}
\multiput(613.00,366.59)(0.798,0.485){11}{\rule{0.729pt}{0.117pt}}
\multiput(613.00,365.17)(9.488,7.000){2}{\rule{0.364pt}{0.400pt}}
\multiput(624.00,373.59)(0.852,0.482){9}{\rule{0.767pt}{0.116pt}}
\multiput(624.00,372.17)(8.409,6.000){2}{\rule{0.383pt}{0.400pt}}
\multiput(634.00,379.59)(0.943,0.482){9}{\rule{0.833pt}{0.116pt}}
\multiput(634.00,378.17)(9.270,6.000){2}{\rule{0.417pt}{0.400pt}}
\multiput(645.00,385.59)(0.721,0.485){11}{\rule{0.671pt}{0.117pt}}
\multiput(645.00,384.17)(8.606,7.000){2}{\rule{0.336pt}{0.400pt}}
\multiput(655.00,392.59)(0.943,0.482){9}{\rule{0.833pt}{0.116pt}}
\multiput(655.00,391.17)(9.270,6.000){2}{\rule{0.417pt}{0.400pt}}
\multiput(666.00,398.59)(0.852,0.482){9}{\rule{0.767pt}{0.116pt}}
\multiput(666.00,397.17)(8.409,6.000){2}{\rule{0.383pt}{0.400pt}}
\multiput(676.00,404.59)(0.798,0.485){11}{\rule{0.729pt}{0.117pt}}
\multiput(676.00,403.17)(9.488,7.000){2}{\rule{0.364pt}{0.400pt}}
\multiput(687.00,411.59)(0.852,0.482){9}{\rule{0.767pt}{0.116pt}}
\multiput(687.00,410.17)(8.409,6.000){2}{\rule{0.383pt}{0.400pt}}
\multiput(697.00,417.59)(0.798,0.485){11}{\rule{0.729pt}{0.117pt}}
\multiput(697.00,416.17)(9.488,7.000){2}{\rule{0.364pt}{0.400pt}}
\multiput(708.00,424.59)(0.852,0.482){9}{\rule{0.767pt}{0.116pt}}
\multiput(708.00,423.17)(8.409,6.000){2}{\rule{0.383pt}{0.400pt}}
\multiput(718.00,430.59)(0.943,0.482){9}{\rule{0.833pt}{0.116pt}}
\multiput(718.00,429.17)(9.270,6.000){2}{\rule{0.417pt}{0.400pt}}
\multiput(729.00,436.59)(0.721,0.485){11}{\rule{0.671pt}{0.117pt}}
\multiput(729.00,435.17)(8.606,7.000){2}{\rule{0.336pt}{0.400pt}}
\multiput(739.00,443.59)(0.943,0.482){9}{\rule{0.833pt}{0.116pt}}
\multiput(739.00,442.17)(9.270,6.000){2}{\rule{0.417pt}{0.400pt}}
\multiput(750.00,449.59)(0.852,0.482){9}{\rule{0.767pt}{0.116pt}}
\multiput(750.00,448.17)(8.409,6.000){2}{\rule{0.383pt}{0.400pt}}
\multiput(760.00,455.59)(0.798,0.485){11}{\rule{0.729pt}{0.117pt}}
\multiput(760.00,454.17)(9.488,7.000){2}{\rule{0.364pt}{0.400pt}}
\multiput(771.00,462.59)(0.852,0.482){9}{\rule{0.767pt}{0.116pt}}
\multiput(771.00,461.17)(8.409,6.000){2}{\rule{0.383pt}{0.400pt}}
\multiput(781.00,468.59)(0.943,0.482){9}{\rule{0.833pt}{0.116pt}}
\multiput(781.00,467.17)(9.270,6.000){2}{\rule{0.417pt}{0.400pt}}
\multiput(792.00,474.59)(0.721,0.485){11}{\rule{0.671pt}{0.117pt}}
\multiput(792.00,473.17)(8.606,7.000){2}{\rule{0.336pt}{0.400pt}}
\multiput(802.00,481.59)(0.943,0.482){9}{\rule{0.833pt}{0.116pt}}
\multiput(802.00,480.17)(9.270,6.000){2}{\rule{0.417pt}{0.400pt}}
\multiput(813.00,487.59)(0.721,0.485){11}{\rule{0.671pt}{0.117pt}}
\multiput(813.00,486.17)(8.606,7.000){2}{\rule{0.336pt}{0.400pt}}
\multiput(823.00,494.59)(0.943,0.482){9}{\rule{0.833pt}{0.116pt}}
\multiput(823.00,493.17)(9.270,6.000){2}{\rule{0.417pt}{0.400pt}}
\multiput(834.00,500.59)(0.852,0.482){9}{\rule{0.767pt}{0.116pt}}
\multiput(834.00,499.17)(8.409,6.000){2}{\rule{0.383pt}{0.400pt}}
\multiput(844.00,506.59)(0.798,0.485){11}{\rule{0.729pt}{0.117pt}}
\multiput(844.00,505.17)(9.488,7.000){2}{\rule{0.364pt}{0.400pt}}
\multiput(855.00,513.59)(0.852,0.482){9}{\rule{0.767pt}{0.116pt}}
\multiput(855.00,512.17)(8.409,6.000){2}{\rule{0.383pt}{0.400pt}}
\multiput(865.00,519.59)(0.943,0.482){9}{\rule{0.833pt}{0.116pt}}
\multiput(865.00,518.17)(9.270,6.000){2}{\rule{0.417pt}{0.400pt}}
\multiput(876.00,525.59)(0.721,0.485){11}{\rule{0.671pt}{0.117pt}}
\multiput(876.00,524.17)(8.606,7.000){2}{\rule{0.336pt}{0.400pt}}
\multiput(886.00,532.59)(0.943,0.482){9}{\rule{0.833pt}{0.116pt}}
\multiput(886.00,531.17)(9.270,6.000){2}{\rule{0.417pt}{0.400pt}}
\multiput(897.00,538.59)(0.852,0.482){9}{\rule{0.767pt}{0.116pt}}
\multiput(897.00,537.17)(8.409,6.000){2}{\rule{0.383pt}{0.400pt}}
\multiput(907.00,544.59)(0.798,0.485){11}{\rule{0.729pt}{0.117pt}}
\multiput(907.00,543.17)(9.488,7.000){2}{\rule{0.364pt}{0.400pt}}
\multiput(918.00,551.59)(0.852,0.482){9}{\rule{0.767pt}{0.116pt}}
\multiput(918.00,550.17)(8.409,6.000){2}{\rule{0.383pt}{0.400pt}}
\multiput(928.00,557.59)(0.943,0.482){9}{\rule{0.833pt}{0.116pt}}
\multiput(928.00,556.17)(9.270,6.000){2}{\rule{0.417pt}{0.400pt}}
\multiput(939.00,563.59)(0.721,0.485){11}{\rule{0.671pt}{0.117pt}}
\multiput(939.00,562.17)(8.606,7.000){2}{\rule{0.336pt}{0.400pt}}
\multiput(949.00,570.59)(0.943,0.482){9}{\rule{0.833pt}{0.116pt}}
\multiput(949.00,569.17)(9.270,6.000){2}{\rule{0.417pt}{0.400pt}}
\multiput(960.00,576.59)(0.721,0.485){11}{\rule{0.671pt}{0.117pt}}
\multiput(960.00,575.17)(8.606,7.000){2}{\rule{0.336pt}{0.400pt}}
\multiput(970.00,583.59)(0.943,0.482){9}{\rule{0.833pt}{0.116pt}}
\multiput(970.00,582.17)(9.270,6.000){2}{\rule{0.417pt}{0.400pt}}
\multiput(981.00,589.59)(0.852,0.482){9}{\rule{0.767pt}{0.116pt}}
\multiput(981.00,588.17)(8.409,6.000){2}{\rule{0.383pt}{0.400pt}}
\multiput(991.00,595.59)(0.798,0.485){11}{\rule{0.729pt}{0.117pt}}
\multiput(991.00,594.17)(9.488,7.000){2}{\rule{0.364pt}{0.400pt}}
\multiput(1002.00,602.59)(0.852,0.482){9}{\rule{0.767pt}{0.116pt}}
\multiput(1002.00,601.17)(8.409,6.000){2}{\rule{0.383pt}{0.400pt}}
\multiput(1012.00,608.59)(0.943,0.482){9}{\rule{0.833pt}{0.116pt}}
\multiput(1012.00,607.17)(9.270,6.000){2}{\rule{0.417pt}{0.400pt}}
\multiput(1023.00,614.59)(0.721,0.485){11}{\rule{0.671pt}{0.117pt}}
\multiput(1023.00,613.17)(8.606,7.000){2}{\rule{0.336pt}{0.400pt}}
\multiput(1033.00,621.59)(0.943,0.482){9}{\rule{0.833pt}{0.116pt}}
\multiput(1033.00,620.17)(9.270,6.000){2}{\rule{0.417pt}{0.400pt}}
\multiput(1044.00,627.59)(0.852,0.482){9}{\rule{0.767pt}{0.116pt}}
\multiput(1044.00,626.17)(8.409,6.000){2}{\rule{0.383pt}{0.400pt}}
\multiput(1054.00,633.59)(0.798,0.485){11}{\rule{0.729pt}{0.117pt}}
\multiput(1054.00,632.17)(9.488,7.000){2}{\rule{0.364pt}{0.400pt}}
\multiput(1065.00,640.59)(0.852,0.482){9}{\rule{0.767pt}{0.116pt}}
\multiput(1065.00,639.17)(8.409,6.000){2}{\rule{0.383pt}{0.400pt}}
\multiput(1075.00,646.59)(0.798,0.485){11}{\rule{0.729pt}{0.117pt}}
\multiput(1075.00,645.17)(9.488,7.000){2}{\rule{0.364pt}{0.400pt}}
\multiput(1086.00,653.59)(0.852,0.482){9}{\rule{0.767pt}{0.116pt}}
\multiput(1086.00,652.17)(8.409,6.000){2}{\rule{0.383pt}{0.400pt}}
\multiput(1096.00,659.59)(0.943,0.482){9}{\rule{0.833pt}{0.116pt}}
\multiput(1096.00,658.17)(9.270,6.000){2}{\rule{0.417pt}{0.400pt}}
\multiput(1107.00,665.59)(0.721,0.485){11}{\rule{0.671pt}{0.117pt}}
\multiput(1107.00,664.17)(8.606,7.000){2}{\rule{0.336pt}{0.400pt}}
\multiput(1117.00,672.59)(0.943,0.482){9}{\rule{0.833pt}{0.116pt}}
\multiput(1117.00,671.17)(9.270,6.000){2}{\rule{0.417pt}{0.400pt}}
\multiput(1128.00,678.59)(0.852,0.482){9}{\rule{0.767pt}{0.116pt}}
\multiput(1128.00,677.17)(8.409,6.000){2}{\rule{0.383pt}{0.400pt}}
\multiput(1138.00,684.59)(0.798,0.485){11}{\rule{0.729pt}{0.117pt}}
\multiput(1138.00,683.17)(9.488,7.000){2}{\rule{0.364pt}{0.400pt}}
\multiput(1149.00,691.59)(0.852,0.482){9}{\rule{0.767pt}{0.116pt}}
\multiput(1149.00,690.17)(8.409,6.000){2}{\rule{0.383pt}{0.400pt}}
\multiput(1159.00,697.59)(0.943,0.482){9}{\rule{0.833pt}{0.116pt}}
\multiput(1159.00,696.17)(9.270,6.000){2}{\rule{0.417pt}{0.400pt}}
\multiput(1170.00,703.59)(0.721,0.485){11}{\rule{0.671pt}{0.117pt}}
\multiput(1170.00,702.17)(8.606,7.000){2}{\rule{0.336pt}{0.400pt}}
\multiput(1180.00,710.59)(0.943,0.482){9}{\rule{0.833pt}{0.116pt}}
\multiput(1180.00,709.17)(9.270,6.000){2}{\rule{0.417pt}{0.400pt}}
\multiput(1191.00,716.59)(0.721,0.485){11}{\rule{0.671pt}{0.117pt}}
\multiput(1191.00,715.17)(8.606,7.000){2}{\rule{0.336pt}{0.400pt}}
\multiput(1201.00,723.59)(0.943,0.482){9}{\rule{0.833pt}{0.116pt}}
\multiput(1201.00,722.17)(9.270,6.000){2}{\rule{0.417pt}{0.400pt}}
\multiput(1212.00,729.59)(0.852,0.482){9}{\rule{0.767pt}{0.116pt}}
\multiput(1212.00,728.17)(8.409,6.000){2}{\rule{0.383pt}{0.400pt}}
\multiput(1222.00,735.59)(0.798,0.485){11}{\rule{0.729pt}{0.117pt}}
\multiput(1222.00,734.17)(9.488,7.000){2}{\rule{0.364pt}{0.400pt}}
\multiput(1233.00,742.59)(0.852,0.482){9}{\rule{0.767pt}{0.116pt}}
\multiput(1233.00,741.17)(8.409,6.000){2}{\rule{0.383pt}{0.400pt}}
\multiput(1243.00,748.59)(0.943,0.482){9}{\rule{0.833pt}{0.116pt}}
\multiput(1243.00,747.17)(9.270,6.000){2}{\rule{0.417pt}{0.400pt}}
\multiput(1254.00,754.59)(0.798,0.485){11}{\rule{0.729pt}{0.117pt}}
\multiput(1254.00,753.17)(9.488,7.000){2}{\rule{0.364pt}{0.400pt}}
\multiput(1265.00,761.59)(0.852,0.482){9}{\rule{0.767pt}{0.116pt}}
\multiput(1265.00,760.17)(8.409,6.000){2}{\rule{0.383pt}{0.400pt}}
\multiput(1275.00,767.59)(0.943,0.482){9}{\rule{0.833pt}{0.116pt}}
\multiput(1275.00,766.17)(9.270,6.000){2}{\rule{0.417pt}{0.400pt}}
\multiput(1286.00,773.59)(0.721,0.485){11}{\rule{0.671pt}{0.117pt}}
\multiput(1286.00,772.17)(8.606,7.000){2}{\rule{0.336pt}{0.400pt}}
\multiput(1296.00,780.59)(0.943,0.482){9}{\rule{0.833pt}{0.116pt}}
\multiput(1296.00,779.17)(9.270,6.000){2}{\rule{0.417pt}{0.400pt}}
\multiput(1307.00,786.59)(0.852,0.482){9}{\rule{0.767pt}{0.116pt}}
\multiput(1307.00,785.17)(8.409,6.000){2}{\rule{0.383pt}{0.400pt}}
\sbox{\plotpoint}{\rule[-0.500pt]{1.000pt}{1.000pt}}%
\put(277,164){\usebox{\plotpoint}}
\put(277.00,164.00){\usebox{\plotpoint}}
\put(294.67,174.88){\usebox{\plotpoint}}
\put(312.34,185.76){\usebox{\plotpoint}}
\put(330.02,196.65){\usebox{\plotpoint}}
\put(347.65,207.59){\usebox{\plotpoint}}
\put(365.27,218.56){\usebox{\plotpoint}}
\put(382.88,229.53){\usebox{\plotpoint}}
\put(400.54,240.44){\usebox{\plotpoint}}
\put(418.21,251.32){\usebox{\plotpoint}}
\put(435.89,262.20){\usebox{\plotpoint}}
\put(453.53,273.12){\usebox{\plotpoint}}
\put(471.15,284.09){\usebox{\plotpoint}}
\put(489.11,294.47){\usebox{\plotpoint}}
\put(506.76,305.32){\usebox{\plotpoint}}
\put(524.41,316.25){\usebox{\plotpoint}}
\put(542.03,327.22){\usebox{\plotpoint}}
\put(559.67,338.15){\usebox{\plotpoint}}
\put(577.34,349.03){\usebox{\plotpoint}}
\put(595.01,359.91){\usebox{\plotpoint}}
\put(612.68,370.81){\usebox{\plotpoint}}
\put(630.29,381.78){\usebox{\plotpoint}}
\put(647.91,392.75){\usebox{\plotpoint}}
\put(665.54,403.70){\usebox{\plotpoint}}
\put(683.21,414.59){\usebox{\plotpoint}}
\put(700.88,425.47){\usebox{\plotpoint}}
\put(718.55,436.35){\usebox{\plotpoint}}
\put(736.18,447.31){\usebox{\plotpoint}}
\put(753.79,458.28){\usebox{\plotpoint}}
\put(771.81,468.57){\usebox{\plotpoint}}
\put(789.37,479.57){\usebox{\plotpoint}}
\put(806.88,490.66){\usebox{\plotpoint}}
\put(824.38,501.75){\usebox{\plotpoint}}
\put(842.03,512.62){\usebox{\plotpoint}}
\put(859.77,523.34){\usebox{\plotpoint}}
\put(877.51,534.05){\usebox{\plotpoint}}
\put(895.12,544.97){\usebox{\plotpoint}}
\put(912.63,556.07){\usebox{\plotpoint}}
\put(930.13,567.16){\usebox{\plotpoint}}
\put(947.73,578.11){\usebox{\plotpoint}}
\put(965.47,588.83){\usebox{\plotpoint}}
\put(983.20,599.54){\usebox{\plotpoint}}
\put(1000.87,610.38){\usebox{\plotpoint}}
\put(1018.37,621.48){\usebox{\plotpoint}}
\put(1035.88,632.57){\usebox{\plotpoint}}
\put(1053.86,642.92){\usebox{\plotpoint}}
\put(1071.48,653.89){\usebox{\plotpoint}}
\put(1089.10,664.86){\usebox{\plotpoint}}
\put(1106.72,675.82){\usebox{\plotpoint}}
\put(1124.39,686.70){\usebox{\plotpoint}}
\put(1142.06,697.59){\usebox{\plotpoint}}
\put(1159.74,708.47){\usebox{\plotpoint}}
\put(1177.36,719.42){\usebox{\plotpoint}}
\put(1194.98,730.39){\usebox{\plotpoint}}
\put(1212.60,741.36){\usebox{\plotpoint}}
\put(1230.26,752.26){\usebox{\plotpoint}}
\put(1247.93,763.14){\usebox{\plotpoint}}
\put(1265.85,773.59){\usebox{\plotpoint}}
\put(1283.41,784.59){\usebox{\plotpoint}}
\put(1300.92,795.68){\usebox{\plotpoint}}
\put(1317,806){\usebox{\plotpoint}}
\sbox{\plotpoint}{\rule[-0.200pt]{0.400pt}{0.400pt}}%
\put(181.0,123.0){\rule[-0.200pt]{0.400pt}{177.543pt}}
\put(181.0,123.0){\rule[-0.200pt]{307.870pt}{0.400pt}}
\put(1459.0,123.0){\rule[-0.200pt]{0.400pt}{177.543pt}}
\put(181.0,860.0){\rule[-0.200pt]{307.870pt}{0.400pt}}
\end{picture}

\end{figure}
 \begin {figure}[p]\caption{Inclinazione di $45'$ con slitta carica,
velocità in ordinata~(\nicefrac{m}{s}) e tempo in ascissa~(s).
I punti sperimentali sono raffigurati con le barre di errore su entrambe le misure. La linea continua è la retta interpolante, la linea tratteggiata quella teorica con il valore atteso di $g$, senza considerare gli attriti.}\label{45cgraf}
\centering
        % GNUPLOT: LaTeX picture
\setlength{\unitlength}{0.240900pt}
\ifx\plotpoint\undefined\newsavebox{\plotpoint}\fi
\begin{picture}(1500,900)(0,0)
\sbox{\plotpoint}{\rule[-0.200pt]{0.400pt}{0.400pt}}%
\put(161,123){\makebox(0,0)[r]{0.25}}
\put(181.0,123.0){\rule[-0.200pt]{4.818pt}{0.400pt}}
\put(161,246){\makebox(0,0)[r]{0.30}}
\put(181.0,246.0){\rule[-0.200pt]{4.818pt}{0.400pt}}
\put(161,369){\makebox(0,0)[r]{0.35}}
\put(181.0,369.0){\rule[-0.200pt]{4.818pt}{0.400pt}}
\put(161,491){\makebox(0,0)[r]{0.40}}
\put(181.0,491.0){\rule[-0.200pt]{4.818pt}{0.400pt}}
\put(161,614){\makebox(0,0)[r]{0.45}}
\put(181.0,614.0){\rule[-0.200pt]{4.818pt}{0.400pt}}
\put(161,737){\makebox(0,0)[r]{0.50}}
\put(181.0,737.0){\rule[-0.200pt]{4.818pt}{0.400pt}}
\put(161,860){\makebox(0,0)[r]{0.55}}
\put(181.0,860.0){\rule[-0.200pt]{4.818pt}{0.400pt}}
\put(181,82){\makebox(0,0){0.0}}
\put(181.0,123.0){\rule[-0.200pt]{0.400pt}{4.818pt}}
\put(437,82){\makebox(0,0){0.5}}
\put(437.0,123.0){\rule[-0.200pt]{0.400pt}{4.818pt}}
\put(692,82){\makebox(0,0){1.0}}
\put(692.0,123.0){\rule[-0.200pt]{0.400pt}{4.818pt}}
\put(948,82){\makebox(0,0){1.5}}
\put(948.0,123.0){\rule[-0.200pt]{0.400pt}{4.818pt}}
\put(1203,82){\makebox(0,0){2.0}}
\put(1203.0,123.0){\rule[-0.200pt]{0.400pt}{4.818pt}}
\put(1459,82){\makebox(0,0){2.5}}
\put(1459.0,123.0){\rule[-0.200pt]{0.400pt}{4.818pt}}
\put(181.0,123.0){\rule[-0.200pt]{0.400pt}{177.543pt}}
\put(181.0,123.0){\rule[-0.200pt]{307.870pt}{0.400pt}}
\put(1459.0,123.0){\rule[-0.200pt]{0.400pt}{177.543pt}}
\put(181.0,860.0){\rule[-0.200pt]{307.870pt}{0.400pt}}
\put(40,491){\makebox(0,0){\rotatebox{90}{velocità $\ms$}}}
\put(820,21){\makebox(0,0){tempo (s)}}
\put(277,164){\usebox{\plotpoint}}
\put(267.0,164.0){\rule[-0.200pt]{4.818pt}{0.400pt}}
\put(267.0,164.0){\rule[-0.200pt]{4.818pt}{0.400pt}}
\put(455.0,274.0){\usebox{\plotpoint}}
\put(445.0,274.0){\rule[-0.200pt]{4.818pt}{0.400pt}}
\put(445.0,275.0){\rule[-0.200pt]{4.818pt}{0.400pt}}
\put(610.0,364.0){\rule[-0.200pt]{0.400pt}{0.482pt}}
\put(600.0,364.0){\rule[-0.200pt]{4.818pt}{0.400pt}}
\put(600.0,366.0){\rule[-0.200pt]{4.818pt}{0.400pt}}
\put(750.0,457.0){\rule[-0.200pt]{0.400pt}{0.964pt}}
\put(740.0,457.0){\rule[-0.200pt]{4.818pt}{0.400pt}}
\put(740.0,461.0){\rule[-0.200pt]{4.818pt}{0.400pt}}
\put(877.0,524.0){\rule[-0.200pt]{0.400pt}{0.964pt}}
\put(867.0,524.0){\rule[-0.200pt]{4.818pt}{0.400pt}}
\put(867.0,528.0){\rule[-0.200pt]{4.818pt}{0.400pt}}
\put(996.0,603.0){\rule[-0.200pt]{0.400pt}{2.650pt}}
\put(986.0,603.0){\rule[-0.200pt]{4.818pt}{0.400pt}}
\put(986.0,614.0){\rule[-0.200pt]{4.818pt}{0.400pt}}
\put(1107.0,665.0){\rule[-0.200pt]{0.400pt}{3.132pt}}
\put(1097.0,665.0){\rule[-0.200pt]{4.818pt}{0.400pt}}
\put(1097.0,678.0){\rule[-0.200pt]{4.818pt}{0.400pt}}
\put(1212.0,731.0){\rule[-0.200pt]{0.400pt}{2.168pt}}
\put(1202.0,731.0){\rule[-0.200pt]{4.818pt}{0.400pt}}
\put(1202.0,740.0){\rule[-0.200pt]{4.818pt}{0.400pt}}
\put(1313.0,762.0){\rule[-0.200pt]{0.400pt}{6.263pt}}
\put(1303.0,762.0){\rule[-0.200pt]{4.818pt}{0.400pt}}
\put(1303.0,788.0){\rule[-0.200pt]{4.818pt}{0.400pt}}
\put(277,164){\usebox{\plotpoint}}
\put(277.0,154.0){\rule[-0.200pt]{0.400pt}{4.818pt}}
\put(277.0,154.0){\rule[-0.200pt]{0.400pt}{4.818pt}}
\put(455,275){\usebox{\plotpoint}}
\put(455.0,265.0){\rule[-0.200pt]{0.400pt}{4.818pt}}
\put(455.0,265.0){\rule[-0.200pt]{0.400pt}{4.818pt}}
\put(610,365){\usebox{\plotpoint}}
\put(610.0,355.0){\rule[-0.200pt]{0.400pt}{4.818pt}}
\put(610.0,355.0){\rule[-0.200pt]{0.400pt}{4.818pt}}
\put(749.0,459.0){\usebox{\plotpoint}}
\put(749.0,449.0){\rule[-0.200pt]{0.400pt}{4.818pt}}
\put(750.0,449.0){\rule[-0.200pt]{0.400pt}{4.818pt}}
\put(877,526){\usebox{\plotpoint}}
\put(877.0,516.0){\rule[-0.200pt]{0.400pt}{4.818pt}}
\put(877.0,516.0){\rule[-0.200pt]{0.400pt}{4.818pt}}
\put(996,608){\usebox{\plotpoint}}
\put(996.0,598.0){\rule[-0.200pt]{0.400pt}{4.818pt}}
\put(996.0,598.0){\rule[-0.200pt]{0.400pt}{4.818pt}}
\put(1107.0,671.0){\usebox{\plotpoint}}
\put(1107.0,661.0){\rule[-0.200pt]{0.400pt}{4.818pt}}
\put(1108.0,661.0){\rule[-0.200pt]{0.400pt}{4.818pt}}
\put(1212.0,736.0){\usebox{\plotpoint}}
\put(1212.0,726.0){\rule[-0.200pt]{0.400pt}{4.818pt}}
\put(1213.0,726.0){\rule[-0.200pt]{0.400pt}{4.818pt}}
\put(1313.0,775.0){\usebox{\plotpoint}}
\put(1313.0,765.0){\rule[-0.200pt]{0.400pt}{4.818pt}}
\put(277,164){\circle*{12}}
\put(455,275){\circle*{12}}
\put(610,365){\circle*{12}}
\put(750,459){\circle*{12}}
\put(877,526){\circle*{12}}
\put(996,608){\circle*{12}}
\put(1107,671){\circle*{12}}
\put(1212,736){\circle*{12}}
\put(1313,775){\circle*{12}}
\put(1314.0,765.0){\rule[-0.200pt]{0.400pt}{4.818pt}}
\put(277,168){\usebox{\plotpoint}}
\multiput(277.00,168.59)(0.852,0.482){9}{\rule{0.767pt}{0.116pt}}
\multiput(277.00,167.17)(8.409,6.000){2}{\rule{0.383pt}{0.400pt}}
\multiput(287.00,174.59)(0.798,0.485){11}{\rule{0.729pt}{0.117pt}}
\multiput(287.00,173.17)(9.488,7.000){2}{\rule{0.364pt}{0.400pt}}
\multiput(298.00,181.59)(0.852,0.482){9}{\rule{0.767pt}{0.116pt}}
\multiput(298.00,180.17)(8.409,6.000){2}{\rule{0.383pt}{0.400pt}}
\multiput(308.00,187.59)(0.943,0.482){9}{\rule{0.833pt}{0.116pt}}
\multiput(308.00,186.17)(9.270,6.000){2}{\rule{0.417pt}{0.400pt}}
\multiput(319.00,193.59)(0.721,0.485){11}{\rule{0.671pt}{0.117pt}}
\multiput(319.00,192.17)(8.606,7.000){2}{\rule{0.336pt}{0.400pt}}
\multiput(329.00,200.59)(0.943,0.482){9}{\rule{0.833pt}{0.116pt}}
\multiput(329.00,199.17)(9.270,6.000){2}{\rule{0.417pt}{0.400pt}}
\multiput(340.00,206.59)(0.852,0.482){9}{\rule{0.767pt}{0.116pt}}
\multiput(340.00,205.17)(8.409,6.000){2}{\rule{0.383pt}{0.400pt}}
\multiput(350.00,212.59)(0.943,0.482){9}{\rule{0.833pt}{0.116pt}}
\multiput(350.00,211.17)(9.270,6.000){2}{\rule{0.417pt}{0.400pt}}
\multiput(361.00,218.59)(0.721,0.485){11}{\rule{0.671pt}{0.117pt}}
\multiput(361.00,217.17)(8.606,7.000){2}{\rule{0.336pt}{0.400pt}}
\multiput(371.00,225.59)(0.943,0.482){9}{\rule{0.833pt}{0.116pt}}
\multiput(371.00,224.17)(9.270,6.000){2}{\rule{0.417pt}{0.400pt}}
\multiput(382.00,231.59)(0.852,0.482){9}{\rule{0.767pt}{0.116pt}}
\multiput(382.00,230.17)(8.409,6.000){2}{\rule{0.383pt}{0.400pt}}
\multiput(392.00,237.59)(0.798,0.485){11}{\rule{0.729pt}{0.117pt}}
\multiput(392.00,236.17)(9.488,7.000){2}{\rule{0.364pt}{0.400pt}}
\multiput(403.00,244.59)(0.852,0.482){9}{\rule{0.767pt}{0.116pt}}
\multiput(403.00,243.17)(8.409,6.000){2}{\rule{0.383pt}{0.400pt}}
\multiput(413.00,250.59)(0.852,0.482){9}{\rule{0.767pt}{0.116pt}}
\multiput(413.00,249.17)(8.409,6.000){2}{\rule{0.383pt}{0.400pt}}
\multiput(423.00,256.59)(0.943,0.482){9}{\rule{0.833pt}{0.116pt}}
\multiput(423.00,255.17)(9.270,6.000){2}{\rule{0.417pt}{0.400pt}}
\multiput(434.00,262.59)(0.721,0.485){11}{\rule{0.671pt}{0.117pt}}
\multiput(434.00,261.17)(8.606,7.000){2}{\rule{0.336pt}{0.400pt}}
\multiput(444.00,269.59)(0.943,0.482){9}{\rule{0.833pt}{0.116pt}}
\multiput(444.00,268.17)(9.270,6.000){2}{\rule{0.417pt}{0.400pt}}
\multiput(455.00,275.59)(0.852,0.482){9}{\rule{0.767pt}{0.116pt}}
\multiput(455.00,274.17)(8.409,6.000){2}{\rule{0.383pt}{0.400pt}}
\multiput(465.00,281.59)(0.798,0.485){11}{\rule{0.729pt}{0.117pt}}
\multiput(465.00,280.17)(9.488,7.000){2}{\rule{0.364pt}{0.400pt}}
\multiput(476.00,288.59)(0.852,0.482){9}{\rule{0.767pt}{0.116pt}}
\multiput(476.00,287.17)(8.409,6.000){2}{\rule{0.383pt}{0.400pt}}
\multiput(486.00,294.59)(0.943,0.482){9}{\rule{0.833pt}{0.116pt}}
\multiput(486.00,293.17)(9.270,6.000){2}{\rule{0.417pt}{0.400pt}}
\multiput(497.00,300.59)(0.852,0.482){9}{\rule{0.767pt}{0.116pt}}
\multiput(497.00,299.17)(8.409,6.000){2}{\rule{0.383pt}{0.400pt}}
\multiput(507.00,306.59)(0.798,0.485){11}{\rule{0.729pt}{0.117pt}}
\multiput(507.00,305.17)(9.488,7.000){2}{\rule{0.364pt}{0.400pt}}
\multiput(518.00,313.59)(0.852,0.482){9}{\rule{0.767pt}{0.116pt}}
\multiput(518.00,312.17)(8.409,6.000){2}{\rule{0.383pt}{0.400pt}}
\multiput(528.00,319.59)(0.943,0.482){9}{\rule{0.833pt}{0.116pt}}
\multiput(528.00,318.17)(9.270,6.000){2}{\rule{0.417pt}{0.400pt}}
\multiput(539.00,325.59)(0.721,0.485){11}{\rule{0.671pt}{0.117pt}}
\multiput(539.00,324.17)(8.606,7.000){2}{\rule{0.336pt}{0.400pt}}
\multiput(549.00,332.59)(0.943,0.482){9}{\rule{0.833pt}{0.116pt}}
\multiput(549.00,331.17)(9.270,6.000){2}{\rule{0.417pt}{0.400pt}}
\multiput(560.00,338.59)(0.852,0.482){9}{\rule{0.767pt}{0.116pt}}
\multiput(560.00,337.17)(8.409,6.000){2}{\rule{0.383pt}{0.400pt}}
\multiput(570.00,344.59)(0.943,0.482){9}{\rule{0.833pt}{0.116pt}}
\multiput(570.00,343.17)(9.270,6.000){2}{\rule{0.417pt}{0.400pt}}
\multiput(581.00,350.59)(0.721,0.485){11}{\rule{0.671pt}{0.117pt}}
\multiput(581.00,349.17)(8.606,7.000){2}{\rule{0.336pt}{0.400pt}}
\multiput(591.00,357.59)(0.943,0.482){9}{\rule{0.833pt}{0.116pt}}
\multiput(591.00,356.17)(9.270,6.000){2}{\rule{0.417pt}{0.400pt}}
\multiput(602.00,363.59)(0.852,0.482){9}{\rule{0.767pt}{0.116pt}}
\multiput(602.00,362.17)(8.409,6.000){2}{\rule{0.383pt}{0.400pt}}
\multiput(612.00,369.59)(0.721,0.485){11}{\rule{0.671pt}{0.117pt}}
\multiput(612.00,368.17)(8.606,7.000){2}{\rule{0.336pt}{0.400pt}}
\multiput(622.00,376.59)(0.943,0.482){9}{\rule{0.833pt}{0.116pt}}
\multiput(622.00,375.17)(9.270,6.000){2}{\rule{0.417pt}{0.400pt}}
\multiput(633.00,382.59)(0.852,0.482){9}{\rule{0.767pt}{0.116pt}}
\multiput(633.00,381.17)(8.409,6.000){2}{\rule{0.383pt}{0.400pt}}
\multiput(643.00,388.59)(0.798,0.485){11}{\rule{0.729pt}{0.117pt}}
\multiput(643.00,387.17)(9.488,7.000){2}{\rule{0.364pt}{0.400pt}}
\multiput(654.00,395.59)(0.852,0.482){9}{\rule{0.767pt}{0.116pt}}
\multiput(654.00,394.17)(8.409,6.000){2}{\rule{0.383pt}{0.400pt}}
\multiput(664.00,401.59)(0.943,0.482){9}{\rule{0.833pt}{0.116pt}}
\multiput(664.00,400.17)(9.270,6.000){2}{\rule{0.417pt}{0.400pt}}
\multiput(675.00,407.59)(0.852,0.482){9}{\rule{0.767pt}{0.116pt}}
\multiput(675.00,406.17)(8.409,6.000){2}{\rule{0.383pt}{0.400pt}}
\multiput(685.00,413.59)(0.798,0.485){11}{\rule{0.729pt}{0.117pt}}
\multiput(685.00,412.17)(9.488,7.000){2}{\rule{0.364pt}{0.400pt}}
\multiput(696.00,420.59)(0.852,0.482){9}{\rule{0.767pt}{0.116pt}}
\multiput(696.00,419.17)(8.409,6.000){2}{\rule{0.383pt}{0.400pt}}
\multiput(706.00,426.59)(0.943,0.482){9}{\rule{0.833pt}{0.116pt}}
\multiput(706.00,425.17)(9.270,6.000){2}{\rule{0.417pt}{0.400pt}}
\multiput(717.00,432.59)(0.721,0.485){11}{\rule{0.671pt}{0.117pt}}
\multiput(717.00,431.17)(8.606,7.000){2}{\rule{0.336pt}{0.400pt}}
\multiput(727.00,439.59)(0.943,0.482){9}{\rule{0.833pt}{0.116pt}}
\multiput(727.00,438.17)(9.270,6.000){2}{\rule{0.417pt}{0.400pt}}
\multiput(738.00,445.59)(0.852,0.482){9}{\rule{0.767pt}{0.116pt}}
\multiput(738.00,444.17)(8.409,6.000){2}{\rule{0.383pt}{0.400pt}}
\multiput(748.00,451.59)(0.943,0.482){9}{\rule{0.833pt}{0.116pt}}
\multiput(748.00,450.17)(9.270,6.000){2}{\rule{0.417pt}{0.400pt}}
\multiput(759.00,457.59)(0.721,0.485){11}{\rule{0.671pt}{0.117pt}}
\multiput(759.00,456.17)(8.606,7.000){2}{\rule{0.336pt}{0.400pt}}
\multiput(769.00,464.59)(0.943,0.482){9}{\rule{0.833pt}{0.116pt}}
\multiput(769.00,463.17)(9.270,6.000){2}{\rule{0.417pt}{0.400pt}}
\multiput(780.00,470.59)(0.852,0.482){9}{\rule{0.767pt}{0.116pt}}
\multiput(780.00,469.17)(8.409,6.000){2}{\rule{0.383pt}{0.400pt}}
\multiput(790.00,476.59)(0.798,0.485){11}{\rule{0.729pt}{0.117pt}}
\multiput(790.00,475.17)(9.488,7.000){2}{\rule{0.364pt}{0.400pt}}
\multiput(801.00,483.59)(0.852,0.482){9}{\rule{0.767pt}{0.116pt}}
\multiput(801.00,482.17)(8.409,6.000){2}{\rule{0.383pt}{0.400pt}}
\multiput(811.00,489.59)(0.852,0.482){9}{\rule{0.767pt}{0.116pt}}
\multiput(811.00,488.17)(8.409,6.000){2}{\rule{0.383pt}{0.400pt}}
\multiput(821.00,495.59)(0.943,0.482){9}{\rule{0.833pt}{0.116pt}}
\multiput(821.00,494.17)(9.270,6.000){2}{\rule{0.417pt}{0.400pt}}
\multiput(832.00,501.59)(0.721,0.485){11}{\rule{0.671pt}{0.117pt}}
\multiput(832.00,500.17)(8.606,7.000){2}{\rule{0.336pt}{0.400pt}}
\multiput(842.00,508.59)(0.943,0.482){9}{\rule{0.833pt}{0.116pt}}
\multiput(842.00,507.17)(9.270,6.000){2}{\rule{0.417pt}{0.400pt}}
\multiput(853.00,514.59)(0.852,0.482){9}{\rule{0.767pt}{0.116pt}}
\multiput(853.00,513.17)(8.409,6.000){2}{\rule{0.383pt}{0.400pt}}
\multiput(863.00,520.59)(0.798,0.485){11}{\rule{0.729pt}{0.117pt}}
\multiput(863.00,519.17)(9.488,7.000){2}{\rule{0.364pt}{0.400pt}}
\multiput(874.00,527.59)(0.852,0.482){9}{\rule{0.767pt}{0.116pt}}
\multiput(874.00,526.17)(8.409,6.000){2}{\rule{0.383pt}{0.400pt}}
\multiput(884.00,533.59)(0.943,0.482){9}{\rule{0.833pt}{0.116pt}}
\multiput(884.00,532.17)(9.270,6.000){2}{\rule{0.417pt}{0.400pt}}
\multiput(895.00,539.59)(0.852,0.482){9}{\rule{0.767pt}{0.116pt}}
\multiput(895.00,538.17)(8.409,6.000){2}{\rule{0.383pt}{0.400pt}}
\multiput(905.00,545.59)(0.798,0.485){11}{\rule{0.729pt}{0.117pt}}
\multiput(905.00,544.17)(9.488,7.000){2}{\rule{0.364pt}{0.400pt}}
\multiput(916.00,552.59)(0.852,0.482){9}{\rule{0.767pt}{0.116pt}}
\multiput(916.00,551.17)(8.409,6.000){2}{\rule{0.383pt}{0.400pt}}
\multiput(926.00,558.59)(0.943,0.482){9}{\rule{0.833pt}{0.116pt}}
\multiput(926.00,557.17)(9.270,6.000){2}{\rule{0.417pt}{0.400pt}}
\multiput(937.00,564.59)(0.721,0.485){11}{\rule{0.671pt}{0.117pt}}
\multiput(937.00,563.17)(8.606,7.000){2}{\rule{0.336pt}{0.400pt}}
\multiput(947.00,571.59)(0.943,0.482){9}{\rule{0.833pt}{0.116pt}}
\multiput(947.00,570.17)(9.270,6.000){2}{\rule{0.417pt}{0.400pt}}
\multiput(958.00,577.59)(0.852,0.482){9}{\rule{0.767pt}{0.116pt}}
\multiput(958.00,576.17)(8.409,6.000){2}{\rule{0.383pt}{0.400pt}}
\multiput(968.00,583.59)(0.943,0.482){9}{\rule{0.833pt}{0.116pt}}
\multiput(968.00,582.17)(9.270,6.000){2}{\rule{0.417pt}{0.400pt}}
\multiput(979.00,589.59)(0.721,0.485){11}{\rule{0.671pt}{0.117pt}}
\multiput(979.00,588.17)(8.606,7.000){2}{\rule{0.336pt}{0.400pt}}
\multiput(989.00,596.59)(0.943,0.482){9}{\rule{0.833pt}{0.116pt}}
\multiput(989.00,595.17)(9.270,6.000){2}{\rule{0.417pt}{0.400pt}}
\multiput(1000.00,602.59)(0.852,0.482){9}{\rule{0.767pt}{0.116pt}}
\multiput(1000.00,601.17)(8.409,6.000){2}{\rule{0.383pt}{0.400pt}}
\multiput(1010.00,608.59)(0.721,0.485){11}{\rule{0.671pt}{0.117pt}}
\multiput(1010.00,607.17)(8.606,7.000){2}{\rule{0.336pt}{0.400pt}}
\multiput(1020.00,615.59)(0.943,0.482){9}{\rule{0.833pt}{0.116pt}}
\multiput(1020.00,614.17)(9.270,6.000){2}{\rule{0.417pt}{0.400pt}}
\multiput(1031.00,621.59)(0.852,0.482){9}{\rule{0.767pt}{0.116pt}}
\multiput(1031.00,620.17)(8.409,6.000){2}{\rule{0.383pt}{0.400pt}}
\multiput(1041.00,627.59)(0.943,0.482){9}{\rule{0.833pt}{0.116pt}}
\multiput(1041.00,626.17)(9.270,6.000){2}{\rule{0.417pt}{0.400pt}}
\multiput(1052.00,633.59)(0.721,0.485){11}{\rule{0.671pt}{0.117pt}}
\multiput(1052.00,632.17)(8.606,7.000){2}{\rule{0.336pt}{0.400pt}}
\multiput(1062.00,640.59)(0.943,0.482){9}{\rule{0.833pt}{0.116pt}}
\multiput(1062.00,639.17)(9.270,6.000){2}{\rule{0.417pt}{0.400pt}}
\multiput(1073.00,646.59)(0.852,0.482){9}{\rule{0.767pt}{0.116pt}}
\multiput(1073.00,645.17)(8.409,6.000){2}{\rule{0.383pt}{0.400pt}}
\multiput(1083.00,652.59)(0.798,0.485){11}{\rule{0.729pt}{0.117pt}}
\multiput(1083.00,651.17)(9.488,7.000){2}{\rule{0.364pt}{0.400pt}}
\multiput(1094.00,659.59)(0.852,0.482){9}{\rule{0.767pt}{0.116pt}}
\multiput(1094.00,658.17)(8.409,6.000){2}{\rule{0.383pt}{0.400pt}}
\multiput(1104.00,665.59)(0.943,0.482){9}{\rule{0.833pt}{0.116pt}}
\multiput(1104.00,664.17)(9.270,6.000){2}{\rule{0.417pt}{0.400pt}}
\multiput(1115.00,671.59)(0.852,0.482){9}{\rule{0.767pt}{0.116pt}}
\multiput(1115.00,670.17)(8.409,6.000){2}{\rule{0.383pt}{0.400pt}}
\multiput(1125.00,677.59)(0.798,0.485){11}{\rule{0.729pt}{0.117pt}}
\multiput(1125.00,676.17)(9.488,7.000){2}{\rule{0.364pt}{0.400pt}}
\multiput(1136.00,684.59)(0.852,0.482){9}{\rule{0.767pt}{0.116pt}}
\multiput(1136.00,683.17)(8.409,6.000){2}{\rule{0.383pt}{0.400pt}}
\multiput(1146.00,690.59)(0.943,0.482){9}{\rule{0.833pt}{0.116pt}}
\multiput(1146.00,689.17)(9.270,6.000){2}{\rule{0.417pt}{0.400pt}}
\multiput(1157.00,696.59)(0.721,0.485){11}{\rule{0.671pt}{0.117pt}}
\multiput(1157.00,695.17)(8.606,7.000){2}{\rule{0.336pt}{0.400pt}}
\multiput(1167.00,703.59)(0.943,0.482){9}{\rule{0.833pt}{0.116pt}}
\multiput(1167.00,702.17)(9.270,6.000){2}{\rule{0.417pt}{0.400pt}}
\multiput(1178.00,709.59)(0.852,0.482){9}{\rule{0.767pt}{0.116pt}}
\multiput(1178.00,708.17)(8.409,6.000){2}{\rule{0.383pt}{0.400pt}}
\multiput(1188.00,715.59)(0.943,0.482){9}{\rule{0.833pt}{0.116pt}}
\multiput(1188.00,714.17)(9.270,6.000){2}{\rule{0.417pt}{0.400pt}}
\multiput(1199.00,721.59)(0.721,0.485){11}{\rule{0.671pt}{0.117pt}}
\multiput(1199.00,720.17)(8.606,7.000){2}{\rule{0.336pt}{0.400pt}}
\multiput(1209.00,728.59)(0.943,0.482){9}{\rule{0.833pt}{0.116pt}}
\multiput(1209.00,727.17)(9.270,6.000){2}{\rule{0.417pt}{0.400pt}}
\multiput(1220.00,734.59)(0.852,0.482){9}{\rule{0.767pt}{0.116pt}}
\multiput(1220.00,733.17)(8.409,6.000){2}{\rule{0.383pt}{0.400pt}}
\multiput(1230.00,740.59)(0.721,0.485){11}{\rule{0.671pt}{0.117pt}}
\multiput(1230.00,739.17)(8.606,7.000){2}{\rule{0.336pt}{0.400pt}}
\multiput(1240.00,747.59)(0.943,0.482){9}{\rule{0.833pt}{0.116pt}}
\multiput(1240.00,746.17)(9.270,6.000){2}{\rule{0.417pt}{0.400pt}}
\multiput(1251.00,753.59)(0.852,0.482){9}{\rule{0.767pt}{0.116pt}}
\multiput(1251.00,752.17)(8.409,6.000){2}{\rule{0.383pt}{0.400pt}}
\multiput(1261.00,759.59)(0.798,0.485){11}{\rule{0.729pt}{0.117pt}}
\multiput(1261.00,758.17)(9.488,7.000){2}{\rule{0.364pt}{0.400pt}}
\multiput(1272.00,766.59)(0.852,0.482){9}{\rule{0.767pt}{0.116pt}}
\multiput(1272.00,765.17)(8.409,6.000){2}{\rule{0.383pt}{0.400pt}}
\multiput(1282.00,772.59)(0.943,0.482){9}{\rule{0.833pt}{0.116pt}}
\multiput(1282.00,771.17)(9.270,6.000){2}{\rule{0.417pt}{0.400pt}}
\multiput(1293.00,778.59)(0.852,0.482){9}{\rule{0.767pt}{0.116pt}}
\multiput(1293.00,777.17)(8.409,6.000){2}{\rule{0.383pt}{0.400pt}}
\multiput(1303.00,784.59)(0.798,0.485){11}{\rule{0.729pt}{0.117pt}}
\multiput(1303.00,783.17)(9.488,7.000){2}{\rule{0.364pt}{0.400pt}}
\sbox{\plotpoint}{\rule[-0.500pt]{1.000pt}{1.000pt}}%
\put(277,170){\usebox{\plotpoint}}
\put(277.00,170.00){\usebox{\plotpoint}}
\put(294.67,180.88){\usebox{\plotpoint}}
\put(312.34,191.76){\usebox{\plotpoint}}
\put(330.06,202.58){\usebox{\plotpoint}}
\put(347.72,213.41){\usebox{\plotpoint}}
\put(365.46,224.12){\usebox{\plotpoint}}
\put(383.20,234.84){\usebox{\plotpoint}}
\put(400.79,245.80){\usebox{\plotpoint}}
\put(418.18,257.11){\usebox{\plotpoint}}
\put(435.79,268.08){\usebox{\plotpoint}}
\put(453.44,279.00){\usebox{\plotpoint}}
\put(471.36,289.47){\usebox{\plotpoint}}
\put(488.86,300.56){\usebox{\plotpoint}}
\put(506.41,311.59){\usebox{\plotpoint}}
\put(524.15,322.30){\usebox{\plotpoint}}
\put(541.89,333.02){\usebox{\plotpoint}}
\put(559.60,343.78){\usebox{\plotpoint}}
\put(577.10,354.87){\usebox{\plotpoint}}
\put(594.61,365.97){\usebox{\plotpoint}}
\put(612.55,376.39){\usebox{\plotpoint}}
\put(630.10,387.42){\usebox{\plotpoint}}
\put(647.60,398.51){\usebox{\plotpoint}}
\put(665.11,409.60){\usebox{\plotpoint}}
\put(682.77,420.44){\usebox{\plotpoint}}
\put(700.51,431.16){\usebox{\plotpoint}}
\put(718.25,441.87){\usebox{\plotpoint}}
\put(735.84,452.82){\usebox{\plotpoint}}
\put(753.60,463.56){\usebox{\plotpoint}}
\put(771.27,474.44){\usebox{\plotpoint}}
\put(788.92,485.35){\usebox{\plotpoint}}
\put(806.54,496.33){\usebox{\plotpoint}}
\put(823.94,507.60){\usebox{\plotpoint}}
\put(841.48,518.64){\usebox{\plotpoint}}
\put(859.22,529.36){\usebox{\plotpoint}}
\put(877.10,539.86){\usebox{\plotpoint}}
\put(894.72,550.82){\usebox{\plotpoint}}
\put(912.39,561.70){\usebox{\plotpoint}}
\put(930.06,572.59){\usebox{\plotpoint}}
\put(947.74,583.47){\usebox{\plotpoint}}
\put(965.37,594.42){\usebox{\plotpoint}}
\put(982.98,605.39){\usebox{\plotpoint}}
\put(1000.60,616.36){\usebox{\plotpoint}}
\put(1018.40,627.04){\usebox{\plotpoint}}
\put(1036.02,638.01){\usebox{\plotpoint}}
\put(1053.63,648.98){\usebox{\plotpoint}}
\put(1071.28,659.90){\usebox{\plotpoint}}
\put(1088.95,670.79){\usebox{\plotpoint}}
\put(1106.62,681.67){\usebox{\plotpoint}}
\put(1124.28,692.57){\usebox{\plotpoint}}
\put(1141.90,703.54){\usebox{\plotpoint}}
\put(1159.82,713.98){\usebox{\plotpoint}}
\put(1177.53,724.74){\usebox{\plotpoint}}
\put(1195.03,735.84){\usebox{\plotpoint}}
\put(1212.54,746.93){\usebox{\plotpoint}}
\put(1230.04,758.03){\usebox{\plotpoint}}
\put(1247.71,768.91){\usebox{\plotpoint}}
\put(1265.56,779.49){\usebox{\plotpoint}}
\put(1283.07,790.58){\usebox{\plotpoint}}
\put(1300.74,801.42){\usebox{\plotpoint}}
\put(1314,809){\usebox{\plotpoint}}
\sbox{\plotpoint}{\rule[-0.200pt]{0.400pt}{0.400pt}}%
\put(181.0,123.0){\rule[-0.200pt]{0.400pt}{177.543pt}}
\put(181.0,123.0){\rule[-0.200pt]{307.870pt}{0.400pt}}
\put(1459.0,123.0){\rule[-0.200pt]{0.400pt}{177.543pt}}
\put(181.0,860.0){\rule[-0.200pt]{307.870pt}{0.400pt}}
\end{picture}

\end{figure}
\begin{table}[p]\caption{Rilevazioni dei tempi di percorrenza~(s) dei vari segmenti di \unit[20]{cm} per la stima dell'attrito. Slitta carica e scarica e con i vari spessori (nessuno, sottile, grosso ed entrambi) davanti all'elettromagnete.}\label{tabattrito}
 \centering \small
 \begin{tabular}{r *4{c} @{\hspace{3.5\tabcolsep}}*4{c}}
& \multicolumn{4}{c}{\textbf{slitta scarica}}
 & \multicolumn{4}{c}{\textbf{slitta carica}}\\
\emph{spessori}
 &  {\nessuno}
&  {\sottile}
&  {\grosso}
&  {\ciccione}
 &  {\nessuno}
&  {\sottile}
&  {\grosso}
&  {\ciccione}\\ \hline
\multirow{5}{*}{\unit[40--60]{cm}} 
&1.1598 &1.3763 &1.6153 &1.7309 &1.6723 &2.0237 &2.4275 &2.5759\\
&1.1661 &1.3806 &1.6199 &1.7547 &1.6877 &2.0213 &2.4175 &2.6245\\
&1.1609 &1.3836 &1.6153 &1.7359 &1.6781 &2.0314 &2.4297 &2.6007\\
&1.1509 &1.3889 &1.6197 &1.7511 &1.6588 &2.0189 &2.3943 &2.5939\\
&1.1523 &1.3855 &1.6068 &1.7616 &1.6588 &2.0399 &2.4305 &2.6169\\[5 pt]
\multirow{5}{*}{\unit[50--70]{cm}}
&1.1612 &1.3890 &1.6354 &1.7528 &1.6781 &2.0698 &2.3803 &2.6177\\
&1.1578 &1.4027 &1.6319 &1.7142 &1.6637 &2.0473 &2.4389 &2.6187\\
&1.1709 &1.4073 &1.6205 &1.7475 &1.6589 &2.0392 &2.3889 &2.5761\\
&1.1655 &1.3770 &1.6317 &1.7149 &1.6753 &2.0550 &2.3928 &2.6167\\
&1.1601 &1.3925 &1.6162 &1.7209 &1.6941 &2.0556 &2.4597 &2.6552\\[5 pt]
\multirow{5}{*}{\unit[60--80]{cm}}
&1.1571 &1.4067 &1.5997 &1.6735 &1.6122 &1.9915 &2.3141 &2.4672\\
&1.1552 &1.3691 &1.5935 &1.6442 &1.6269 &1.9360 &2.3453 &2.4971\\
&1.1622 &1.3783 &1.5322 &1.6472 &1.6353 &1.9484 &2.2295 &2.4539\\
&1.1616 &1.3755 &1.5477 &1.6331 &1.6305 &1.9698 &2.2782 &2.4133\\
&1.1669 &1.3727 &1.5422 &1.6246 &1.6405 &1.9712 &2.2344 &2.4684\\[5 pt]
\multirow{5}{*}{\unit[70--90]{cm}}
&1.1443 &1.3455 &1.5189 &1.6497 &1.6239 &1.8449 &2.1932 &2.3533\\
&1.1617 &1.3295 &1.5426 &1.6124 &1.6453 &1.9107 &2.2118 &2.4287\\
&1.1572 &1.3007 &1.4971 &1.6227 &1.6393 &1.9423 &2.2069 &2.3942\\
&1.1309 &1.3581 &1.5233 &1.6950 &1.6356 &1.8931 &2.2155 &2.4138\\
&1.1304 &1.3287 &1.5235 &1.6143 &1.6267 &1.9146 &2.1841 &2.4440\\[5 pt]
\multirow{5}{*}{\unit[80--100]{cm}}
&1.1327 &1.3157 &1.4995 &1.6222 &1.6181 &1.9011 &2.1912 &2.3175\\
&1.1251 &1.3265 &1.5099 &1.6385 &1.5740 &1.8784 &2.1999 &2.2837\\
&1.1266 &1.3045 &1.4945 &1.6100 &1.5813 &1.9155 &2.1357 &2.2745\\
&1.1499 &1.3367 &1.4913 &1.5961 &1.5525 &1.9077 &2.1686 &2.3509\\
&1.1287 &1.3866 &1.5065 &1.6102 &1.5651 &1.8491 &2.1542 &2.2693\\[5 pt]
\multirow{5}{*}{\unit[90--110]{cm}}
&1.1198 &1.3365 &1.4485 &1.5043 &1.5644 &1.8248 &2.0749 &2.1599\\
&1.1107 &1.3398 &1.5150 &1.5001 &1.5519 &1.8209 &2.1331 &2.2087\\
&1.1309 &1.3000 &1.4797 &1.5242 &1.5455 &1.8635 &2.0251 &2.1009\\
&1.1592 &1.3401 &1.4471 &1.5165 &1.5583 &1.8357 &2.0464 &2.2227\\
&1.1143 &1.3118 &1.4865 &1.5371 &1.5847 &1.8448 &2.0625 &2.2044\\[5 pt]
\multirow{5}{*}{\unit[100--120]{cm}}
&1.0995 &1.2523 &1.4125 &1.4765 &1.5542 &1.7498 &1.9711 &2.1170\\
&1.0981 &1.2854 &1.4335 &1.4597 &1.5195 &1.8111 &2.0238 &2.0840\\
&1.0933 &1.2773 &1.4543 &1.4610 &1.5301 &1.8367 &2.0329 &2.1569\\
&1.1187 &1.2383 &1.4784 &1.4843 &1.5564 &1.7919 &1.9589 &2.1756\\
&1.0925 &1.2333 &1.4097 &1.5073 &1.5690 &1.8051 &1.9611 &2.2094\\[5 pt]
\multirow{5}{*}{\unit[110--130]{cm}}
&1.5133 &1.2731 &1.3770 &1.0079 &1.5435 &1.7191 &1.9875 &2.0419\\
&1.5203 &1.2849 &1.4433 &1.1093 &1.5441 &1.7494 &1.9532 &2.0008\\
&1.4817 &1.2617 &1.4336 &1.1119 &1.5319 &1.7454 &1.9241 &1.9903\\
&1.5327 &1.2417 &1.4251 &1.1249 &1.5564 &1.7605 &2.0190 &1.9605\\
&1.5799 &1.2554 &1.3967 &1.1106 &1.5305 &1.7275 &1.9141 &2.0731
\end{tabular}
\end{table}
\begin{table}[p]\caption{Slitta scarica. Medie sui tempi con errore sulla media e velocità di percorrenza dei vari segmenti di \unit[20]{cm}, con vari spessori (nessuno, sottile, grosso ed entrambi) davanti all'elettromagnete. Gli errori sulla velocità sono stati ricavati con la formula di propagazione.}\label{tabtvscarica}
 \centering \small
 \begin{tabular}{r @{\hspace{3\tabcolsep}} *4{r@{ $\pm$ }l}}
& \multicolumn{8}{c}{tempi (s)}\\
 & \multicolumn{2}{c}{ {\nessuno}}
& \multicolumn{2}{c}{ {\sottile}}
& \multicolumn{2}{c}{ {\grosso}}
& \multicolumn{2}{c}{ {\ciccione}}\\[2 pt]
\unit[40--60]{cm}&1.1580 &0.0063 &1.3830 &0.0048 &1.6154 &0.0053 &1.7468 &0.0130\\
\unit[50--70]{cm}&1.1631 &0.0052 &1.3937 &0.0119 &1.6271 &0.0083 &1.7301 &0.0186\\
\unit[60--80]{cm}&1.1606 &0.0046 &1.3805 &0.0151 &1.5631 &0.0312 &1.6445 &0.0185\\
\unit[70--90]{cm}&1.1449 &0.0145 &1.3325 &0.0216 &1.5211 &0.0162 &1.6388 &0.0348\\
\unit[80--100]{cm}&1.1326 &0.0101 &1.3340 &0.0318 &1.5003 &0.0078 &1.6154 &0.0159\\
\unit[90--110]{cm}&1.1270 &0.0196 &1.3256 &0.0186 &1.4754 &0.0284 &1.5164 &0.0150\\
\unit[100--120]{cm}&1.1004 &0.0107 &1.2573 &0.0232 &1.4377 &0.0290 &1.4778 &0.0195\\
\unit[110--130]{cm}& 1.0929 &0.0479&1.2634 &0.0165 &1.4151 &0.0275 &1.5256 &0.0357\\
\\
& \multicolumn{8}{c}{velocità (\unitfrac{m}{s})}\\
 & \multicolumn{2}{c}{ {\nessuno}}
& \multicolumn{2}{c}{ {\sottile}}
& \multicolumn{2}{c}{ {\grosso}}
& \multicolumn{2}{c}{ {\ciccione}}\\[2 pt]
\unit[40--60]{cm}&0.1727 &0.0011 &0.1446 &0.0007 &0.1238 &0.0007 &0.1145 &0.0015\\
\unit[50--70]{cm}&0.1720 &0.0009 &0.1435 &0.0017 &0.1229 &0.0010 &0.1156 &0.0022\\
\unit[60--80]{cm}&0.1723 &0.0008 &0.1449 &0.0022 &0.1280 &0.0040 &0.1216 &0.0023\\
\unit[70--90]{cm}&0.1747 &0.0025 &0.1501 &0.0032 &0.1315 &0.0021 &0.1220 &0.0042\\
\unit[80--100]{cm}&0.1766 &0.0018 &0.1499 &0.0048 &0.1333 &0.0010 &0.1238 &0.0020\\
\unit[90--110]{cm}&0.1775 &0.0035 &0.1509 &0.0028 &0.1356 &0.0039 &0.1319 &0.0020\\
\unit[100--120]{cm}&0.1817 &0.0019 &0.1591 &0.0037 &0.1391 &0.0040 &0.1353 &0.0026\\
\unit[110--130]{cm}&0.1311 &0.0047 &0.1583 &0.0026 &0.1413 &0.0039 &0.1830 &0.0088
\end{tabular}
\end{table}
\begin{table}[p]\caption{Slitta carica. Medie sui tempi con errore sulla media e velocità di percorrenza dei vari segmenti di \unit[20]{cm}, con vari spessori (nessuno, sottile, grosso ed entrambi) davanti all'elettromagnete. Gli errori sulla velocità sono stati ricavati con la formula di propagazione.}\label{tabtvcarica}
 \centering \small
 \begin{tabular}{r @{\hspace{3\tabcolsep}} *4{r@{ $\pm$ }l}}
& \multicolumn{8}{c}{tempi (s)}\\
 & \multicolumn{2}{c}{ {\nessuno}}
& \multicolumn{2}{c}{ {\sottile}}
& \multicolumn{2}{c}{ {\grosso}}
& \multicolumn{2}{c}{ {\ciccione}}\\[2 pt]
\unit[40--60]{cm}&1.6711 &0.0125 &2.0270 &0.0086 &2.4199 &0.0152 &2.6024 &0.0192\\
\unit[50--70]{cm}&1.6740 &0.0138 &2.0534 &0.0113 &2.4121 &0.0350 &2.6169 &0.0280\\
\unit[60--80]{cm}&1.6291 &0.0107 &1.9634 &0.0216 &2.2803 &0.0501 &2.4600 &0.0305\\
\unit[70--90]{cm}&1.6342 &0.0089 &1.9011 &0.0360 &2.2023 &0.0132 &2.4068 &0.0351\\
\unit[80--100]{cm}&1.5782 &0.0248 &1.8904 &0.0269 &2.1699 &0.0263 &2.2992 &0.0345\\
\unit[90--110]{cm}&1.5610 &0.0150 &1.8379 &0.0171 &2.0684 &0.0407 &2.1793 &0.0498\\
\unit[100--120]{cm}&1.5458 &0.0204 &1.7989 &0.0319 &1.9896 &0.0359 &2.1486 &0.0492\\
\unit[110--130]{cm}&1.5413 &0.0106 &1.7404 &0.0168 &1.9596 &0.0438 &2.0133 &0.0443\\
\\
& \multicolumn{8}{c}{velocità (\unitfrac{m}{s})}\\
 & \multicolumn{2}{c}{ {\nessuno}}
& \multicolumn{2}{c}{ {\sottile}}
& \multicolumn{2}{c}{ {\grosso}}
& \multicolumn{2}{c}{ {\ciccione}}\\[2 pt]
\unit[40--60]{cm}&0.1197 &0.0015 &0.0987 &0.0008 &0.0826 &0.0013 &0.0769 &0.0015\\
\unit[50--70]{cm}&0.1195 &0.0016 &0.0974 &0.0011 &0.0829 &0.0029 &0.0764 &0.0021\\
\unit[60--80]{cm}&0.1228 &0.0013 &0.1019 &0.0022 &0.0877 &0.0044 &0.0813 &0.0025\\
\unit[70--90]{cm}&0.1224 &0.0011 &0.1052 &0.0038 &0.0908 &0.0012 &0.0831 &0.0029\\
\unit[80--100]{cm}&0.1267 &0.0031 &0.1058 &0.0028 &0.0922 &0.0024 &0.0870 &0.0030\\
\unit[90--110]{cm}&0.1281 &0.0019 &0.1088 &0.0019 &0.0967 &0.0039 &0.0918 &0.0046\\
\unit[100--120]{cm}&0.1294 &0.0026 &0.1112 &0.0035 &0.1005 &0.0036 &0.0931 &0.0046\\
\unit[110--130]{cm}&0.1298 &0.0014 &0.1149 &0.0019 &0.1021 &0.0045 &0.0993 &0.0044
\end{tabular}








\end{table}
\clearpage
 \begin {figure}[p]\caption{Slitta scarica, nessuno spessore. Tratteggiata la retta interpolante.}\label{scarico0}
\centering
        % GNUPLOT: LaTeX picture
\setlength{\unitlength}{0.240900pt}
\ifx\plotpoint\undefined\newsavebox{\plotpoint}\fi
\begin{picture}(1500,900)(0,0)
\sbox{\plotpoint}{\rule[-0.200pt]{0.400pt}{0.400pt}}%
\put(181,123){\makebox(0,0)[r]{0.165}}
\put(201.0,123.0){\rule[-0.200pt]{4.818pt}{0.400pt}}
\put(181,228){\makebox(0,0)[r]{0.170}}
\put(201.0,228.0){\rule[-0.200pt]{4.818pt}{0.400pt}}
\put(181,334){\makebox(0,0)[r]{0.175}}
\put(201.0,334.0){\rule[-0.200pt]{4.818pt}{0.400pt}}
\put(181,439){\makebox(0,0)[r]{0.180}}
\put(201.0,439.0){\rule[-0.200pt]{4.818pt}{0.400pt}}
\put(181,544){\makebox(0,0)[r]{0.185}}
\put(201.0,544.0){\rule[-0.200pt]{4.818pt}{0.400pt}}
\put(181,649){\makebox(0,0)[r]{0.190}}
\put(201.0,649.0){\rule[-0.200pt]{4.818pt}{0.400pt}}
\put(181,755){\makebox(0,0)[r]{0.195}}
\put(201.0,755.0){\rule[-0.200pt]{4.818pt}{0.400pt}}
\put(181,860){\makebox(0,0)[r]{0.200}}
\put(201.0,860.0){\rule[-0.200pt]{4.818pt}{0.400pt}}
\put(280,82){\makebox(0,0){ 50}}
\put(280.0,123.0){\rule[-0.200pt]{0.400pt}{4.818pt}}
\put(437,82){\makebox(0,0){ 60}}
\put(437.0,123.0){\rule[-0.200pt]{0.400pt}{4.818pt}}
\put(594,82){\makebox(0,0){ 70}}
\put(594.0,123.0){\rule[-0.200pt]{0.400pt}{4.818pt}}
\put(751,82){\makebox(0,0){ 80}}
\put(751.0,123.0){\rule[-0.200pt]{0.400pt}{4.818pt}}
\put(909,82){\makebox(0,0){ 90}}
\put(909.0,123.0){\rule[-0.200pt]{0.400pt}{4.818pt}}
\put(1066,82){\makebox(0,0){ 100}}
\put(1066.0,123.0){\rule[-0.200pt]{0.400pt}{4.818pt}}
\put(1223,82){\makebox(0,0){ 110}}
\put(1223.0,123.0){\rule[-0.200pt]{0.400pt}{4.818pt}}
\put(1380,82){\makebox(0,0){ 120}}
\put(1380.0,123.0){\rule[-0.200pt]{0.400pt}{4.818pt}}
\put(201.0,123.0){\rule[-0.200pt]{0.400pt}{177.543pt}}
\put(201.0,123.0){\rule[-0.200pt]{303.052pt}{0.400pt}}
\put(1459.0,123.0){\rule[-0.200pt]{0.400pt}{177.543pt}}
\put(201.0,860.0){\rule[-0.200pt]{303.052pt}{0.400pt}}
\put(40,491){\makebox(0,0){\rotatebox{90}{velocità \ms}}}
\put(830,21){\makebox(0,0){posizione (cm)}}
\put(280.0,262.0){\rule[-0.200pt]{0.400pt}{11.081pt}}
\put(270.0,262.0){\rule[-0.200pt]{4.818pt}{0.400pt}}
\put(270.0,308.0){\rule[-0.200pt]{4.818pt}{0.400pt}}
\put(437.0,251.0){\rule[-0.200pt]{0.400pt}{9.154pt}}
\put(427.0,251.0){\rule[-0.200pt]{4.818pt}{0.400pt}}
\put(427.0,289.0){\rule[-0.200pt]{4.818pt}{0.400pt}}
\put(594.0,260.0){\rule[-0.200pt]{0.400pt}{8.191pt}}
\put(584.0,260.0){\rule[-0.200pt]{4.818pt}{0.400pt}}
\put(584.0,294.0){\rule[-0.200pt]{4.818pt}{0.400pt}}
\put(751.0,275.0){\rule[-0.200pt]{0.400pt}{25.294pt}}
\put(741.0,275.0){\rule[-0.200pt]{4.818pt}{0.400pt}}
\put(741.0,380.0){\rule[-0.200pt]{4.818pt}{0.400pt}}
\put(909.0,329.0){\rule[-0.200pt]{0.400pt}{18.308pt}}
\put(899.0,329.0){\rule[-0.200pt]{4.818pt}{0.400pt}}
\put(899.0,405.0){\rule[-0.200pt]{4.818pt}{0.400pt}}
\put(1066.0,313.0){\rule[-0.200pt]{0.400pt}{35.412pt}}
\put(1056.0,313.0){\rule[-0.200pt]{4.818pt}{0.400pt}}
\put(1056.0,460.0){\rule[-0.200pt]{4.818pt}{0.400pt}}
\put(1223.0,435.0){\rule[-0.200pt]{0.400pt}{19.272pt}}
\put(1213.0,435.0){\rule[-0.200pt]{4.818pt}{0.400pt}}
\put(1213.0,515.0){\rule[-0.200pt]{4.818pt}{0.400pt}}
\put(1380.0,317.0){\rule[-0.200pt]{0.400pt}{89.133pt}}
\put(1370.0,317.0){\rule[-0.200pt]{4.818pt}{0.400pt}}
\put(280,285){\circle*{12}}
\put(437,270){\circle*{12}}
\put(594,277){\circle*{12}}
\put(751,327){\circle*{12}}
\put(909,367){\circle*{12}}
\put(1066,386){\circle*{12}}
\put(1223,475){\circle*{12}}
\put(1380,502){\circle*{12}}
\put(1370.0,687.0){\rule[-0.200pt]{4.818pt}{0.400pt}}
\put(201,223){\usebox{\plotpoint}}
\put(201.00,223.00){\usebox{\plotpoint}}
\put(221.31,227.22){\usebox{\plotpoint}}
\put(241.59,231.60){\usebox{\plotpoint}}
\put(261.82,236.27){\usebox{\plotpoint}}
\put(282.06,240.78){\usebox{\plotpoint}}
\put(302.40,244.86){\usebox{\plotpoint}}
\put(322.57,249.75){\usebox{\plotpoint}}
\put(342.81,254.30){\usebox{\plotpoint}}
\put(363.16,258.35){\usebox{\plotpoint}}
\put(383.39,263.01){\usebox{\plotpoint}}
\put(403.75,266.96){\usebox{\plotpoint}}
\put(423.98,271.61){\usebox{\plotpoint}}
\put(444.15,276.50){\usebox{\plotpoint}}
\put(464.51,280.46){\usebox{\plotpoint}}
\put(484.77,284.94){\usebox{\plotpoint}}
\put(504.96,289.76){\usebox{\plotpoint}}
\put(525.26,294.04){\usebox{\plotpoint}}
\put(545.55,298.36){\usebox{\plotpoint}}
\put(565.77,303.02){\usebox{\plotpoint}}
\put(586.00,307.62){\usebox{\plotpoint}}
\put(606.35,311.62){\usebox{\plotpoint}}
\put(626.52,316.51){\usebox{\plotpoint}}
\put(646.76,321.13){\usebox{\plotpoint}}
\put(667.12,325.10){\usebox{\plotpoint}}
\put(687.34,329.77){\usebox{\plotpoint}}
\put(707.52,334.63){\usebox{\plotpoint}}
\put(727.92,338.37){\usebox{\plotpoint}}
\put(748.09,343.25){\usebox{\plotpoint}}
\put(768.32,347.92){\usebox{\plotpoint}}
\put(788.71,351.68){\usebox{\plotpoint}}
\put(808.90,356.51){\usebox{\plotpoint}}
\put(829.10,361.27){\usebox{\plotpoint}}
\put(849.48,365.11){\usebox{\plotpoint}}
\put(869.67,369.92){\usebox{\plotpoint}}
\put(889.87,374.66){\usebox{\plotpoint}}
\put(910.24,378.58){\usebox{\plotpoint}}
\put(930.45,383.26){\usebox{\plotpoint}}
\put(950.67,387.92){\usebox{\plotpoint}}
\put(970.96,392.22){\usebox{\plotpoint}}
\put(991.25,396.52){\usebox{\plotpoint}}
\put(1011.44,401.36){\usebox{\plotpoint}}
\put(1031.72,405.73){\usebox{\plotpoint}}
\put(1052.00,410.00){\usebox{\plotpoint}}
\put(1072.23,414.67){\usebox{\plotpoint}}
\put(1092.43,419.37){\usebox{\plotpoint}}
\put(1112.81,423.26){\usebox{\plotpoint}}
\put(1133.01,428.00){\usebox{\plotpoint}}
\put(1153.20,432.82){\usebox{\plotpoint}}
\put(1173.58,436.65){\usebox{\plotpoint}}
\put(1193.78,441.41){\usebox{\plotpoint}}
\put(1214.01,446.08){\usebox{\plotpoint}}
\put(1234.38,450.01){\usebox{\plotpoint}}
\put(1254.60,454.68){\usebox{\plotpoint}}
\put(1274.77,459.56){\usebox{\plotpoint}}
\put(1295.18,463.27){\usebox{\plotpoint}}
\put(1315.37,468.09){\usebox{\plotpoint}}
\put(1335.57,472.82){\usebox{\plotpoint}}
\put(1355.93,476.82){\usebox{\plotpoint}}
\put(1376.17,481.42){\usebox{\plotpoint}}
\put(1396.36,486.21){\usebox{\plotpoint}}
\put(1416.75,490.02){\usebox{\plotpoint}}
\put(1436.96,494.74){\usebox{\plotpoint}}
\put(1457.14,499.57){\usebox{\plotpoint}}
\put(1459,500){\usebox{\plotpoint}}
\put(201.0,123.0){\rule[-0.200pt]{0.400pt}{177.543pt}}
\put(201.0,123.0){\rule[-0.200pt]{303.052pt}{0.400pt}}
\put(1459.0,123.0){\rule[-0.200pt]{0.400pt}{177.543pt}}
\put(201.0,860.0){\rule[-0.200pt]{303.052pt}{0.400pt}}
\end{picture}

\end{figure}
 \begin {figure}[p]\caption{Slitta scarica, spessore sottile. Tratteggiata la retta interpolante.}\label{scaricos}
\centering
        % GNUPLOT: LaTeX picture
\setlength{\unitlength}{0.240900pt}
\ifx\plotpoint\undefined\newsavebox{\plotpoint}\fi
\begin{picture}(1500,900)(0,0)
\sbox{\plotpoint}{\rule[-0.200pt]{0.400pt}{0.400pt}}%
\put(181,123){\makebox(0,0)[r]{0.140}}
\put(201.0,123.0){\rule[-0.200pt]{4.818pt}{0.400pt}}
\put(181,270){\makebox(0,0)[r]{0.145}}
\put(201.0,270.0){\rule[-0.200pt]{4.818pt}{0.400pt}}
\put(181,418){\makebox(0,0)[r]{0.150}}
\put(201.0,418.0){\rule[-0.200pt]{4.818pt}{0.400pt}}
\put(181,565){\makebox(0,0)[r]{0.155}}
\put(201.0,565.0){\rule[-0.200pt]{4.818pt}{0.400pt}}
\put(181,713){\makebox(0,0)[r]{0.160}}
\put(201.0,713.0){\rule[-0.200pt]{4.818pt}{0.400pt}}
\put(181,860){\makebox(0,0)[r]{0.165}}
\put(201.0,860.0){\rule[-0.200pt]{4.818pt}{0.400pt}}
\put(280,82){\makebox(0,0){ 50}}
\put(280.0,123.0){\rule[-0.200pt]{0.400pt}{4.818pt}}
\put(437,82){\makebox(0,0){ 60}}
\put(437.0,123.0){\rule[-0.200pt]{0.400pt}{4.818pt}}
\put(594,82){\makebox(0,0){ 70}}
\put(594.0,123.0){\rule[-0.200pt]{0.400pt}{4.818pt}}
\put(751,82){\makebox(0,0){ 80}}
\put(751.0,123.0){\rule[-0.200pt]{0.400pt}{4.818pt}}
\put(909,82){\makebox(0,0){ 90}}
\put(909.0,123.0){\rule[-0.200pt]{0.400pt}{4.818pt}}
\put(1066,82){\makebox(0,0){ 100}}
\put(1066.0,123.0){\rule[-0.200pt]{0.400pt}{4.818pt}}
\put(1223,82){\makebox(0,0){ 110}}
\put(1223.0,123.0){\rule[-0.200pt]{0.400pt}{4.818pt}}
\put(1380,82){\makebox(0,0){ 120}}
\put(1380.0,123.0){\rule[-0.200pt]{0.400pt}{4.818pt}}
\put(201.0,123.0){\rule[-0.200pt]{0.400pt}{177.543pt}}
\put(201.0,123.0){\rule[-0.200pt]{303.052pt}{0.400pt}}
\put(1459.0,123.0){\rule[-0.200pt]{0.400pt}{177.543pt}}
\put(201.0,860.0){\rule[-0.200pt]{303.052pt}{0.400pt}}
\put(40,491){\makebox(0,0){\rotatebox{90}{velocità \ms}}}
\put(830,21){\makebox(0,0){posizione (cm)}}
\put(280.0,238.0){\rule[-0.200pt]{0.400pt}{9.877pt}}
\put(270.0,238.0){\rule[-0.200pt]{4.818pt}{0.400pt}}
\put(270.0,279.0){\rule[-0.200pt]{4.818pt}{0.400pt}}
\put(437.0,176.0){\rule[-0.200pt]{0.400pt}{24.090pt}}
\put(427.0,176.0){\rule[-0.200pt]{4.818pt}{0.400pt}}
\put(427.0,276.0){\rule[-0.200pt]{4.818pt}{0.400pt}}
\put(594.0,203.0){\rule[-0.200pt]{0.400pt}{31.076pt}}
\put(584.0,203.0){\rule[-0.200pt]{4.818pt}{0.400pt}}
\put(584.0,332.0){\rule[-0.200pt]{4.818pt}{0.400pt}}
\put(751.0,326.0){\rule[-0.200pt]{0.400pt}{45.530pt}}
\put(741.0,326.0){\rule[-0.200pt]{4.818pt}{0.400pt}}
\put(741.0,515.0){\rule[-0.200pt]{4.818pt}{0.400pt}}
\put(909.0,273.0){\rule[-0.200pt]{0.400pt}{68.175pt}}
\put(899.0,273.0){\rule[-0.200pt]{4.818pt}{0.400pt}}
\put(899.0,556.0){\rule[-0.200pt]{4.818pt}{0.400pt}}
\put(1066.0,362.0){\rule[-0.200pt]{0.400pt}{39.748pt}}
\put(1056.0,362.0){\rule[-0.200pt]{4.818pt}{0.400pt}}
\put(1056.0,527.0){\rule[-0.200pt]{4.818pt}{0.400pt}}
\put(1223.0,577.0){\rule[-0.200pt]{0.400pt}{52.516pt}}
\put(1213.0,577.0){\rule[-0.200pt]{4.818pt}{0.400pt}}
\put(1213.0,795.0){\rule[-0.200pt]{4.818pt}{0.400pt}}
\put(1380.0,586.0){\rule[-0.200pt]{0.400pt}{36.858pt}}
\put(1370.0,586.0){\rule[-0.200pt]{4.818pt}{0.400pt}}
\put(280,259){\circle*{12}}
\put(437,226){\circle*{12}}
\put(594,267){\circle*{12}}
\put(751,421){\circle*{12}}
\put(909,415){\circle*{12}}
\put(1066,444){\circle*{12}}
\put(1223,686){\circle*{12}}
\put(1380,662){\circle*{12}}
\put(1370.0,739.0){\rule[-0.200pt]{4.818pt}{0.400pt}}
\put(201,153){\usebox{\plotpoint}}
\put(201.00,153.00){\usebox{\plotpoint}}
\put(219.94,161.48){\usebox{\plotpoint}}
\put(238.89,169.95){\usebox{\plotpoint}}
\put(258.09,177.81){\usebox{\plotpoint}}
\put(277.13,186.06){\usebox{\plotpoint}}
\put(296.14,194.36){\usebox{\plotpoint}}
\put(315.37,202.17){\usebox{\plotpoint}}
\put(334.39,210.46){\usebox{\plotpoint}}
\put(353.24,219.09){\usebox{\plotpoint}}
\put(372.43,226.97){\usebox{\plotpoint}}
\put(391.62,234.85){\usebox{\plotpoint}}
\put(410.67,243.08){\usebox{\plotpoint}}
\put(429.86,250.95){\usebox{\plotpoint}}
\put(448.71,259.58){\usebox{\plotpoint}}
\put(467.73,267.88){\usebox{\plotpoint}}
\put(486.84,275.92){\usebox{\plotpoint}}
\put(505.94,283.98){\usebox{\plotpoint}}
\put(525.05,292.03){\usebox{\plotpoint}}
\put(544.16,300.07){\usebox{\plotpoint}}
\put(563.18,308.38){\usebox{\plotpoint}}
\put(582.03,317.01){\usebox{\plotpoint}}
\put(601.40,324.46){\usebox{\plotpoint}}
\put(620.25,333.10){\usebox{\plotpoint}}
\put(639.44,340.97){\usebox{\plotpoint}}
\put(658.48,349.22){\usebox{\plotpoint}}
\put(677.50,357.50){\usebox{\plotpoint}}
\put(696.52,365.78){\usebox{\plotpoint}}
\put(715.75,373.60){\usebox{\plotpoint}}
\put(734.77,381.89){\usebox{\plotpoint}}
\put(753.81,390.14){\usebox{\plotpoint}}
\put(773.01,398.00){\usebox{\plotpoint}}
\put(791.95,406.48){\usebox{\plotpoint}}
\put(811.25,414.11){\usebox{\plotpoint}}
\put(830.19,422.58){\usebox{\plotpoint}}
\put(849.14,431.05){\usebox{\plotpoint}}
\put(868.24,439.12){\usebox{\plotpoint}}
\put(887.36,447.14){\usebox{\plotpoint}}
\put(906.55,455.02){\usebox{\plotpoint}}
\put(925.59,463.27){\usebox{\plotpoint}}
\put(944.61,471.54){\usebox{\plotpoint}}
\put(963.46,480.18){\usebox{\plotpoint}}
\put(982.65,488.07){\usebox{\plotpoint}}
\put(1001.84,495.94){\usebox{\plotpoint}}
\put(1020.89,504.18){\usebox{\plotpoint}}
\put(1040.09,512.04){\usebox{\plotpoint}}
\put(1058.94,520.67){\usebox{\plotpoint}}
\put(1077.95,528.98){\usebox{\plotpoint}}
\put(1097.19,536.76){\usebox{\plotpoint}}
\put(1116.20,545.08){\usebox{\plotpoint}}
\put(1135.30,553.15){\usebox{\plotpoint}}
\put(1154.41,561.19){\usebox{\plotpoint}}
\put(1173.36,569.65){\usebox{\plotpoint}}
\put(1192.30,578.12){\usebox{\plotpoint}}
\put(1211.68,585.57){\usebox{\plotpoint}}
\put(1230.52,594.20){\usebox{\plotpoint}}
\put(1249.71,602.10){\usebox{\plotpoint}}
\put(1268.75,610.35){\usebox{\plotpoint}}
\put(1287.78,618.61){\usebox{\plotpoint}}
\put(1307.15,626.07){\usebox{\plotpoint}}
\put(1326.00,634.69){\usebox{\plotpoint}}
\put(1345.01,643.00){\usebox{\plotpoint}}
\put(1364.05,651.25){\usebox{\plotpoint}}
\put(1383.25,659.12){\usebox{\plotpoint}}
\put(1402.11,667.73){\usebox{\plotpoint}}
\put(1421.47,675.22){\usebox{\plotpoint}}
\put(1440.42,683.67){\usebox{\plotpoint}}
\put(1459,692){\usebox{\plotpoint}}
\put(201.0,123.0){\rule[-0.200pt]{0.400pt}{177.543pt}}
\put(201.0,123.0){\rule[-0.200pt]{303.052pt}{0.400pt}}
\put(1459.0,123.0){\rule[-0.200pt]{0.400pt}{177.543pt}}
\put(201.0,860.0){\rule[-0.200pt]{303.052pt}{0.400pt}}
\end{picture}

\end{figure}
 \begin {figure}[p]\caption{Slitta scarica, spessore grosso. Tratteggiata la retta interpolante.}\label{scaricog}
\centering
        % GNUPLOT: LaTeX picture
\setlength{\unitlength}{0.240900pt}
\ifx\plotpoint\undefined\newsavebox{\plotpoint}\fi
\begin{picture}(1500,900)(0,0)
\sbox{\plotpoint}{\rule[-0.200pt]{0.400pt}{0.400pt}}%
\put(181,123){\makebox(0,0)[r]{0.120}}
\put(201.0,123.0){\rule[-0.200pt]{4.818pt}{0.400pt}}
\put(181,246){\makebox(0,0)[r]{0.125}}
\put(201.0,246.0){\rule[-0.200pt]{4.818pt}{0.400pt}}
\put(181,369){\makebox(0,0)[r]{0.130}}
\put(201.0,369.0){\rule[-0.200pt]{4.818pt}{0.400pt}}
\put(181,492){\makebox(0,0)[r]{0.135}}
\put(201.0,492.0){\rule[-0.200pt]{4.818pt}{0.400pt}}
\put(181,614){\makebox(0,0)[r]{0.140}}
\put(201.0,614.0){\rule[-0.200pt]{4.818pt}{0.400pt}}
\put(181,737){\makebox(0,0)[r]{0.145}}
\put(201.0,737.0){\rule[-0.200pt]{4.818pt}{0.400pt}}
\put(181,860){\makebox(0,0)[r]{0.150}}
\put(201.0,860.0){\rule[-0.200pt]{4.818pt}{0.400pt}}
\put(280,82){\makebox(0,0){ 50}}
\put(280.0,123.0){\rule[-0.200pt]{0.400pt}{4.818pt}}
\put(437,82){\makebox(0,0){ 60}}
\put(437.0,123.0){\rule[-0.200pt]{0.400pt}{4.818pt}}
\put(594,82){\makebox(0,0){ 70}}
\put(594.0,123.0){\rule[-0.200pt]{0.400pt}{4.818pt}}
\put(751,82){\makebox(0,0){ 80}}
\put(751.0,123.0){\rule[-0.200pt]{0.400pt}{4.818pt}}
\put(909,82){\makebox(0,0){ 90}}
\put(909.0,123.0){\rule[-0.200pt]{0.400pt}{4.818pt}}
\put(1066,82){\makebox(0,0){ 100}}
\put(1066.0,123.0){\rule[-0.200pt]{0.400pt}{4.818pt}}
\put(1223,82){\makebox(0,0){ 110}}
\put(1223.0,123.0){\rule[-0.200pt]{0.400pt}{4.818pt}}
\put(1380,82){\makebox(0,0){ 120}}
\put(1380.0,123.0){\rule[-0.200pt]{0.400pt}{4.818pt}}
\put(201.0,123.0){\rule[-0.200pt]{0.400pt}{177.543pt}}
\put(201.0,123.0){\rule[-0.200pt]{303.052pt}{0.400pt}}
\put(1459.0,123.0){\rule[-0.200pt]{0.400pt}{177.543pt}}
\put(201.0,860.0){\rule[-0.200pt]{303.052pt}{0.400pt}}
\put(40,491){\makebox(0,0){\rotatebox{90}{velocità \ms}}}
\put(830,21){\makebox(0,0){posizione (cm)}}
\put(280.0,199.0){\rule[-0.200pt]{0.400pt}{8.431pt}}
\put(270.0,199.0){\rule[-0.200pt]{4.818pt}{0.400pt}}
\put(270.0,234.0){\rule[-0.200pt]{4.818pt}{0.400pt}}
\put(437.0,170.0){\rule[-0.200pt]{0.400pt}{11.804pt}}
\put(427.0,170.0){\rule[-0.200pt]{4.818pt}{0.400pt}}
\put(427.0,219.0){\rule[-0.200pt]{4.818pt}{0.400pt}}
\put(594.0,221.0){\rule[-0.200pt]{0.400pt}{47.457pt}}
\put(584.0,221.0){\rule[-0.200pt]{4.818pt}{0.400pt}}
\put(584.0,418.0){\rule[-0.200pt]{4.818pt}{0.400pt}}
\put(751.0,354.0){\rule[-0.200pt]{0.400pt}{24.813pt}}
\put(741.0,354.0){\rule[-0.200pt]{4.818pt}{0.400pt}}
\put(741.0,457.0){\rule[-0.200pt]{4.818pt}{0.400pt}}
\put(909.0,425.0){\rule[-0.200pt]{0.400pt}{11.804pt}}
\put(899.0,425.0){\rule[-0.200pt]{4.818pt}{0.400pt}}
\put(899.0,474.0){\rule[-0.200pt]{4.818pt}{0.400pt}}
\put(1066.0,410.0){\rule[-0.200pt]{0.400pt}{46.253pt}}
\put(1056.0,410.0){\rule[-0.200pt]{4.818pt}{0.400pt}}
\put(1056.0,602.0){\rule[-0.200pt]{4.818pt}{0.400pt}}
\put(1223.0,494.0){\rule[-0.200pt]{0.400pt}{47.216pt}}
\put(1213.0,494.0){\rule[-0.200pt]{4.818pt}{0.400pt}}
\put(1213.0,690.0){\rule[-0.200pt]{4.818pt}{0.400pt}}
\put(1380.0,550.0){\rule[-0.200pt]{0.400pt}{46.253pt}}
\put(1370.0,550.0){\rule[-0.200pt]{4.818pt}{0.400pt}}
\put(280,216){\circle*{12}}
\put(437,194){\circle*{12}}
\put(594,320){\circle*{12}}
\put(751,406){\circle*{12}}
\put(909,450){\circle*{12}}
\put(1066,506){\circle*{12}}
\put(1223,592){\circle*{12}}
\put(1380,646){\circle*{12}}
\put(1370.0,742.0){\rule[-0.200pt]{4.818pt}{0.400pt}}
\put(201,149){\usebox{\plotpoint}}
\put(201.00,149.00){\usebox{\plotpoint}}
\put(219.94,157.48){\usebox{\plotpoint}}
\put(238.89,165.95){\usebox{\plotpoint}}
\put(258.26,173.41){\usebox{\plotpoint}}
\put(277.11,182.04){\usebox{\plotpoint}}
\put(296.30,189.91){\usebox{\plotpoint}}
\put(315.35,198.14){\usebox{\plotpoint}}
\put(334.54,206.02){\usebox{\plotpoint}}
\put(353.58,214.27){\usebox{\plotpoint}}
\put(372.61,222.54){\usebox{\plotpoint}}
\put(391.98,229.99){\usebox{\plotpoint}}
\put(410.83,238.63){\usebox{\plotpoint}}
\put(430.19,246.10){\usebox{\plotpoint}}
\put(449.05,254.71){\usebox{\plotpoint}}
\put(468.06,263.02){\usebox{\plotpoint}}
\put(487.36,270.65){\usebox{\plotpoint}}
\put(506.31,279.12){\usebox{\plotpoint}}
\put(525.40,287.20){\usebox{\plotpoint}}
\put(544.53,295.20){\usebox{\plotpoint}}
\put(563.71,303.10){\usebox{\plotpoint}}
\put(582.76,311.35){\usebox{\plotpoint}}
\put(601.79,319.61){\usebox{\plotpoint}}
\put(621.00,327.46){\usebox{\plotpoint}}
\put(640.03,335.71){\usebox{\plotpoint}}
\put(659.24,343.57){\usebox{\plotpoint}}
\put(678.28,351.80){\usebox{\plotpoint}}
\put(697.29,360.12){\usebox{\plotpoint}}
\put(716.53,367.90){\usebox{\plotpoint}}
\put(735.53,376.22){\usebox{\plotpoint}}
\put(754.56,384.49){\usebox{\plotpoint}}
\put(773.78,392.30){\usebox{\plotpoint}}
\put(792.86,400.43){\usebox{\plotpoint}}
\put(811.98,408.45){\usebox{\plotpoint}}
\put(830.94,416.89){\usebox{\plotpoint}}
\put(850.22,424.56){\usebox{\plotpoint}}
\put(869.18,432.99){\usebox{\plotpoint}}
\put(888.14,441.44){\usebox{\plotpoint}}
\put(907.51,448.89){\usebox{\plotpoint}}
\put(926.36,457.52){\usebox{\plotpoint}}
\put(945.73,464.97){\usebox{\plotpoint}}
\put(964.58,473.61){\usebox{\plotpoint}}
\put(983.74,481.57){\usebox{\plotpoint}}
\put(1002.95,489.40){\usebox{\plotpoint}}
\put(1021.98,497.68){\usebox{\plotpoint}}
\put(1041.16,505.58){\usebox{\plotpoint}}
\put(1060.06,514.10){\usebox{\plotpoint}}
\put(1079.37,521.68){\usebox{\plotpoint}}
\put(1098.28,530.18){\usebox{\plotpoint}}
\put(1117.29,538.50){\usebox{\plotpoint}}
\put(1136.58,546.16){\usebox{\plotpoint}}
\put(1155.54,554.59){\usebox{\plotpoint}}
\put(1174.58,562.79){\usebox{\plotpoint}}
\put(1193.76,570.68){\usebox{\plotpoint}}
\put(1212.91,578.65){\usebox{\plotpoint}}
\put(1232.00,586.77){\usebox{\plotpoint}}
\put(1251.15,594.76){\usebox{\plotpoint}}
\put(1270.19,603.01){\usebox{\plotpoint}}
\put(1289.26,611.18){\usebox{\plotpoint}}
\put(1308.56,618.78){\usebox{\plotpoint}}
\put(1327.48,627.26){\usebox{\plotpoint}}
\put(1346.47,635.61){\usebox{\plotpoint}}
\put(1365.73,643.36){\usebox{\plotpoint}}
\put(1384.72,651.72){\usebox{\plotpoint}}
\put(1403.73,660.03){\usebox{\plotpoint}}
\put(1422.98,667.76){\usebox{\plotpoint}}
\put(1442.01,676.00){\usebox{\plotpoint}}
\put(1459,683){\usebox{\plotpoint}}
\put(201.0,123.0){\rule[-0.200pt]{0.400pt}{177.543pt}}
\put(201.0,123.0){\rule[-0.200pt]{303.052pt}{0.400pt}}
\put(1459.0,123.0){\rule[-0.200pt]{0.400pt}{177.543pt}}
\put(201.0,860.0){\rule[-0.200pt]{303.052pt}{0.400pt}}
\end{picture}

\end{figure}
 \begin {figure}[p]\caption{Slitta scarica, entrambi gli spessori. Tratteggiata la retta interpolante.}\label{scaricogg}
\centering
        % GNUPLOT: LaTeX picture
\setlength{\unitlength}{0.240900pt}
\ifx\plotpoint\undefined\newsavebox{\plotpoint}\fi
\begin{picture}(1500,900)(0,0)
\sbox{\plotpoint}{\rule[-0.200pt]{0.400pt}{0.400pt}}%
\put(181,123){\makebox(0,0)[r]{0.110}}
\put(201.0,123.0){\rule[-0.200pt]{4.818pt}{0.400pt}}
\put(181,246){\makebox(0,0)[r]{0.115}}
\put(201.0,246.0){\rule[-0.200pt]{4.818pt}{0.400pt}}
\put(181,369){\makebox(0,0)[r]{0.120}}
\put(201.0,369.0){\rule[-0.200pt]{4.818pt}{0.400pt}}
\put(181,491){\makebox(0,0)[r]{0.125}}
\put(201.0,491.0){\rule[-0.200pt]{4.818pt}{0.400pt}}
\put(181,614){\makebox(0,0)[r]{0.130}}
\put(201.0,614.0){\rule[-0.200pt]{4.818pt}{0.400pt}}
\put(181,737){\makebox(0,0)[r]{0.135}}
\put(201.0,737.0){\rule[-0.200pt]{4.818pt}{0.400pt}}
\put(181,860){\makebox(0,0)[r]{0.140}}
\put(201.0,860.0){\rule[-0.200pt]{4.818pt}{0.400pt}}
\put(280,82){\makebox(0,0){ 50}}
\put(280.0,123.0){\rule[-0.200pt]{0.400pt}{4.818pt}}
\put(437,82){\makebox(0,0){ 60}}
\put(437.0,123.0){\rule[-0.200pt]{0.400pt}{4.818pt}}
\put(594,82){\makebox(0,0){ 70}}
\put(594.0,123.0){\rule[-0.200pt]{0.400pt}{4.818pt}}
\put(751,82){\makebox(0,0){ 80}}
\put(751.0,123.0){\rule[-0.200pt]{0.400pt}{4.818pt}}
\put(909,82){\makebox(0,0){ 90}}
\put(909.0,123.0){\rule[-0.200pt]{0.400pt}{4.818pt}}
\put(1066,82){\makebox(0,0){ 100}}
\put(1066.0,123.0){\rule[-0.200pt]{0.400pt}{4.818pt}}
\put(1223,82){\makebox(0,0){ 110}}
\put(1223.0,123.0){\rule[-0.200pt]{0.400pt}{4.818pt}}
\put(1380,82){\makebox(0,0){ 120}}
\put(1380.0,123.0){\rule[-0.200pt]{0.400pt}{4.818pt}}
\put(201.0,123.0){\rule[-0.200pt]{0.400pt}{177.543pt}}
\put(201.0,123.0){\rule[-0.200pt]{303.052pt}{0.400pt}}
\put(1459.0,123.0){\rule[-0.200pt]{0.400pt}{177.543pt}}
\put(201.0,860.0){\rule[-0.200pt]{303.052pt}{0.400pt}}
\put(40,491){\makebox(0,0){\rotatebox{90}{velocità \ms}}}
\put(830,21){\makebox(0,0){posizione (cm)}}
\put(280.0,197.0){\rule[-0.200pt]{0.400pt}{17.586pt}}
\put(270.0,197.0){\rule[-0.200pt]{4.818pt}{0.400pt}}
\put(270.0,270.0){\rule[-0.200pt]{4.818pt}{0.400pt}}
\put(437.0,207.0){\rule[-0.200pt]{0.400pt}{26.017pt}}
\put(427.0,207.0){\rule[-0.200pt]{4.818pt}{0.400pt}}
\put(427.0,315.0){\rule[-0.200pt]{4.818pt}{0.400pt}}
\put(594.0,351.0){\rule[-0.200pt]{0.400pt}{27.222pt}}
\put(584.0,351.0){\rule[-0.200pt]{4.818pt}{0.400pt}}
\put(584.0,464.0){\rule[-0.200pt]{4.818pt}{0.400pt}}
\put(751.0,315.0){\rule[-0.200pt]{0.400pt}{49.625pt}}
\put(741.0,315.0){\rule[-0.200pt]{4.818pt}{0.400pt}}
\put(741.0,521.0){\rule[-0.200pt]{4.818pt}{0.400pt}}
\put(909.0,413.0){\rule[-0.200pt]{0.400pt}{23.608pt}}
\put(899.0,413.0){\rule[-0.200pt]{4.818pt}{0.400pt}}
\put(899.0,511.0){\rule[-0.200pt]{4.818pt}{0.400pt}}
\put(1066.0,612.0){\rule[-0.200pt]{0.400pt}{23.608pt}}
\put(1056.0,612.0){\rule[-0.200pt]{4.818pt}{0.400pt}}
\put(1056.0,710.0){\rule[-0.200pt]{4.818pt}{0.400pt}}
\put(1223.0,681.0){\rule[-0.200pt]{0.400pt}{30.594pt}}
\put(1213.0,681.0){\rule[-0.200pt]{4.818pt}{0.400pt}}
\put(1213.0,808.0){\rule[-0.200pt]{4.818pt}{0.400pt}}
\put(1380.0,526.0){\rule[-0.200pt]{0.400pt}{55.648pt}}
\put(1370.0,526.0){\rule[-0.200pt]{4.818pt}{0.400pt}}
\put(280,234){\circle*{12}}
\put(437,261){\circle*{12}}
\put(594,408){\circle*{12}}
\put(751,418){\circle*{12}}
\put(909,462){\circle*{12}}
\put(1066,661){\circle*{12}}
\put(1223,745){\circle*{12}}
\put(1380,641){\circle*{12}}
\put(1370.0,757.0){\rule[-0.200pt]{4.818pt}{0.400pt}}
\put(201,221){\usebox{\plotpoint}}
\put(201.00,221.00){\usebox{\plotpoint}}
\put(220.30,228.63){\usebox{\plotpoint}}
\put(239.25,237.10){\usebox{\plotpoint}}
\put(258.44,244.97){\usebox{\plotpoint}}
\put(277.48,253.22){\usebox{\plotpoint}}
\put(296.50,261.50){\usebox{\plotpoint}}
\put(315.72,269.33){\usebox{\plotpoint}}
\put(334.75,277.60){\usebox{\plotpoint}}
\put(353.60,286.23){\usebox{\plotpoint}}
\put(372.97,293.68){\usebox{\plotpoint}}
\put(391.98,301.99){\usebox{\plotpoint}}
\put(411.02,310.24){\usebox{\plotpoint}}
\put(430.22,318.11){\usebox{\plotpoint}}
\put(449.08,326.72){\usebox{\plotpoint}}
\put(468.44,334.20){\usebox{\plotpoint}}
\put(487.39,342.66){\usebox{\plotpoint}}
\put(506.33,351.13){\usebox{\plotpoint}}
\put(525.63,358.76){\usebox{\plotpoint}}
\put(544.58,367.22){\usebox{\plotpoint}}
\put(563.76,375.12){\usebox{\plotpoint}}
\put(582.81,383.37){\usebox{\plotpoint}}
\put(601.84,391.63){\usebox{\plotpoint}}
\put(621.05,399.48){\usebox{\plotpoint}}
\put(640.08,407.72){\usebox{\plotpoint}}
\put(658.93,416.36){\usebox{\plotpoint}}
\put(678.31,423.81){\usebox{\plotpoint}}
\put(697.31,432.13){\usebox{\plotpoint}}
\put(716.35,440.39){\usebox{\plotpoint}}
\put(735.54,448.27){\usebox{\plotpoint}}
\put(754.41,456.85){\usebox{\plotpoint}}
\put(773.76,464.35){\usebox{\plotpoint}}
\put(792.72,472.80){\usebox{\plotpoint}}
\put(811.67,481.26){\usebox{\plotpoint}}
\put(830.96,488.90){\usebox{\plotpoint}}
\put(849.91,497.35){\usebox{\plotpoint}}
\put(868.98,505.49){\usebox{\plotpoint}}
\put(888.10,513.51){\usebox{\plotpoint}}
\put(907.14,521.75){\usebox{\plotpoint}}
\put(926.35,529.62){\usebox{\plotpoint}}
\put(945.39,537.84){\usebox{\plotpoint}}
\put(964.24,546.48){\usebox{\plotpoint}}
\put(983.61,553.93){\usebox{\plotpoint}}
\put(1002.62,562.26){\usebox{\plotpoint}}
\put(1021.65,570.53){\usebox{\plotpoint}}
\put(1040.83,578.42){\usebox{\plotpoint}}
\put(1059.72,586.97){\usebox{\plotpoint}}
\put(1079.05,594.52){\usebox{\plotpoint}}
\put(1097.94,603.05){\usebox{\plotpoint}}
\put(1116.95,611.37){\usebox{\plotpoint}}
\put(1136.24,619.02){\usebox{\plotpoint}}
\put(1155.20,627.46){\usebox{\plotpoint}}
\put(1174.25,635.63){\usebox{\plotpoint}}
\put(1193.38,643.64){\usebox{\plotpoint}}
\put(1212.43,651.86){\usebox{\plotpoint}}
\put(1231.62,659.75){\usebox{\plotpoint}}
\put(1250.67,667.95){\usebox{\plotpoint}}
\put(1269.52,676.59){\usebox{\plotpoint}}
\put(1288.90,684.04){\usebox{\plotpoint}}
\put(1307.89,692.37){\usebox{\plotpoint}}
\put(1326.92,700.66){\usebox{\plotpoint}}
\put(1346.10,708.55){\usebox{\plotpoint}}
\put(1365.00,717.08){\usebox{\plotpoint}}
\put(1384.32,724.66){\usebox{\plotpoint}}
\put(1403.22,733.16){\usebox{\plotpoint}}
\put(1422.23,741.47){\usebox{\plotpoint}}
\put(1441.52,749.13){\usebox{\plotpoint}}
\put(1459,757){\usebox{\plotpoint}}
\put(201.0,123.0){\rule[-0.200pt]{0.400pt}{177.543pt}}
\put(201.0,123.0){\rule[-0.200pt]{303.052pt}{0.400pt}}
\put(1459.0,123.0){\rule[-0.200pt]{0.400pt}{177.543pt}}
\put(201.0,860.0){\rule[-0.200pt]{303.052pt}{0.400pt}}
\end{picture}

\end{figure}
 \begin {figure}[p]\caption{Slitta carica, nessuno spessore. Tratteggiata la retta interpolante.}\label{carico0}
\centering
        % GNUPLOT: LaTeX picture
\setlength{\unitlength}{0.240900pt}
\ifx\plotpoint\undefined\newsavebox{\plotpoint}\fi
\begin{picture}(1500,900)(0,0)
\sbox{\plotpoint}{\rule[-0.200pt]{0.400pt}{0.400pt}}%
\put(181,123){\makebox(0,0)[r]{0.115}}
\put(201.0,123.0){\rule[-0.200pt]{4.818pt}{0.400pt}}
\put(181,307){\makebox(0,0)[r]{0.120}}
\put(201.0,307.0){\rule[-0.200pt]{4.818pt}{0.400pt}}
\put(181,491){\makebox(0,0)[r]{0.125}}
\put(201.0,491.0){\rule[-0.200pt]{4.818pt}{0.400pt}}
\put(181,676){\makebox(0,0)[r]{0.130}}
\put(201.0,676.0){\rule[-0.200pt]{4.818pt}{0.400pt}}
\put(181,860){\makebox(0,0)[r]{0.135}}
\put(201.0,860.0){\rule[-0.200pt]{4.818pt}{0.400pt}}
\put(280,82){\makebox(0,0){ 50}}
\put(280.0,123.0){\rule[-0.200pt]{0.400pt}{4.818pt}}
\put(437,82){\makebox(0,0){ 60}}
\put(437.0,123.0){\rule[-0.200pt]{0.400pt}{4.818pt}}
\put(594,82){\makebox(0,0){ 70}}
\put(594.0,123.0){\rule[-0.200pt]{0.400pt}{4.818pt}}
\put(751,82){\makebox(0,0){ 80}}
\put(751.0,123.0){\rule[-0.200pt]{0.400pt}{4.818pt}}
\put(909,82){\makebox(0,0){ 90}}
\put(909.0,123.0){\rule[-0.200pt]{0.400pt}{4.818pt}}
\put(1066,82){\makebox(0,0){ 100}}
\put(1066.0,123.0){\rule[-0.200pt]{0.400pt}{4.818pt}}
\put(1223,82){\makebox(0,0){ 110}}
\put(1223.0,123.0){\rule[-0.200pt]{0.400pt}{4.818pt}}
\put(1380,82){\makebox(0,0){ 120}}
\put(1380.0,123.0){\rule[-0.200pt]{0.400pt}{4.818pt}}
\put(201.0,123.0){\rule[-0.200pt]{0.400pt}{177.543pt}}
\put(201.0,123.0){\rule[-0.200pt]{303.052pt}{0.400pt}}
\put(1459.0,123.0){\rule[-0.200pt]{0.400pt}{177.543pt}}
\put(201.0,860.0){\rule[-0.200pt]{303.052pt}{0.400pt}}
\put(40,491){\makebox(0,0){\rotatebox{90}{velocità \ms}}}
\put(830,21){\makebox(0,0){posizione (cm)}}
\put(280.0,241.0){\rule[-0.200pt]{0.400pt}{26.499pt}}
\put(270.0,241.0){\rule[-0.200pt]{4.818pt}{0.400pt}}
\put(270.0,351.0){\rule[-0.200pt]{4.818pt}{0.400pt}}
\put(437.0,230.0){\rule[-0.200pt]{0.400pt}{28.426pt}}
\put(427.0,230.0){\rule[-0.200pt]{4.818pt}{0.400pt}}
\put(427.0,348.0){\rule[-0.200pt]{4.818pt}{0.400pt}}
\put(594.0,363.0){\rule[-0.200pt]{0.400pt}{22.885pt}}
\put(584.0,363.0){\rule[-0.200pt]{4.818pt}{0.400pt}}
\put(584.0,458.0){\rule[-0.200pt]{4.818pt}{0.400pt}}
\put(751.0,355.0){\rule[-0.200pt]{0.400pt}{19.513pt}}
\put(741.0,355.0){\rule[-0.200pt]{4.818pt}{0.400pt}}
\put(741.0,436.0){\rule[-0.200pt]{4.818pt}{0.400pt}}
\put(909.0,440.0){\rule[-0.200pt]{0.400pt}{54.925pt}}
\put(899.0,440.0){\rule[-0.200pt]{4.818pt}{0.400pt}}
\put(899.0,668.0){\rule[-0.200pt]{4.818pt}{0.400pt}}
\put(1066.0,536.0){\rule[-0.200pt]{0.400pt}{33.726pt}}
\put(1056.0,536.0){\rule[-0.200pt]{4.818pt}{0.400pt}}
\put(1056.0,676.0){\rule[-0.200pt]{4.818pt}{0.400pt}}
\put(1223.0,558.0){\rule[-0.200pt]{0.400pt}{46.012pt}}
\put(1213.0,558.0){\rule[-0.200pt]{4.818pt}{0.400pt}}
\put(1213.0,749.0){\rule[-0.200pt]{4.818pt}{0.400pt}}
\put(1380.0,617.0){\rule[-0.200pt]{0.400pt}{24.813pt}}
\put(1370.0,617.0){\rule[-0.200pt]{4.818pt}{0.400pt}}
\put(280,296){\circle*{12}}
\put(437,289){\circle*{12}}
\put(594,410){\circle*{12}}
\put(751,396){\circle*{12}}
\put(909,554){\circle*{12}}
\put(1066,606){\circle*{12}}
\put(1223,654){\circle*{12}}
\put(1380,668){\circle*{12}}
\put(1370.0,720.0){\rule[-0.200pt]{4.818pt}{0.400pt}}
\put(201,238){\usebox{\plotpoint}}
\put(201.00,238.00){\usebox{\plotpoint}}
\put(220.30,245.63){\usebox{\plotpoint}}
\put(239.61,253.23){\usebox{\plotpoint}}
\put(258.98,260.69){\usebox{\plotpoint}}
\put(278.22,268.47){\usebox{\plotpoint}}
\put(297.59,275.92){\usebox{\plotpoint}}
\put(316.83,283.70){\usebox{\plotpoint}}
\put(336.20,291.16){\usebox{\plotpoint}}
\put(355.77,298.07){\usebox{\plotpoint}}
\put(375.14,305.52){\usebox{\plotpoint}}
\put(394.49,313.04){\usebox{\plotpoint}}
\put(413.75,320.75){\usebox{\plotpoint}}
\put(433.09,328.29){\usebox{\plotpoint}}
\put(452.36,335.99){\usebox{\plotpoint}}
\put(471.74,343.44){\usebox{\plotpoint}}
\put(491.00,351.17){\usebox{\plotpoint}}
\put(510.35,358.67){\usebox{\plotpoint}}
\put(529.60,366.42){\usebox{\plotpoint}}
\put(548.96,373.91){\usebox{\plotpoint}}
\put(568.33,381.36){\usebox{\plotpoint}}
\put(587.57,389.14){\usebox{\plotpoint}}
\put(606.94,396.59){\usebox{\plotpoint}}
\put(626.18,404.38){\usebox{\plotpoint}}
\put(645.55,411.83){\usebox{\plotpoint}}
\put(664.79,419.61){\usebox{\plotpoint}}
\put(684.16,427.06){\usebox{\plotpoint}}
\put(703.46,434.69){\usebox{\plotpoint}}
\put(722.77,442.30){\usebox{\plotpoint}}
\put(742.06,449.94){\usebox{\plotpoint}}
\put(761.38,457.53){\usebox{\plotpoint}}
\put(780.75,464.98){\usebox{\plotpoint}}
\put(799.99,472.77){\usebox{\plotpoint}}
\put(819.36,480.22){\usebox{\plotpoint}}
\put(838.60,488.00){\usebox{\plotpoint}}
\put(857.98,495.45){\usebox{\plotpoint}}
\put(877.21,503.24){\usebox{\plotpoint}}
\put(896.59,510.69){\usebox{\plotpoint}}
\put(916.23,517.34){\usebox{\plotpoint}}
\put(935.50,525.04){\usebox{\plotpoint}}
\put(954.83,532.60){\usebox{\plotpoint}}
\put(974.11,540.27){\usebox{\plotpoint}}
\put(993.48,547.72){\usebox{\plotpoint}}
\put(1012.74,555.47){\usebox{\plotpoint}}
\put(1032.09,562.96){\usebox{\plotpoint}}
\put(1051.34,570.73){\usebox{\plotpoint}}
\put(1070.71,578.19){\usebox{\plotpoint}}
\put(1089.94,585.98){\usebox{\plotpoint}}
\put(1109.32,593.43){\usebox{\plotpoint}}
\put(1128.69,600.88){\usebox{\plotpoint}}
\put(1147.93,608.66){\usebox{\plotpoint}}
\put(1167.30,616.12){\usebox{\plotpoint}}
\put(1186.54,623.90){\usebox{\plotpoint}}
\put(1205.91,631.35){\usebox{\plotpoint}}
\put(1225.20,639.00){\usebox{\plotpoint}}
\put(1244.52,646.58){\usebox{\plotpoint}}
\put(1263.81,654.25){\usebox{\plotpoint}}
\put(1283.13,661.82){\usebox{\plotpoint}}
\put(1302.50,669.27){\usebox{\plotpoint}}
\put(1321.74,677.05){\usebox{\plotpoint}}
\put(1341.11,684.51){\usebox{\plotpoint}}
\put(1360.35,692.29){\usebox{\plotpoint}}
\put(1379.72,699.74){\usebox{\plotpoint}}
\put(1398.96,707.52){\usebox{\plotpoint}}
\put(1418.33,714.97){\usebox{\plotpoint}}
\put(1437.67,722.53){\usebox{\plotpoint}}
\put(1457.21,729.45){\usebox{\plotpoint}}
\put(1459,730){\usebox{\plotpoint}}
\put(201.0,123.0){\rule[-0.200pt]{0.400pt}{177.543pt}}
\put(201.0,123.0){\rule[-0.200pt]{303.052pt}{0.400pt}}
\put(1459.0,123.0){\rule[-0.200pt]{0.400pt}{177.543pt}}
\put(201.0,860.0){\rule[-0.200pt]{303.052pt}{0.400pt}}
\end{picture}

\end{figure}
 \begin {figure}[p]\caption{Slitta carica, spessore sottile. Tratteggiata la retta interpolante.}\label{caricos}
\centering
        % GNUPLOT: LaTeX picture
\setlength{\unitlength}{0.240900pt}
\ifx\plotpoint\undefined\newsavebox{\plotpoint}\fi
\begin{picture}(1500,900)(0,0)
\sbox{\plotpoint}{\rule[-0.200pt]{0.400pt}{0.400pt}}%
\put(181,123){\makebox(0,0)[r]{0.090}}
\put(201.0,123.0){\rule[-0.200pt]{4.818pt}{0.400pt}}
\put(181,246){\makebox(0,0)[r]{0.095}}
\put(201.0,246.0){\rule[-0.200pt]{4.818pt}{0.400pt}}
\put(181,369){\makebox(0,0)[r]{0.100}}
\put(201.0,369.0){\rule[-0.200pt]{4.818pt}{0.400pt}}
\put(181,492){\makebox(0,0)[r]{0.105}}
\put(201.0,492.0){\rule[-0.200pt]{4.818pt}{0.400pt}}
\put(181,614){\makebox(0,0)[r]{0.110}}
\put(201.0,614.0){\rule[-0.200pt]{4.818pt}{0.400pt}}
\put(181,737){\makebox(0,0)[r]{0.115}}
\put(201.0,737.0){\rule[-0.200pt]{4.818pt}{0.400pt}}
\put(181,860){\makebox(0,0)[r]{0.120}}
\put(201.0,860.0){\rule[-0.200pt]{4.818pt}{0.400pt}}
\put(280,82){\makebox(0,0){ 50}}
\put(280.0,123.0){\rule[-0.200pt]{0.400pt}{4.818pt}}
\put(437,82){\makebox(0,0){ 60}}
\put(437.0,123.0){\rule[-0.200pt]{0.400pt}{4.818pt}}
\put(594,82){\makebox(0,0){ 70}}
\put(594.0,123.0){\rule[-0.200pt]{0.400pt}{4.818pt}}
\put(751,82){\makebox(0,0){ 80}}
\put(751.0,123.0){\rule[-0.200pt]{0.400pt}{4.818pt}}
\put(909,82){\makebox(0,0){ 90}}
\put(909.0,123.0){\rule[-0.200pt]{0.400pt}{4.818pt}}
\put(1066,82){\makebox(0,0){ 100}}
\put(1066.0,123.0){\rule[-0.200pt]{0.400pt}{4.818pt}}
\put(1223,82){\makebox(0,0){ 110}}
\put(1223.0,123.0){\rule[-0.200pt]{0.400pt}{4.818pt}}
\put(1380,82){\makebox(0,0){ 120}}
\put(1380.0,123.0){\rule[-0.200pt]{0.400pt}{4.818pt}}
\put(201.0,123.0){\rule[-0.200pt]{0.400pt}{177.543pt}}
\put(201.0,123.0){\rule[-0.200pt]{303.052pt}{0.400pt}}
\put(1459.0,123.0){\rule[-0.200pt]{0.400pt}{177.543pt}}
\put(201.0,860.0){\rule[-0.200pt]{303.052pt}{0.400pt}}
\put(40,491){\makebox(0,0){\rotatebox{90}{velocità \ms}}}
\put(830,21){\makebox(0,0){posizione (cm)}}
\put(280.0,317.0){\rule[-0.200pt]{0.400pt}{9.395pt}}
\put(270.0,317.0){\rule[-0.200pt]{4.818pt}{0.400pt}}
\put(270.0,356.0){\rule[-0.200pt]{4.818pt}{0.400pt}}
\put(437.0,278.0){\rule[-0.200pt]{0.400pt}{13.009pt}}
\put(427.0,278.0){\rule[-0.200pt]{4.818pt}{0.400pt}}
\put(427.0,332.0){\rule[-0.200pt]{4.818pt}{0.400pt}}
\put(594.0,361.0){\rule[-0.200pt]{0.400pt}{26.017pt}}
\put(584.0,361.0){\rule[-0.200pt]{4.818pt}{0.400pt}}
\put(584.0,469.0){\rule[-0.200pt]{4.818pt}{0.400pt}}
\put(751.0,403.0){\rule[-0.200pt]{0.400pt}{45.048pt}}
\put(741.0,403.0){\rule[-0.200pt]{4.818pt}{0.400pt}}
\put(741.0,590.0){\rule[-0.200pt]{4.818pt}{0.400pt}}
\put(909.0,442.0){\rule[-0.200pt]{0.400pt}{33.244pt}}
\put(899.0,442.0){\rule[-0.200pt]{4.818pt}{0.400pt}}
\put(899.0,580.0){\rule[-0.200pt]{4.818pt}{0.400pt}}
\put(1066.0,538.0){\rule[-0.200pt]{0.400pt}{22.645pt}}
\put(1056.0,538.0){\rule[-0.200pt]{4.818pt}{0.400pt}}
\put(1056.0,632.0){\rule[-0.200pt]{4.818pt}{0.400pt}}
\put(1223.0,558.0){\rule[-0.200pt]{0.400pt}{41.435pt}}
\put(1213.0,558.0){\rule[-0.200pt]{4.818pt}{0.400pt}}
\put(1213.0,730.0){\rule[-0.200pt]{4.818pt}{0.400pt}}
\put(1380.0,688.0){\rule[-0.200pt]{0.400pt}{22.404pt}}
\put(1370.0,688.0){\rule[-0.200pt]{4.818pt}{0.400pt}}
\put(280,337){\circle*{12}}
\put(437,305){\circle*{12}}
\put(594,415){\circle*{12}}
\put(751,496){\circle*{12}}
\put(909,511){\circle*{12}}
\put(1066,585){\circle*{12}}
\put(1223,644){\circle*{12}}
\put(1380,735){\circle*{12}}
\put(1370.0,781.0){\rule[-0.200pt]{4.818pt}{0.400pt}}
\put(201,265){\usebox{\plotpoint}}
\put(201.00,265.00){\usebox{\plotpoint}}
\put(220.30,272.63){\usebox{\plotpoint}}
\put(239.63,280.19){\usebox{\plotpoint}}
\put(259.29,286.80){\usebox{\plotpoint}}
\put(278.53,294.59){\usebox{\plotpoint}}
\put(297.90,302.04){\usebox{\plotpoint}}
\put(317.19,309.67){\usebox{\plotpoint}}
\put(336.81,316.39){\usebox{\plotpoint}}
\put(356.05,324.17){\usebox{\plotpoint}}
\put(375.42,331.62){\usebox{\plotpoint}}
\put(395.07,338.28){\usebox{\plotpoint}}
\put(414.34,345.98){\usebox{\plotpoint}}
\put(433.67,353.53){\usebox{\plotpoint}}
\put(453.21,360.45){\usebox{\plotpoint}}
\put(472.63,367.78){\usebox{\plotpoint}}
\put(491.88,375.53){\usebox{\plotpoint}}
\put(511.24,383.01){\usebox{\plotpoint}}
\put(530.80,389.93){\usebox{\plotpoint}}
\put(550.18,397.38){\usebox{\plotpoint}}
\put(569.55,404.83){\usebox{\plotpoint}}
\put(588.95,412.14){\usebox{\plotpoint}}
\put(608.46,419.19){\usebox{\plotpoint}}
\put(627.70,426.96){\usebox{\plotpoint}}
\put(647.09,434.36){\usebox{\plotpoint}}
\put(666.64,441.32){\usebox{\plotpoint}}
\put(686.01,448.77){\usebox{\plotpoint}}
\put(705.29,456.46){\usebox{\plotpoint}}
\put(724.69,463.83){\usebox{\plotpoint}}
\put(744.20,470.83){\usebox{\plotpoint}}
\put(763.54,478.36){\usebox{\plotpoint}}
\put(782.91,485.81){\usebox{\plotpoint}}
\put(802.48,492.72){\usebox{\plotpoint}}
\put(821.85,500.17){\usebox{\plotpoint}}
\put(841.09,507.96){\usebox{\plotpoint}}
\put(860.46,515.41){\usebox{\plotpoint}}
\put(880.03,522.32){\usebox{\plotpoint}}
\put(899.40,529.77){\usebox{\plotpoint}}
\put(918.71,537.38){\usebox{\plotpoint}}
\put(938.31,544.12){\usebox{\plotpoint}}
\put(957.61,551.76){\usebox{\plotpoint}}
\put(976.93,559.36){\usebox{\plotpoint}}
\put(996.47,566.30){\usebox{\plotpoint}}
\put(1015.84,573.71){\usebox{\plotpoint}}
\put(1035.21,581.16){\usebox{\plotpoint}}
\put(1054.45,588.94){\usebox{\plotpoint}}
\put(1074.04,595.78){\usebox{\plotpoint}}
\put(1093.37,603.30){\usebox{\plotpoint}}
\put(1112.74,610.75){\usebox{\plotpoint}}
\put(1132.16,618.05){\usebox{\plotpoint}}
\put(1151.68,625.11){\usebox{\plotpoint}}
\put(1171.00,632.67){\usebox{\plotpoint}}
\put(1190.29,640.34){\usebox{\plotpoint}}
\put(1209.97,646.91){\usebox{\plotpoint}}
\put(1229.21,654.67){\usebox{\plotpoint}}
\put(1248.58,662.14){\usebox{\plotpoint}}
\put(1267.82,669.92){\usebox{\plotpoint}}
\put(1287.49,676.50){\usebox{\plotpoint}}
\put(1306.86,683.95){\usebox{\plotpoint}}
\put(1326.10,691.73){\usebox{\plotpoint}}
\put(1345.77,698.32){\usebox{\plotpoint}}
\put(1365.02,706.08){\usebox{\plotpoint}}
\put(1384.37,713.57){\usebox{\plotpoint}}
\put(1403.63,721.32){\usebox{\plotpoint}}
\put(1423.31,727.89){\usebox{\plotpoint}}
\put(1442.58,735.58){\usebox{\plotpoint}}
\put(1459,742){\usebox{\plotpoint}}
\put(201.0,123.0){\rule[-0.200pt]{0.400pt}{177.543pt}}
\put(201.0,123.0){\rule[-0.200pt]{303.052pt}{0.400pt}}
\put(1459.0,123.0){\rule[-0.200pt]{0.400pt}{177.543pt}}
\put(201.0,860.0){\rule[-0.200pt]{303.052pt}{0.400pt}}
\end{picture}

\end{figure}
 \begin {figure}[p]\caption{Slitta carica, spessore grosso. Tratteggiata la retta interpolante.}\label{caricog}
\centering
        % GNUPLOT: LaTeX picture
\setlength{\unitlength}{0.240900pt}
\ifx\plotpoint\undefined\newsavebox{\plotpoint}\fi
\begin{picture}(1500,900)(0,0)
\sbox{\plotpoint}{\rule[-0.200pt]{0.400pt}{0.400pt}}%
\put(181,123){\makebox(0,0)[r]{0.075}}
\put(201.0,123.0){\rule[-0.200pt]{4.818pt}{0.400pt}}
\put(181,215){\makebox(0,0)[r]{0.080}}
\put(201.0,215.0){\rule[-0.200pt]{4.818pt}{0.400pt}}
\put(181,307){\makebox(0,0)[r]{0.085}}
\put(201.0,307.0){\rule[-0.200pt]{4.818pt}{0.400pt}}
\put(181,399){\makebox(0,0)[r]{0.090}}
\put(201.0,399.0){\rule[-0.200pt]{4.818pt}{0.400pt}}
\put(181,492){\makebox(0,0)[r]{0.095}}
\put(201.0,492.0){\rule[-0.200pt]{4.818pt}{0.400pt}}
\put(181,584){\makebox(0,0)[r]{0.100}}
\put(201.0,584.0){\rule[-0.200pt]{4.818pt}{0.400pt}}
\put(181,676){\makebox(0,0)[r]{0.105}}
\put(201.0,676.0){\rule[-0.200pt]{4.818pt}{0.400pt}}
\put(181,768){\makebox(0,0)[r]{0.110}}
\put(201.0,768.0){\rule[-0.200pt]{4.818pt}{0.400pt}}
\put(181,860){\makebox(0,0)[r]{0.115}}
\put(201.0,860.0){\rule[-0.200pt]{4.818pt}{0.400pt}}
\put(280,82){\makebox(0,0){ 50}}
\put(280.0,123.0){\rule[-0.200pt]{0.400pt}{4.818pt}}
\put(437,82){\makebox(0,0){ 60}}
\put(437.0,123.0){\rule[-0.200pt]{0.400pt}{4.818pt}}
\put(594,82){\makebox(0,0){ 70}}
\put(594.0,123.0){\rule[-0.200pt]{0.400pt}{4.818pt}}
\put(751,82){\makebox(0,0){ 80}}
\put(751.0,123.0){\rule[-0.200pt]{0.400pt}{4.818pt}}
\put(909,82){\makebox(0,0){ 90}}
\put(909.0,123.0){\rule[-0.200pt]{0.400pt}{4.818pt}}
\put(1066,82){\makebox(0,0){ 100}}
\put(1066.0,123.0){\rule[-0.200pt]{0.400pt}{4.818pt}}
\put(1223,82){\makebox(0,0){ 110}}
\put(1223.0,123.0){\rule[-0.200pt]{0.400pt}{4.818pt}}
\put(1380,82){\makebox(0,0){ 120}}
\put(1380.0,123.0){\rule[-0.200pt]{0.400pt}{4.818pt}}
\put(201.0,123.0){\rule[-0.200pt]{0.400pt}{177.543pt}}
\put(201.0,123.0){\rule[-0.200pt]{303.052pt}{0.400pt}}
\put(1459.0,123.0){\rule[-0.200pt]{0.400pt}{177.543pt}}
\put(201.0,860.0){\rule[-0.200pt]{303.052pt}{0.400pt}}
\put(40,491){\makebox(0,0){\rotatebox{90}{velocità \ms}}}
\put(830,21){\makebox(0,0){posizione (cm)}}
\put(280.0,239.0){\rule[-0.200pt]{0.400pt}{11.563pt}}
\put(270.0,239.0){\rule[-0.200pt]{4.818pt}{0.400pt}}
\put(270.0,287.0){\rule[-0.200pt]{4.818pt}{0.400pt}}
\put(437.0,215.0){\rule[-0.200pt]{0.400pt}{25.776pt}}
\put(427.0,215.0){\rule[-0.200pt]{4.818pt}{0.400pt}}
\put(427.0,322.0){\rule[-0.200pt]{4.818pt}{0.400pt}}
\put(594.0,276.0){\rule[-0.200pt]{0.400pt}{39.026pt}}
\put(584.0,276.0){\rule[-0.200pt]{4.818pt}{0.400pt}}
\put(584.0,438.0){\rule[-0.200pt]{4.818pt}{0.400pt}}
\put(751.0,392.0){\rule[-0.200pt]{0.400pt}{10.600pt}}
\put(741.0,392.0){\rule[-0.200pt]{4.818pt}{0.400pt}}
\put(741.0,436.0){\rule[-0.200pt]{4.818pt}{0.400pt}}
\put(909.0,396.0){\rule[-0.200pt]{0.400pt}{21.199pt}}
\put(899.0,396.0){\rule[-0.200pt]{4.818pt}{0.400pt}}
\put(899.0,484.0){\rule[-0.200pt]{4.818pt}{0.400pt}}
\put(1066.0,451.0){\rule[-0.200pt]{0.400pt}{34.690pt}}
\put(1056.0,451.0){\rule[-0.200pt]{4.818pt}{0.400pt}}
\put(1056.0,595.0){\rule[-0.200pt]{4.818pt}{0.400pt}}
\put(1223.0,527.0){\rule[-0.200pt]{0.400pt}{31.799pt}}
\put(1213.0,527.0){\rule[-0.200pt]{4.818pt}{0.400pt}}
\put(1213.0,659.0){\rule[-0.200pt]{4.818pt}{0.400pt}}
\put(1380.0,539.0){\rule[-0.200pt]{0.400pt}{39.989pt}}
\put(1370.0,539.0){\rule[-0.200pt]{4.818pt}{0.400pt}}
\put(280,263){\circle*{12}}
\put(437,269){\circle*{12}}
\put(594,357){\circle*{12}}
\put(751,414){\circle*{12}}
\put(909,440){\circle*{12}}
\put(1066,523){\circle*{12}}
\put(1223,593){\circle*{12}}
\put(1380,622){\circle*{12}}
\put(1370.0,705.0){\rule[-0.200pt]{4.818pt}{0.400pt}}
\put(201,205){\usebox{\plotpoint}}
\put(201.00,205.00){\usebox{\plotpoint}}
\put(220.48,212.16){\usebox{\plotpoint}}
\put(239.94,219.36){\usebox{\plotpoint}}
\put(259.49,226.30){\usebox{\plotpoint}}
\put(278.90,233.58){\usebox{\plotpoint}}
\put(298.53,240.28){\usebox{\plotpoint}}
\put(317.84,247.87){\usebox{\plotpoint}}
\put(337.45,254.63){\usebox{\plotpoint}}
\put(357.01,261.54){\usebox{\plotpoint}}
\put(376.38,268.99){\usebox{\plotpoint}}
\put(396.02,275.67){\usebox{\plotpoint}}
\put(415.57,282.56){\usebox{\plotpoint}}
\put(434.92,290.05){\usebox{\plotpoint}}
\put(454.51,296.85){\usebox{\plotpoint}}
\put(474.04,303.86){\usebox{\plotpoint}}
\put(493.44,311.17){\usebox{\plotpoint}}
\put(512.97,318.15){\usebox{\plotpoint}}
\put(532.39,325.43){\usebox{\plotpoint}}
\put(552.03,332.09){\usebox{\plotpoint}}
\put(571.43,339.48){\usebox{\plotpoint}}
\put(590.97,346.45){\usebox{\plotpoint}}
\put(610.62,353.09){\usebox{\plotpoint}}
\put(629.88,360.80){\usebox{\plotpoint}}
\put(649.52,367.47){\usebox{\plotpoint}}
\put(669.06,374.40){\usebox{\plotpoint}}
\put(688.48,381.72){\usebox{\plotpoint}}
\put(708.03,388.68){\usebox{\plotpoint}}
\put(727.55,395.71){\usebox{\plotpoint}}
\put(746.96,402.98){\usebox{\plotpoint}}
\put(766.48,409.99){\usebox{\plotpoint}}
\put(786.01,417.00){\usebox{\plotpoint}}
\put(805.57,423.91){\usebox{\plotpoint}}
\put(824.96,431.32){\usebox{\plotpoint}}
\put(844.51,438.27){\usebox{\plotpoint}}
\put(863.92,445.64){\usebox{\plotpoint}}
\put(883.45,452.63){\usebox{\plotpoint}}
\put(903.13,459.20){\usebox{\plotpoint}}
\put(922.39,466.91){\usebox{\plotpoint}}
\put(942.04,473.56){\usebox{\plotpoint}}
\put(961.59,480.53){\usebox{\plotpoint}}
\put(980.98,487.92){\usebox{\plotpoint}}
\put(1000.63,494.58){\usebox{\plotpoint}}
\put(1020.04,501.86){\usebox{\plotpoint}}
\put(1039.57,508.84){\usebox{\plotpoint}}
\put(1058.98,516.15){\usebox{\plotpoint}}
\put(1078.50,523.17){\usebox{\plotpoint}}
\put(1098.06,530.10){\usebox{\plotpoint}}
\put(1117.46,537.45){\usebox{\plotpoint}}
\put(1137.02,544.34){\usebox{\plotpoint}}
\put(1156.65,551.02){\usebox{\plotpoint}}
\put(1175.92,558.72){\usebox{\plotpoint}}
\put(1195.56,565.37){\usebox{\plotpoint}}
\put(1215.18,572.13){\usebox{\plotpoint}}
\put(1234.48,579.72){\usebox{\plotpoint}}
\put(1254.11,586.42){\usebox{\plotpoint}}
\put(1273.53,593.70){\usebox{\plotpoint}}
\put(1293.07,600.64){\usebox{\plotpoint}}
\put(1312.53,607.84){\usebox{\plotpoint}}
\put(1332.01,615.00){\usebox{\plotpoint}}
\put(1351.61,621.76){\usebox{\plotpoint}}
\put(1370.95,629.29){\usebox{\plotpoint}}
\put(1390.52,636.13){\usebox{\plotpoint}}
\put(1410.15,642.83){\usebox{\plotpoint}}
\put(1429.52,650.28){\usebox{\plotpoint}}
\put(1449.09,657.19){\usebox{\plotpoint}}
\put(1459,661){\usebox{\plotpoint}}
\put(201.0,123.0){\rule[-0.200pt]{0.400pt}{177.543pt}}
\put(201.0,123.0){\rule[-0.200pt]{303.052pt}{0.400pt}}
\put(1459.0,123.0){\rule[-0.200pt]{0.400pt}{177.543pt}}
\put(201.0,860.0){\rule[-0.200pt]{303.052pt}{0.400pt}}
\end{picture}

\end{figure}
 \begin {figure}[p]\caption{Slitta carica, entrambi gli spessori. Tratteggiata la retta interpolante.}\label{caricogg}
\centering
        % GNUPLOT: LaTeX picture
\setlength{\unitlength}{0.240900pt}
\ifx\plotpoint\undefined\newsavebox{\plotpoint}\fi
\begin{picture}(1500,900)(0,0)
\sbox{\plotpoint}{\rule[-0.200pt]{0.400pt}{0.400pt}}%
\put(181,123){\makebox(0,0)[r]{0.070}}
\put(201.0,123.0){\rule[-0.200pt]{4.818pt}{0.400pt}}
\put(181,215){\makebox(0,0)[r]{0.075}}
\put(201.0,215.0){\rule[-0.200pt]{4.818pt}{0.400pt}}
\put(181,307){\makebox(0,0)[r]{0.080}}
\put(201.0,307.0){\rule[-0.200pt]{4.818pt}{0.400pt}}
\put(181,399){\makebox(0,0)[r]{0.085}}
\put(201.0,399.0){\rule[-0.200pt]{4.818pt}{0.400pt}}
\put(181,492){\makebox(0,0)[r]{0.090}}
\put(201.0,492.0){\rule[-0.200pt]{4.818pt}{0.400pt}}
\put(181,584){\makebox(0,0)[r]{0.095}}
\put(201.0,584.0){\rule[-0.200pt]{4.818pt}{0.400pt}}
\put(181,676){\makebox(0,0)[r]{0.100}}
\put(201.0,676.0){\rule[-0.200pt]{4.818pt}{0.400pt}}
\put(181,768){\makebox(0,0)[r]{0.105}}
\put(201.0,768.0){\rule[-0.200pt]{4.818pt}{0.400pt}}
\put(181,860){\makebox(0,0)[r]{0.110}}
\put(201.0,860.0){\rule[-0.200pt]{4.818pt}{0.400pt}}
\put(280,82){\makebox(0,0){ 50}}
\put(280.0,123.0){\rule[-0.200pt]{0.400pt}{4.818pt}}
\put(437,82){\makebox(0,0){ 60}}
\put(437.0,123.0){\rule[-0.200pt]{0.400pt}{4.818pt}}
\put(594,82){\makebox(0,0){ 70}}
\put(594.0,123.0){\rule[-0.200pt]{0.400pt}{4.818pt}}
\put(751,82){\makebox(0,0){ 80}}
\put(751.0,123.0){\rule[-0.200pt]{0.400pt}{4.818pt}}
\put(909,82){\makebox(0,0){ 90}}
\put(909.0,123.0){\rule[-0.200pt]{0.400pt}{4.818pt}}
\put(1066,82){\makebox(0,0){ 100}}
\put(1066.0,123.0){\rule[-0.200pt]{0.400pt}{4.818pt}}
\put(1223,82){\makebox(0,0){ 110}}
\put(1223.0,123.0){\rule[-0.200pt]{0.400pt}{4.818pt}}
\put(1380,82){\makebox(0,0){ 120}}
\put(1380.0,123.0){\rule[-0.200pt]{0.400pt}{4.818pt}}
\put(201.0,123.0){\rule[-0.200pt]{0.400pt}{177.543pt}}
\put(201.0,123.0){\rule[-0.200pt]{303.052pt}{0.400pt}}
\put(1459.0,123.0){\rule[-0.200pt]{0.400pt}{177.543pt}}
\put(201.0,860.0){\rule[-0.200pt]{303.052pt}{0.400pt}}
\put(40,491){\makebox(0,0){\rotatebox{90}{velocità \ms}}}
\put(830,21){\makebox(0,0){posizione (cm)}}
\put(280.0,222.0){\rule[-0.200pt]{0.400pt}{13.490pt}}
\put(270.0,222.0){\rule[-0.200pt]{4.818pt}{0.400pt}}
\put(270.0,278.0){\rule[-0.200pt]{4.818pt}{0.400pt}}
\put(437.0,202.0){\rule[-0.200pt]{0.400pt}{18.790pt}}
\put(427.0,202.0){\rule[-0.200pt]{4.818pt}{0.400pt}}
\put(427.0,280.0){\rule[-0.200pt]{4.818pt}{0.400pt}}
\put(594.0,285.0){\rule[-0.200pt]{0.400pt}{22.163pt}}
\put(584.0,285.0){\rule[-0.200pt]{4.818pt}{0.400pt}}
\put(584.0,377.0){\rule[-0.200pt]{4.818pt}{0.400pt}}
\put(751.0,311.0){\rule[-0.200pt]{0.400pt}{25.776pt}}
\put(741.0,311.0){\rule[-0.200pt]{4.818pt}{0.400pt}}
\put(741.0,418.0){\rule[-0.200pt]{4.818pt}{0.400pt}}
\put(909.0,381.0){\rule[-0.200pt]{0.400pt}{26.499pt}}
\put(899.0,381.0){\rule[-0.200pt]{4.818pt}{0.400pt}}
\put(899.0,491.0){\rule[-0.200pt]{4.818pt}{0.400pt}}
\put(1066.0,440.0){\rule[-0.200pt]{0.400pt}{40.712pt}}
\put(1056.0,440.0){\rule[-0.200pt]{4.818pt}{0.400pt}}
\put(1056.0,609.0){\rule[-0.200pt]{4.818pt}{0.400pt}}
\put(1223.0,464.0){\rule[-0.200pt]{0.400pt}{40.712pt}}
\put(1213.0,464.0){\rule[-0.200pt]{4.818pt}{0.400pt}}
\put(1213.0,633.0){\rule[-0.200pt]{4.818pt}{0.400pt}}
\put(1380.0,582.0){\rule[-0.200pt]{0.400pt}{39.026pt}}
\put(1370.0,582.0){\rule[-0.200pt]{4.818pt}{0.400pt}}
\put(280,250){\circle*{12}}
\put(437,241){\circle*{12}}
\put(594,331){\circle*{12}}
\put(751,364){\circle*{12}}
\put(909,436){\circle*{12}}
\put(1066,525){\circle*{12}}
\put(1223,549){\circle*{12}}
\put(1380,663){\circle*{12}}
\put(1370.0,744.0){\rule[-0.200pt]{4.818pt}{0.400pt}}
\put(201,178){\usebox{\plotpoint}}
\put(201.00,178.00){\usebox{\plotpoint}}
\put(220.30,185.63){\usebox{\plotpoint}}
\put(239.61,193.23){\usebox{\plotpoint}}
\put(259.15,200.20){\usebox{\plotpoint}}
\put(278.53,207.59){\usebox{\plotpoint}}
\put(297.90,215.04){\usebox{\plotpoint}}
\put(317.14,222.82){\usebox{\plotpoint}}
\put(336.51,230.27){\usebox{\plotpoint}}
\put(355.75,238.06){\usebox{\plotpoint}}
\put(375.34,244.87){\usebox{\plotpoint}}
\put(394.77,252.15){\usebox{\plotpoint}}
\put(414.04,259.86){\usebox{\plotpoint}}
\put(433.37,267.40){\usebox{\plotpoint}}
\put(452.65,275.09){\usebox{\plotpoint}}
\put(472.11,282.27){\usebox{\plotpoint}}
\put(491.58,289.41){\usebox{\plotpoint}}
\put(510.93,296.90){\usebox{\plotpoint}}
\put(530.18,304.66){\usebox{\plotpoint}}
\put(549.54,312.13){\usebox{\plotpoint}}
\put(568.92,319.58){\usebox{\plotpoint}}
\put(588.30,326.94){\usebox{\plotpoint}}
\put(607.83,333.94){\usebox{\plotpoint}}
\put(627.07,341.72){\usebox{\plotpoint}}
\put(646.44,349.18){\usebox{\plotpoint}}
\put(665.68,356.95){\usebox{\plotpoint}}
\put(685.05,364.41){\usebox{\plotpoint}}
\put(704.55,371.52){\usebox{\plotpoint}}
\put(723.99,378.77){\usebox{\plotpoint}}
\put(743.27,386.45){\usebox{\plotpoint}}
\put(762.60,394.00){\usebox{\plotpoint}}
\put(781.97,401.45){\usebox{\plotpoint}}
\put(801.21,409.24){\usebox{\plotpoint}}
\put(820.82,416.02){\usebox{\plotpoint}}
\put(840.13,423.59){\usebox{\plotpoint}}
\put(859.50,431.04){\usebox{\plotpoint}}
\put(878.74,438.82){\usebox{\plotpoint}}
\put(898.11,446.27){\usebox{\plotpoint}}
\put(917.43,453.85){\usebox{\plotpoint}}
\put(937.00,460.69){\usebox{\plotpoint}}
\put(956.34,468.22){\usebox{\plotpoint}}
\put(975.64,475.86){\usebox{\plotpoint}}
\put(995.01,483.31){\usebox{\plotpoint}}
\put(1014.25,491.10){\usebox{\plotpoint}}
\put(1033.62,498.55){\usebox{\plotpoint}}
\put(1053.19,505.46){\usebox{\plotpoint}}
\put(1072.56,512.91){\usebox{\plotpoint}}
\put(1091.80,520.69){\usebox{\plotpoint}}
\put(1111.17,528.14){\usebox{\plotpoint}}
\put(1130.52,535.64){\usebox{\plotpoint}}
\put(1149.99,542.77){\usebox{\plotpoint}}
\put(1169.43,550.01){\usebox{\plotpoint}}
\put(1188.70,557.73){\usebox{\plotpoint}}
\put(1208.07,565.18){\usebox{\plotpoint}}
\put(1227.34,572.89){\usebox{\plotpoint}}
\put(1246.68,580.41){\usebox{\plotpoint}}
\put(1266.22,587.41){\usebox{\plotpoint}}
\put(1285.62,594.78){\usebox{\plotpoint}}
\put(1304.99,602.23){\usebox{\plotpoint}}
\put(1324.23,610.01){\usebox{\plotpoint}}
\put(1343.60,617.46){\usebox{\plotpoint}}
\put(1362.84,625.25){\usebox{\plotpoint}}
\put(1382.50,631.85){\usebox{\plotpoint}}
\put(1401.75,639.60){\usebox{\plotpoint}}
\put(1421.13,647.05){\usebox{\plotpoint}}
\put(1440.43,654.68){\usebox{\plotpoint}}
\put(1459,662){\usebox{\plotpoint}}
\put(201.0,123.0){\rule[-0.200pt]{0.400pt}{177.543pt}}
\put(201.0,123.0){\rule[-0.200pt]{303.052pt}{0.400pt}}
\put(1459.0,123.0){\rule[-0.200pt]{0.400pt}{177.543pt}}
\put(201.0,860.0){\rule[-0.200pt]{303.052pt}{0.400pt}}
\end{picture}

\end{figure}
\end{document}
