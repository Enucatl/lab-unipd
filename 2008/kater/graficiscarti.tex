\documentclass[italian,a4paper]{article}
\usepackage[tight,nice]{units}
\usepackage{babel,amsmath,amssymb,amsthm,graphicx,url,wrapfig,multirow}
\usepackage[text={6in,9in},centering]{geometry}
\usepackage[utf8x]{inputenc}
\usepackage[T1]{fontenc}
\usepackage{ae,aecompl}
\usepackage[Euler]{upgreek}
\usepackage[footnotesize,bf]{caption}
\usepackage[usenames]{color}
\include{pstricks}
% \include{pstricks}
\frenchspacing
\pagestyle{plain}
%------------- eliminare prime e ultime linee isolate
\clubpenalty=9999%
\widowpenalty=9999
%--- definizione numerazioni
\renewcommand{\theequation}{\thesection.\arabic{equation}}
\renewcommand{\thefigure}{\thesection.\arabic{figure}}
\renewcommand{\thetable}{\thesection.\arabic{table}}
\addto\captionsitalian{%
  \renewcommand{\figurename}%
{Grafico}%
}
%
%------------- ridefinizione simbolo per elenchi puntati: en dash
%\renewcommand{\labelitemi}{\textbf{--}}
\renewcommand{\labelenumi}{\textbf{\arabic{enumi}.}}
\setlength{\abovecaptionskip}{\baselineskip}   % 0.5cm as an example
\setlength{\floatsep}{2\baselineskip}
\setlength{\belowcaptionskip}{\baselineskip}   % 0.5cm as an example
%------------- nuovi environment senza spazi
%\newenvironment{packed_item}{
%\begin{itemize}
%  \setlength{\itemsep}{1pt}
%  \setlength{\parskip}{0pt}
%  \setlength{\parsep}{0pt}
%}{\end{itemize}}
%\newenvironment{packed_enum}{
%\begin{enumerate}
%  \setlength{\itemsep}{1pt}
%  \setlength{\parskip}{0pt}
%  \setlength{\parsep}{0pt}
%}{\end{enumerate}}
%\newenvironment{packed_description}{
%\begin{enumerate}
%   \setlength{\itemsep}{1pt}
%   \setlength{\parskip}{0pt}
%   \setlength{\parsep}{0pt}
% }{\end{enumerate}}
%--------- comandi insiemi numeri complessi, naturali, reali e altre abbreviazioni
\newcommand{\micro}{\ensuremath{\upmu}} %prefisso micro
\newcommand{\e}{\mathrm{e}} %numero di nepero
\newcommand{\di}{\mathrm{d}} %simbolo di differenziale
\renewcommand{\leq}{\leqslant}
\renewcommand{\pi}{\uppi} % costante pi greco
\renewcommand{\tau}{\uptau} %momento della forza
\newcommand{\coloneqq}{\mathrel{\mathop:}=} % := ``per definizione''
\newcommand{\ms}{(\unitfrac{m}{s})}
%--------- porzione dedicata ai float in una pagina:
\renewcommand{\textfraction}{0.05}
\renewcommand{\topfraction}{0.95}
\renewcommand{\bottomfraction}{0.95}
\renewcommand{\floatpagefraction}{0.35}
\setcounter{totalnumber}{5}
%---------
%
%---------
\begin{document}
Da un calcolo numerico si trova, per un angolo di $5^\circ$:
\begin{equation*}
 I(5^\circ)=\dfrac 2 \pi \int_0^{5^\circ} \dfrac{\di x}{\sqrt{\cos x - \cos 5^\circ}} = 1.0004058\dots
\end{equation*}
Quindi l'approssimazione usata ($I(\alpha) \equiv 1$) risulta buona a meno di un'approssimazione dell'ordine di $4\cdot10^{-4}$. I seguenti grafici confermano questa influenza dell'errore sistematico.
\begin{figure}[h]
\centering
\caption{Prima serie di cento misure, cronometro automatico. Scarti $x_i - \bar{x}$ rispetto al numero di oscillazioni.}\label{scarti1}
 % GNUPLOT: LaTeX picture using PSTRICKS macros
% Define new PST objects, if not already defined
\ifx\PSTloaded\undefined
\def\PSTloaded{t}
\psset{arrowsize=.1 3.2 1.4 .3}
\psset{dotsize=.08}
\catcode`@=11

\newpsobject{PST@Border}{psline}{linewidth=.0015,linestyle=solid}
\newpsobject{PST@Axes}{psline}{linewidth=.0015,linestyle=dotted,dotsep=.004}
\newpsobject{PST@Solid}{psline}{linewidth=.0015,linestyle=solid}
\newpsobject{PST@Dashed}{psline}{linewidth=.0015,linestyle=dashed,dash=.01 .01}
\newpsobject{PST@Dotted}{psline}{linewidth=.0025,linestyle=dotted,dotsep=.008}
\newpsobject{PST@LongDash}{psline}{linewidth=.0015,linestyle=dashed,dash=.02 .01}
\newpsobject{PST@Diamond}{psdots}{linewidth=.001,linestyle=solid,dotstyle=*}
\newpsobject{PST@Filldiamond}{psdots}{linewidth=.001,linestyle=solid,dotstyle=square*,dotangle=45}
\newpsobject{PST@Cross}{psdots}{linewidth=.001,linestyle=solid,dotstyle=+,dotangle=45}
\newpsobject{PST@Plus}{psdots}{linewidth=.001,linestyle=solid,dotstyle=+}
\newpsobject{PST@Square}{psdots}{linewidth=.001,linestyle=solid,dotstyle=square}
\newpsobject{PST@Circle}{psdots}{linewidth=.001,linestyle=solid,dotstyle=o}
\newpsobject{PST@Triangle}{psdots}{linewidth=.001,linestyle=solid,dotstyle=triangle}
\newpsobject{PST@Pentagon}{psdots}{linewidth=.001,linestyle=solid,dotstyle=pentagon}
\newpsobject{PST@Fillsquare}{psdots}{linewidth=.001,linestyle=solid,dotstyle=square*}
\newpsobject{PST@Fillcircle}{psdots}{linewidth=.001,linestyle=solid,dotstyle=*}
\newpsobject{PST@Filltriangle}{psdots}{linewidth=.001,linestyle=solid,dotstyle=triangle*}
\newpsobject{PST@Fillpentagon}{psdots}{linewidth=.001,linestyle=solid,dotstyle=pentagon*}
\newpsobject{PST@Arrow}{psline}{linewidth=.001,linestyle=solid}
\catcode`@=12

\fi
\psset{unit=5.0in,xunit=5.0in,yunit=3.0in}
\pspicture(0.000000,0.000000)(1.000000,1.000000)
\ifx\nofigs\undefined
\catcode`@=11

\PST@Border(0.2070,0.1260)
(0.2220,0.1260)

\rput[r](0.1910,0.1260){-0.0008}
\PST@Border(0.2070,0.2102)
(0.2220,0.2102)

\rput[r](0.1910,0.2102){-0.0006}
\PST@Border(0.2070,0.2944)
(0.2220,0.2944)

\rput[r](0.1910,0.2944){-0.0004}
\PST@Border(0.2070,0.3786)
(0.2220,0.3786)

\rput[r](0.1910,0.3786){-0.0002}
\PST@Border(0.2070,0.4628)
(0.2220,0.4628)

\rput[r](0.1910,0.4628){0.0000}
\PST@Border(0.2070,0.5470)
(0.2220,0.5470)

\rput[r](0.1910,0.5470){0.0002}
\PST@Border(0.2070,0.6312)
(0.2220,0.6312)

\rput[r](0.1910,0.6312){0.0004}
\PST@Border(0.2070,0.7154)
(0.2220,0.7154)

\rput[r](0.1910,0.7154){0.0006}
\PST@Border(0.2070,0.7996)
(0.2220,0.7996)

\rput[r](0.1910,0.7996){0.0008}
\PST@Border(0.2070,0.8838)
(0.2220,0.8838)

\rput[r](0.1910,0.8838){0.0010}
\PST@Border(0.2070,0.9680)
(0.2220,0.9680)

\rput[r](0.1910,0.9680){0.0012}
\PST@Border(0.2070,0.1260)
(0.2070,0.1460)

\rput(0.2070,0.0840){ 0}
\PST@Border(0.2821,0.1260)
(0.2821,0.1460)

\rput(0.2821,0.0840){ 10}
\PST@Border(0.3572,0.1260)
(0.3572,0.1460)

\rput(0.3572,0.0840){ 20}
\PST@Border(0.4323,0.1260)
(0.4323,0.1460)

\rput(0.4323,0.0840){ 30}
\PST@Border(0.5074,0.1260)
(0.5074,0.1460)

\rput(0.5074,0.0840){ 40}
\PST@Border(0.5825,0.1260)
(0.5825,0.1460)

\rput(0.5825,0.0840){ 50}
\PST@Border(0.6576,0.1260)
(0.6576,0.1460)

\rput(0.6576,0.0840){ 60}
\PST@Border(0.7327,0.1260)
(0.7327,0.1460)

\rput(0.7327,0.0840){ 70}
\PST@Border(0.8078,0.1260)
(0.8078,0.1460)

\rput(0.8078,0.0840){ 80}
\PST@Border(0.8829,0.1260)
(0.8829,0.1460)

\rput(0.8829,0.0840){ 90}
\PST@Border(0.9580,0.1260)
(0.9580,0.1460)

\rput(0.9580,0.0840){ 100}
\PST@Border(0.2070,0.9680)
(0.2070,0.1260)
(0.9580,0.1260)
(0.9580,0.9680)
(0.2070,0.9680)

\rput{L}(0.0420,0.5470){scarti (\unit{s})}
\rput(0.5825,0.0210){oscillazioni}
\PST@Diamond(0.2145,0.9259)
\PST@Diamond(0.2220,0.9259)
\PST@Diamond(0.2295,0.9259)
\PST@Diamond(0.2370,0.9259)
\PST@Diamond(0.2445,0.9259)
\PST@Diamond(0.2521,0.8417)
\PST@Diamond(0.2596,0.8417)
\PST@Diamond(0.2671,0.8417)
\PST@Diamond(0.2746,0.7996)
\PST@Diamond(0.2821,0.8417)
\PST@Diamond(0.2896,0.7996)
\PST@Diamond(0.2971,0.7996)
\PST@Diamond(0.3046,0.7575)
\PST@Diamond(0.3121,0.7575)
\PST@Diamond(0.3196,0.7154)
\PST@Diamond(0.3272,0.7154)
\PST@Diamond(0.3347,0.7154)
\PST@Diamond(0.3422,0.7575)
\PST@Diamond(0.3497,0.7154)
\PST@Diamond(0.3572,0.6733)
\PST@Diamond(0.3647,0.6733)
\PST@Diamond(0.3722,0.7154)
\PST@Diamond(0.3797,0.6733)
\PST@Diamond(0.3872,0.6312)
\PST@Diamond(0.3947,0.6312)
\PST@Diamond(0.4023,0.6312)
\PST@Diamond(0.4098,0.5891)
\PST@Diamond(0.4173,0.6312)
\PST@Diamond(0.4248,0.5470)
\PST@Diamond(0.4323,0.5470)
\PST@Diamond(0.4398,0.5470)
\PST@Diamond(0.4473,0.5891)
\PST@Diamond(0.4548,0.5470)
\PST@Diamond(0.4623,0.5049)
\PST@Diamond(0.4698,0.5470)
\PST@Diamond(0.4774,0.5049)
\PST@Diamond(0.4849,0.4628)
\PST@Diamond(0.4924,0.4628)
\PST@Diamond(0.4999,0.5049)
\PST@Diamond(0.5074,0.4207)
\PST@Diamond(0.5149,0.4207)
\PST@Diamond(0.5224,0.4628)
\PST@Diamond(0.5299,0.4207)
\PST@Diamond(0.5374,0.3786)
\PST@Diamond(0.5449,0.4207)
\PST@Diamond(0.5525,0.4207)
\PST@Diamond(0.5600,0.3365)
\PST@Diamond(0.5675,0.3786)
\PST@Diamond(0.5750,0.3786)
\PST@Diamond(0.5825,0.3365)
\PST@Diamond(0.5900,0.3365)
\PST@Diamond(0.5975,0.3365)
\PST@Diamond(0.6050,0.2944)
\PST@Diamond(0.6125,0.3365)
\PST@Diamond(0.6200,0.2944)
\PST@Diamond(0.6276,0.2944)
\PST@Diamond(0.6351,0.3365)
\PST@Diamond(0.6426,0.2944)
\PST@Diamond(0.6501,0.2944)
\PST@Diamond(0.6576,0.3786)
\PST@Diamond(0.6651,0.2944)
\PST@Diamond(0.6726,0.2944)
\PST@Diamond(0.6801,0.3365)
\PST@Diamond(0.6876,0.2944)
\PST@Diamond(0.6951,0.3365)
\PST@Diamond(0.7027,0.3365)
\PST@Diamond(0.7102,0.2944)
\PST@Diamond(0.7177,0.3365)
\PST@Diamond(0.7252,0.3365)
\PST@Diamond(0.7327,0.3365)
\PST@Diamond(0.7402,0.2944)
\PST@Diamond(0.7477,0.2523)
\PST@Diamond(0.7552,0.3786)
\PST@Diamond(0.7627,0.3365)
\PST@Diamond(0.7702,0.2523)
\PST@Diamond(0.7778,0.3365)
\PST@Diamond(0.7853,0.2523)
\PST@Diamond(0.7928,0.3365)
\PST@Diamond(0.8003,0.2944)
\PST@Diamond(0.8078,0.2523)
\PST@Diamond(0.8153,0.2944)
\PST@Diamond(0.8228,0.2944)
\PST@Diamond(0.8303,0.2944)
\PST@Diamond(0.8378,0.2944)
\PST@Diamond(0.8453,0.2523)
\PST@Diamond(0.8529,0.2944)
\PST@Diamond(0.8604,0.2944)
\PST@Diamond(0.8679,0.2102)
\PST@Diamond(0.8754,0.3786)
\PST@Diamond(0.8829,0.2523)
\PST@Diamond(0.8904,0.2523)
\PST@Diamond(0.8979,0.2102)
\PST@Diamond(0.9054,0.2944)
\PST@Diamond(0.9129,0.2102)
\PST@Diamond(0.9204,0.2523)
\PST@Diamond(0.9280,0.1681)
\PST@Diamond(0.9355,0.2523)
\PST@Diamond(0.9430,0.2523)
\PST@Diamond(0.9505,0.2102)
\PST@Diamond(0.9580,0.2523)
\PST@Border(0.2070,0.9680)
(0.2070,0.1260)
(0.9580,0.1260)
(0.9580,0.9680)
(0.2070,0.9680)

\catcode`@=12
\fi
\endpspicture

\end{figure}
\begin{figure}[h]
\centering
\caption{Seconda serie di cento misure, cronometro automatico. Scarti $x_i - \bar{x}$ rispetto al numero di oscillazioni.}\label{scarti2}
 % GNUPLOT: LaTeX picture using PSTRICKS macros
% Define new PST objects, if not already defined
\ifx\PSTloaded\undefined
\def\PSTloaded{t}
\psset{arrowsize=.1 3.2 1.4 .3}
\psset{dotsize=.08}
\catcode`@=11

\newpsobject{PST@Border}{psline}{linewidth=.0015,linestyle=solid}
\newpsobject{PST@Axes}{psline}{linewidth=.0015,linestyle=dotted,dotsep=.004}
\newpsobject{PST@Solid}{psline}{linewidth=.0015,linestyle=solid}
\newpsobject{PST@Dashed}{psline}{linewidth=.0015,linestyle=dashed,dash=.01 .01}
\newpsobject{PST@Dotted}{psline}{linewidth=.0025,linestyle=dotted,dotsep=.008}
\newpsobject{PST@LongDash}{psline}{linewidth=.0015,linestyle=dashed,dash=.02 .01}
\newpsobject{PST@Diamond}{psdots}{linewidth=.001,linestyle=solid,dotstyle=*}
\newpsobject{PST@Filldiamond}{psdots}{linewidth=.001,linestyle=solid,dotstyle=square*,dotangle=45}
\newpsobject{PST@Cross}{psdots}{linewidth=.001,linestyle=solid,dotstyle=+,dotangle=45}
\newpsobject{PST@Plus}{psdots}{linewidth=.001,linestyle=solid,dotstyle=+}
\newpsobject{PST@Square}{psdots}{linewidth=.001,linestyle=solid,dotstyle=square}
\newpsobject{PST@Circle}{psdots}{linewidth=.001,linestyle=solid,dotstyle=o}
\newpsobject{PST@Triangle}{psdots}{linewidth=.001,linestyle=solid,dotstyle=triangle}
\newpsobject{PST@Pentagon}{psdots}{linewidth=.001,linestyle=solid,dotstyle=pentagon}
\newpsobject{PST@Fillsquare}{psdots}{linewidth=.001,linestyle=solid,dotstyle=square*}
\newpsobject{PST@Fillcircle}{psdots}{linewidth=.001,linestyle=solid,dotstyle=*}
\newpsobject{PST@Filltriangle}{psdots}{linewidth=.001,linestyle=solid,dotstyle=triangle*}
\newpsobject{PST@Fillpentagon}{psdots}{linewidth=.001,linestyle=solid,dotstyle=pentagon*}
\newpsobject{PST@Arrow}{psline}{linewidth=.001,linestyle=solid}
\catcode`@=12

\fi
\psset{unit=5.0in,xunit=5.0in,yunit=3.0in}
\pspicture(0.000000,0.000000)(1.000000,1.000000)
\ifx\nofigs\undefined
\catcode`@=11

\PST@Border(0.2070,0.1260)
(0.2220,0.1260)

\rput[r](0.1910,0.1260){-0.0010}
\PST@Border(0.2070,0.2944)
(0.2220,0.2944)

\rput[r](0.1910,0.2944){-0.0005}
\PST@Border(0.2070,0.4628)
(0.2220,0.4628)

\rput[r](0.1910,0.4628){0.0000}
\PST@Border(0.2070,0.6312)
(0.2220,0.6312)

\rput[r](0.1910,0.6312){0.0005}
\PST@Border(0.2070,0.7996)
(0.2220,0.7996)

\rput[r](0.1910,0.7996){0.0010}
\PST@Border(0.2070,0.9680)
(0.2220,0.9680)

\rput[r](0.1910,0.9680){0.0015}
\PST@Border(0.2070,0.1260)
(0.2070,0.1460)

\rput(0.2070,0.0840){ 0}
\PST@Border(0.2821,0.1260)
(0.2821,0.1460)

\rput(0.2821,0.0840){ 10}
\PST@Border(0.3572,0.1260)
(0.3572,0.1460)

\rput(0.3572,0.0840){ 20}
\PST@Border(0.4323,0.1260)
(0.4323,0.1460)

\rput(0.4323,0.0840){ 30}
\PST@Border(0.5074,0.1260)
(0.5074,0.1460)

\rput(0.5074,0.0840){ 40}
\PST@Border(0.5825,0.1260)
(0.5825,0.1460)

\rput(0.5825,0.0840){ 50}
\PST@Border(0.6576,0.1260)
(0.6576,0.1460)

\rput(0.6576,0.0840){ 60}
\PST@Border(0.7327,0.1260)
(0.7327,0.1460)

\rput(0.7327,0.0840){ 70}
\PST@Border(0.8078,0.1260)
(0.8078,0.1460)

\rput(0.8078,0.0840){ 80}
\PST@Border(0.8829,0.1260)
(0.8829,0.1460)

\rput(0.8829,0.0840){ 90}
\PST@Border(0.9580,0.1260)
(0.9580,0.1460)

\rput(0.9580,0.0840){ 100}
\PST@Border(0.2070,0.9680)
(0.2070,0.1260)
(0.9580,0.1260)
(0.9580,0.9680)
(0.2070,0.9680)

\rput{L}(0.0420,0.5470){scarti (\unit{s})}
\rput(0.5825,0.0210){oscillazioni}
\PST@Diamond(0.2145,0.8333)
\PST@Diamond(0.2220,0.8333)
\PST@Diamond(0.2295,0.8333)
\PST@Diamond(0.2370,0.8333)
\PST@Diamond(0.2445,0.8333)
\PST@Diamond(0.2521,0.7996)
\PST@Diamond(0.2596,0.7659)
\PST@Diamond(0.2671,0.7996)
\PST@Diamond(0.2746,0.7996)
\PST@Diamond(0.2821,0.8333)
\PST@Diamond(0.2896,0.7322)
\PST@Diamond(0.2971,0.7659)
\PST@Diamond(0.3046,0.7996)
\PST@Diamond(0.3121,0.7659)
\PST@Diamond(0.3196,0.6986)
\PST@Diamond(0.3272,0.7659)
\PST@Diamond(0.3347,0.7322)
\PST@Diamond(0.3422,0.7659)
\PST@Diamond(0.3497,0.6986)
\PST@Diamond(0.3572,0.7322)
\PST@Diamond(0.3647,0.7322)
\PST@Diamond(0.3722,0.7322)
\PST@Diamond(0.3797,0.8333)
\PST@Diamond(0.3872,0.6649)
\PST@Diamond(0.3947,0.6649)
\PST@Diamond(0.4023,0.6649)
\PST@Diamond(0.4098,0.6986)
\PST@Diamond(0.4173,0.6649)
\PST@Diamond(0.4248,0.7322)
\PST@Diamond(0.4323,0.5975)
\PST@Diamond(0.4398,0.6649)
\PST@Diamond(0.4473,0.6649)
\PST@Diamond(0.4548,0.6649)
\PST@Diamond(0.4623,0.6312)
\PST@Diamond(0.4698,0.6649)
\PST@Diamond(0.4774,0.6649)
\PST@Diamond(0.4849,0.6312)
\PST@Diamond(0.4924,0.5638)
\PST@Diamond(0.4999,0.5975)
\PST@Diamond(0.5074,0.5975)
\PST@Diamond(0.5149,0.5975)
\PST@Diamond(0.5224,0.5638)
\PST@Diamond(0.5299,0.5975)
\PST@Diamond(0.5374,0.5638)
\PST@Diamond(0.5449,0.5638)
\PST@Diamond(0.5525,0.5302)
\PST@Diamond(0.5600,0.5638)
\PST@Diamond(0.5675,0.5302)
\PST@Diamond(0.5750,0.5302)
\PST@Diamond(0.5825,0.4965)
\PST@Diamond(0.5900,0.4291)
\PST@Diamond(0.5975,0.4965)
\PST@Diamond(0.6050,0.4291)
\PST@Diamond(0.6125,0.4291)
\PST@Diamond(0.6200,0.3618)
\PST@Diamond(0.6276,0.4291)
\PST@Diamond(0.6351,0.4291)
\PST@Diamond(0.6426,0.3618)
\PST@Diamond(0.6501,0.3618)
\PST@Diamond(0.6576,0.3618)
\PST@Diamond(0.6651,0.2944)
\PST@Diamond(0.6726,0.3618)
\PST@Diamond(0.6801,0.3281)
\PST@Diamond(0.6876,0.2607)
\PST@Diamond(0.6951,0.3281)
\PST@Diamond(0.7027,0.1260)
\PST@Diamond(0.7102,0.3954)
\PST@Diamond(0.7177,0.2607)
\PST@Diamond(0.7252,0.1934)
\PST@Diamond(0.7327,0.3281)
\PST@Diamond(0.7402,0.2270)
\PST@Diamond(0.7477,0.2270)
\PST@Diamond(0.7552,0.2270)
\PST@Diamond(0.7627,0.2607)
\PST@Diamond(0.7702,0.2270)
\PST@Diamond(0.7778,0.2944)
\PST@Diamond(0.7853,0.2270)
\PST@Diamond(0.7928,0.2270)
\PST@Diamond(0.8003,0.2270)
\PST@Diamond(0.8078,0.1934)
\PST@Diamond(0.8153,0.2270)
\PST@Diamond(0.8228,0.1597)
\PST@Diamond(0.8303,0.2944)
\PST@Diamond(0.8378,0.2270)
\PST@Diamond(0.8453,0.2270)
\PST@Diamond(0.8529,0.1934)
\PST@Diamond(0.8604,0.2270)
\PST@Diamond(0.8679,0.1260)
\PST@Diamond(0.8754,0.2607)
\PST@Diamond(0.8829,0.1934)
\PST@Diamond(0.8904,0.1597)
\PST@Diamond(0.8979,0.1934)
\PST@Diamond(0.9054,0.2270)
\PST@Diamond(0.9129,0.1597)
\PST@Diamond(0.9204,0.1934)
\PST@Diamond(0.9280,0.1597)
\PST@Diamond(0.9355,0.2270)
\PST@Diamond(0.9430,0.1260)
\PST@Diamond(0.9505,0.1934)
\PST@Diamond(0.9580,0.1934)
\PST@Border(0.2070,0.9680)
(0.2070,0.1260)
(0.9580,0.1260)
(0.9580,0.9680)
(0.2070,0.9680)

\catcode`@=12
\fi
\endpspicture

\end{figure}
\begin{figure}[h]
\centering
\caption{Terza serie di cento misure, cronometro automatico. Scarti $x_i - \bar{x}$ rispetto al numero di oscillazioni.}\label{scarti3}
 % GNUPLOT: LaTeX picture using PSTRICKS macros
% Define new PST objects, if not already defined
\ifx\PSTloaded\undefined
\def\PSTloaded{t}
\psset{arrowsize=.1 3.2 1.4 .3}
\psset{dotsize=.08}
\catcode`@=11

\newpsobject{PST@Border}{psline}{linewidth=.0015,linestyle=solid}
\newpsobject{PST@Axes}{psline}{linewidth=.0015,linestyle=dotted,dotsep=.004}
\newpsobject{PST@Solid}{psline}{linewidth=.0015,linestyle=solid}
\newpsobject{PST@Dashed}{psline}{linewidth=.0015,linestyle=dashed,dash=.01 .01}
\newpsobject{PST@Dotted}{psline}{linewidth=.0025,linestyle=dotted,dotsep=.008}
\newpsobject{PST@LongDash}{psline}{linewidth=.0015,linestyle=dashed,dash=.02 .01}
\newpsobject{PST@Diamond}{psdots}{linewidth=.001,linestyle=solid,dotstyle=*}
\newpsobject{PST@Filldiamond}{psdots}{linewidth=.001,linestyle=solid,dotstyle=square*,dotangle=45}
\newpsobject{PST@Cross}{psdots}{linewidth=.001,linestyle=solid,dotstyle=+,dotangle=45}
\newpsobject{PST@Plus}{psdots}{linewidth=.001,linestyle=solid,dotstyle=+}
\newpsobject{PST@Square}{psdots}{linewidth=.001,linestyle=solid,dotstyle=square}
\newpsobject{PST@Circle}{psdots}{linewidth=.001,linestyle=solid,dotstyle=o}
\newpsobject{PST@Triangle}{psdots}{linewidth=.001,linestyle=solid,dotstyle=triangle}
\newpsobject{PST@Pentagon}{psdots}{linewidth=.001,linestyle=solid,dotstyle=pentagon}
\newpsobject{PST@Fillsquare}{psdots}{linewidth=.001,linestyle=solid,dotstyle=square*}
\newpsobject{PST@Fillcircle}{psdots}{linewidth=.001,linestyle=solid,dotstyle=*}
\newpsobject{PST@Filltriangle}{psdots}{linewidth=.001,linestyle=solid,dotstyle=triangle*}
\newpsobject{PST@Fillpentagon}{psdots}{linewidth=.001,linestyle=solid,dotstyle=pentagon*}
\newpsobject{PST@Arrow}{psline}{linewidth=.001,linestyle=solid}
\catcode`@=12

\fi
\psset{unit=5.0in,xunit=5.0in,yunit=3.0in}
\pspicture(0.000000,0.000000)(1.000000,1.000000)
\ifx\nofigs\undefined
\catcode`@=11

\PST@Border(0.2070,0.1260)
(0.2220,0.1260)

\rput[r](0.1910,0.1260){-0.0010}
\PST@Border(0.2070,0.2944)
(0.2220,0.2944)

\rput[r](0.1910,0.2944){-0.0005}
\PST@Border(0.2070,0.4628)
(0.2220,0.4628)

\rput[r](0.1910,0.4628){0.0000}
\PST@Border(0.2070,0.6312)
(0.2220,0.6312)

\rput[r](0.1910,0.6312){0.0005}
\PST@Border(0.2070,0.7996)
(0.2220,0.7996)

\rput[r](0.1910,0.7996){0.0010}
\PST@Border(0.2070,0.9680)
(0.2220,0.9680)

\rput[r](0.1910,0.9680){0.0015}
\PST@Border(0.2070,0.1260)
(0.2070,0.1460)

\rput(0.2070,0.0840){ 0}
\PST@Border(0.2821,0.1260)
(0.2821,0.1460)

\rput(0.2821,0.0840){ 10}
\PST@Border(0.3572,0.1260)
(0.3572,0.1460)

\rput(0.3572,0.0840){ 20}
\PST@Border(0.4323,0.1260)
(0.4323,0.1460)

\rput(0.4323,0.0840){ 30}
\PST@Border(0.5074,0.1260)
(0.5074,0.1460)

\rput(0.5074,0.0840){ 40}
\PST@Border(0.5825,0.1260)
(0.5825,0.1460)

\rput(0.5825,0.0840){ 50}
\PST@Border(0.6576,0.1260)
(0.6576,0.1460)

\rput(0.6576,0.0840){ 60}
\PST@Border(0.7327,0.1260)
(0.7327,0.1460)

\rput(0.7327,0.0840){ 70}
\PST@Border(0.8078,0.1260)
(0.8078,0.1460)

\rput(0.8078,0.0840){ 80}
\PST@Border(0.8829,0.1260)
(0.8829,0.1460)

\rput(0.8829,0.0840){ 90}
\PST@Border(0.9580,0.1260)
(0.9580,0.1460)

\rput(0.9580,0.0840){ 100}
\PST@Border(0.2070,0.9680)
(0.2070,0.1260)
(0.9580,0.1260)
(0.9580,0.9680)
(0.2070,0.9680)

\rput{L}(0.0420,0.5470){scarti (\unit{s})}
\rput(0.5825,0.0210){oscillazioni}
\PST@Diamond(0.2145,0.8670)
\PST@Diamond(0.2220,0.8333)
\PST@Diamond(0.2295,0.7996)
\PST@Diamond(0.2370,0.8333)
\PST@Diamond(0.2445,0.8670)
\PST@Diamond(0.2521,0.7659)
\PST@Diamond(0.2596,0.8333)
\PST@Diamond(0.2671,0.8333)
\PST@Diamond(0.2746,0.7659)
\PST@Diamond(0.2821,0.7659)
\PST@Diamond(0.2896,0.7659)
\PST@Diamond(0.2971,0.7659)
\PST@Diamond(0.3046,0.7322)
\PST@Diamond(0.3121,0.7659)
\PST@Diamond(0.3196,0.7659)
\PST@Diamond(0.3272,0.7322)
\PST@Diamond(0.3347,0.6986)
\PST@Diamond(0.3422,0.6649)
\PST@Diamond(0.3497,0.7322)
\PST@Diamond(0.3572,0.6649)
\PST@Diamond(0.3647,0.5975)
\PST@Diamond(0.3722,0.6312)
\PST@Diamond(0.3797,0.5975)
\PST@Diamond(0.3872,0.6312)
\PST@Diamond(0.3947,0.6312)
\PST@Diamond(0.4023,0.5638)
\PST@Diamond(0.4098,0.5975)
\PST@Diamond(0.4173,0.5975)
\PST@Diamond(0.4248,0.5302)
\PST@Diamond(0.4323,0.5975)
\PST@Diamond(0.4398,0.5638)
\PST@Diamond(0.4473,0.5302)
\PST@Diamond(0.4548,0.4965)
\PST@Diamond(0.4623,0.4965)
\PST@Diamond(0.4698,0.4965)
\PST@Diamond(0.4774,0.5302)
\PST@Diamond(0.4849,0.4291)
\PST@Diamond(0.4924,0.4291)
\PST@Diamond(0.4999,0.5302)
\PST@Diamond(0.5074,0.4291)
\PST@Diamond(0.5149,0.4965)
\PST@Diamond(0.5224,0.4628)
\PST@Diamond(0.5299,0.4628)
\PST@Diamond(0.5374,0.4628)
\PST@Diamond(0.5449,0.4291)
\PST@Diamond(0.5525,0.4628)
\PST@Diamond(0.5600,0.3954)
\PST@Diamond(0.5675,0.4291)
\PST@Diamond(0.5750,0.4291)
\PST@Diamond(0.5825,0.3954)
\PST@Diamond(0.5900,0.3954)
\PST@Diamond(0.5975,0.4291)
\PST@Diamond(0.6050,0.4291)
\PST@Diamond(0.6125,0.3954)
\PST@Diamond(0.6200,0.4291)
\PST@Diamond(0.6276,0.3281)
\PST@Diamond(0.6351,0.4291)
\PST@Diamond(0.6426,0.3954)
\PST@Diamond(0.6501,0.3618)
\PST@Diamond(0.6576,0.3281)
\PST@Diamond(0.6651,0.3954)
\PST@Diamond(0.6726,0.3618)
\PST@Diamond(0.6801,0.3281)
\PST@Diamond(0.6876,0.3618)
\PST@Diamond(0.6951,0.3281)
\PST@Diamond(0.7027,0.3618)
\PST@Diamond(0.7102,0.3618)
\PST@Diamond(0.7177,0.2944)
\PST@Diamond(0.7252,0.3281)
\PST@Diamond(0.7327,0.3618)
\PST@Diamond(0.7402,0.3618)
\PST@Diamond(0.7477,0.3281)
\PST@Diamond(0.7552,0.3281)
\PST@Diamond(0.7627,0.2607)
\PST@Diamond(0.7702,0.3281)
\PST@Diamond(0.7778,0.2607)
\PST@Diamond(0.7853,0.3281)
\PST@Diamond(0.7928,0.2607)
\PST@Diamond(0.8003,0.1934)
\PST@Diamond(0.8078,0.3618)
\PST@Diamond(0.8153,0.2607)
\PST@Diamond(0.8228,0.2270)
\PST@Diamond(0.8303,0.1934)
\PST@Diamond(0.8378,0.3281)
\PST@Diamond(0.8453,0.2944)
\PST@Diamond(0.8529,0.2270)
\PST@Diamond(0.8604,0.2270)
\PST@Diamond(0.8679,0.2607)
\PST@Diamond(0.8754,0.2607)
\PST@Diamond(0.8829,0.2607)
\PST@Diamond(0.8904,0.1934)
\PST@Diamond(0.8979,0.2607)
\PST@Diamond(0.9054,0.1597)
\PST@Diamond(0.9129,0.2944)
\PST@Diamond(0.9204,0.1597)
\PST@Diamond(0.9280,0.2607)
\PST@Diamond(0.9355,0.1934)
\PST@Diamond(0.9430,0.1597)
\PST@Diamond(0.9505,0.2270)
\PST@Diamond(0.9580,0.1934)
\PST@Border(0.2070,0.9680)
(0.2070,0.1260)
(0.9580,0.1260)
(0.9580,0.9680)
(0.2070,0.9680)

\catcode`@=12
\fi
\endpspicture

\end{figure}
\begin{figure}[h]
\centering
\caption{Quarta serie di cento misure, cronometro automatico. Scarti $x_i - \bar{x}$ rispetto al numero di oscillazioni.}\label{scarti4}
 % GNUPLOT: LaTeX picture using PSTRICKS macros
% Define new PST objects, if not already defined
\ifx\PSTloaded\undefined
\def\PSTloaded{t}
\psset{arrowsize=.1 3.2 1.4 .3}
\psset{dotsize=.08}
\catcode`@=11

\newpsobject{PST@Border}{psline}{linewidth=.0015,linestyle=solid}
\newpsobject{PST@Axes}{psline}{linewidth=.0015,linestyle=dotted,dotsep=.004}
\newpsobject{PST@Solid}{psline}{linewidth=.0015,linestyle=solid}
\newpsobject{PST@Dashed}{psline}{linewidth=.0015,linestyle=dashed,dash=.01 .01}
\newpsobject{PST@Dotted}{psline}{linewidth=.0025,linestyle=dotted,dotsep=.008}
\newpsobject{PST@LongDash}{psline}{linewidth=.0015,linestyle=dashed,dash=.02 .01}
\newpsobject{PST@Diamond}{psdots}{linewidth=.001,linestyle=solid,dotstyle=*}
\newpsobject{PST@Filldiamond}{psdots}{linewidth=.001,linestyle=solid,dotstyle=square*,dotangle=45}
\newpsobject{PST@Cross}{psdots}{linewidth=.001,linestyle=solid,dotstyle=+,dotangle=45}
\newpsobject{PST@Plus}{psdots}{linewidth=.001,linestyle=solid,dotstyle=+}
\newpsobject{PST@Square}{psdots}{linewidth=.001,linestyle=solid,dotstyle=square}
\newpsobject{PST@Circle}{psdots}{linewidth=.001,linestyle=solid,dotstyle=o}
\newpsobject{PST@Triangle}{psdots}{linewidth=.001,linestyle=solid,dotstyle=triangle}
\newpsobject{PST@Pentagon}{psdots}{linewidth=.001,linestyle=solid,dotstyle=pentagon}
\newpsobject{PST@Fillsquare}{psdots}{linewidth=.001,linestyle=solid,dotstyle=square*}
\newpsobject{PST@Fillcircle}{psdots}{linewidth=.001,linestyle=solid,dotstyle=*}
\newpsobject{PST@Filltriangle}{psdots}{linewidth=.001,linestyle=solid,dotstyle=triangle*}
\newpsobject{PST@Fillpentagon}{psdots}{linewidth=.001,linestyle=solid,dotstyle=pentagon*}
\newpsobject{PST@Arrow}{psline}{linewidth=.001,linestyle=solid}
\catcode`@=12

\fi
\psset{unit=5.0in,xunit=5.0in,yunit=3.0in}
\pspicture(0.000000,0.000000)(1.000000,1.000000)
\ifx\nofigs\undefined
\catcode`@=11

\PST@Border(0.2070,0.1260)
(0.2220,0.1260)

\rput[r](0.1910,0.1260){-0.0008}
\PST@Border(0.2070,0.2196)
(0.2220,0.2196)

\rput[r](0.1910,0.2196){-0.0006}
\PST@Border(0.2070,0.3131)
(0.2220,0.3131)

\rput[r](0.1910,0.3131){-0.0004}
\PST@Border(0.2070,0.4067)
(0.2220,0.4067)

\rput[r](0.1910,0.4067){-0.0002}
\PST@Border(0.2070,0.5002)
(0.2220,0.5002)

\rput[r](0.1910,0.5002){0.0000}
\PST@Border(0.2070,0.5938)
(0.2220,0.5938)

\rput[r](0.1910,0.5938){0.0002}
\PST@Border(0.2070,0.6873)
(0.2220,0.6873)

\rput[r](0.1910,0.6873){0.0004}
\PST@Border(0.2070,0.7809)
(0.2220,0.7809)

\rput[r](0.1910,0.7809){0.0006}
\PST@Border(0.2070,0.8744)
(0.2220,0.8744)

\rput[r](0.1910,0.8744){0.0008}
\PST@Border(0.2070,0.9680)
(0.2220,0.9680)

\rput[r](0.1910,0.9680){0.0010}
\PST@Border(0.2070,0.1260)
(0.2070,0.1460)

\rput(0.2070,0.0840){ 0}
\PST@Border(0.2821,0.1260)
(0.2821,0.1460)

\rput(0.2821,0.0840){ 10}
\PST@Border(0.3572,0.1260)
(0.3572,0.1460)

\rput(0.3572,0.0840){ 20}
\PST@Border(0.4323,0.1260)
(0.4323,0.1460)

\rput(0.4323,0.0840){ 30}
\PST@Border(0.5074,0.1260)
(0.5074,0.1460)

\rput(0.5074,0.0840){ 40}
\PST@Border(0.5825,0.1260)
(0.5825,0.1460)

\rput(0.5825,0.0840){ 50}
\PST@Border(0.6576,0.1260)
(0.6576,0.1460)

\rput(0.6576,0.0840){ 60}
\PST@Border(0.7327,0.1260)
(0.7327,0.1460)

\rput(0.7327,0.0840){ 70}
\PST@Border(0.8078,0.1260)
(0.8078,0.1460)

\rput(0.8078,0.0840){ 80}
\PST@Border(0.8829,0.1260)
(0.8829,0.1460)

\rput(0.8829,0.0840){ 90}
\PST@Border(0.9580,0.1260)
(0.9580,0.1460)

\rput(0.9580,0.0840){ 100}
\PST@Border(0.2070,0.9680)
(0.2070,0.1260)
(0.9580,0.1260)
(0.9580,0.9680)
(0.2070,0.9680)

\rput{L}(0.0420,0.5470){scarti (\unit{s})}
\rput(0.5825,0.0210){oscillazioni}
\PST@Diamond(0.2145,0.9212)
\PST@Diamond(0.2220,0.9680)
\PST@Diamond(0.2295,0.9212)
\PST@Diamond(0.2370,0.9680)
\PST@Diamond(0.2445,0.9212)
\PST@Diamond(0.2521,0.8744)
\PST@Diamond(0.2596,0.9212)
\PST@Diamond(0.2671,0.8744)
\PST@Diamond(0.2746,0.8744)
\PST@Diamond(0.2821,0.8744)
\PST@Diamond(0.2896,0.8277)
\PST@Diamond(0.2971,0.8277)
\PST@Diamond(0.3046,0.8277)
\PST@Diamond(0.3121,0.8277)
\PST@Diamond(0.3196,0.7341)
\PST@Diamond(0.3272,0.8277)
\PST@Diamond(0.3347,0.8277)
\PST@Diamond(0.3422,0.7341)
\PST@Diamond(0.3497,0.7341)
\PST@Diamond(0.3572,0.7809)
\PST@Diamond(0.3647,0.6873)
\PST@Diamond(0.3722,0.6873)
\PST@Diamond(0.3797,0.6873)
\PST@Diamond(0.3872,0.7341)
\PST@Diamond(0.3947,0.5938)
\PST@Diamond(0.4023,0.5938)
\PST@Diamond(0.4098,0.5470)
\PST@Diamond(0.4173,0.5938)
\PST@Diamond(0.4248,0.5470)
\PST@Diamond(0.4323,0.4534)
\PST@Diamond(0.4398,0.5938)
\PST@Diamond(0.4473,0.4067)
\PST@Diamond(0.4548,0.4067)
\PST@Diamond(0.4623,0.4067)
\PST@Diamond(0.4698,0.4067)
\PST@Diamond(0.4774,0.4534)
\PST@Diamond(0.4849,0.5002)
\PST@Diamond(0.4924,0.4534)
\PST@Diamond(0.4999,0.4534)
\PST@Diamond(0.5074,0.4534)
\PST@Diamond(0.5149,0.3599)
\PST@Diamond(0.5224,0.4534)
\PST@Diamond(0.5299,0.4067)
\PST@Diamond(0.5374,0.5470)
\PST@Diamond(0.5449,0.4534)
\PST@Diamond(0.5525,0.3599)
\PST@Diamond(0.5600,0.4067)
\PST@Diamond(0.5675,0.4067)
\PST@Diamond(0.5750,0.4067)
\PST@Diamond(0.5825,0.4067)
\PST@Diamond(0.5900,0.4534)
\PST@Diamond(0.5975,0.3599)
\PST@Diamond(0.6050,0.4534)
\PST@Diamond(0.6125,0.3599)
\PST@Diamond(0.6200,0.3599)
\PST@Diamond(0.6276,0.4067)
\PST@Diamond(0.6351,0.4067)
\PST@Diamond(0.6426,0.3599)
\PST@Diamond(0.6501,0.3599)
\PST@Diamond(0.6576,0.3131)
\PST@Diamond(0.6651,0.4534)
\PST@Diamond(0.6726,0.3599)
\PST@Diamond(0.6801,0.3599)
\PST@Diamond(0.6876,0.3131)
\PST@Diamond(0.6951,0.3131)
\PST@Diamond(0.7027,0.4067)
\PST@Diamond(0.7102,0.2663)
\PST@Diamond(0.7177,0.4067)
\PST@Diamond(0.7252,0.3599)
\PST@Diamond(0.7327,0.2663)
\PST@Diamond(0.7402,0.3599)
\PST@Diamond(0.7477,0.3599)
\PST@Diamond(0.7552,0.2663)
\PST@Diamond(0.7627,0.3599)
\PST@Diamond(0.7702,0.3131)
\PST@Diamond(0.7778,0.2663)
\PST@Diamond(0.7853,0.3131)
\PST@Diamond(0.7928,0.3599)
\PST@Diamond(0.8003,0.2663)
\PST@Diamond(0.8078,0.4067)
\PST@Diamond(0.8153,0.3131)
\PST@Diamond(0.8228,0.3131)
\PST@Diamond(0.8303,0.3131)
\PST@Diamond(0.8378,0.2663)
\PST@Diamond(0.8453,0.3131)
\PST@Diamond(0.8529,0.2663)
\PST@Diamond(0.8604,0.2663)
\PST@Diamond(0.8679,0.3131)
\PST@Diamond(0.8754,0.3131)
\PST@Diamond(0.8829,0.2196)
\PST@Diamond(0.8904,0.2663)
\PST@Diamond(0.8979,0.3599)
\PST@Diamond(0.9054,0.2663)
\PST@Diamond(0.9129,0.2663)
\PST@Diamond(0.9204,0.2196)
\PST@Diamond(0.9280,0.3599)
\PST@Diamond(0.9355,0.1728)
\PST@Diamond(0.9430,0.2196)
\PST@Diamond(0.9505,0.1728)
\PST@Diamond(0.9580,0.2663)
\PST@Border(0.2070,0.9680)
(0.2070,0.1260)
(0.9580,0.1260)
(0.9580,0.9680)
(0.2070,0.9680)

\catcode`@=12
\fi
\endpspicture

\end{figure}
\begin{figure}[h]
\centering
\caption{Quinta serie di cento misure, cronometro automatico. Scarti $x_i - \bar{x}$ rispetto al numero di oscillazioni.}\label{scarti5}
 % GNUPLOT: LaTeX picture using PSTRICKS macros
% Define new PST objects, if not already defined
\ifx\PSTloaded\undefined
\def\PSTloaded{t}
\psset{arrowsize=.1 3.2 1.4 .3}
\psset{dotsize=.08}
\catcode`@=11

\newpsobject{PST@Border}{psline}{linewidth=.0015,linestyle=solid}
\newpsobject{PST@Axes}{psline}{linewidth=.0015,linestyle=dotted,dotsep=.004}
\newpsobject{PST@Solid}{psline}{linewidth=.0015,linestyle=solid}
\newpsobject{PST@Dashed}{psline}{linewidth=.0015,linestyle=dashed,dash=.01 .01}
\newpsobject{PST@Dotted}{psline}{linewidth=.0025,linestyle=dotted,dotsep=.008}
\newpsobject{PST@LongDash}{psline}{linewidth=.0015,linestyle=dashed,dash=.02 .01}
\newpsobject{PST@Diamond}{psdots}{linewidth=.001,linestyle=solid,dotstyle=*}
\newpsobject{PST@Filldiamond}{psdots}{linewidth=.001,linestyle=solid,dotstyle=square*,dotangle=45}
\newpsobject{PST@Cross}{psdots}{linewidth=.001,linestyle=solid,dotstyle=+,dotangle=45}
\newpsobject{PST@Plus}{psdots}{linewidth=.001,linestyle=solid,dotstyle=+}
\newpsobject{PST@Square}{psdots}{linewidth=.001,linestyle=solid,dotstyle=square}
\newpsobject{PST@Circle}{psdots}{linewidth=.001,linestyle=solid,dotstyle=o}
\newpsobject{PST@Triangle}{psdots}{linewidth=.001,linestyle=solid,dotstyle=triangle}
\newpsobject{PST@Pentagon}{psdots}{linewidth=.001,linestyle=solid,dotstyle=pentagon}
\newpsobject{PST@Fillsquare}{psdots}{linewidth=.001,linestyle=solid,dotstyle=square*}
\newpsobject{PST@Fillcircle}{psdots}{linewidth=.001,linestyle=solid,dotstyle=*}
\newpsobject{PST@Filltriangle}{psdots}{linewidth=.001,linestyle=solid,dotstyle=triangle*}
\newpsobject{PST@Fillpentagon}{psdots}{linewidth=.001,linestyle=solid,dotstyle=pentagon*}
\newpsobject{PST@Arrow}{psline}{linewidth=.001,linestyle=solid}
\catcode`@=12

\fi
\psset{unit=5.0in,xunit=5.0in,yunit=3.0in}
\pspicture(0.000000,0.000000)(1.000000,1.000000)
\ifx\nofigs\undefined
\catcode`@=11

\PST@Border(0.2070,0.1260)
(0.2220,0.1260)

\rput[r](0.1910,0.1260){-0.0010}
\PST@Border(0.2070,0.2196)
(0.2220,0.2196)

\rput[r](0.1910,0.2196){-0.0008}
\PST@Border(0.2070,0.3131)
(0.2220,0.3131)

\rput[r](0.1910,0.3131){-0.0006}
\PST@Border(0.2070,0.4067)
(0.2220,0.4067)

\rput[r](0.1910,0.4067){-0.0004}
\PST@Border(0.2070,0.5002)
(0.2220,0.5002)

\rput[r](0.1910,0.5002){-0.0002}
\PST@Border(0.2070,0.5938)
(0.2220,0.5938)

\rput[r](0.1910,0.5938){0.0000}
\PST@Border(0.2070,0.6873)
(0.2220,0.6873)

\rput[r](0.1910,0.6873){0.0002}
\PST@Border(0.2070,0.7809)
(0.2220,0.7809)

\rput[r](0.1910,0.7809){0.0004}
\PST@Border(0.2070,0.8744)
(0.2220,0.8744)

\rput[r](0.1910,0.8744){0.0006}
\PST@Border(0.2070,0.9680)
(0.2220,0.9680)

\rput[r](0.1910,0.9680){0.0008}
\PST@Border(0.2070,0.1260)
(0.2070,0.1460)

\rput(0.2070,0.0840){ 0}
\PST@Border(0.2821,0.1260)
(0.2821,0.1460)

\rput(0.2821,0.0840){ 10}
\PST@Border(0.3572,0.1260)
(0.3572,0.1460)

\rput(0.3572,0.0840){ 20}
\PST@Border(0.4323,0.1260)
(0.4323,0.1460)

\rput(0.4323,0.0840){ 30}
\PST@Border(0.5074,0.1260)
(0.5074,0.1460)

\rput(0.5074,0.0840){ 40}
\PST@Border(0.5825,0.1260)
(0.5825,0.1460)

\rput(0.5825,0.0840){ 50}
\PST@Border(0.6576,0.1260)
(0.6576,0.1460)

\rput(0.6576,0.0840){ 60}
\PST@Border(0.7327,0.1260)
(0.7327,0.1460)

\rput(0.7327,0.0840){ 70}
\PST@Border(0.8078,0.1260)
(0.8078,0.1460)

\rput(0.8078,0.0840){ 80}
\PST@Border(0.8829,0.1260)
(0.8829,0.1460)

\rput(0.8829,0.0840){ 90}
\PST@Border(0.9580,0.1260)
(0.9580,0.1460)

\rput(0.9580,0.0840){ 100}
\PST@Border(0.2070,0.9680)
(0.2070,0.1260)
(0.9580,0.1260)
(0.9580,0.9680)
(0.2070,0.9680)

\rput{L}(0.0420,0.5470){scarti (\unit{s})}
\rput(0.5825,0.0210){oscillazioni}
\PST@Diamond(0.2145,0.7809)
\PST@Diamond(0.2220,0.9680)
\PST@Diamond(0.2295,0.9212)
\PST@Diamond(0.2370,0.8744)
\PST@Diamond(0.2445,0.8744)
\PST@Diamond(0.2521,0.9212)
\PST@Diamond(0.2596,0.8744)
\PST@Diamond(0.2671,0.7809)
\PST@Diamond(0.2746,0.8744)
\PST@Diamond(0.2821,0.8277)
\PST@Diamond(0.2896,0.8277)
\PST@Diamond(0.2971,0.8277)
\PST@Diamond(0.3046,0.7809)
\PST@Diamond(0.3121,0.8277)
\PST@Diamond(0.3196,0.8744)
\PST@Diamond(0.3272,0.7809)
\PST@Diamond(0.3347,0.8277)
\PST@Diamond(0.3422,0.8277)
\PST@Diamond(0.3497,0.7809)
\PST@Diamond(0.3572,0.8277)
\PST@Diamond(0.3647,0.7809)
\PST@Diamond(0.3722,0.7809)
\PST@Diamond(0.3797,0.8277)
\PST@Diamond(0.3872,0.8277)
\PST@Diamond(0.3947,0.7809)
\PST@Diamond(0.4023,0.7809)
\PST@Diamond(0.4098,0.7809)
\PST@Diamond(0.4173,0.7809)
\PST@Diamond(0.4248,0.7341)
\PST@Diamond(0.4323,0.7341)
\PST@Diamond(0.4398,0.7809)
\PST@Diamond(0.4473,0.7809)
\PST@Diamond(0.4548,0.7809)
\PST@Diamond(0.4623,0.7341)
\PST@Diamond(0.4698,0.7341)
\PST@Diamond(0.4774,0.7341)
\PST@Diamond(0.4849,0.6873)
\PST@Diamond(0.4924,0.7341)
\PST@Diamond(0.4999,0.7341)
\PST@Diamond(0.5074,0.6873)
\PST@Diamond(0.5149,0.7341)
\PST@Diamond(0.5224,0.6873)
\PST@Diamond(0.5299,0.6873)
\PST@Diamond(0.5374,0.7341)
\PST@Diamond(0.5449,0.6406)
\PST@Diamond(0.5525,0.6406)
\PST@Diamond(0.5600,0.7341)
\PST@Diamond(0.5675,0.6406)
\PST@Diamond(0.5750,0.6873)
\PST@Diamond(0.5825,0.6873)
\PST@Diamond(0.5900,0.6873)
\PST@Diamond(0.5975,0.6406)
\PST@Diamond(0.6050,0.6406)
\PST@Diamond(0.6125,0.6873)
\PST@Diamond(0.6200,0.5938)
\PST@Diamond(0.6276,0.6406)
\PST@Diamond(0.6351,0.6406)
\PST@Diamond(0.6426,0.6406)
\PST@Diamond(0.6501,0.6406)
\PST@Diamond(0.6576,0.5938)
\PST@Diamond(0.6651,0.5938)
\PST@Diamond(0.6726,0.5938)
\PST@Diamond(0.6801,0.5470)
\PST@Diamond(0.6876,0.5470)
\PST@Diamond(0.6951,0.5470)
\PST@Diamond(0.7027,0.5002)
\PST@Diamond(0.7102,0.5470)
\PST@Diamond(0.7177,0.5470)
\PST@Diamond(0.7252,0.5002)
\PST@Diamond(0.7327,0.5002)
\PST@Diamond(0.7402,0.5002)
\PST@Diamond(0.7477,0.5002)
\PST@Diamond(0.7552,0.5002)
\PST@Diamond(0.7627,0.4534)
\PST@Diamond(0.7702,0.5470)
\PST@Diamond(0.7778,0.4067)
\PST@Diamond(0.7853,0.5002)
\PST@Diamond(0.7928,0.4067)
\PST@Diamond(0.8003,0.4534)
\PST@Diamond(0.8078,0.4067)
\PST@Diamond(0.8153,0.4067)
\PST@Diamond(0.8228,0.4067)
\PST@Diamond(0.8303,0.4067)
\PST@Diamond(0.8378,0.3599)
\PST@Diamond(0.8453,0.3599)
\PST@Diamond(0.8529,0.4067)
\PST@Diamond(0.8604,0.2663)
\PST@Diamond(0.8679,0.3131)
\PST@Diamond(0.8754,0.3131)
\PST@Diamond(0.8829,0.3131)
\PST@Diamond(0.8904,0.2663)
\PST@Diamond(0.8979,0.2663)
\PST@Diamond(0.9054,0.2196)
\PST@Diamond(0.9129,0.2196)
\PST@Diamond(0.9204,0.2663)
\PST@Diamond(0.9280,0.1728)
\PST@Diamond(0.9355,0.1728)
\PST@Diamond(0.9430,0.2196)
\PST@Diamond(0.9505,0.2196)
\PST@Diamond(0.9580,0.1260)
\PST@Border(0.2070,0.9680)
(0.2070,0.1260)
(0.9580,0.1260)
(0.9580,0.9680)
(0.2070,0.9680)

\catcode`@=12
\fi
\endpspicture

\end{figure}
\end{document}
