\documentclass[italian,a4paper]{article}
\usepackage[tight,nice]{units}
\usepackage{babel,amsmath,amssymb,amsthm,graphicx,url}
\usepackage[text={6in,9in},centering]{geometry}
\usepackage[utf8x]{inputenc}
\usepackage[T1]{fontenc}
\usepackage{ae,aecompl}
\usepackage[Euler]{upgreek}
\usepackage[footnotesize,bf]{caption}
\usepackage[usenames]{color}
\include{pstricks}
\frenchspacing
\pagestyle{plain}
%------------- eliminare prime e ultime linee isolate
\clubpenalty=9999%
\widowpenalty=9999
%--- definizione numerazioni
\renewcommand{\theequation}{\thesection.\arabic{equation}}
\renewcommand{\thefigure}{\thesection.\arabic{figure}}
\renewcommand{\thetable}{\thesection.\arabic{table}}
\addto\captionsitalian{%
  \renewcommand{\figurename}%
{Grafico}%
}
%
%------------- ridefinizione simbolo per elenchi puntati: en dash
%\renewcommand{\labelitemi}{\textbf{--}}
\renewcommand{\labelenumi}{\textbf{\arabic{enumi}.}}
\setlength{\abovecaptionskip}{\baselineskip}   % 0.5cm as an example
\setlength{\floatsep}{2\baselineskip}
\setlength{\belowcaptionskip}{\baselineskip}   % 0.5cm as an example
%------------- nuovi environment senza spazi
%\newenvironment{packed_item}{
%\begin{itemize}
%  \setlength{\itemsep}{1pt}
%  \setlength{\parskip}{0pt}
%  \setlength{\parsep}{0pt}
%}{\end{itemize}}
%\newenvironment{packed_enum}{
%\begin{enumerate}
%  \setlength{\itemsep}{1pt}
%  \setlength{\parskip}{0pt}
%  \setlength{\parsep}{0pt}
%}{\end{enumerate}}
%\newenvironment{packed_description}{
%\begin{enumerate}
%   \setlength{\itemsep}{1pt}
%   \setlength{\parskip}{0pt}
%   \setlength{\parsep}{0pt}
% }{\end{enumerate}}
%--------- comandi insiemi numeri complessi, naturali, reali e altre abbreviazioni
\newcommand{\micro}{\ensuremath{\upmu}} %prefisso micro
\newcommand{\e}{\mathrm{e}} %numero di nepero
\newcommand{\di}{\mathrm{d}} %simbolo di differenziale
\renewcommand{\leq}{\leqslant}
\renewcommand{\pi}{\uppi} % costante pi greco
\renewcommand{\tau}{\uptau} %momento della forza
\newcommand{\coloneqq}{\mathrel{\mathop:}=} % := ``per definizione''
\newcommand{\ms}{(\unitfrac{m}{s})}
%--------- porzione dedicata ai float in una pagina:
\renewcommand{\textfraction}{0.05}
\renewcommand{\topfraction}{0.95}
\renewcommand{\bottomfraction}{0.95}
\renewcommand{\floatpagefraction}{0.35}
\setcounter{totalnumber}{5}
%---------
%
%---------
\begin{document}
\title{Relazione di laboratorio: pendolo di Kater}
\author{\normalsize Ilaria Brivio (582116)\\%
\normalsize \url{brivio.ilaria@tiscali.it}%
\and %
\normalsize Matteo Abis (584206)\\ %
\normalsize \url{webmaster@latinblog.org}}
\date{\today}
\maketitle
%------------------
\section{Obiettivo dell'esperienza}
Obiettivo dell'esperienza è stimare nel modo più accurato il valore dell'accelerazione di gravità $g$, confrontandolo con il valore atteso di $\unitfrac[9.806\pm 0.001]{m}{s^2}$.
\section{Descrizione dell'apparato strumentale}
Lo strumento è un pendolo fisico, costituito da una barra di metallo con due masse, una fissa e una mobile. Il pendolo può oscillare sostenuto da due coltelli rispetto a due assi $O$ e $O'$ paralleli tra loro, separati da una distanza $OO'=\unit[994.5\pm0.2]{mm}$. \`E possibile posizionare la massa mobile in modo che i periodi rispetto agli assi $O$ e $O'$ siano uguali, e determinare questa posizione sarà il primo obiettivo dell'esperienza. Come errore statistico sulla posizione della massa lungo la scala graduata si assume $\delta/3 = \unit[0.3]{mm}$, dove $\delta$ è l'errore massimo dovuto alla sensibilità dello strumento. Le misure sono state effettuate con un cronometro manuale di sensibilità $\unit[10^3]{s^{-1}}$ e uno automatico di sensibilità $\unit[10^4]{s^{-1}}$.

\section{Descrizione della metodologia di misura}
Si è posizionata la massa mobile $m$ a distanza $10, 20, \dots, \unit[90]{cm}$ dall'asse $O$, ed è stato misurato con cronometro manuale il tempo impiegato a compiere dieci oscillazioni consecutive rispetto a ciascun asse. Con un semplice calcolo trigonometrico si è fissata un'ampiezza massima di oscillazione di circa $5^\circ$ in modo da ridurre gli errori sistematici introdotti dall'approssimazione per angoli piccoli. \`E stata riportata in grafico la durata delle dieci oscillazioni contro la posizione della massa lungo la scala graduata (vedi grafico~\ref{course}). L'intersezione delle due curve è stata individuata per interpolazione dei due soli punti più vicini. Prendiamo questa prima approssimazione $x_1$. \`E stato fatto uno ``zoom'' a passi di \unit[1]{cm}, posizionando $m$ in $x_1 - \unit[2]{cm},\dots,x_1 + \unit[2]{cm}$ e misurando 50 periodi di oscillazione. Si è determinato con la stessa procedura un punto $x_2$. Un'altra iterazione con passi di $\unit[0.5]{cm}$, ovvero con la massa in posizione $x_2 - \unit[0.5]{cm},x_2,x_2 + \unit[0.5]{cm}$ per cento oscillazioni con cronometro automatico ha fornito, attraverso un'interpolazione lineare, il valore definitivo $x_f$. Una volta posizionata la massa in $x_f$ sono state osservate cinque serie di cento oscillazioni consecutive, sempre con il cronometro automatico. La media pesata dei valori medi di ciascuna serie ha determinato il periodo $\bar{t}$ usato per calcolare l'accelerazione $g$.
\section{Risultati sperimentali ed elaborazione dati}
Come noto, il periodo del pendolo è $t=2\pi\sqrt{\ell_r/g}$. Da questa si potrà ricavare $g=\frac{4\pi^2\ell_r}{t^2}$. Bisogna solo stimare la lunghezza ridotta $\ell_r$. Sia $h$ la distanza tra $m$ (massa mobile) e $O$, $h'$ la distanza tra $m$ e $O'$. La lunghezza ridotta sarà dunque $\ell_r^{(O)} = I_c/mh +h$ rispetto ad $O$ e $\ell_r^{(O')} = I_c/mh' +h'$. Uguagliando le due si ottengono due soluzioni: $h=h'$ e $h=\frac{I_c}{mh'}$. La prima significa che il baricentro è punto medio di $OO'$, cosa che nel nostro caso non accade, quindi dobbiamo considerare la seconda. Il significato fisico di questa soluzione è $OO' = h + h' = \ell_r^{(O)} = \ell_r^{(O')}$ che fornisce quindi una facile stima per la lunghezza ridotta.
\begin{table}[h]\caption{misure rispettivamente di dieci, cinquanta e cento oscillazioni (\unit{s}), con massa mobile nelle posizioni indicate (\unit{cm}) e risultati delle interpolazioni sotto ciascuna serie.}\label{cinzia}
\centering
\begin{tabular}{rcccrcccrcc}
 $x$	&$O$	&$O'$	&	&$x_1$	&$O$	&$O'$	&	&$x_2$	&$O$	&$O'$\\\cline{1-3}\cline{5-7}\cline{9-11}
10	&21.787	&20.339	&	&-2	&102.091&100.462&	&-0.5	&200.364&200.207\\
20	&20.074	&20.019	&	&-1	&101.416&100.305&	&0	&199.800&199.924\\
30	&19.263	&19.831	&	&0	&100.773&100.179&	&0.5	&199.321&199.799\\
40	&19.027	&19.636	&	&1	&100.245&100.030&	&	&	& \\
50	&18.993	&19.450	&	&2	&99.730	&99.910	&	&\multicolumn{3}{c}{$x_f=\unit[21.32\pm0.07]{cm}$}\\
60	&19.135	&19.535	&	&	&	&	&	&	&	& \\
70	&19.597	&19.727	&	&\multicolumn{3}{c}{$x_2=\unit[21.5]{cm}$}&	&	&	& \\
80	&20.154	&20.101	&	&	&	&	&	&	&	& \\
90	&20.737	&20.531	&	&	&	&	&	&	&	& \\
\\
\multicolumn{3}{c}{$x_1=\unit[20]{cm}$}
\end{tabular}
\end{table}\\
I dati ottenuti nell'ultima fase ($x_2$) sono abbastanza fitti da giustificare un'interpolazione lineare $t=ax+b$ rispetto ai tre punti per $O$ e $t=cx+d$ rispetto a $O'$. La posizione definitiva sarà quindi $x_f = \frac{b-d}{c-a}$ con errore calcolato per propagazione:
\begin{equation*}
 \sigma_{x_f}^2=\left(\dfrac{1}{c-a}\right)^2\left[\sigma_b^2+\sigma_d^2+\left(\dfrac{b-d}{c-a}\right)^2(\sigma_a^2+\sigma_c^2)\right]
\end{equation*}
Dall'interpolazione risulta:
\begin{table}[h]
\centering
 \begin{tabular}{cc}
  $O$ &  $O'$ \\\hline
$a=-1.039\pm0.047$ & $c=-0.410\pm0.087$\\
$b=199.827\pm0.019$ & $d=199.970\pm0.035$\\
 \end{tabular}
\end{table}\\
Da cui il valore riportato in tabella~\ref{cinzia} come $x_f$. Fissando la massa mobile in questa posizione, sono state misurate con il cronometro automatico cinque serie di cento oscillazioni consecutive (vedi tabella~\ref{auto}). Le medie delle cinque serie sono:
\begin{table}[h]\caption{medie con errore sulla media delle cinque serie (\unit{s}).}\label{medie}
\centering
\begin{tabular}{cr@{$\pm$}l}
 serie & $t$&$\sigma_{t}$\\\hline
 1 & 1.99980&0.00005\\
 2 & 2.00054&0.00007\\
 3 & 2.00019&0.00006\\
 4 & 2.00026&0.00005\\
 5 & 2.00044&0.00004\\
\end{tabular}
\end{table}\\
La media pesata dei cinque valori è:
\begin{equation*}
 \bar{t}=\left(\sum_i \dfrac{t_i}{\sigma_{t_i}^2} \right)\left(\sum_i \dfrac{1}{\sigma_{t_i}^2} \right)^{-1} \qquad \sigma_{\bar{t}} = \left(\sum_i \dfrac{1}{\sigma_{t_i}^2} \right)^{-\nicefrac 1 2}
\end{equation*}
\begin{equation*}
\bar{t} = \unit[2.00023\pm0.00002]{s}\\
\end{equation*}
Utilizzando questo valore per il calcolo di $g$:
\begin{equation*}
g=\frac{4\pi^2\ell_r}{t^2} = \unitfrac[9.8091\pm0.0020]{m}{s^2}
\end{equation*}
Che ha compatibilità 1.4 con il valore atteso a Padova $g=9.806\pm0.001$.
\section{Analisi degli errori sistematici}
Sebbene la compatibilità di $g$ con il valore atteso si possa ritenere soddisfacente, le medie riportate in tabella~\ref{medie} sono scarsamente compatibili tra loro. Ciò fa pensare ad un'importante influenza di errori sistematici.  L'errore sistematico principale è la dipendenza del periodo dall'ampiezza massima di oscillazione $\alpha$. Infatti, la formula esatta per il periodo è:
\begin{equation*}
 t=2\pi\sqrt{\dfrac{\ell_r}{g}}I(\alpha) \qquad \text{dove} \qquad I(\alpha) =\dfrac 2 \pi \int_0^{\alpha} \dfrac{\di x}{\sqrt{\cos x - \cos \alpha}}
\end{equation*}
Che tende a 1 per $\alpha \ll \unit[1]{rad}$. Per $\alpha = 5^\circ$, come nel nostro caso, si ha, da un calcolo numerico:
\begin{equation*}
\dfrac 2 \pi \int_0^{5^\circ} \dfrac{\di x}{\sqrt{\cos x - \cos 5^\circ}} = 1.0004058\dots
\end{equation*}
Quindi la formula usata nei calcoli al paragrafo precedente ($I(\alpha) \equiv 1$) risulta buona a meno di un'approssimazione dell'ordine di $4\cdot10^{-4}$. Poiché l'errore statistico sulle misure automatiche risulta dello stesso ordine di grandezza, dovrebbe essere possibile osservare questo errore sistematico nel corso dell'esperimento. A tal fine sono stati riportati in grafico gli scarti dalla media (con segno) sulle serie di cento oscillazioni. Da questi è evidente la distribuzione non casuale degli errori e, di conseguenza, un'influenza significativa dell'errore sistematico.
Altre fonti di errore sistematico derivano dalla presenza del coltello su cui oscilla il pendolo e dall'attrito viscoso, ma si possono quantificare in un ordine di grandezza di $10^{-5}$ e sono perciò trascurabili a meno di approssimazioni del 10\%.
\begin{figure}[hp]
\centering
\caption{Prima serie di cento misure, cronometro automatico. Scarti $t_i - \bar{t}$ rispetto al numero di oscillazioni.}\label{scarti1}
 % GNUPLOT: LaTeX picture using PSTRICKS macros
% Define new PST objects, if not already defined
\ifx\PSTloaded\undefined
\def\PSTloaded{t}
\psset{arrowsize=.1 3.2 1.4 .3}
\psset{dotsize=.08}
\catcode`@=11

\newpsobject{PST@Border}{psline}{linewidth=.0015,linestyle=solid}
\newpsobject{PST@Axes}{psline}{linewidth=.0015,linestyle=dotted,dotsep=.004}
\newpsobject{PST@Solid}{psline}{linewidth=.0015,linestyle=solid}
\newpsobject{PST@Dashed}{psline}{linewidth=.0015,linestyle=dashed,dash=.01 .01}
\newpsobject{PST@Dotted}{psline}{linewidth=.0025,linestyle=dotted,dotsep=.008}
\newpsobject{PST@LongDash}{psline}{linewidth=.0015,linestyle=dashed,dash=.02 .01}
\newpsobject{PST@Diamond}{psdots}{linewidth=.001,linestyle=solid,dotstyle=*}
\newpsobject{PST@Filldiamond}{psdots}{linewidth=.001,linestyle=solid,dotstyle=square*,dotangle=45}
\newpsobject{PST@Cross}{psdots}{linewidth=.001,linestyle=solid,dotstyle=+,dotangle=45}
\newpsobject{PST@Plus}{psdots}{linewidth=.001,linestyle=solid,dotstyle=+}
\newpsobject{PST@Square}{psdots}{linewidth=.001,linestyle=solid,dotstyle=square}
\newpsobject{PST@Circle}{psdots}{linewidth=.001,linestyle=solid,dotstyle=o}
\newpsobject{PST@Triangle}{psdots}{linewidth=.001,linestyle=solid,dotstyle=triangle}
\newpsobject{PST@Pentagon}{psdots}{linewidth=.001,linestyle=solid,dotstyle=pentagon}
\newpsobject{PST@Fillsquare}{psdots}{linewidth=.001,linestyle=solid,dotstyle=square*}
\newpsobject{PST@Fillcircle}{psdots}{linewidth=.001,linestyle=solid,dotstyle=*}
\newpsobject{PST@Filltriangle}{psdots}{linewidth=.001,linestyle=solid,dotstyle=triangle*}
\newpsobject{PST@Fillpentagon}{psdots}{linewidth=.001,linestyle=solid,dotstyle=pentagon*}
\newpsobject{PST@Arrow}{psline}{linewidth=.001,linestyle=solid}
\catcode`@=12

\fi
\psset{unit=5.0in,xunit=5.0in,yunit=3.0in}
\pspicture(0.000000,0.000000)(1.000000,1.000000)
\ifx\nofigs\undefined
\catcode`@=11

\PST@Border(0.2070,0.1260)
(0.2220,0.1260)

\rput[r](0.1910,0.1260){-0.0008}
\PST@Border(0.2070,0.2102)
(0.2220,0.2102)

\rput[r](0.1910,0.2102){-0.0006}
\PST@Border(0.2070,0.2944)
(0.2220,0.2944)

\rput[r](0.1910,0.2944){-0.0004}
\PST@Border(0.2070,0.3786)
(0.2220,0.3786)

\rput[r](0.1910,0.3786){-0.0002}
\PST@Border(0.2070,0.4628)
(0.2220,0.4628)

\rput[r](0.1910,0.4628){0.0000}
\PST@Border(0.2070,0.5470)
(0.2220,0.5470)

\rput[r](0.1910,0.5470){0.0002}
\PST@Border(0.2070,0.6312)
(0.2220,0.6312)

\rput[r](0.1910,0.6312){0.0004}
\PST@Border(0.2070,0.7154)
(0.2220,0.7154)

\rput[r](0.1910,0.7154){0.0006}
\PST@Border(0.2070,0.7996)
(0.2220,0.7996)

\rput[r](0.1910,0.7996){0.0008}
\PST@Border(0.2070,0.8838)
(0.2220,0.8838)

\rput[r](0.1910,0.8838){0.0010}
\PST@Border(0.2070,0.9680)
(0.2220,0.9680)

\rput[r](0.1910,0.9680){0.0012}
\PST@Border(0.2070,0.1260)
(0.2070,0.1460)

\rput(0.2070,0.0840){ 0}
\PST@Border(0.2821,0.1260)
(0.2821,0.1460)

\rput(0.2821,0.0840){ 10}
\PST@Border(0.3572,0.1260)
(0.3572,0.1460)

\rput(0.3572,0.0840){ 20}
\PST@Border(0.4323,0.1260)
(0.4323,0.1460)

\rput(0.4323,0.0840){ 30}
\PST@Border(0.5074,0.1260)
(0.5074,0.1460)

\rput(0.5074,0.0840){ 40}
\PST@Border(0.5825,0.1260)
(0.5825,0.1460)

\rput(0.5825,0.0840){ 50}
\PST@Border(0.6576,0.1260)
(0.6576,0.1460)

\rput(0.6576,0.0840){ 60}
\PST@Border(0.7327,0.1260)
(0.7327,0.1460)

\rput(0.7327,0.0840){ 70}
\PST@Border(0.8078,0.1260)
(0.8078,0.1460)

\rput(0.8078,0.0840){ 80}
\PST@Border(0.8829,0.1260)
(0.8829,0.1460)

\rput(0.8829,0.0840){ 90}
\PST@Border(0.9580,0.1260)
(0.9580,0.1460)

\rput(0.9580,0.0840){ 100}
\PST@Border(0.2070,0.9680)
(0.2070,0.1260)
(0.9580,0.1260)
(0.9580,0.9680)
(0.2070,0.9680)

\rput{L}(0.0420,0.5470){scarti (\unit{s})}
\rput(0.5825,0.0210){oscillazioni}
\PST@Diamond(0.2145,0.9259)
\PST@Diamond(0.2220,0.9259)
\PST@Diamond(0.2295,0.9259)
\PST@Diamond(0.2370,0.9259)
\PST@Diamond(0.2445,0.9259)
\PST@Diamond(0.2521,0.8417)
\PST@Diamond(0.2596,0.8417)
\PST@Diamond(0.2671,0.8417)
\PST@Diamond(0.2746,0.7996)
\PST@Diamond(0.2821,0.8417)
\PST@Diamond(0.2896,0.7996)
\PST@Diamond(0.2971,0.7996)
\PST@Diamond(0.3046,0.7575)
\PST@Diamond(0.3121,0.7575)
\PST@Diamond(0.3196,0.7154)
\PST@Diamond(0.3272,0.7154)
\PST@Diamond(0.3347,0.7154)
\PST@Diamond(0.3422,0.7575)
\PST@Diamond(0.3497,0.7154)
\PST@Diamond(0.3572,0.6733)
\PST@Diamond(0.3647,0.6733)
\PST@Diamond(0.3722,0.7154)
\PST@Diamond(0.3797,0.6733)
\PST@Diamond(0.3872,0.6312)
\PST@Diamond(0.3947,0.6312)
\PST@Diamond(0.4023,0.6312)
\PST@Diamond(0.4098,0.5891)
\PST@Diamond(0.4173,0.6312)
\PST@Diamond(0.4248,0.5470)
\PST@Diamond(0.4323,0.5470)
\PST@Diamond(0.4398,0.5470)
\PST@Diamond(0.4473,0.5891)
\PST@Diamond(0.4548,0.5470)
\PST@Diamond(0.4623,0.5049)
\PST@Diamond(0.4698,0.5470)
\PST@Diamond(0.4774,0.5049)
\PST@Diamond(0.4849,0.4628)
\PST@Diamond(0.4924,0.4628)
\PST@Diamond(0.4999,0.5049)
\PST@Diamond(0.5074,0.4207)
\PST@Diamond(0.5149,0.4207)
\PST@Diamond(0.5224,0.4628)
\PST@Diamond(0.5299,0.4207)
\PST@Diamond(0.5374,0.3786)
\PST@Diamond(0.5449,0.4207)
\PST@Diamond(0.5525,0.4207)
\PST@Diamond(0.5600,0.3365)
\PST@Diamond(0.5675,0.3786)
\PST@Diamond(0.5750,0.3786)
\PST@Diamond(0.5825,0.3365)
\PST@Diamond(0.5900,0.3365)
\PST@Diamond(0.5975,0.3365)
\PST@Diamond(0.6050,0.2944)
\PST@Diamond(0.6125,0.3365)
\PST@Diamond(0.6200,0.2944)
\PST@Diamond(0.6276,0.2944)
\PST@Diamond(0.6351,0.3365)
\PST@Diamond(0.6426,0.2944)
\PST@Diamond(0.6501,0.2944)
\PST@Diamond(0.6576,0.3786)
\PST@Diamond(0.6651,0.2944)
\PST@Diamond(0.6726,0.2944)
\PST@Diamond(0.6801,0.3365)
\PST@Diamond(0.6876,0.2944)
\PST@Diamond(0.6951,0.3365)
\PST@Diamond(0.7027,0.3365)
\PST@Diamond(0.7102,0.2944)
\PST@Diamond(0.7177,0.3365)
\PST@Diamond(0.7252,0.3365)
\PST@Diamond(0.7327,0.3365)
\PST@Diamond(0.7402,0.2944)
\PST@Diamond(0.7477,0.2523)
\PST@Diamond(0.7552,0.3786)
\PST@Diamond(0.7627,0.3365)
\PST@Diamond(0.7702,0.2523)
\PST@Diamond(0.7778,0.3365)
\PST@Diamond(0.7853,0.2523)
\PST@Diamond(0.7928,0.3365)
\PST@Diamond(0.8003,0.2944)
\PST@Diamond(0.8078,0.2523)
\PST@Diamond(0.8153,0.2944)
\PST@Diamond(0.8228,0.2944)
\PST@Diamond(0.8303,0.2944)
\PST@Diamond(0.8378,0.2944)
\PST@Diamond(0.8453,0.2523)
\PST@Diamond(0.8529,0.2944)
\PST@Diamond(0.8604,0.2944)
\PST@Diamond(0.8679,0.2102)
\PST@Diamond(0.8754,0.3786)
\PST@Diamond(0.8829,0.2523)
\PST@Diamond(0.8904,0.2523)
\PST@Diamond(0.8979,0.2102)
\PST@Diamond(0.9054,0.2944)
\PST@Diamond(0.9129,0.2102)
\PST@Diamond(0.9204,0.2523)
\PST@Diamond(0.9280,0.1681)
\PST@Diamond(0.9355,0.2523)
\PST@Diamond(0.9430,0.2523)
\PST@Diamond(0.9505,0.2102)
\PST@Diamond(0.9580,0.2523)
\PST@Border(0.2070,0.9680)
(0.2070,0.1260)
(0.9580,0.1260)
(0.9580,0.9680)
(0.2070,0.9680)

\catcode`@=12
\fi
\endpspicture

\end{figure}
\begin{figure}[hp]
\centering
\caption{Seconda serie di cento misure, cronometro automatico. Scarti $t_i - \bar{t}$ rispetto al numero di oscillazioni.}\label{scarti2}
 % GNUPLOT: LaTeX picture using PSTRICKS macros
% Define new PST objects, if not already defined
\ifx\PSTloaded\undefined
\def\PSTloaded{t}
\psset{arrowsize=.1 3.2 1.4 .3}
\psset{dotsize=.08}
\catcode`@=11

\newpsobject{PST@Border}{psline}{linewidth=.0015,linestyle=solid}
\newpsobject{PST@Axes}{psline}{linewidth=.0015,linestyle=dotted,dotsep=.004}
\newpsobject{PST@Solid}{psline}{linewidth=.0015,linestyle=solid}
\newpsobject{PST@Dashed}{psline}{linewidth=.0015,linestyle=dashed,dash=.01 .01}
\newpsobject{PST@Dotted}{psline}{linewidth=.0025,linestyle=dotted,dotsep=.008}
\newpsobject{PST@LongDash}{psline}{linewidth=.0015,linestyle=dashed,dash=.02 .01}
\newpsobject{PST@Diamond}{psdots}{linewidth=.001,linestyle=solid,dotstyle=*}
\newpsobject{PST@Filldiamond}{psdots}{linewidth=.001,linestyle=solid,dotstyle=square*,dotangle=45}
\newpsobject{PST@Cross}{psdots}{linewidth=.001,linestyle=solid,dotstyle=+,dotangle=45}
\newpsobject{PST@Plus}{psdots}{linewidth=.001,linestyle=solid,dotstyle=+}
\newpsobject{PST@Square}{psdots}{linewidth=.001,linestyle=solid,dotstyle=square}
\newpsobject{PST@Circle}{psdots}{linewidth=.001,linestyle=solid,dotstyle=o}
\newpsobject{PST@Triangle}{psdots}{linewidth=.001,linestyle=solid,dotstyle=triangle}
\newpsobject{PST@Pentagon}{psdots}{linewidth=.001,linestyle=solid,dotstyle=pentagon}
\newpsobject{PST@Fillsquare}{psdots}{linewidth=.001,linestyle=solid,dotstyle=square*}
\newpsobject{PST@Fillcircle}{psdots}{linewidth=.001,linestyle=solid,dotstyle=*}
\newpsobject{PST@Filltriangle}{psdots}{linewidth=.001,linestyle=solid,dotstyle=triangle*}
\newpsobject{PST@Fillpentagon}{psdots}{linewidth=.001,linestyle=solid,dotstyle=pentagon*}
\newpsobject{PST@Arrow}{psline}{linewidth=.001,linestyle=solid}
\catcode`@=12

\fi
\psset{unit=5.0in,xunit=5.0in,yunit=3.0in}
\pspicture(0.000000,0.000000)(1.000000,1.000000)
\ifx\nofigs\undefined
\catcode`@=11

\PST@Border(0.2070,0.1260)
(0.2220,0.1260)

\rput[r](0.1910,0.1260){-0.0010}
\PST@Border(0.2070,0.2944)
(0.2220,0.2944)

\rput[r](0.1910,0.2944){-0.0005}
\PST@Border(0.2070,0.4628)
(0.2220,0.4628)

\rput[r](0.1910,0.4628){0.0000}
\PST@Border(0.2070,0.6312)
(0.2220,0.6312)

\rput[r](0.1910,0.6312){0.0005}
\PST@Border(0.2070,0.7996)
(0.2220,0.7996)

\rput[r](0.1910,0.7996){0.0010}
\PST@Border(0.2070,0.9680)
(0.2220,0.9680)

\rput[r](0.1910,0.9680){0.0015}
\PST@Border(0.2070,0.1260)
(0.2070,0.1460)

\rput(0.2070,0.0840){ 0}
\PST@Border(0.2821,0.1260)
(0.2821,0.1460)

\rput(0.2821,0.0840){ 10}
\PST@Border(0.3572,0.1260)
(0.3572,0.1460)

\rput(0.3572,0.0840){ 20}
\PST@Border(0.4323,0.1260)
(0.4323,0.1460)

\rput(0.4323,0.0840){ 30}
\PST@Border(0.5074,0.1260)
(0.5074,0.1460)

\rput(0.5074,0.0840){ 40}
\PST@Border(0.5825,0.1260)
(0.5825,0.1460)

\rput(0.5825,0.0840){ 50}
\PST@Border(0.6576,0.1260)
(0.6576,0.1460)

\rput(0.6576,0.0840){ 60}
\PST@Border(0.7327,0.1260)
(0.7327,0.1460)

\rput(0.7327,0.0840){ 70}
\PST@Border(0.8078,0.1260)
(0.8078,0.1460)

\rput(0.8078,0.0840){ 80}
\PST@Border(0.8829,0.1260)
(0.8829,0.1460)

\rput(0.8829,0.0840){ 90}
\PST@Border(0.9580,0.1260)
(0.9580,0.1460)

\rput(0.9580,0.0840){ 100}
\PST@Border(0.2070,0.9680)
(0.2070,0.1260)
(0.9580,0.1260)
(0.9580,0.9680)
(0.2070,0.9680)

\rput{L}(0.0420,0.5470){scarti (\unit{s})}
\rput(0.5825,0.0210){oscillazioni}
\PST@Diamond(0.2145,0.8333)
\PST@Diamond(0.2220,0.8333)
\PST@Diamond(0.2295,0.8333)
\PST@Diamond(0.2370,0.8333)
\PST@Diamond(0.2445,0.8333)
\PST@Diamond(0.2521,0.7996)
\PST@Diamond(0.2596,0.7659)
\PST@Diamond(0.2671,0.7996)
\PST@Diamond(0.2746,0.7996)
\PST@Diamond(0.2821,0.8333)
\PST@Diamond(0.2896,0.7322)
\PST@Diamond(0.2971,0.7659)
\PST@Diamond(0.3046,0.7996)
\PST@Diamond(0.3121,0.7659)
\PST@Diamond(0.3196,0.6986)
\PST@Diamond(0.3272,0.7659)
\PST@Diamond(0.3347,0.7322)
\PST@Diamond(0.3422,0.7659)
\PST@Diamond(0.3497,0.6986)
\PST@Diamond(0.3572,0.7322)
\PST@Diamond(0.3647,0.7322)
\PST@Diamond(0.3722,0.7322)
\PST@Diamond(0.3797,0.8333)
\PST@Diamond(0.3872,0.6649)
\PST@Diamond(0.3947,0.6649)
\PST@Diamond(0.4023,0.6649)
\PST@Diamond(0.4098,0.6986)
\PST@Diamond(0.4173,0.6649)
\PST@Diamond(0.4248,0.7322)
\PST@Diamond(0.4323,0.5975)
\PST@Diamond(0.4398,0.6649)
\PST@Diamond(0.4473,0.6649)
\PST@Diamond(0.4548,0.6649)
\PST@Diamond(0.4623,0.6312)
\PST@Diamond(0.4698,0.6649)
\PST@Diamond(0.4774,0.6649)
\PST@Diamond(0.4849,0.6312)
\PST@Diamond(0.4924,0.5638)
\PST@Diamond(0.4999,0.5975)
\PST@Diamond(0.5074,0.5975)
\PST@Diamond(0.5149,0.5975)
\PST@Diamond(0.5224,0.5638)
\PST@Diamond(0.5299,0.5975)
\PST@Diamond(0.5374,0.5638)
\PST@Diamond(0.5449,0.5638)
\PST@Diamond(0.5525,0.5302)
\PST@Diamond(0.5600,0.5638)
\PST@Diamond(0.5675,0.5302)
\PST@Diamond(0.5750,0.5302)
\PST@Diamond(0.5825,0.4965)
\PST@Diamond(0.5900,0.4291)
\PST@Diamond(0.5975,0.4965)
\PST@Diamond(0.6050,0.4291)
\PST@Diamond(0.6125,0.4291)
\PST@Diamond(0.6200,0.3618)
\PST@Diamond(0.6276,0.4291)
\PST@Diamond(0.6351,0.4291)
\PST@Diamond(0.6426,0.3618)
\PST@Diamond(0.6501,0.3618)
\PST@Diamond(0.6576,0.3618)
\PST@Diamond(0.6651,0.2944)
\PST@Diamond(0.6726,0.3618)
\PST@Diamond(0.6801,0.3281)
\PST@Diamond(0.6876,0.2607)
\PST@Diamond(0.6951,0.3281)
\PST@Diamond(0.7027,0.1260)
\PST@Diamond(0.7102,0.3954)
\PST@Diamond(0.7177,0.2607)
\PST@Diamond(0.7252,0.1934)
\PST@Diamond(0.7327,0.3281)
\PST@Diamond(0.7402,0.2270)
\PST@Diamond(0.7477,0.2270)
\PST@Diamond(0.7552,0.2270)
\PST@Diamond(0.7627,0.2607)
\PST@Diamond(0.7702,0.2270)
\PST@Diamond(0.7778,0.2944)
\PST@Diamond(0.7853,0.2270)
\PST@Diamond(0.7928,0.2270)
\PST@Diamond(0.8003,0.2270)
\PST@Diamond(0.8078,0.1934)
\PST@Diamond(0.8153,0.2270)
\PST@Diamond(0.8228,0.1597)
\PST@Diamond(0.8303,0.2944)
\PST@Diamond(0.8378,0.2270)
\PST@Diamond(0.8453,0.2270)
\PST@Diamond(0.8529,0.1934)
\PST@Diamond(0.8604,0.2270)
\PST@Diamond(0.8679,0.1260)
\PST@Diamond(0.8754,0.2607)
\PST@Diamond(0.8829,0.1934)
\PST@Diamond(0.8904,0.1597)
\PST@Diamond(0.8979,0.1934)
\PST@Diamond(0.9054,0.2270)
\PST@Diamond(0.9129,0.1597)
\PST@Diamond(0.9204,0.1934)
\PST@Diamond(0.9280,0.1597)
\PST@Diamond(0.9355,0.2270)
\PST@Diamond(0.9430,0.1260)
\PST@Diamond(0.9505,0.1934)
\PST@Diamond(0.9580,0.1934)
\PST@Border(0.2070,0.9680)
(0.2070,0.1260)
(0.9580,0.1260)
(0.9580,0.9680)
(0.2070,0.9680)

\catcode`@=12
\fi
\endpspicture

\end{figure}
\begin{figure}[hp]
\centering
\caption{Terza serie di cento misure, cronometro automatico. Scarti $t_i - \bar{t}$ rispetto al numero di oscillazioni.}\label{scarti3}
 % GNUPLOT: LaTeX picture using PSTRICKS macros
% Define new PST objects, if not already defined
\ifx\PSTloaded\undefined
\def\PSTloaded{t}
\psset{arrowsize=.1 3.2 1.4 .3}
\psset{dotsize=.08}
\catcode`@=11

\newpsobject{PST@Border}{psline}{linewidth=.0015,linestyle=solid}
\newpsobject{PST@Axes}{psline}{linewidth=.0015,linestyle=dotted,dotsep=.004}
\newpsobject{PST@Solid}{psline}{linewidth=.0015,linestyle=solid}
\newpsobject{PST@Dashed}{psline}{linewidth=.0015,linestyle=dashed,dash=.01 .01}
\newpsobject{PST@Dotted}{psline}{linewidth=.0025,linestyle=dotted,dotsep=.008}
\newpsobject{PST@LongDash}{psline}{linewidth=.0015,linestyle=dashed,dash=.02 .01}
\newpsobject{PST@Diamond}{psdots}{linewidth=.001,linestyle=solid,dotstyle=*}
\newpsobject{PST@Filldiamond}{psdots}{linewidth=.001,linestyle=solid,dotstyle=square*,dotangle=45}
\newpsobject{PST@Cross}{psdots}{linewidth=.001,linestyle=solid,dotstyle=+,dotangle=45}
\newpsobject{PST@Plus}{psdots}{linewidth=.001,linestyle=solid,dotstyle=+}
\newpsobject{PST@Square}{psdots}{linewidth=.001,linestyle=solid,dotstyle=square}
\newpsobject{PST@Circle}{psdots}{linewidth=.001,linestyle=solid,dotstyle=o}
\newpsobject{PST@Triangle}{psdots}{linewidth=.001,linestyle=solid,dotstyle=triangle}
\newpsobject{PST@Pentagon}{psdots}{linewidth=.001,linestyle=solid,dotstyle=pentagon}
\newpsobject{PST@Fillsquare}{psdots}{linewidth=.001,linestyle=solid,dotstyle=square*}
\newpsobject{PST@Fillcircle}{psdots}{linewidth=.001,linestyle=solid,dotstyle=*}
\newpsobject{PST@Filltriangle}{psdots}{linewidth=.001,linestyle=solid,dotstyle=triangle*}
\newpsobject{PST@Fillpentagon}{psdots}{linewidth=.001,linestyle=solid,dotstyle=pentagon*}
\newpsobject{PST@Arrow}{psline}{linewidth=.001,linestyle=solid}
\catcode`@=12

\fi
\psset{unit=5.0in,xunit=5.0in,yunit=3.0in}
\pspicture(0.000000,0.000000)(1.000000,1.000000)
\ifx\nofigs\undefined
\catcode`@=11

\PST@Border(0.2070,0.1260)
(0.2220,0.1260)

\rput[r](0.1910,0.1260){-0.0010}
\PST@Border(0.2070,0.2944)
(0.2220,0.2944)

\rput[r](0.1910,0.2944){-0.0005}
\PST@Border(0.2070,0.4628)
(0.2220,0.4628)

\rput[r](0.1910,0.4628){0.0000}
\PST@Border(0.2070,0.6312)
(0.2220,0.6312)

\rput[r](0.1910,0.6312){0.0005}
\PST@Border(0.2070,0.7996)
(0.2220,0.7996)

\rput[r](0.1910,0.7996){0.0010}
\PST@Border(0.2070,0.9680)
(0.2220,0.9680)

\rput[r](0.1910,0.9680){0.0015}
\PST@Border(0.2070,0.1260)
(0.2070,0.1460)

\rput(0.2070,0.0840){ 0}
\PST@Border(0.2821,0.1260)
(0.2821,0.1460)

\rput(0.2821,0.0840){ 10}
\PST@Border(0.3572,0.1260)
(0.3572,0.1460)

\rput(0.3572,0.0840){ 20}
\PST@Border(0.4323,0.1260)
(0.4323,0.1460)

\rput(0.4323,0.0840){ 30}
\PST@Border(0.5074,0.1260)
(0.5074,0.1460)

\rput(0.5074,0.0840){ 40}
\PST@Border(0.5825,0.1260)
(0.5825,0.1460)

\rput(0.5825,0.0840){ 50}
\PST@Border(0.6576,0.1260)
(0.6576,0.1460)

\rput(0.6576,0.0840){ 60}
\PST@Border(0.7327,0.1260)
(0.7327,0.1460)

\rput(0.7327,0.0840){ 70}
\PST@Border(0.8078,0.1260)
(0.8078,0.1460)

\rput(0.8078,0.0840){ 80}
\PST@Border(0.8829,0.1260)
(0.8829,0.1460)

\rput(0.8829,0.0840){ 90}
\PST@Border(0.9580,0.1260)
(0.9580,0.1460)

\rput(0.9580,0.0840){ 100}
\PST@Border(0.2070,0.9680)
(0.2070,0.1260)
(0.9580,0.1260)
(0.9580,0.9680)
(0.2070,0.9680)

\rput{L}(0.0420,0.5470){scarti (\unit{s})}
\rput(0.5825,0.0210){oscillazioni}
\PST@Diamond(0.2145,0.8670)
\PST@Diamond(0.2220,0.8333)
\PST@Diamond(0.2295,0.7996)
\PST@Diamond(0.2370,0.8333)
\PST@Diamond(0.2445,0.8670)
\PST@Diamond(0.2521,0.7659)
\PST@Diamond(0.2596,0.8333)
\PST@Diamond(0.2671,0.8333)
\PST@Diamond(0.2746,0.7659)
\PST@Diamond(0.2821,0.7659)
\PST@Diamond(0.2896,0.7659)
\PST@Diamond(0.2971,0.7659)
\PST@Diamond(0.3046,0.7322)
\PST@Diamond(0.3121,0.7659)
\PST@Diamond(0.3196,0.7659)
\PST@Diamond(0.3272,0.7322)
\PST@Diamond(0.3347,0.6986)
\PST@Diamond(0.3422,0.6649)
\PST@Diamond(0.3497,0.7322)
\PST@Diamond(0.3572,0.6649)
\PST@Diamond(0.3647,0.5975)
\PST@Diamond(0.3722,0.6312)
\PST@Diamond(0.3797,0.5975)
\PST@Diamond(0.3872,0.6312)
\PST@Diamond(0.3947,0.6312)
\PST@Diamond(0.4023,0.5638)
\PST@Diamond(0.4098,0.5975)
\PST@Diamond(0.4173,0.5975)
\PST@Diamond(0.4248,0.5302)
\PST@Diamond(0.4323,0.5975)
\PST@Diamond(0.4398,0.5638)
\PST@Diamond(0.4473,0.5302)
\PST@Diamond(0.4548,0.4965)
\PST@Diamond(0.4623,0.4965)
\PST@Diamond(0.4698,0.4965)
\PST@Diamond(0.4774,0.5302)
\PST@Diamond(0.4849,0.4291)
\PST@Diamond(0.4924,0.4291)
\PST@Diamond(0.4999,0.5302)
\PST@Diamond(0.5074,0.4291)
\PST@Diamond(0.5149,0.4965)
\PST@Diamond(0.5224,0.4628)
\PST@Diamond(0.5299,0.4628)
\PST@Diamond(0.5374,0.4628)
\PST@Diamond(0.5449,0.4291)
\PST@Diamond(0.5525,0.4628)
\PST@Diamond(0.5600,0.3954)
\PST@Diamond(0.5675,0.4291)
\PST@Diamond(0.5750,0.4291)
\PST@Diamond(0.5825,0.3954)
\PST@Diamond(0.5900,0.3954)
\PST@Diamond(0.5975,0.4291)
\PST@Diamond(0.6050,0.4291)
\PST@Diamond(0.6125,0.3954)
\PST@Diamond(0.6200,0.4291)
\PST@Diamond(0.6276,0.3281)
\PST@Diamond(0.6351,0.4291)
\PST@Diamond(0.6426,0.3954)
\PST@Diamond(0.6501,0.3618)
\PST@Diamond(0.6576,0.3281)
\PST@Diamond(0.6651,0.3954)
\PST@Diamond(0.6726,0.3618)
\PST@Diamond(0.6801,0.3281)
\PST@Diamond(0.6876,0.3618)
\PST@Diamond(0.6951,0.3281)
\PST@Diamond(0.7027,0.3618)
\PST@Diamond(0.7102,0.3618)
\PST@Diamond(0.7177,0.2944)
\PST@Diamond(0.7252,0.3281)
\PST@Diamond(0.7327,0.3618)
\PST@Diamond(0.7402,0.3618)
\PST@Diamond(0.7477,0.3281)
\PST@Diamond(0.7552,0.3281)
\PST@Diamond(0.7627,0.2607)
\PST@Diamond(0.7702,0.3281)
\PST@Diamond(0.7778,0.2607)
\PST@Diamond(0.7853,0.3281)
\PST@Diamond(0.7928,0.2607)
\PST@Diamond(0.8003,0.1934)
\PST@Diamond(0.8078,0.3618)
\PST@Diamond(0.8153,0.2607)
\PST@Diamond(0.8228,0.2270)
\PST@Diamond(0.8303,0.1934)
\PST@Diamond(0.8378,0.3281)
\PST@Diamond(0.8453,0.2944)
\PST@Diamond(0.8529,0.2270)
\PST@Diamond(0.8604,0.2270)
\PST@Diamond(0.8679,0.2607)
\PST@Diamond(0.8754,0.2607)
\PST@Diamond(0.8829,0.2607)
\PST@Diamond(0.8904,0.1934)
\PST@Diamond(0.8979,0.2607)
\PST@Diamond(0.9054,0.1597)
\PST@Diamond(0.9129,0.2944)
\PST@Diamond(0.9204,0.1597)
\PST@Diamond(0.9280,0.2607)
\PST@Diamond(0.9355,0.1934)
\PST@Diamond(0.9430,0.1597)
\PST@Diamond(0.9505,0.2270)
\PST@Diamond(0.9580,0.1934)
\PST@Border(0.2070,0.9680)
(0.2070,0.1260)
(0.9580,0.1260)
(0.9580,0.9680)
(0.2070,0.9680)

\catcode`@=12
\fi
\endpspicture

\end{figure}
\begin{figure}[hp]
\centering
\caption{Quarta serie di cento misure, cronometro automatico. Scarti $t_i - \bar{t}$ rispetto al numero di oscillazioni.}\label{scarti4}
 % GNUPLOT: LaTeX picture using PSTRICKS macros
% Define new PST objects, if not already defined
\ifx\PSTloaded\undefined
\def\PSTloaded{t}
\psset{arrowsize=.1 3.2 1.4 .3}
\psset{dotsize=.08}
\catcode`@=11

\newpsobject{PST@Border}{psline}{linewidth=.0015,linestyle=solid}
\newpsobject{PST@Axes}{psline}{linewidth=.0015,linestyle=dotted,dotsep=.004}
\newpsobject{PST@Solid}{psline}{linewidth=.0015,linestyle=solid}
\newpsobject{PST@Dashed}{psline}{linewidth=.0015,linestyle=dashed,dash=.01 .01}
\newpsobject{PST@Dotted}{psline}{linewidth=.0025,linestyle=dotted,dotsep=.008}
\newpsobject{PST@LongDash}{psline}{linewidth=.0015,linestyle=dashed,dash=.02 .01}
\newpsobject{PST@Diamond}{psdots}{linewidth=.001,linestyle=solid,dotstyle=*}
\newpsobject{PST@Filldiamond}{psdots}{linewidth=.001,linestyle=solid,dotstyle=square*,dotangle=45}
\newpsobject{PST@Cross}{psdots}{linewidth=.001,linestyle=solid,dotstyle=+,dotangle=45}
\newpsobject{PST@Plus}{psdots}{linewidth=.001,linestyle=solid,dotstyle=+}
\newpsobject{PST@Square}{psdots}{linewidth=.001,linestyle=solid,dotstyle=square}
\newpsobject{PST@Circle}{psdots}{linewidth=.001,linestyle=solid,dotstyle=o}
\newpsobject{PST@Triangle}{psdots}{linewidth=.001,linestyle=solid,dotstyle=triangle}
\newpsobject{PST@Pentagon}{psdots}{linewidth=.001,linestyle=solid,dotstyle=pentagon}
\newpsobject{PST@Fillsquare}{psdots}{linewidth=.001,linestyle=solid,dotstyle=square*}
\newpsobject{PST@Fillcircle}{psdots}{linewidth=.001,linestyle=solid,dotstyle=*}
\newpsobject{PST@Filltriangle}{psdots}{linewidth=.001,linestyle=solid,dotstyle=triangle*}
\newpsobject{PST@Fillpentagon}{psdots}{linewidth=.001,linestyle=solid,dotstyle=pentagon*}
\newpsobject{PST@Arrow}{psline}{linewidth=.001,linestyle=solid}
\catcode`@=12

\fi
\psset{unit=5.0in,xunit=5.0in,yunit=3.0in}
\pspicture(0.000000,0.000000)(1.000000,1.000000)
\ifx\nofigs\undefined
\catcode`@=11

\PST@Border(0.2070,0.1260)
(0.2220,0.1260)

\rput[r](0.1910,0.1260){-0.0008}
\PST@Border(0.2070,0.2196)
(0.2220,0.2196)

\rput[r](0.1910,0.2196){-0.0006}
\PST@Border(0.2070,0.3131)
(0.2220,0.3131)

\rput[r](0.1910,0.3131){-0.0004}
\PST@Border(0.2070,0.4067)
(0.2220,0.4067)

\rput[r](0.1910,0.4067){-0.0002}
\PST@Border(0.2070,0.5002)
(0.2220,0.5002)

\rput[r](0.1910,0.5002){0.0000}
\PST@Border(0.2070,0.5938)
(0.2220,0.5938)

\rput[r](0.1910,0.5938){0.0002}
\PST@Border(0.2070,0.6873)
(0.2220,0.6873)

\rput[r](0.1910,0.6873){0.0004}
\PST@Border(0.2070,0.7809)
(0.2220,0.7809)

\rput[r](0.1910,0.7809){0.0006}
\PST@Border(0.2070,0.8744)
(0.2220,0.8744)

\rput[r](0.1910,0.8744){0.0008}
\PST@Border(0.2070,0.9680)
(0.2220,0.9680)

\rput[r](0.1910,0.9680){0.0010}
\PST@Border(0.2070,0.1260)
(0.2070,0.1460)

\rput(0.2070,0.0840){ 0}
\PST@Border(0.2821,0.1260)
(0.2821,0.1460)

\rput(0.2821,0.0840){ 10}
\PST@Border(0.3572,0.1260)
(0.3572,0.1460)

\rput(0.3572,0.0840){ 20}
\PST@Border(0.4323,0.1260)
(0.4323,0.1460)

\rput(0.4323,0.0840){ 30}
\PST@Border(0.5074,0.1260)
(0.5074,0.1460)

\rput(0.5074,0.0840){ 40}
\PST@Border(0.5825,0.1260)
(0.5825,0.1460)

\rput(0.5825,0.0840){ 50}
\PST@Border(0.6576,0.1260)
(0.6576,0.1460)

\rput(0.6576,0.0840){ 60}
\PST@Border(0.7327,0.1260)
(0.7327,0.1460)

\rput(0.7327,0.0840){ 70}
\PST@Border(0.8078,0.1260)
(0.8078,0.1460)

\rput(0.8078,0.0840){ 80}
\PST@Border(0.8829,0.1260)
(0.8829,0.1460)

\rput(0.8829,0.0840){ 90}
\PST@Border(0.9580,0.1260)
(0.9580,0.1460)

\rput(0.9580,0.0840){ 100}
\PST@Border(0.2070,0.9680)
(0.2070,0.1260)
(0.9580,0.1260)
(0.9580,0.9680)
(0.2070,0.9680)

\rput{L}(0.0420,0.5470){scarti (\unit{s})}
\rput(0.5825,0.0210){oscillazioni}
\PST@Diamond(0.2145,0.9212)
\PST@Diamond(0.2220,0.9680)
\PST@Diamond(0.2295,0.9212)
\PST@Diamond(0.2370,0.9680)
\PST@Diamond(0.2445,0.9212)
\PST@Diamond(0.2521,0.8744)
\PST@Diamond(0.2596,0.9212)
\PST@Diamond(0.2671,0.8744)
\PST@Diamond(0.2746,0.8744)
\PST@Diamond(0.2821,0.8744)
\PST@Diamond(0.2896,0.8277)
\PST@Diamond(0.2971,0.8277)
\PST@Diamond(0.3046,0.8277)
\PST@Diamond(0.3121,0.8277)
\PST@Diamond(0.3196,0.7341)
\PST@Diamond(0.3272,0.8277)
\PST@Diamond(0.3347,0.8277)
\PST@Diamond(0.3422,0.7341)
\PST@Diamond(0.3497,0.7341)
\PST@Diamond(0.3572,0.7809)
\PST@Diamond(0.3647,0.6873)
\PST@Diamond(0.3722,0.6873)
\PST@Diamond(0.3797,0.6873)
\PST@Diamond(0.3872,0.7341)
\PST@Diamond(0.3947,0.5938)
\PST@Diamond(0.4023,0.5938)
\PST@Diamond(0.4098,0.5470)
\PST@Diamond(0.4173,0.5938)
\PST@Diamond(0.4248,0.5470)
\PST@Diamond(0.4323,0.4534)
\PST@Diamond(0.4398,0.5938)
\PST@Diamond(0.4473,0.4067)
\PST@Diamond(0.4548,0.4067)
\PST@Diamond(0.4623,0.4067)
\PST@Diamond(0.4698,0.4067)
\PST@Diamond(0.4774,0.4534)
\PST@Diamond(0.4849,0.5002)
\PST@Diamond(0.4924,0.4534)
\PST@Diamond(0.4999,0.4534)
\PST@Diamond(0.5074,0.4534)
\PST@Diamond(0.5149,0.3599)
\PST@Diamond(0.5224,0.4534)
\PST@Diamond(0.5299,0.4067)
\PST@Diamond(0.5374,0.5470)
\PST@Diamond(0.5449,0.4534)
\PST@Diamond(0.5525,0.3599)
\PST@Diamond(0.5600,0.4067)
\PST@Diamond(0.5675,0.4067)
\PST@Diamond(0.5750,0.4067)
\PST@Diamond(0.5825,0.4067)
\PST@Diamond(0.5900,0.4534)
\PST@Diamond(0.5975,0.3599)
\PST@Diamond(0.6050,0.4534)
\PST@Diamond(0.6125,0.3599)
\PST@Diamond(0.6200,0.3599)
\PST@Diamond(0.6276,0.4067)
\PST@Diamond(0.6351,0.4067)
\PST@Diamond(0.6426,0.3599)
\PST@Diamond(0.6501,0.3599)
\PST@Diamond(0.6576,0.3131)
\PST@Diamond(0.6651,0.4534)
\PST@Diamond(0.6726,0.3599)
\PST@Diamond(0.6801,0.3599)
\PST@Diamond(0.6876,0.3131)
\PST@Diamond(0.6951,0.3131)
\PST@Diamond(0.7027,0.4067)
\PST@Diamond(0.7102,0.2663)
\PST@Diamond(0.7177,0.4067)
\PST@Diamond(0.7252,0.3599)
\PST@Diamond(0.7327,0.2663)
\PST@Diamond(0.7402,0.3599)
\PST@Diamond(0.7477,0.3599)
\PST@Diamond(0.7552,0.2663)
\PST@Diamond(0.7627,0.3599)
\PST@Diamond(0.7702,0.3131)
\PST@Diamond(0.7778,0.2663)
\PST@Diamond(0.7853,0.3131)
\PST@Diamond(0.7928,0.3599)
\PST@Diamond(0.8003,0.2663)
\PST@Diamond(0.8078,0.4067)
\PST@Diamond(0.8153,0.3131)
\PST@Diamond(0.8228,0.3131)
\PST@Diamond(0.8303,0.3131)
\PST@Diamond(0.8378,0.2663)
\PST@Diamond(0.8453,0.3131)
\PST@Diamond(0.8529,0.2663)
\PST@Diamond(0.8604,0.2663)
\PST@Diamond(0.8679,0.3131)
\PST@Diamond(0.8754,0.3131)
\PST@Diamond(0.8829,0.2196)
\PST@Diamond(0.8904,0.2663)
\PST@Diamond(0.8979,0.3599)
\PST@Diamond(0.9054,0.2663)
\PST@Diamond(0.9129,0.2663)
\PST@Diamond(0.9204,0.2196)
\PST@Diamond(0.9280,0.3599)
\PST@Diamond(0.9355,0.1728)
\PST@Diamond(0.9430,0.2196)
\PST@Diamond(0.9505,0.1728)
\PST@Diamond(0.9580,0.2663)
\PST@Border(0.2070,0.9680)
(0.2070,0.1260)
(0.9580,0.1260)
(0.9580,0.9680)
(0.2070,0.9680)

\catcode`@=12
\fi
\endpspicture

\end{figure}
\begin{figure}[hp]
\centering
\caption{Quinta serie di cento misure, cronometro automatico. Scarti $t_i - \bar{t}$ rispetto al numero di oscillazioni.}\label{scarti5}
 % GNUPLOT: LaTeX picture using PSTRICKS macros
% Define new PST objects, if not already defined
\ifx\PSTloaded\undefined
\def\PSTloaded{t}
\psset{arrowsize=.1 3.2 1.4 .3}
\psset{dotsize=.08}
\catcode`@=11

\newpsobject{PST@Border}{psline}{linewidth=.0015,linestyle=solid}
\newpsobject{PST@Axes}{psline}{linewidth=.0015,linestyle=dotted,dotsep=.004}
\newpsobject{PST@Solid}{psline}{linewidth=.0015,linestyle=solid}
\newpsobject{PST@Dashed}{psline}{linewidth=.0015,linestyle=dashed,dash=.01 .01}
\newpsobject{PST@Dotted}{psline}{linewidth=.0025,linestyle=dotted,dotsep=.008}
\newpsobject{PST@LongDash}{psline}{linewidth=.0015,linestyle=dashed,dash=.02 .01}
\newpsobject{PST@Diamond}{psdots}{linewidth=.001,linestyle=solid,dotstyle=*}
\newpsobject{PST@Filldiamond}{psdots}{linewidth=.001,linestyle=solid,dotstyle=square*,dotangle=45}
\newpsobject{PST@Cross}{psdots}{linewidth=.001,linestyle=solid,dotstyle=+,dotangle=45}
\newpsobject{PST@Plus}{psdots}{linewidth=.001,linestyle=solid,dotstyle=+}
\newpsobject{PST@Square}{psdots}{linewidth=.001,linestyle=solid,dotstyle=square}
\newpsobject{PST@Circle}{psdots}{linewidth=.001,linestyle=solid,dotstyle=o}
\newpsobject{PST@Triangle}{psdots}{linewidth=.001,linestyle=solid,dotstyle=triangle}
\newpsobject{PST@Pentagon}{psdots}{linewidth=.001,linestyle=solid,dotstyle=pentagon}
\newpsobject{PST@Fillsquare}{psdots}{linewidth=.001,linestyle=solid,dotstyle=square*}
\newpsobject{PST@Fillcircle}{psdots}{linewidth=.001,linestyle=solid,dotstyle=*}
\newpsobject{PST@Filltriangle}{psdots}{linewidth=.001,linestyle=solid,dotstyle=triangle*}
\newpsobject{PST@Fillpentagon}{psdots}{linewidth=.001,linestyle=solid,dotstyle=pentagon*}
\newpsobject{PST@Arrow}{psline}{linewidth=.001,linestyle=solid}
\catcode`@=12

\fi
\psset{unit=5.0in,xunit=5.0in,yunit=3.0in}
\pspicture(0.000000,0.000000)(1.000000,1.000000)
\ifx\nofigs\undefined
\catcode`@=11

\PST@Border(0.2070,0.1260)
(0.2220,0.1260)

\rput[r](0.1910,0.1260){-0.0010}
\PST@Border(0.2070,0.2196)
(0.2220,0.2196)

\rput[r](0.1910,0.2196){-0.0008}
\PST@Border(0.2070,0.3131)
(0.2220,0.3131)

\rput[r](0.1910,0.3131){-0.0006}
\PST@Border(0.2070,0.4067)
(0.2220,0.4067)

\rput[r](0.1910,0.4067){-0.0004}
\PST@Border(0.2070,0.5002)
(0.2220,0.5002)

\rput[r](0.1910,0.5002){-0.0002}
\PST@Border(0.2070,0.5938)
(0.2220,0.5938)

\rput[r](0.1910,0.5938){0.0000}
\PST@Border(0.2070,0.6873)
(0.2220,0.6873)

\rput[r](0.1910,0.6873){0.0002}
\PST@Border(0.2070,0.7809)
(0.2220,0.7809)

\rput[r](0.1910,0.7809){0.0004}
\PST@Border(0.2070,0.8744)
(0.2220,0.8744)

\rput[r](0.1910,0.8744){0.0006}
\PST@Border(0.2070,0.9680)
(0.2220,0.9680)

\rput[r](0.1910,0.9680){0.0008}
\PST@Border(0.2070,0.1260)
(0.2070,0.1460)

\rput(0.2070,0.0840){ 0}
\PST@Border(0.2821,0.1260)
(0.2821,0.1460)

\rput(0.2821,0.0840){ 10}
\PST@Border(0.3572,0.1260)
(0.3572,0.1460)

\rput(0.3572,0.0840){ 20}
\PST@Border(0.4323,0.1260)
(0.4323,0.1460)

\rput(0.4323,0.0840){ 30}
\PST@Border(0.5074,0.1260)
(0.5074,0.1460)

\rput(0.5074,0.0840){ 40}
\PST@Border(0.5825,0.1260)
(0.5825,0.1460)

\rput(0.5825,0.0840){ 50}
\PST@Border(0.6576,0.1260)
(0.6576,0.1460)

\rput(0.6576,0.0840){ 60}
\PST@Border(0.7327,0.1260)
(0.7327,0.1460)

\rput(0.7327,0.0840){ 70}
\PST@Border(0.8078,0.1260)
(0.8078,0.1460)

\rput(0.8078,0.0840){ 80}
\PST@Border(0.8829,0.1260)
(0.8829,0.1460)

\rput(0.8829,0.0840){ 90}
\PST@Border(0.9580,0.1260)
(0.9580,0.1460)

\rput(0.9580,0.0840){ 100}
\PST@Border(0.2070,0.9680)
(0.2070,0.1260)
(0.9580,0.1260)
(0.9580,0.9680)
(0.2070,0.9680)

\rput{L}(0.0420,0.5470){scarti (\unit{s})}
\rput(0.5825,0.0210){oscillazioni}
\PST@Diamond(0.2145,0.7809)
\PST@Diamond(0.2220,0.9680)
\PST@Diamond(0.2295,0.9212)
\PST@Diamond(0.2370,0.8744)
\PST@Diamond(0.2445,0.8744)
\PST@Diamond(0.2521,0.9212)
\PST@Diamond(0.2596,0.8744)
\PST@Diamond(0.2671,0.7809)
\PST@Diamond(0.2746,0.8744)
\PST@Diamond(0.2821,0.8277)
\PST@Diamond(0.2896,0.8277)
\PST@Diamond(0.2971,0.8277)
\PST@Diamond(0.3046,0.7809)
\PST@Diamond(0.3121,0.8277)
\PST@Diamond(0.3196,0.8744)
\PST@Diamond(0.3272,0.7809)
\PST@Diamond(0.3347,0.8277)
\PST@Diamond(0.3422,0.8277)
\PST@Diamond(0.3497,0.7809)
\PST@Diamond(0.3572,0.8277)
\PST@Diamond(0.3647,0.7809)
\PST@Diamond(0.3722,0.7809)
\PST@Diamond(0.3797,0.8277)
\PST@Diamond(0.3872,0.8277)
\PST@Diamond(0.3947,0.7809)
\PST@Diamond(0.4023,0.7809)
\PST@Diamond(0.4098,0.7809)
\PST@Diamond(0.4173,0.7809)
\PST@Diamond(0.4248,0.7341)
\PST@Diamond(0.4323,0.7341)
\PST@Diamond(0.4398,0.7809)
\PST@Diamond(0.4473,0.7809)
\PST@Diamond(0.4548,0.7809)
\PST@Diamond(0.4623,0.7341)
\PST@Diamond(0.4698,0.7341)
\PST@Diamond(0.4774,0.7341)
\PST@Diamond(0.4849,0.6873)
\PST@Diamond(0.4924,0.7341)
\PST@Diamond(0.4999,0.7341)
\PST@Diamond(0.5074,0.6873)
\PST@Diamond(0.5149,0.7341)
\PST@Diamond(0.5224,0.6873)
\PST@Diamond(0.5299,0.6873)
\PST@Diamond(0.5374,0.7341)
\PST@Diamond(0.5449,0.6406)
\PST@Diamond(0.5525,0.6406)
\PST@Diamond(0.5600,0.7341)
\PST@Diamond(0.5675,0.6406)
\PST@Diamond(0.5750,0.6873)
\PST@Diamond(0.5825,0.6873)
\PST@Diamond(0.5900,0.6873)
\PST@Diamond(0.5975,0.6406)
\PST@Diamond(0.6050,0.6406)
\PST@Diamond(0.6125,0.6873)
\PST@Diamond(0.6200,0.5938)
\PST@Diamond(0.6276,0.6406)
\PST@Diamond(0.6351,0.6406)
\PST@Diamond(0.6426,0.6406)
\PST@Diamond(0.6501,0.6406)
\PST@Diamond(0.6576,0.5938)
\PST@Diamond(0.6651,0.5938)
\PST@Diamond(0.6726,0.5938)
\PST@Diamond(0.6801,0.5470)
\PST@Diamond(0.6876,0.5470)
\PST@Diamond(0.6951,0.5470)
\PST@Diamond(0.7027,0.5002)
\PST@Diamond(0.7102,0.5470)
\PST@Diamond(0.7177,0.5470)
\PST@Diamond(0.7252,0.5002)
\PST@Diamond(0.7327,0.5002)
\PST@Diamond(0.7402,0.5002)
\PST@Diamond(0.7477,0.5002)
\PST@Diamond(0.7552,0.5002)
\PST@Diamond(0.7627,0.4534)
\PST@Diamond(0.7702,0.5470)
\PST@Diamond(0.7778,0.4067)
\PST@Diamond(0.7853,0.5002)
\PST@Diamond(0.7928,0.4067)
\PST@Diamond(0.8003,0.4534)
\PST@Diamond(0.8078,0.4067)
\PST@Diamond(0.8153,0.4067)
\PST@Diamond(0.8228,0.4067)
\PST@Diamond(0.8303,0.4067)
\PST@Diamond(0.8378,0.3599)
\PST@Diamond(0.8453,0.3599)
\PST@Diamond(0.8529,0.4067)
\PST@Diamond(0.8604,0.2663)
\PST@Diamond(0.8679,0.3131)
\PST@Diamond(0.8754,0.3131)
\PST@Diamond(0.8829,0.3131)
\PST@Diamond(0.8904,0.2663)
\PST@Diamond(0.8979,0.2663)
\PST@Diamond(0.9054,0.2196)
\PST@Diamond(0.9129,0.2196)
\PST@Diamond(0.9204,0.2663)
\PST@Diamond(0.9280,0.1728)
\PST@Diamond(0.9355,0.1728)
\PST@Diamond(0.9430,0.2196)
\PST@Diamond(0.9505,0.2196)
\PST@Diamond(0.9580,0.1260)
\PST@Border(0.2070,0.9680)
(0.2070,0.1260)
(0.9580,0.1260)
(0.9580,0.9680)
(0.2070,0.9680)

\catcode`@=12
\fi
\endpspicture

\end{figure}\\
Si può provare a correggere la stima di $g$ ottenuta considerando quindi questo contributo. Prendendo i primi dieci dati di ciascuna serie, in cui l'angolo si può ritenere ancora sufficientemente vicino al valore iniziale di $5^\circ$, si ottengono le seguenti medie:
\begin{table}[h]\caption{medie con errore sulla media delle prime dieci misure delle cinque serie (\unit{s}).}\label{medie2}
\centering
\begin{tabular}{cr@{$\pm$}l}
 serie & $t$&$\sigma_{t}$\\\hline
 1 & 2.00079&0.00012\\
 2 & 2.00155&0.00007\\
 3 & 2.00125&0.00012\\
 4 & 2.00118&0.00008\\
 5 & 2.00099&0.00013\\
\end{tabular}
\end{table}\\
La cui media pesata è $\bar{t}=\unit[2.00125\pm0.00004]{s}$. Inserendo quindi il fattore correttivo per $I(5^\circ)$ prima calcolato dall'integrale:
\begin{equation*}
 g=\dfrac{4\pi^2 \ell_r}{\bar{t}^2}[I(5^\circ)]^2 = \unitfrac[9.8071\pm0.0020]{m}{s^2}
\end{equation*}
Si vede subito che la compatibilità con il valore atteso è notevolmente migliorata, e infatti vale $\lambda = 0.47$.
\section{Conclusioni}
L'esperimento ha fornito un valore di $g$ soddisfacente, soprattutto visto che la massa mobile è stata posizionata con una procedura piuttosto approssimata. Sarebbe stato opportuno procedere con passi di \unit[0.1]{cm}. Inoltre, l'uso del cronometro automatico evidenzia l'importanza degli errori sistematici dovuti all'approssimazione per angoli piccoli, ma con un'adeguata correzione si possono dare diverse stime dell'accelerazione di gravità più compatibili con il valore atteso.
\section{Appendice}
\begin{figure}[hp]\caption{Tempi per dieci oscillazioni rispetto agli assi $O$ (cerchi) e $O'$ (croci), con massa in posizione indicata in ascissa lungo la scala graduata. L'intersezione delle due rette segnate è stata assunta come posizione $x_1$.}\label{course}
\centering
 % GNUPLOT: LaTeX picture using PSTRICKS macros
% Define new PST objects, if not already defined
\ifx\PSTloaded\undefined
\def\PSTloaded{t}
\psset{arrowsize=.01 3.2 1.4 .3}
\psset{dotsize=.08}
\catcode`@=11

\newpsobject{PST@Border}{psline}{linewidth=.0015,linestyle=solid}
\newpsobject{PST@Axes}{psline}{linewidth=.0015,linestyle=dotted,dotsep=.004}
\newpsobject{PST@Solid}{psline}{linewidth=.0015,linestyle=solid}
\newpsobject{PST@Dashed}{psline}{linewidth=.0015,linestyle=dashed,dash=.01 .01}
\newpsobject{PST@Dotted}{psline}{linewidth=.0025,linestyle=dotted,dotsep=.008}
\newpsobject{PST@LongDash}{psline}{linewidth=.0015,linestyle=dashed,dash=.02 .01}
\newpsobject{PST@Diamond}{psdots}{linewidth=.001,linestyle=solid,dotstyle=square,dotangle=45}
\newpsobject{PST@Filldiamond}{psdots}{linewidth=.001,linestyle=solid,dotstyle=square*,dotangle=45}
\newpsobject{PST@Cross}{psdots}{linewidth=.001,linestyle=solid,dotstyle=+,dotangle=45}
\newpsobject{PST@Plus}{psdots}{linewidth=.001,linestyle=solid,dotstyle=+}
\newpsobject{PST@Square}{psdots}{linewidth=.001,linestyle=solid,dotstyle=square}
\newpsobject{PST@Circle}{psdots}{linewidth=.001,linestyle=solid,dotstyle=o}
\newpsobject{PST@Triangle}{psdots}{linewidth=.001,linestyle=solid,dotstyle=triangle}
\newpsobject{PST@Pentagon}{psdots}{linewidth=.001,linestyle=solid,dotstyle=pentagon}
\newpsobject{PST@Fillsquare}{psdots}{linewidth=.001,linestyle=solid,dotstyle=square*}
\newpsobject{PST@Fillcircle}{psdots}{linewidth=.001,linestyle=solid,dotstyle=*}
\newpsobject{PST@Filltriangle}{psdots}{linewidth=.001,linestyle=solid,dotstyle=triangle*}
\newpsobject{PST@Fillpentagon}{psdots}{linewidth=.001,linestyle=solid,dotstyle=pentagon*}
\newpsobject{PST@Arrow}{psline}{linewidth=.001,linestyle=solid}
\catcode`@=12

\fi
\psset{unit=5.0in,xunit=5.0in,yunit=3.0in}
\pspicture(0.000000,0.000000)(1.000000,1.000000)
\ifx\nofigs\undefined
\catcode`@=11

\PST@Border(0.1590,0.1260)
(0.1740,0.1260)

\rput[r](0.1430,0.1260){18.5}
\PST@Border(0.1590,0.2463)
(0.1740,0.2463)

\rput[r](0.1430,0.2463){19.0}
\PST@Border(0.1590,0.3666)
(0.1740,0.3666)

\rput[r](0.1430,0.3666){19.5}
\PST@Border(0.1590,0.4869)
(0.1740,0.4869)

\rput[r](0.1430,0.4869){20.0}
\PST@Border(0.1590,0.6071)
(0.1740,0.6071)

\rput[r](0.1430,0.6071){20.5}
\PST@Border(0.1590,0.7274)
(0.1740,0.7274)

\rput[r](0.1430,0.7274){21.0}
\PST@Border(0.1590,0.8477)
(0.1740,0.8477)

\rput[r](0.1430,0.8477){21.5}
\PST@Border(0.1590,0.9680)
(0.1740,0.9680)

\rput[r](0.1430,0.9680){22.0}
\PST@Border(0.1590,0.1260)
(0.1590,0.1460)

\rput(0.1590,0.0840){10}
\PST@Border(0.2575,0.1260)
(0.2575,0.1460)

\rput(0.2575,0.0840){20}
\PST@Border(0.3560,0.1260)
(0.3560,0.1460)

\rput(0.3560,0.0840){30}
\PST@Border(0.4545,0.1260)
(0.4545,0.1460)

\rput(0.4545,0.0840){40}
\PST@Border(0.5530,0.1260)
(0.5530,0.1460)

\rput(0.5530,0.0840){50}
\PST@Border(0.6515,0.1260)
(0.6515,0.1460)

\rput(0.6515,0.0840){60}
\PST@Border(0.7500,0.1260)
(0.7500,0.1460)

\rput(0.7500,0.0840){70}
\PST@Border(0.8485,0.1260)
(0.8485,0.1460)

\rput(0.8485,0.0840){80}
\PST@Border(0.9470,0.1260)
(0.9470,0.1460)

\rput(0.9470,0.0840){90}
\PST@Border(0.1590,0.9680)
(0.1590,0.1260)
(0.9470,0.1260)
(0.9470,0.9680)
(0.1590,0.9680)

\rput{L}(0.0420,0.5470){tempo ($\unit{s})$}
\rput(0.5530,0.0210){posizione (\unit{cm})}
\PST@Diamond(0.1590,0.9168)
\PST@Diamond(0.2575,0.5047)
\psline(0.2575,0.5047)(0.3560,0.3096)
\PST@Diamond(0.3560,0.3096)
\PST@Diamond(0.4545,0.2528)
\PST@Diamond(0.5530,0.2446)
\PST@Diamond(0.6515,0.2788)
\PST@Diamond(0.7500,0.3899)
\PST@Diamond(0.8485,0.5239)
\PST@Diamond(0.9470,0.6642)
\PST@Plus(0.1590,0.5684)
\PST@Plus(0.2575,0.4914)
\psline(0.2575,0.4914)(0.3560,0.4462)
\PST@Plus(0.3560,0.4462)
\PST@Plus(0.4545,0.3993)
\PST@Plus(0.5530,0.3545)
\PST@Plus(0.6515,0.3750)
\PST@Plus(0.7500,0.4212)
\PST@Plus(0.8485,0.5112)
\PST@Plus(0.9470,0.6146)
\PST@Border(0.1590,0.9680)
(0.1590,0.1260)
(0.9470,0.1260)
(0.9470,0.9680)
(0.1590,0.9680)

\catcode`@=12
\fi
\endpspicture

\end{figure}
\begin{figure}[hp]\caption{Tempi per cinquanta oscillazioni rispetto agli assi $O$ (cerchi) e $O'$ (croci), con massa in posizione indicata in ascissa rispetto alla posizione $x_1$. L'intersezione delle due rette segnate è stata assunta come posizione $x_2$.}\label{medium}
\centering
 % GNUPLOT: LaTeX picture using PSTRICKS macros
% Define new PST objects, if not already defined
\ifx\PSTloaded\undefined
\def\PSTloaded{t}
\psset{arrowsize=.01 3.2 1.4 .3}
\psset{dotsize=.08}
\catcode`@=11

\newpsobject{PST@Border}{psline}{linewidth=.0015,linestyle=solid}
\newpsobject{PST@Axes}{psline}{linewidth=.0015,linestyle=dotted,dotsep=.004}
\newpsobject{PST@Solid}{psline}{linewidth=.0015,linestyle=solid}
\newpsobject{PST@Dashed}{psline}{linewidth=.0015,linestyle=dashed,dash=.01 .01}
\newpsobject{PST@Dotted}{psline}{linewidth=.0025,linestyle=dotted,dotsep=.008}
\newpsobject{PST@LongDash}{psline}{linewidth=.0015,linestyle=dashed,dash=.02 .01}
\newpsobject{PST@Diamond}{psdots}{linewidth=.001,linestyle=solid,dotstyle=square,dotangle=45}
\newpsobject{PST@Filldiamond}{psdots}{linewidth=.001,linestyle=solid,dotstyle=square*,dotangle=45}
\newpsobject{PST@Cross}{psdots}{linewidth=.001,linestyle=solid,dotstyle=+,dotangle=45}
\newpsobject{PST@Plus}{psdots}{linewidth=.001,linestyle=solid,dotstyle=+}
\newpsobject{PST@Square}{psdots}{linewidth=.001,linestyle=solid,dotstyle=square}
\newpsobject{PST@Circle}{psdots}{linewidth=.001,linestyle=solid,dotstyle=o}
\newpsobject{PST@Triangle}{psdots}{linewidth=.001,linestyle=solid,dotstyle=triangle}
\newpsobject{PST@Pentagon}{psdots}{linewidth=.001,linestyle=solid,dotstyle=pentagon}
\newpsobject{PST@Fillsquare}{psdots}{linewidth=.001,linestyle=solid,dotstyle=square*}
\newpsobject{PST@Fillcircle}{psdots}{linewidth=.001,linestyle=solid,dotstyle=*}
\newpsobject{PST@Filltriangle}{psdots}{linewidth=.001,linestyle=solid,dotstyle=triangle*}
\newpsobject{PST@Fillpentagon}{psdots}{linewidth=.001,linestyle=solid,dotstyle=pentagon*}
\newpsobject{PST@Arrow}{psline}{linewidth=.001,linestyle=solid}
\catcode`@=12

\fi
\psset{unit=5.0in,xunit=5.0in,yunit=3.0in}
\pspicture(0.000000,0.000000)(1.000000,1.000000)
\ifx\nofigs\undefined
\catcode`@=11

\PST@Border(0.1750,0.1260)
(0.1900,0.1260)

\rput[r](0.1590,0.1260){99.5}
\PST@Border(0.1750,0.2663)
(0.1900,0.2663)

\rput[r](0.1590,0.2663){100.0}
\PST@Border(0.1750,0.4067)
(0.1900,0.4067)

\rput[r](0.1590,0.4067){100.5}
\PST@Border(0.1750,0.5470)
(0.1900,0.5470)

\rput[r](0.1590,0.5470){101.0}
\PST@Border(0.1750,0.6873)
(0.1900,0.6873)

\rput[r](0.1590,0.6873){101.5}
\PST@Border(0.1750,0.8277)
(0.1900,0.8277)

\rput[r](0.1590,0.8277){102.0}
\PST@Border(0.1750,0.9680)
(0.1900,0.9680)

\rput[r](0.1590,0.9680){102.5}
\PST@Border(0.1750,0.1260)
(0.1750,0.1460)

\rput(0.1750,0.0840){-3}
\PST@Border(0.3037,0.1260)
(0.3037,0.1460)

\rput(0.3037,0.0840){-2}
\PST@Border(0.4323,0.1260)
(0.4323,0.1460)

\rput(0.4323,0.0840){-1}
\PST@Border(0.5610,0.1260)
(0.5610,0.1460)

\rput(0.5610,0.0840){0}
\PST@Border(0.6897,0.1260)
(0.6897,0.1460)

\rput(0.6897,0.0840){1}
\PST@Border(0.8183,0.1260)
(0.8183,0.1460)

\rput(0.8183,0.0840){2}
\PST@Border(0.9470,0.1260)
(0.9470,0.1460)

\rput(0.9470,0.0840){3}
\PST@Border(0.1750,0.9680)
(0.1750,0.1260)
(0.9470,0.1260)
(0.9470,0.9680)
(0.1750,0.9680)

\rput{L}(0.0420,0.5470){tempo ($\unit{s})$}
\rput(0.5610,0.0210){posizione (\unit{cm})}
\PST@Diamond(0.3037,0.8532)
\PST@Diamond(0.4323,0.6638)
\PST@Diamond(0.5610,0.4833)
\PST@Diamond(0.6897,0.3351)
\PST@Diamond(0.8183,0.1906)
\psline(0.6897,0.3351)(0.8183,0.1906)
\PST@Plus(0.3037,0.3960)
\PST@Plus(0.4323,0.3519)
\PST@Plus(0.5610,0.3166)
\PST@Plus(0.6897,0.2748)
\PST@Plus(0.8183,0.2411)
\psline(0.6897,0.2748)(0.8183,0.2411)
\PST@Border(0.1750,0.9680)
(0.1750,0.1260)
(0.9470,0.1260)
(0.9470,0.9680)
(0.1750,0.9680)

\catcode`@=12
\fi
\endpspicture

\end{figure}
\begin{figure}[hp]\caption{Tempi per cento oscillazioni rispetto agli assi $O$ (cerchi) e $O'$ (croci), con massa in posizione indicata in ascissa rispetto alla posizione $x_2$. L'intersezione delle due rette segnate è stata assunta come posizione $x_f$.}\label{fine}
\centering
 % GNUPLOT: LaTeX picture using PSTRICKS macros
% Define new PST objects, if not already defined
\ifx\PSTloaded\undefined
\def\PSTloaded{t}
\psset{arrowsize=.01 3.2 1.4 .3}
\psset{dotsize=.08}
\catcode`@=11

\newpsobject{PST@Border}{psline}{linewidth=.0015,linestyle=solid}
\newpsobject{PST@Axes}{psline}{linewidth=.0015,linestyle=dotted,dotsep=.004}
\newpsobject{PST@Solid}{psline}{linewidth=.0015,linestyle=solid}
\newpsobject{PST@Dashed}{psline}{linewidth=.0015,linestyle=dashed,dash=.01 .01}
\newpsobject{PST@Dotted}{psline}{linewidth=.0015,linestyle=dashed,dash=.01 .01}
\newpsobject{PST@LongDash}{psline}{linewidth=.0015,linestyle=dashed,dash=.02 .01}
\newpsobject{PST@Diamond}{psdots}{linewidth=.001,linestyle=solid,dotstyle=square,dotangle=45}
\newpsobject{PST@Filldiamond}{psdots}{linewidth=.001,linestyle=solid,dotstyle=square*,dotangle=45}
\newpsobject{PST@Cross}{psdots}{linewidth=.001,linestyle=solid,dotstyle=+,dotangle=45}
\newpsobject{PST@Plus}{psdots}{linewidth=.001,linestyle=solid,dotstyle=+}
\newpsobject{PST@Square}{psdots}{linewidth=.001,linestyle=solid,dotstyle=square}
\newpsobject{PST@Circle}{psdots}{linewidth=.001,linestyle=solid,dotstyle=o}
\newpsobject{PST@Triangle}{psdots}{linewidth=.001,linestyle=solid,dotstyle=triangle}
\newpsobject{PST@Pentagon}{psdots}{linewidth=.001,linestyle=solid,dotstyle=pentagon}
\newpsobject{PST@Fillsquare}{psdots}{linewidth=.001,linestyle=solid,dotstyle=square*}
\newpsobject{PST@Fillcircle}{psdots}{linewidth=.001,linestyle=solid,dotstyle=*}
\newpsobject{PST@Filltriangle}{psdots}{linewidth=.001,linestyle=solid,dotstyle=triangle*}
\newpsobject{PST@Fillpentagon}{psdots}{linewidth=.001,linestyle=solid,dotstyle=pentagon*}
\newpsobject{PST@Arrow}{psline}{linewidth=.001,linestyle=solid}
\catcode`@=12

\fi
\psset{unit=5.0in,xunit=5.0in,yunit=3.0in}
\pspicture(0.000000,0.000000)(1.000000,1.000000)
\ifx\nofigs\undefined
\catcode`@=11

\PST@Border(0.1750,0.1260)
(0.1900,0.1260)

\rput[r](0.1590,0.1260){198.5}
\PST@Border(0.1750,0.2944)
(0.1900,0.2944)

\rput[r](0.1590,0.2944){199.0}
\PST@Border(0.1750,0.4628)
(0.1900,0.4628)

\rput[r](0.1590,0.4628){199.5}
\PST@Border(0.1750,0.6312)
(0.1900,0.6312)

\rput[r](0.1590,0.6312){200.0}
\PST@Border(0.1750,0.7996)
(0.1900,0.7996)

\rput[r](0.1590,0.7996){200.5}
\PST@Border(0.1750,0.9680)
(0.1900,0.9680)

\rput[r](0.1590,0.9680){201.0}
\PST@Border(0.1750,0.1260)
(0.1750,0.1460)

\rput(0.1750,0.0840){-1.0}
\PST@Border(0.3680,0.1260)
(0.3680,0.1460)

\rput(0.3680,0.0840){-0.5}
\PST@Border(0.5610,0.1260)
(0.5610,0.1460)

\rput(0.5610,0.0840){0.0}
\PST@Border(0.7540,0.1260)
(0.7540,0.1460)

\rput(0.7540,0.0840){0.5}
\PST@Border(0.9470,0.1260)
(0.9470,0.1460)

\rput(0.9470,0.0840){1.0}
\PST@Border(0.1750,0.9680)
(0.1750,0.1260)
(0.9470,0.1260)
(0.9470,0.9680)
(0.1750,0.9680)

\rput{L}(0.0420,0.5470){tempo ($\unit{s})$}
\rput(0.5610,0.0210){posizione (\unit{cm})}
\PST@Diamond(0.3680,0.7538)
\PST@Diamond(0.5610,0.5638)
\PST@Diamond(0.7540,0.4025)
\PST@Plus(0.3680,0.7009)
\PST@Plus(0.5610,0.6056)
\PST@Plus(0.7540,0.5635)
\PST@Dotted(0.1750,0.9247)
(0.1750,0.9247)
(0.1828,0.9176)
(0.1906,0.9105)
(0.1984,0.9034)
(0.2062,0.8963)
(0.2140,0.8892)
(0.2218,0.8821)
(0.2296,0.8750)
(0.2374,0.8679)
(0.2452,0.8608)
(0.2530,0.8537)
(0.2608,0.8466)
(0.2686,0.8395)
(0.2764,0.8324)
(0.2842,0.8253)
(0.2920,0.8182)
(0.2998,0.8111)
(0.3076,0.8040)
(0.3154,0.7969)
(0.3232,0.7898)
(0.3310,0.7827)
(0.3388,0.7756)
(0.3466,0.7685)
(0.3544,0.7614)
(0.3622,0.7543)
(0.3699,0.7472)
(0.3777,0.7402)
(0.3855,0.7331)
(0.3933,0.7260)
(0.4011,0.7189)
(0.4089,0.7118)
(0.4167,0.7047)
(0.4245,0.6976)
(0.4323,0.6905)
(0.4401,0.6834)
(0.4479,0.6763)
(0.4557,0.6692)
(0.4635,0.6621)
(0.4713,0.6550)
(0.4791,0.6479)
(0.4869,0.6408)
(0.4947,0.6337)
(0.5025,0.6266)
(0.5103,0.6195)
(0.5181,0.6124)
(0.5259,0.6053)
(0.5337,0.5982)
(0.5415,0.5911)
(0.5493,0.5840)
(0.5571,0.5769)
(0.5649,0.5698)
(0.5727,0.5627)
(0.5805,0.5556)
(0.5883,0.5485)
(0.5961,0.5414)
(0.6039,0.5344)
(0.6117,0.5273)
(0.6195,0.5202)
(0.6273,0.5131)
(0.6351,0.5060)
(0.6429,0.4989)
(0.6507,0.4918)
(0.6585,0.4847)
(0.6663,0.4776)
(0.6741,0.4705)
(0.6819,0.4634)
(0.6897,0.4563)
(0.6975,0.4492)
(0.7053,0.4421)
(0.7131,0.4350)
(0.7209,0.4279)
(0.7287,0.4208)
(0.7365,0.4137)
(0.7443,0.4066)
(0.7521,0.3995)
(0.7598,0.3924)
(0.7676,0.3853)
(0.7754,0.3782)
(0.7832,0.3711)
(0.7910,0.3640)
(0.7988,0.3569)
(0.8066,0.3498)
(0.8144,0.3427)
(0.8222,0.3356)
(0.8300,0.3285)
(0.8378,0.3215)
(0.8456,0.3144)
(0.8534,0.3073)
(0.8612,0.3002)
(0.8690,0.2931)
(0.8768,0.2860)
(0.8846,0.2789)
(0.8924,0.2718)
(0.9002,0.2647)
(0.9080,0.2576)
(0.9158,0.2505)
(0.9236,0.2434)
(0.9314,0.2363)
(0.9392,0.2292)
(0.9470,0.2221)

\PST@LongDash(0.1750,0.7608)
(0.1750,0.7608)
(0.1828,0.7580)
(0.1906,0.7552)
(0.1984,0.7524)
(0.2062,0.7497)
(0.2140,0.7469)
(0.2218,0.7441)
(0.2296,0.7413)
(0.2374,0.7385)
(0.2452,0.7358)
(0.2530,0.7330)
(0.2608,0.7302)
(0.2686,0.7274)
(0.2764,0.7247)
(0.2842,0.7219)
(0.2920,0.7191)
(0.2998,0.7163)
(0.3076,0.7136)
(0.3154,0.7108)
(0.3232,0.7080)
(0.3310,0.7052)
(0.3388,0.7025)
(0.3466,0.6997)
(0.3544,0.6969)
(0.3622,0.6941)
(0.3699,0.6914)
(0.3777,0.6886)
(0.3855,0.6858)
(0.3933,0.6830)
(0.4011,0.6803)
(0.4089,0.6775)
(0.4167,0.6747)
(0.4245,0.6719)
(0.4323,0.6691)
(0.4401,0.6664)
(0.4479,0.6636)
(0.4557,0.6608)
(0.4635,0.6580)
(0.4713,0.6553)
(0.4791,0.6525)
(0.4869,0.6497)
(0.4947,0.6469)
(0.5025,0.6442)
(0.5103,0.6414)
(0.5181,0.6386)
(0.5259,0.6358)
(0.5337,0.6331)
(0.5415,0.6303)
(0.5493,0.6275)
(0.5571,0.6247)
(0.5649,0.6220)
(0.5727,0.6192)
(0.5805,0.6164)
(0.5883,0.6136)
(0.5961,0.6108)
(0.6039,0.6081)
(0.6117,0.6053)
(0.6195,0.6025)
(0.6273,0.5997)
(0.6351,0.5970)
(0.6429,0.5942)
(0.6507,0.5914)
(0.6585,0.5886)
(0.6663,0.5859)
(0.6741,0.5831)
(0.6819,0.5803)
(0.6897,0.5775)
(0.6975,0.5748)
(0.7053,0.5720)
(0.7131,0.5692)
(0.7209,0.5664)
(0.7287,0.5637)
(0.7365,0.5609)
(0.7443,0.5581)
(0.7521,0.5553)
(0.7598,0.5526)
(0.7676,0.5498)
(0.7754,0.5470)
(0.7832,0.5442)
(0.7910,0.5414)
(0.7988,0.5387)
(0.8066,0.5359)
(0.8144,0.5331)
(0.8222,0.5303)
(0.8300,0.5276)
(0.8378,0.5248)
(0.8456,0.5220)
(0.8534,0.5192)
(0.8612,0.5165)
(0.8690,0.5137)
(0.8768,0.5109)
(0.8846,0.5081)
(0.8924,0.5054)
(0.9002,0.5026)
(0.9080,0.4998)
(0.9158,0.4970)
(0.9236,0.4943)
(0.9314,0.4915)
(0.9392,0.4887)
(0.9470,0.4859)

\PST@Border(0.1750,0.9680)
(0.1750,0.1260)
(0.9470,0.1260)
(0.9470,0.9680)
(0.1750,0.9680)

\catcode`@=12
\fi
\endpspicture

\end{figure}
\begin{table}[p]\caption{Cinque serie di rilevazioni automatiche dei periodi (\unit{s}) con massa mobile in posizione $x_f$ (segue).}\label{auto}
\centering \small
\begin{tabular}{r*5c}
\emph{serie} &\emph{1} &\emph{2} &\emph{3} &\emph{4} &\emph{5}\\\hline
1 &2.0009 &2.0016 &2.0014 &2.0012 &2.0008\\
2 &2.0009 &2.0016 &2.0013 &2.0013 &2.0012\\
3 &2.0009 &2.0016 &2.0012 &2.0012 &2.0011\\
4 &2.0009 &2.0016 &2.0013 &2.0013 &2.0010\\
5 &2.0009 &2.0016 &2.0014 &2.0012 &2.0010\\
6 &2.0007 &2.0015 &2.0011 &2.0011 &2.0011\\
7 &2.0007 &2.0014 &2.0013 &2.0012 &2.0010\\
8 &2.0007 &2.0015 &2.0013 &2.0011 &2.0008\\
9 &2.0006 &2.0015 &2.0011 &2.0011 &2.0010\\
10 &2.0007 &2.0016 &2.0011 &2.0011 &2.0009\\
11 &2.0006 &2.0013 &2.0011 &2.0010 &2.0009\\
12 &2.0006 &2.0014 &2.0011 &2.0010 &2.0009\\
13 &2.0005 &2.0015 &2.0010 &2.0010 &2.0008\\
14 &2.0005 &2.0014 &2.0011 &2.0010 &2.0009\\
15 &2.0004 &2.0012 &2.0011 &2.0008 &2.0010\\
16 &2.0004 &2.0014 &2.0010 &2.0010 &2.0008\\
17 &2.0004 &2.0013 &2.0009 &2.0010 &2.0009\\
18 &2.0005 &2.0014 &2.0008 &2.0008 &2.0009\\
19 &2.0004 &2.0012 &2.0010 &2.0008 &2.0008\\
20 &2.0003 &2.0013 &2.0008 &2.0009 &2.0009\\
21 &2.0003 &2.0013 &2.0006 &2.0007 &2.0008\\
22 &2.0004 &2.0013 &2.0007 &2.0007 &2.0008\\
23 &2.0003 &2.0016 &2.0006 &2.0007 &2.0009\\
24 &2.0002 &2.0011 &2.0007 &2.0008 &2.0009\\
25 &2.0002 &2.0011 &2.0007 &2.0005 &2.0008\\
26 &2.0002 &2.0011 &2.0005 &2.0005 &2.0008\\
27 &2.0001 &2.0012 &2.0006 &2.0004 &2.0008\\
28 &2.0002 &2.0011 &2.0006 &2.0005 &2.0008\\
29 &2.0000 &2.0013 &2.0004 &2.0004 &2.0007\\
30 &2.0000 &2.0009 &2.0006 &2.0002 &2.0007\\
31 &2.0000 &2.0011 &2.0005 &2.0005 &2.0008\\
32 &2.0001 &2.0011 &2.0004 &2.0001 &2.0008\\
33 &2.0000 &2.0011 &2.0003 &2.0001 &2.0008\\
34 &1.9999 &2.0010 &2.0003 &2.0001 &2.0007\\
35 &2.0000 &2.0011 &2.0003 &2.0001 &2.0007\\
36 &1.9999 &2.0011 &2.0004 &2.0002 &2.0007\\
37 &1.9998 &2.0010 &2.0001 &2.0003 &2.0006\\
38 &1.9998 &2.0008 &2.0001 &2.0002 &2.0007\\
39 &1.9999 &2.0009 &2.0004 &2.0002 &2.0007\\
40 &1.9997 &2.0009 &2.0001 &2.0002 &2.0006\\
41 &1.9997 &2.0009 &2.0003 &2.0000 &2.0007\\
42 &1.9998 &2.0008 &2.0002 &2.0002 &2.0006\\
43 &1.9997 &2.0009 &2.0002 &2.0001 &2.0006\\
44 &1.9996 &2.0008 &2.0002 &2.0004 &2.0007\\
45 &1.9997 &2.0008 &2.0001 &2.0002 &2.0005\\
46 &1.9997 &2.0007 &2.0002 &2.0000 &2.0005\\
47 &1.9995 &2.0008 &2.0000 &2.0001 &2.0007\\
48 &1.9996 &2.0007 &2.0001 &2.0001 &2.0005\\
49 &1.9996 &2.0007 &2.0001 &2.0001 &2.0006\\
50 &1.9995 &2.0006 &2.0000 &2.0001 &2.0006
\end{tabular}
\end{table}
\begin{table}[p]\caption{Cinque serie di rilevazioni automatiche dei periodi (\unit{s}) con massa mobile in posizione $x_f$.}
\centering \small
\begin{tabular}{r*5c}
\emph{serie} &\emph{1} &\emph{2} &\emph{3} &\emph{4} &\emph{5}\\\hline
51 &1.9995 &2.0004 &2.0000 &2.0002 &2.0006\\
52 &1.9995 &2.0006 &2.0001 &2.0000 &2.0005\\
53 &1.9994 &2.0004 &2.0001 &2.0002 &2.0005\\
54 &1.9995 &2.0004 &2.0000 &2.0000 &2.0006\\
55 &1.9994 &2.0002 &2.0001 &2.0000 &2.0004\\
56 &1.9994 &2.0004 &1.9998 &2.0001 &2.0005\\
57 &1.9995 &2.0004 &2.0001 &2.0001 &2.0005\\
58 &1.9994 &2.0002 &2.0000 &2.0000 &2.0005\\
59 &1.9994 &2.0002 &1.9999 &2.0000 &2.0005\\
60 &1.9996 &2.0002 &1.9998 &1.9999 &2.0004\\
61 &1.9994 &2.0000 &2.0000 &2.0002 &2.0004\\
62 &1.9994 &2.0002 &1.9999 &2.0000 &2.0004\\
63 &1.9995 &2.0001 &1.9998 &2.0000 &2.0003\\
64 &1.9994 &1.9999 &1.9999 &1.9999 &2.0003\\
65 &1.9995 &2.0001 &1.9998 &1.9999 &2.0003\\
66 &1.9995 &1.9995 &1.9999 &2.0001 &2.0002\\
67 &1.9994 &2.0003 &1.9999 &1.9998 &2.0003\\
68 &1.9995 &1.9999 &1.9997 &2.0001 &2.0003\\
69 &1.9995 &1.9997 &1.9998 &2.0000 &2.0002\\
70 &1.9995 &2.0001 &1.9999 &1.9998 &2.0002\\
71 &1.9994 &1.9998 &1.9999 &2.0000 &2.0002\\
72 &1.9993 &1.9998 &1.9998 &2.0000 &2.0002\\
73 &1.9996 &1.9998 &1.9998 &1.9998 &2.0002\\
74 &1.9995 &1.9999 &1.9996 &2.0000 &2.0001\\
75 &1.9993 &1.9998 &1.9998 &1.9999 &2.0003\\
76 &1.9995 &2.0000 &1.9996 &1.9998 &2.0000\\
77 &1.9993 &1.9998 &1.9998 &1.9999 &2.0002\\
78 &1.9995 &1.9998 &1.9996 &2.0000 &2.0000\\
79 &1.9994 &1.9998 &1.9994 &1.9998 &2.0001\\
80 &1.9993 &1.9997 &1.9999 &2.0001 &2.0000\\
81 &1.9994 &1.9998 &1.9996 &1.9999 &2.0000\\
82 &1.9994 &1.9996 &1.9995 &1.9999 &2.0000\\
83 &1.9994 &2.0000 &1.9994 &1.9999 &2.0000\\
84 &1.9994 &1.9998 &1.9998 &1.9998 &1.9999\\
85 &1.9993 &1.9998 &1.9997 &1.9999 &1.9999\\
86 &1.9994 &1.9997 &1.9995 &1.9998 &2.0000\\
87 &1.9994 &1.9998 &1.9995 &1.9998 &1.9997\\
88 &1.9992 &1.9995 &1.9996 &1.9999 &1.9998\\
89 &1.9996 &1.9999 &1.9996 &1.9999 &1.9998\\
90 &1.9993 &1.9997 &1.9996 &1.9997 &1.9998\\
91 &1.9993 &1.9996 &1.9994 &1.9998 &1.9997\\
92 &1.9992 &1.9997 &1.9996 &2.0000 &1.9997\\
93 &1.9994 &1.9998 &1.9993 &1.9998 &1.9996\\
94 &1.9992 &1.9996 &1.9997 &1.9998 &1.9996\\
95 &1.9993 &1.9997 &1.9993 &1.9997 &1.9997\\
96 &1.9991 &1.9996 &1.9996 &2.0000 &1.9995\\
97 &1.9993 &1.9998 &1.9994 &1.9996 &1.9995\\
98 &1.9993 &1.9995 &1.9993 &1.9997 &1.9996\\
99 &1.9992 &1.9997 &1.9995 &1.9996 &1.9996\\
100 &1.9993 &1.9997 &1.9994 &1.9998 &1.9994
\end{tabular}

\end{table}
\end{document}
