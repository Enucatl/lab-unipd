\documentclass[italian,a4paper]{article}
\usepackage[tight,nice]{units}
\usepackage{babel,amsmath,amssymb,amsthm,graphicx,url,wrapfig,multirow}
\usepackage[text={6in,9in},centering]{geometry}
\usepackage[utf8x]{inputenc}
\usepackage[T1]{fontenc}
\usepackage{ae,aecompl}
\usepackage[Euler]{upgreek}
\usepackage[footnotesize,bf]{caption}
\usepackage[usenames]{color}
\include{pstricks}
% \include{pstricks}
\frenchspacing
\pagestyle{plain}
%------------- eliminare prime e ultime linee isolate
\clubpenalty=9999%
\widowpenalty=9999
%--- definizione numerazioni
\renewcommand{\theequation}{\thesection.\arabic{equation}}
\renewcommand{\thefigure}{\thesection.\arabic{figure}}
\renewcommand{\thetable}{\thesection.\arabic{table}}
\addto\captionsitalian{%
  \renewcommand{\figurename}%
{Grafico}%
}
%
%------------- ridefinizione simbolo per elenchi puntati: en dash
%\renewcommand{\labelitemi}{\textbf{--}}
\renewcommand{\labelenumi}{\textbf{\arabic{enumi}.}}
\setlength{\abovecaptionskip}{\baselineskip}   % 0.5cm as an example
\setlength{\floatsep}{2\baselineskip}
\setlength{\belowcaptionskip}{\baselineskip}   % 0.5cm as an example
%------------- nuovi environment senza spazi
%\newenvironment{packed_item}{
%\begin{itemize}
%  \setlength{\itemsep}{1pt}
%  \setlength{\parskip}{0pt}
%  \setlength{\parsep}{0pt}
%}{\end{itemize}}
%\newenvironment{packed_enum}{
%\begin{enumerate}
%  \setlength{\itemsep}{1pt}
%  \setlength{\parskip}{0pt}
%  \setlength{\parsep}{0pt}
%}{\end{enumerate}}
%\newenvironment{packed_description}{
%\begin{enumerate}
%   \setlength{\itemsep}{1pt}
%   \setlength{\parskip}{0pt}
%   \setlength{\parsep}{0pt}
% }{\end{enumerate}}
%--------- comandi insiemi numeri complessi, naturali, reali e altre abbreviazioni
\newcommand{\micro}{\ensuremath{\upmu}} %prefisso micro
\newcommand{\e}{\mathrm{e}} %numero di nepero
\newcommand{\di}{\mathrm{d}} %simbolo di differenziale
\renewcommand{\leq}{\leqslant}
\renewcommand{\pi}{\uppi} % costante pi greco
\renewcommand{\tau}{\uptau} %momento della forza
\newcommand{\coloneqq}{\mathrel{\mathop:}=} % := ``per definizione''
\newcommand{\ms}{(\unitfrac{m}{s})}
%--------- porzione dedicata ai float in una pagina:
\renewcommand{\textfraction}{0.05}
\renewcommand{\topfraction}{0.95}
\renewcommand{\bottomfraction}{0.95}
\renewcommand{\floatpagefraction}{0.35}
\setcounter{totalnumber}{5}
%---------
%
%---------
\begin{document}
\title{Relazione di laboratorio: l'estensimetro}
\author{\normalsize Ilaria Brivio (582116)\\%
\normalsize \url{brivio.ilaria@tiscali.it}%
\and %
\normalsize Matteo Abis (584206)\\ %
\normalsize \url{webmaster@latinblog.org}}
\date{\today}
\maketitle
%------------------
\section{Obiettivo dell'esperienza}
Obiettivo dell'esperienza è la verifica della legge di Hooke $F = \frac{YS}{\ell} \Delta x$, che prevede una dipendenza lineare dell'allungamento di un corpo elastico dal modulo della forza applicata.
\section{Descrizione dell'apparato strumentale}
Sono stati utilizzati dieci estensimetri.
L'estensimetro è uno strumento che permette di applicare una forza a un filo cilindrico di materiale,
lunghezza e diametro noti e misurarne l'allungamento attraverso un minimetro di sensibilità~$\unit[10^5]{m^{-1}}$. La forza applicata è regolabile mediante la rotazione di una ghiera collegata a un dinamometro, la cui scala è in grammi-peso.
Per l'elaborazione dei dati tali valori sono stati convertiti in Newton e moltiplicati per quattro, per l'effetto della leva. Come errore sulla forza è stata stimata la più piccola sottosuddivisione apprezzabile ad occhio nudo, che corrisponde a~$\unit[0.05]{N}$.
\begin{table}[h]\caption{Caratteristiche degli estensimetri utilizzati: materiale, lunghezza $\ell$ e diametro $d$.}
\centering
 \begin{tabular}{*2c *2{r@{$\pm$}l}}
  n. & materiale & \multicolumn{2}{c}{$\ell \pm \sigma_\ell (\unit{m})$} & \multicolumn{2}{c}{$d \pm \sigma_d (\unit{mm})$}\\\hline
1 & tungsteno & 1.000 & 0.002 &0.250&0.005\\
3 & ottone & 1.000 & 0.002 &0.500&0.005\\
6 & acciaio & 0.950 & 0.002 &0.305&0.003\\
7 & acciaio & 0.950 & 0.002 &0.330&0.003\\
8 & acciaio & 0.950 & 0.002 &0.356&0.004\\
9 & acciaio & 0.950 & 0.002 &0.381&0.004\\
14 & acciaio & 0.800 & 0.002 &0.279&0.003\\
16 & acciaio & 0.600 & 0.002 &0.279&0.003\\
17 & acciaio & 0.500 & 0.002 &0.279&0.003\\
18 & acciaio & 0.400 & 0.002 &0.279&0.003
 \end{tabular}
\end{table}
\section{Descrizione della metodologia di misura}
Su ciascun estensimetro sono stati misurati gli allungamenti $\Delta x$ corrispondenti ad una forza di $200$, $300$, $\dots$, $1100$~grammi-peso\footnote{Questi sono i valori letti sul dinamometro. Le forze applicate sono $\Delta F=0$, $3.92$, $7.84$, $11.77$, $15.69$, $19.61$, $23.53$, $27.46$, $31.38$, $35.30$~\unit{N}, ovvero la forza, in Newton, moltiplicata per quattro.}. Lo zero del minimetro è stato posizionato in modo da coincidere con la lunghezza del filo con forza \unit[200]{grammi-peso} sul dinamometro. La misura è stata ripetuta partendo da 1100 fino a 200~grammi-peso, ovvero accorciando progressivamente il filo. Sono stati scelti quattro estensimetri con filo di acciaio di uguale lunghezza e diverso spessore, altri quattro con stesso spessore e lunghezza variabile e due di diverso materiale.
\section{Risultati sperimentali ed elaborazione dati}
Per prima cosa sono stati riportati i dati sperimentali in grafico, con la variazione della forza applicata $\Delta F = F_{\mathrm{app}}-\unit[200]{g}$ in ascissa e il corrispondente allungamento $\Delta x$ in ordinata. Per la stima del modulo di Young su ogni estensimetro sono stati seguiti due approcci differenti:
\begin{enumerate}
 \item Sono stati interpolati i dati in allungamento, con coefficiente angolare $K_1 \pm \sigma_{K_1}$ e in accorciamento $K_2 \pm \sigma_{K_2}$. \`E stata calcolata la media pesata $\bar{K}\pm\sigma_{\bar{K}}$ dei due valori e da questa è stato ricavato il valore di $\widetilde{Y}=4\ell/\pi d^2 \bar{K}$ con il relativo errore $\sigma_{\widetilde{Y}}$ (vedi formula~{}). I risultati sono riportati in tabella, dove i prefissi \micro\ e M indicano rispettivamente ordini di grandezza $10^{-6}$ e $10^9$.\\
\begin{table}[h]
\centering
 \begin{tabular}{c *4{r@{$\pm$}l}}
  n. &  \multicolumn{2}{c}{$K_1\,(\unitfrac{\micro m}{N})$}&
\multicolumn{2}{c}{$K_2\,(\unitfrac{\micro m}{N})$}&
\multicolumn{2}{c}{$\bar{K}\,(\unitfrac{\micro m}{N})$}&
\multicolumn{2}{c}{$\widetilde{Y}\,(\unitfrac{MN}{m^2})$}\\\hline
1  & 54.0&0.4 &54.1&0.3 &54.1&0.2 &377&2\\
3  & 51.4&0.1 &54.4&0.2 &54.4&0.1 & 99&2\\
6  & 63.8&0.2 &63.7&0.3 &63.8&0.2 &204&4\\
7  & 54.6&0.5 &54.7&0.5 &54.7&0.4 &203&4\\
8  & 44.8&0.1 &44.7&0.2 &44.8&0.1 &213&4\\
9  & 42.0&0.6 &42.4&0.2 &42.3&0.2 &197&4\\
14 & 64.2&0.1 &63.7&0.3 &64.1&0.1 &204&4\\
16 & 47.1&0.2 &47.0&0.2 &47.1&0.1 &208&4\\
17 & 39.8&0.2 &39.7&0.2 &39.7&0.1 &206&4\\
18 & 33.5&0.2 &33.4&0.2 &33.4&0.1 &196&4\\
 \end{tabular}
\end{table}
\item Sono stati calcolati direttamente da $K_1$ e $K_2$ due valori $Y_1$ e $Y_2$, con i rispettivi errori per propagazione, e poi la loro media pesata $\bar{Y}\pm\sigma_{\bar{Y}}$.\\
\begin{table}[h]
\centering
 \begin{tabular}{c *3{r@{$\pm$}l}}
  n. &
\multicolumn{2}{c}{$Y_1\,(\unitfrac{MN}{m^2})$}&
\multicolumn{2}{c}{$Y_2\,(\unitfrac{MN}{m^2})$}&
\multicolumn{2}{c}{$\bar{Y}\,(\unitfrac{MN}{m^2})$}\\\hline
1  &377&2 &376&2 &377&1\\
3  &99&2 &99&2 & 99&1\\
6  &204&4 &204&4 &204&3\\
7  &203&5 &203&5 &203&3\\
8  &213&4 &213&4 &213&3\\
9  &198&5 &197&4 &197&3\\
14 &204&4 &205&4 &205&3\\
16 &208&4 &209&4 &208&3\\
17 &206&4 &206&4 &206&3\\
18 &195&4 &196&4 &196&3\\
 \end{tabular}
\end{table}
\end{enumerate}
Come si vede dalle tabelle, la compatibilità tra $\bar{Y}$ e $\widetilde{Y}$ è ovunque uguale a zero, tranne che per l'estensimetro n. 14 ($\lambda = 0.2$).
\section{Discussione dei risultati}
I risultati sperimentali sono in pieno accordo con la legge di Hooke, come evidente dai grafici~\ref{kes} e~\ref{kel}. Dei due metodi di elaborazione dei dati appare migliore quello che calcola il modulo di Young come media pesata dei due $Y_1$ e $Y_2$, perché l'errore finale risulta minore. Ciò può essere dovuto al fatto che la media pesata permette di ridurre l'errore statistico in modo più forte rispetto alla propagazione degli errori.

\section{Conclusioni}
Come mostrano in particolare i grafici~\ref{P} e~\ref{R}, la legge di Hooke risulta verificata. Il modulo di Young per l'acciaio, facendo una media dall'ultima tabella risulta $204\pm3 \cdot 10^9 \unitfrac{N}{m^2}$, pienamente compatibile con il valore atteso di $205\pm1 \cdot 10^9 \unitfrac{N}{m^2}$. Anche il valore per l'ottone ha $\lambda = 0.8$ con $96\pm2\cdot 10^9 \unitfrac{N}{m^2}$ e lo stesso vale per il tungsteno $\lambda = 0.13$ con $379\pm15\cdot 10^9 \unitfrac{N}{m^2}$. Per quanto riguarda l'elaborazione dei dati, risulta più conveniente apeplicare prima la propagazione degli errori e successivamente l'operazione di media pesata al fine di avere un errore statistico minore sulla misura finale.
\newpage
\section{Appendice}
\centering
\begin{figure}[h]\caption{Grafico relativo all'estensimetro n.1, con filo di tungsteno. In ascissa sono riportati i valori di $\Delta F~(\unit{N})$ e in ordinata la corrispondente variazione di lunghezza del filo $\Delta x~(\unit{m})$. Le barre di errore su entrambe le misure sono troppo piccole per poter essere rappresentate su questa scala.}\label{1}
% GNUPLOT: LaTeX picture using PSTRICKS macros
% Define new PST objects, if not already defined
\ifx\PSTloaded\undefined
\def\PSTloaded{t}
\psset{arrowsize=.01 3.2 1.4 .3}
\psset{dotsize=.125}
\catcode`@=11

\newpsobject{PST@Border}{psline}{linewidth=.0015,linestyle=solid}
\newpsobject{PST@Axes}{psline}{linewidth=.0015,linestyle=dotted,dotsep=.004}
\newpsobject{PST@Solid}{psline}{linewidth=.0015,linestyle=solid}
\newpsobject{PST@Dashed}{psline}{linewidth=.0015,linestyle=dashed,dash=.01 .01}
\newpsobject{PST@Dotted}{psline}{linewidth=.0025,linestyle=dotted,dotsep=.008}
\newpsobject{PST@LongDash}{psline}{linewidth=.0015,linestyle=dashed,dash=.02 .01}
\newpsobject{PST@Diamond}{psdots}{linewidth=.001,linestyle=solid,dotstyle=square,dotangle=45}
\newpsobject{PST@Filldiamond}{psdots}{linewidth=.001,linestyle=solid,dotstyle=square*,dotangle=45}
\newpsobject{PST@Cross}{psdots}{linewidth=.001,linestyle=solid,dotstyle=+,dotangle=45}
\newpsobject{PST@Plus}{psdots}{linewidth=.001,linestyle=solid,dotstyle=+}
\newpsobject{PST@Square}{psdots}{linewidth=.001,linestyle=solid,dotstyle=square}
\newpsobject{PST@Circle}{psdots}{linewidth=.001,linestyle=solid,dotstyle=o}
\newpsobject{PST@Triangle}{psdots}{linewidth=.001,linestyle=solid,dotstyle=triangle}
\newpsobject{PST@Pentagon}{psdots}{linewidth=.001,linestyle=solid,dotstyle=pentagon}
\newpsobject{PST@Fillsquare}{psdots}{linewidth=.001,linestyle=solid,dotstyle=square*}
\newpsobject{PST@Fillcircle}{psdots}{linewidth=.001,linestyle=solid,dotstyle=*}
\newpsobject{PST@Filltriangle}{psdots}{linewidth=.001,linestyle=solid,dotstyle=triangle*}
\newpsobject{PST@Fillpentagon}{psdots}{linewidth=.001,linestyle=solid,dotstyle=pentagon*}
\newpsobject{PST@Arrow}{psline}{linewidth=.001,linestyle=solid}
\catcode`@=12

\fi
\psset{unit=5.0in,xunit=5.0in,yunit=3.0in}
\pspicture(0.000000,0.000000)(1.000000,1.000000)
\ifx\nofigs\undefined
\catcode`@=11

\PST@Border(0.2070,0.1260)
(0.2220,0.1260)

\rput[r](0.1910,0.1260){-0.0002}
\PST@Border(0.2070,0.2025)
(0.2220,0.2025)

\rput[r](0.1910,0.2025){0.0000}
\PST@Border(0.2070,0.2791)
(0.2220,0.2791)

\rput[r](0.1910,0.2791){0.0002}
\PST@Border(0.2070,0.3556)
(0.2220,0.3556)

\rput[r](0.1910,0.3556){0.0004}
\PST@Border(0.2070,0.4322)
(0.2220,0.4322)

\rput[r](0.1910,0.4322){0.0006}
\PST@Border(0.2070,0.5087)
(0.2220,0.5087)

\rput[r](0.1910,0.5087){0.0008}
\PST@Border(0.2070,0.5853)
(0.2220,0.5853)

\rput[r](0.1910,0.5853){0.0010}
\PST@Border(0.2070,0.6618)
(0.2220,0.6618)

\rput[r](0.1910,0.6618){0.0012}
\PST@Border(0.2070,0.7384)
(0.2220,0.7384)

\rput[r](0.1910,0.7384){0.0014}
\PST@Border(0.2070,0.8149)
(0.2220,0.8149)

\rput[r](0.1910,0.8149){0.0016}
\PST@Border(0.2070,0.8915)
(0.2220,0.8915)

\rput[r](0.1910,0.8915){0.0018}
\PST@Border(0.2070,0.9680)
(0.2220,0.9680)

\rput[r](0.1910,0.9680){0.0020}
\PST@Border(0.2070,0.1260)
(0.2070,0.1460)

\rput(0.2070,0.0840){ 0}
\PST@Border(0.3009,0.1260)
(0.3009,0.1460)

\rput(0.3009,0.0840){ 5}
\PST@Border(0.3948,0.1260)
(0.3948,0.1460)

\rput(0.3948,0.0840){ 10}
\PST@Border(0.4886,0.1260)
(0.4886,0.1460)

\rput(0.4886,0.0840){ 15}
\PST@Border(0.5825,0.1260)
(0.5825,0.1460)

\rput(0.5825,0.0840){ 20}
\PST@Border(0.6764,0.1260)
(0.6764,0.1460)

\rput(0.6764,0.0840){ 25}
\PST@Border(0.7703,0.1260)
(0.7703,0.1460)

\rput(0.7703,0.0840){ 30}
\PST@Border(0.8641,0.1260)
(0.8641,0.1460)

\rput(0.8641,0.0840){ 35}
\PST@Border(0.9580,0.1260)
(0.9580,0.1460)

\rput(0.9580,0.0840){ 40}
\PST@Border(0.2070,0.9680)
(0.2070,0.1260)
(0.9580,0.1260)
(0.9580,0.9680)
(0.2070,0.9680)

\rput{L}(0.0420,0.5470){$\Delta x$}
\rput(0.5825,0.0210){$\Delta F$}
\rput[r](0.6710,0.9270){punti allungamento}
\PST@Circle(0.2070,0.2025)
\PST@Circle(0.2806,0.2867)
\PST@Circle(0.3542,0.3633)
\PST@Circle(0.4280,0.4437)
\PST@Circle(0.5016,0.5317)
\PST@Circle(0.5752,0.5929)
\PST@Circle(0.6488,0.6924)
\PST@Circle(0.7226,0.7690)
\PST@Circle(0.7962,0.8532)
\PST@Circle(0.8698,0.9336)
\PST@Circle(0.7265,0.9270)
\rput[r](0.6710,0.8850){punti accorciamento}
\PST@Cross(0.2070,0.2025)
\PST@Cross(0.2806,0.2867)
\PST@Cross(0.3542,0.3709)
\PST@Cross(0.4280,0.4551)
\PST@Cross(0.5016,0.5355)
\PST@Cross(0.5752,0.6159)
\PST@Cross(0.6488,0.6963)
\PST@Cross(0.7226,0.7728)
\PST@Cross(0.7962,0.8608)
\PST@Cross(0.8698,0.9336)
\PST@Cross(0.7265,0.8850)
\rput[r](0.6710,0.8430){interpolazione allungamento}
\PST@Dashed(0.6870,0.8430)
(0.7660,0.8430)

\PST@Dashed(0.2070,0.2020)
(0.2070,0.2020)
(0.2137,0.2094)
(0.2204,0.2168)
(0.2271,0.2241)
(0.2338,0.2315)
(0.2405,0.2389)
(0.2472,0.2462)
(0.2539,0.2536)
(0.2606,0.2610)
(0.2673,0.2684)
(0.2739,0.2757)
(0.2806,0.2831)
(0.2873,0.2905)
(0.2940,0.2978)
(0.3007,0.3052)
(0.3074,0.3126)
(0.3141,0.3200)
(0.3208,0.3273)
(0.3275,0.3347)
(0.3342,0.3421)
(0.3409,0.3494)
(0.3476,0.3568)
(0.3543,0.3642)
(0.3610,0.3716)
(0.3677,0.3789)
(0.3744,0.3863)
(0.3811,0.3937)
(0.3878,0.4010)
(0.3944,0.4084)
(0.4011,0.4158)
(0.4078,0.4232)
(0.4145,0.4305)
(0.4212,0.4379)
(0.4279,0.4453)
(0.4346,0.4526)
(0.4413,0.4600)
(0.4480,0.4674)
(0.4547,0.4748)
(0.4614,0.4821)
(0.4681,0.4895)
(0.4748,0.4969)
(0.4815,0.5042)
(0.4882,0.5116)
(0.4949,0.5190)
(0.5016,0.5264)
(0.5083,0.5337)
(0.5149,0.5411)
(0.5216,0.5485)
(0.5283,0.5558)
(0.5350,0.5632)
(0.5417,0.5706)
(0.5484,0.5780)
(0.5551,0.5853)
(0.5618,0.5927)
(0.5685,0.6001)
(0.5752,0.6074)
(0.5819,0.6148)
(0.5886,0.6222)
(0.5953,0.6296)
(0.6020,0.6369)
(0.6087,0.6443)
(0.6154,0.6517)
(0.6221,0.6590)
(0.6288,0.6664)
(0.6354,0.6738)
(0.6421,0.6812)
(0.6488,0.6885)
(0.6555,0.6959)
(0.6622,0.7033)
(0.6689,0.7106)
(0.6756,0.7180)
(0.6823,0.7254)
(0.6890,0.7328)
(0.6957,0.7401)
(0.7024,0.7475)
(0.7091,0.7549)
(0.7158,0.7622)
(0.7225,0.7696)
(0.7292,0.7770)
(0.7359,0.7844)
(0.7426,0.7917)
(0.7493,0.7991)
(0.7560,0.8065)
(0.7626,0.8138)
(0.7693,0.8212)
(0.7760,0.8286)
(0.7827,0.8360)
(0.7894,0.8433)
(0.7961,0.8507)
(0.8028,0.8581)
(0.8095,0.8654)
(0.8162,0.8728)
(0.8229,0.8802)
(0.8296,0.8876)
(0.8363,0.8949)
(0.8430,0.9023)
(0.8497,0.9097)
(0.8564,0.9170)
(0.8631,0.9244)
(0.8698,0.9318)

\rput[r](0.6710,0.8010){interpolazione accorciamento}
\PST@Dotted(0.6870,0.8010)
(0.7660,0.8010)

\PST@Dotted(0.2070,0.2073)
(0.2070,0.2073)
(0.2137,0.2147)
(0.2204,0.2221)
(0.2271,0.2295)
(0.2338,0.2369)
(0.2405,0.2442)
(0.2472,0.2516)
(0.2539,0.2590)
(0.2606,0.2664)
(0.2673,0.2738)
(0.2739,0.2812)
(0.2806,0.2886)
(0.2873,0.2960)
(0.2940,0.3034)
(0.3007,0.3107)
(0.3074,0.3181)
(0.3141,0.3255)
(0.3208,0.3329)
(0.3275,0.3403)
(0.3342,0.3477)
(0.3409,0.3551)
(0.3476,0.3625)
(0.3543,0.3698)
(0.3610,0.3772)
(0.3677,0.3846)
(0.3744,0.3920)
(0.3811,0.3994)
(0.3878,0.4068)
(0.3944,0.4142)
(0.4011,0.4216)
(0.4078,0.4290)
(0.4145,0.4363)
(0.4212,0.4437)
(0.4279,0.4511)
(0.4346,0.4585)
(0.4413,0.4659)
(0.4480,0.4733)
(0.4547,0.4807)
(0.4614,0.4881)
(0.4681,0.4954)
(0.4748,0.5028)
(0.4815,0.5102)
(0.4882,0.5176)
(0.4949,0.5250)
(0.5016,0.5324)
(0.5083,0.5398)
(0.5149,0.5472)
(0.5216,0.5546)
(0.5283,0.5619)
(0.5350,0.5693)
(0.5417,0.5767)
(0.5484,0.5841)
(0.5551,0.5915)
(0.5618,0.5989)
(0.5685,0.6063)
(0.5752,0.6137)
(0.5819,0.6210)
(0.5886,0.6284)
(0.5953,0.6358)
(0.6020,0.6432)
(0.6087,0.6506)
(0.6154,0.6580)
(0.6221,0.6654)
(0.6288,0.6728)
(0.6354,0.6802)
(0.6421,0.6875)
(0.6488,0.6949)
(0.6555,0.7023)
(0.6622,0.7097)
(0.6689,0.7171)
(0.6756,0.7245)
(0.6823,0.7319)
(0.6890,0.7393)
(0.6957,0.7466)
(0.7024,0.7540)
(0.7091,0.7614)
(0.7158,0.7688)
(0.7225,0.7762)
(0.7292,0.7836)
(0.7359,0.7910)
(0.7426,0.7984)
(0.7493,0.8058)
(0.7560,0.8131)
(0.7626,0.8205)
(0.7693,0.8279)
(0.7760,0.8353)
(0.7827,0.8427)
(0.7894,0.8501)
(0.7961,0.8575)
(0.8028,0.8649)
(0.8095,0.8722)
(0.8162,0.8796)
(0.8229,0.8870)
(0.8296,0.8944)
(0.8363,0.9018)
(0.8430,0.9092)
(0.8497,0.9166)
(0.8564,0.9240)
(0.8631,0.9314)
(0.8698,0.9387)

\PST@Border(0.2070,0.9680)
(0.2070,0.1260)
(0.9580,0.1260)
(0.9580,0.9680)
(0.2070,0.9680)

\catcode`@=12
\fi
\endpspicture

\end{figure}
\begin{figure}[p]\caption{Grafico relativo all'estensimetro n.3, con filo di ottone. In ascissa sono riportati i valori di $\Delta F~(\unit{N})$ e in ordinata la corrispondente variazione di lunghezza del filo $\Delta x~(\unit{m})$. Le barre di errore su entrambe le misure sono troppo piccole per poter essere rappresentate su questa scala.}\label{3}
% GNUPLOT: LaTeX picture using PSTRICKS macros
% Define new PST objects, if not already defined
\ifx\PSTloaded\undefined
\def\PSTloaded{t}
\psset{arrowsize=.01 3.2 1.4 .3}
\psset{dotsize=.08}
\catcode`@=11

\newpsobject{PST@Border}{psline}{linewidth=.0015,linestyle=solid}
\newpsobject{PST@Axes}{psline}{linewidth=.0015,linestyle=dotted,dotsep=.004}
\newpsobject{PST@Solid}{psline}{linewidth=.0015,linestyle=solid}
\newpsobject{PST@Dashed}{psline}{linewidth=.0015,linestyle=dashed,dash=.01 .01}
\newpsobject{PST@Dotted}{psline}{linewidth=.0025,linestyle=dotted,dotsep=.008}
\newpsobject{PST@LongDash}{psline}{linewidth=.0015,linestyle=dashed,dash=.02 .01}
\newpsobject{PST@Diamond}{psdots}{linewidth=.001,linestyle=solid,dotstyle=square,dotangle=45}
\newpsobject{PST@Filldiamond}{psdots}{linewidth=.001,linestyle=solid,dotstyle=square*,dotangle=45}
\newpsobject{PST@Cross}{psdots}{linewidth=.001,linestyle=solid,dotstyle=+,dotangle=45}
\newpsobject{PST@Plus}{psdots}{linewidth=.001,linestyle=solid,dotstyle=+}
\newpsobject{PST@Square}{psdots}{linewidth=.001,linestyle=solid,dotstyle=square}
\newpsobject{PST@Circle}{psdots}{linewidth=.001,linestyle=solid,dotstyle=o}
\newpsobject{PST@Triangle}{psdots}{linewidth=.001,linestyle=solid,dotstyle=triangle}
\newpsobject{PST@Pentagon}{psdots}{linewidth=.001,linestyle=solid,dotstyle=pentagon}
\newpsobject{PST@Fillsquare}{psdots}{linewidth=.001,linestyle=solid,dotstyle=square*}
\newpsobject{PST@Fillcircle}{psdots}{linewidth=.001,linestyle=solid,dotstyle=*}
\newpsobject{PST@Filltriangle}{psdots}{linewidth=.001,linestyle=solid,dotstyle=triangle*}
\newpsobject{PST@Fillpentagon}{psdots}{linewidth=.001,linestyle=solid,dotstyle=pentagon*}
\newpsobject{PST@Arrow}{psline}{linewidth=.001,linestyle=solid}
\catcode`@=12

\fi
\psset{unit=5.0in,xunit=5.0in,yunit=3.0in}
\pspicture(0.000000,0.000000)(1.000000,1.000000)
\ifx\nofigs\undefined
\catcode`@=11

\PST@Border(0.1010,0.0840)
(0.1160,0.0840)

\rput[r](0.0850,0.0840){1.4}
\PST@Border(0.1010,0.2103)
(0.1160,0.2103)

\rput[r](0.0850,0.2103){1.6}
\PST@Border(0.1010,0.3366)
(0.1160,0.3366)

\rput[r](0.0850,0.3366){1.8}
\PST@Border(0.1010,0.4629)
(0.1160,0.4629)

\rput[r](0.0850,0.4629){2.0}
\PST@Border(0.1010,0.5891)
(0.1160,0.5891)

\rput[r](0.0850,0.5891){2.2}
\PST@Border(0.1010,0.7154)
(0.1160,0.7154)

\rput[r](0.0850,0.7154){2.4}
\PST@Border(0.1010,0.8417)
(0.1160,0.8417)

\rput[r](0.0850,0.8417){2.6}
\PST@Border(0.1010,0.9680)
(0.1160,0.9680)

\rput[r](0.0850,0.9680){2.8}
\PST@Border(0.1010,0.0840)
(0.1010,0.1040)

\rput(0.1010,0.0420){ 22}
\PST@Border(0.1962,0.0840)
(0.1962,0.1040)

\rput(0.1962,0.0420){ 24}
\PST@Border(0.2914,0.0840)
(0.2914,0.1040)

\rput(0.2914,0.0420){ 26}
\PST@Border(0.3867,0.0840)
(0.3867,0.1040)

\rput(0.3867,0.0420){ 28}
\PST@Border(0.4819,0.0840)
(0.4819,0.1040)

\rput(0.4819,0.0420){ 30}
\PST@Border(0.5771,0.0840)
(0.5771,0.1040)

\rput(0.5771,0.0420){ 32}
\PST@Border(0.6723,0.0840)
(0.6723,0.1040)

\rput(0.6723,0.0420){ 34}
\PST@Border(0.7676,0.0840)
(0.7676,0.1040)

\rput(0.7676,0.0420){ 36}
\PST@Border(0.8628,0.0840)
(0.8628,0.1040)

\rput(0.8628,0.0420){ 38}
\PST@Border(0.9580,0.0840)
(0.9580,0.1040)

\rput(0.9580,0.0420){ 40}
\PST@Border(0.1010,0.9680)
(0.1010,0.0840)
(0.9580,0.0840)
(0.9580,0.9680)
(0.1010,0.9680)

\PST@Solid(0.1010,0.1459)
(0.1010,0.1459)
(0.1097,0.1536)
(0.1183,0.1614)
(0.1270,0.1692)
(0.1356,0.1770)
(0.1443,0.1847)
(0.1529,0.1925)
(0.1616,0.2003)
(0.1703,0.2081)
(0.1789,0.2159)
(0.1876,0.2236)
(0.1962,0.2314)
(0.2049,0.2392)
(0.2135,0.2470)
(0.2222,0.2548)
(0.2308,0.2625)
(0.2395,0.2703)
(0.2482,0.2781)
(0.2568,0.2859)
(0.2655,0.2937)
(0.2741,0.3014)
(0.2828,0.3092)
(0.2914,0.3170)
(0.3001,0.3248)
(0.3088,0.3326)
(0.3174,0.3403)
(0.3261,0.3481)
(0.3347,0.3559)
(0.3434,0.3637)
(0.3520,0.3714)
(0.3607,0.3792)
(0.3694,0.3870)
(0.3780,0.3948)
(0.3867,0.4026)
(0.3953,0.4103)
(0.4040,0.4181)
(0.4126,0.4259)
(0.4213,0.4337)
(0.4299,0.4415)
(0.4386,0.4492)
(0.4473,0.4570)
(0.4559,0.4648)
(0.4646,0.4726)
(0.4732,0.4804)
(0.4819,0.4881)
(0.4905,0.4959)
(0.4992,0.5037)
(0.5079,0.5115)
(0.5165,0.5192)
(0.5252,0.5270)
(0.5338,0.5348)
(0.5425,0.5426)
(0.5511,0.5504)
(0.5598,0.5581)
(0.5685,0.5659)
(0.5771,0.5737)
(0.5858,0.5815)
(0.5944,0.5893)
(0.6031,0.5970)
(0.6117,0.6048)
(0.6204,0.6126)
(0.6291,0.6204)
(0.6377,0.6282)
(0.6464,0.6359)
(0.6550,0.6437)
(0.6637,0.6515)
(0.6723,0.6593)
(0.6810,0.6671)
(0.6896,0.6748)
(0.6983,0.6826)
(0.7070,0.6904)
(0.7156,0.6982)
(0.7243,0.7059)
(0.7329,0.7137)
(0.7416,0.7215)
(0.7502,0.7293)
(0.7589,0.7371)
(0.7676,0.7448)
(0.7762,0.7526)
(0.7849,0.7604)
(0.7935,0.7682)
(0.8022,0.7760)
(0.8108,0.7837)
(0.8195,0.7915)
(0.8282,0.7993)
(0.8368,0.8071)
(0.8455,0.8149)
(0.8541,0.8226)
(0.8628,0.8304)
(0.8714,0.8382)
(0.8801,0.8460)
(0.8887,0.8538)
(0.8974,0.8615)
(0.9061,0.8693)
(0.9147,0.8771)
(0.9234,0.8849)
(0.9320,0.8926)
(0.9407,0.9004)
(0.9493,0.9082)
(0.9580,0.9160)

\PST@Diamond(0.9580,0.9156)
\PST@Diamond(0.8771,0.8436)
\PST@Diamond(0.7676,0.7464)
\PST@Diamond(0.6723,0.6580)
\PST@Diamond(0.5771,0.5734)
\PST@Diamond(0.4819,0.4875)
\PST@Diamond(0.3867,0.4035)
\PST@Diamond(0.2962,0.3202)
\PST@Diamond(0.1962,0.2330)
\PST@Diamond(0.1010,0.1452)
\PST@Border(0.1010,0.9680)
(0.1010,0.0840)
(0.9580,0.0840)
(0.9580,0.9680)
(0.1010,0.9680)

\catcode`@=12
\fi
\endpspicture

\end{figure}
\begin{figure}[p]\caption{Grafico relativo all'estensimetro n.6. In ascissa sono riportati i valori di $\Delta F~(\unit{N})$ e in ordinata la corrispondente variazione di lunghezza del filo $\Delta x~(\unit{m})$. Le barre di errore su entrambe le misure sono troppo piccole per poter essere rappresentate su questa scala.}\label{6}
% GNUPLOT: LaTeX picture using PSTRICKS macros
% Define new PST objects, if not already defined
\ifx\PSTloaded\undefined
\def\PSTloaded{t}
\psset{arrowsize=.01 3.2 1.4 .3}
\psset{dotsize=.125}
\catcode`@=11

\newpsobject{PST@Border}{psline}{linewidth=.0015,linestyle=solid}
\newpsobject{PST@Axes}{psline}{linewidth=.0015,linestyle=dotted,dotsep=.004}
\newpsobject{PST@Solid}{psline}{linewidth=.0015,linestyle=solid}
\newpsobject{PST@Dashed}{psline}{linewidth=.0015,linestyle=dashed,dash=.01 .01}
\newpsobject{PST@Dotted}{psline}{linewidth=.0025,linestyle=dotted,dotsep=.008}
\newpsobject{PST@LongDash}{psline}{linewidth=.0015,linestyle=dashed,dash=.02 .01}
\newpsobject{PST@Diamond}{psdots}{linewidth=.001,linestyle=solid,dotstyle=square,dotangle=45}
\newpsobject{PST@Filldiamond}{psdots}{linewidth=.001,linestyle=solid,dotstyle=square*,dotangle=45}
\newpsobject{PST@Cross}{psdots}{linewidth=.001,linestyle=solid,dotstyle=+,dotangle=45}
\newpsobject{PST@Plus}{psdots}{linewidth=.001,linestyle=solid,dotstyle=+}
\newpsobject{PST@Square}{psdots}{linewidth=.001,linestyle=solid,dotstyle=square}
\newpsobject{PST@Circle}{psdots}{linewidth=.001,linestyle=solid,dotstyle=o}
\newpsobject{PST@Triangle}{psdots}{linewidth=.001,linestyle=solid,dotstyle=triangle}
\newpsobject{PST@Pentagon}{psdots}{linewidth=.001,linestyle=solid,dotstyle=pentagon}
\newpsobject{PST@Fillsquare}{psdots}{linewidth=.001,linestyle=solid,dotstyle=square*}
\newpsobject{PST@Fillcircle}{psdots}{linewidth=.001,linestyle=solid,dotstyle=*}
\newpsobject{PST@Filltriangle}{psdots}{linewidth=.001,linestyle=solid,dotstyle=triangle*}
\newpsobject{PST@Fillpentagon}{psdots}{linewidth=.001,linestyle=solid,dotstyle=pentagon*}
\newpsobject{PST@Arrow}{psline}{linewidth=.001,linestyle=solid}
\catcode`@=12

\fi
\psset{unit=5.0in,xunit=5.0in,yunit=3.0in}
\pspicture(0.000000,0.000000)(1.000000,1.000000)
\ifx\nofigs\undefined
\catcode`@=11

\PST@Border(0.2070,0.1260)
(0.2220,0.1260)

\rput[r](0.1910,0.1260){-0.0005}
\PST@Border(0.2070,0.2663)
(0.2220,0.2663)

\rput[r](0.1910,0.2663){0.0000}
\PST@Border(0.2070,0.4067)
(0.2220,0.4067)

\rput[r](0.1910,0.4067){0.0005}
\PST@Border(0.2070,0.5470)
(0.2220,0.5470)

\rput[r](0.1910,0.5470){0.0010}
\PST@Border(0.2070,0.6873)
(0.2220,0.6873)

\rput[r](0.1910,0.6873){0.0015}
\PST@Border(0.2070,0.8277)
(0.2220,0.8277)

\rput[r](0.1910,0.8277){0.0020}
\PST@Border(0.2070,0.9680)
(0.2220,0.9680)

\rput[r](0.1910,0.9680){0.0025}
\PST@Border(0.2070,0.1260)
(0.2070,0.1460)

\rput(0.2070,0.0840){ 0}
\PST@Border(0.3009,0.1260)
(0.3009,0.1460)

\rput(0.3009,0.0840){ 5}
\PST@Border(0.3948,0.1260)
(0.3948,0.1460)

\rput(0.3948,0.0840){ 10}
\PST@Border(0.4886,0.1260)
(0.4886,0.1460)

\rput(0.4886,0.0840){ 15}
\PST@Border(0.5825,0.1260)
(0.5825,0.1460)

\rput(0.5825,0.0840){ 20}
\PST@Border(0.6764,0.1260)
(0.6764,0.1460)

\rput(0.6764,0.0840){ 25}
\PST@Border(0.7703,0.1260)
(0.7703,0.1460)

\rput(0.7703,0.0840){ 30}
\PST@Border(0.8641,0.1260)
(0.8641,0.1460)

\rput(0.8641,0.0840){ 35}
\PST@Border(0.9580,0.1260)
(0.9580,0.1460)

\rput(0.9580,0.0840){ 40}
\PST@Border(0.2070,0.9680)
(0.2070,0.1260)
(0.9580,0.1260)
(0.9580,0.9680)
(0.2070,0.9680)

\rput{L}(0.0420,0.5470){$\Delta x$}
\rput(0.5825,0.0210){$\Delta F$}
\rput[r](0.6710,0.9270){punti allungamento}
\PST@Circle(0.2070,0.2663)
\PST@Circle(0.2806,0.3365)
\PST@Circle(0.3542,0.4067)
\PST@Circle(0.4280,0.4768)
\PST@Circle(0.5016,0.5470)
\PST@Circle(0.5752,0.6200)
\PST@Circle(0.6488,0.6901)
\PST@Circle(0.7226,0.7603)
\PST@Circle(0.7962,0.8305)
\PST@Circle(0.8698,0.8950)
\PST@Circle(0.7265,0.9270)
\rput[r](0.6710,0.8850){punti accorciamento}
\PST@Cross(0.2070,0.2635)
\PST@Cross(0.2806,0.3421)
\PST@Cross(0.3542,0.4095)
\PST@Cross(0.4280,0.4768)
\PST@Cross(0.5016,0.5470)
\PST@Cross(0.5752,0.6200)
\PST@Cross(0.6488,0.6901)
\PST@Cross(0.7226,0.7603)
\PST@Cross(0.7962,0.8305)
\PST@Cross(0.8698,0.8950)
\PST@Cross(0.7265,0.8850)
\rput[r](0.6710,0.8430){interpolazione allungamento}
\PST@Dashed(0.6870,0.8430)
(0.7660,0.8430)

\PST@Dashed(0.2070,0.2667)
(0.2070,0.2667)
(0.2137,0.2731)
(0.2204,0.2794)
(0.2271,0.2858)
(0.2338,0.2922)
(0.2405,0.2986)
(0.2472,0.3050)
(0.2539,0.3114)
(0.2606,0.3178)
(0.2673,0.3242)
(0.2739,0.3306)
(0.2806,0.3369)
(0.2873,0.3433)
(0.2940,0.3497)
(0.3007,0.3561)
(0.3074,0.3625)
(0.3141,0.3689)
(0.3208,0.3753)
(0.3275,0.3817)
(0.3342,0.3881)
(0.3409,0.3944)
(0.3476,0.4008)
(0.3543,0.4072)
(0.3610,0.4136)
(0.3677,0.4200)
(0.3744,0.4264)
(0.3811,0.4328)
(0.3878,0.4392)
(0.3944,0.4456)
(0.4011,0.4519)
(0.4078,0.4583)
(0.4145,0.4647)
(0.4212,0.4711)
(0.4279,0.4775)
(0.4346,0.4839)
(0.4413,0.4903)
(0.4480,0.4967)
(0.4547,0.5031)
(0.4614,0.5095)
(0.4681,0.5158)
(0.4748,0.5222)
(0.4815,0.5286)
(0.4882,0.5350)
(0.4949,0.5414)
(0.5016,0.5478)
(0.5083,0.5542)
(0.5149,0.5606)
(0.5216,0.5670)
(0.5283,0.5733)
(0.5350,0.5797)
(0.5417,0.5861)
(0.5484,0.5925)
(0.5551,0.5989)
(0.5618,0.6053)
(0.5685,0.6117)
(0.5752,0.6181)
(0.5819,0.6245)
(0.5886,0.6308)
(0.5953,0.6372)
(0.6020,0.6436)
(0.6087,0.6500)
(0.6154,0.6564)
(0.6221,0.6628)
(0.6288,0.6692)
(0.6354,0.6756)
(0.6421,0.6820)
(0.6488,0.6883)
(0.6555,0.6947)
(0.6622,0.7011)
(0.6689,0.7075)
(0.6756,0.7139)
(0.6823,0.7203)
(0.6890,0.7267)
(0.6957,0.7331)
(0.7024,0.7395)
(0.7091,0.7458)
(0.7158,0.7522)
(0.7225,0.7586)
(0.7292,0.7650)
(0.7359,0.7714)
(0.7426,0.7778)
(0.7493,0.7842)
(0.7560,0.7906)
(0.7626,0.7970)
(0.7693,0.8033)
(0.7760,0.8097)
(0.7827,0.8161)
(0.7894,0.8225)
(0.7961,0.8289)
(0.8028,0.8353)
(0.8095,0.8417)
(0.8162,0.8481)
(0.8229,0.8545)
(0.8296,0.8609)
(0.8363,0.8672)
(0.8430,0.8736)
(0.8497,0.8800)
(0.8564,0.8864)
(0.8631,0.8928)
(0.8698,0.8992)

\rput[r](0.6710,0.8010){interpolazione accorciamento}
\PST@Dotted(0.6870,0.8010)
(0.7660,0.8010)

\PST@Dotted(0.2070,0.2680)
(0.2070,0.2680)
(0.2137,0.2744)
(0.2204,0.2807)
(0.2271,0.2871)
(0.2338,0.2935)
(0.2405,0.2999)
(0.2472,0.3062)
(0.2539,0.3126)
(0.2606,0.3190)
(0.2673,0.3254)
(0.2739,0.3317)
(0.2806,0.3381)
(0.2873,0.3445)
(0.2940,0.3508)
(0.3007,0.3572)
(0.3074,0.3636)
(0.3141,0.3700)
(0.3208,0.3763)
(0.3275,0.3827)
(0.3342,0.3891)
(0.3409,0.3955)
(0.3476,0.4018)
(0.3543,0.4082)
(0.3610,0.4146)
(0.3677,0.4210)
(0.3744,0.4273)
(0.3811,0.4337)
(0.3878,0.4401)
(0.3944,0.4465)
(0.4011,0.4528)
(0.4078,0.4592)
(0.4145,0.4656)
(0.4212,0.4719)
(0.4279,0.4783)
(0.4346,0.4847)
(0.4413,0.4911)
(0.4480,0.4974)
(0.4547,0.5038)
(0.4614,0.5102)
(0.4681,0.5166)
(0.4748,0.5229)
(0.4815,0.5293)
(0.4882,0.5357)
(0.4949,0.5421)
(0.5016,0.5484)
(0.5083,0.5548)
(0.5149,0.5612)
(0.5216,0.5676)
(0.5283,0.5739)
(0.5350,0.5803)
(0.5417,0.5867)
(0.5484,0.5930)
(0.5551,0.5994)
(0.5618,0.6058)
(0.5685,0.6122)
(0.5752,0.6185)
(0.5819,0.6249)
(0.5886,0.6313)
(0.5953,0.6377)
(0.6020,0.6440)
(0.6087,0.6504)
(0.6154,0.6568)
(0.6221,0.6632)
(0.6288,0.6695)
(0.6354,0.6759)
(0.6421,0.6823)
(0.6488,0.6887)
(0.6555,0.6950)
(0.6622,0.7014)
(0.6689,0.7078)
(0.6756,0.7141)
(0.6823,0.7205)
(0.6890,0.7269)
(0.6957,0.7333)
(0.7024,0.7396)
(0.7091,0.7460)
(0.7158,0.7524)
(0.7225,0.7588)
(0.7292,0.7651)
(0.7359,0.7715)
(0.7426,0.7779)
(0.7493,0.7843)
(0.7560,0.7906)
(0.7626,0.7970)
(0.7693,0.8034)
(0.7760,0.8098)
(0.7827,0.8161)
(0.7894,0.8225)
(0.7961,0.8289)
(0.8028,0.8352)
(0.8095,0.8416)
(0.8162,0.8480)
(0.8229,0.8544)
(0.8296,0.8607)
(0.8363,0.8671)
(0.8430,0.8735)
(0.8497,0.8799)
(0.8564,0.8862)
(0.8631,0.8926)
(0.8698,0.8990)

\PST@Border(0.2070,0.9680)
(0.2070,0.1260)
(0.9580,0.1260)
(0.9580,0.9680)
(0.2070,0.9680)

\catcode`@=12
\fi
\endpspicture

\end{figure}
\begin{figure}[p]\caption{Grafico relativo all'estensimetro n.7. In ascissa sono riportati i valori di $\Delta F~(\unit{N})$ e in ordinata la corrispondente variazione di lunghezza del filo $\Delta x~(\unit{m})$. Le barre di errore su entrambe le misure sono troppo piccole per poter essere rappresentate su questa scala.}\label{7}
% GNUPLOT: LaTeX picture using PSTRICKS macros
% Define new PST objects, if not already defined
\ifx\PSTloaded\undefined
\def\PSTloaded{t}
\psset{arrowsize=.01 3.2 1.4 .3}
\psset{dotsize=.125}
\catcode`@=11

\newpsobject{PST@Border}{psline}{linewidth=.0015,linestyle=solid}
\newpsobject{PST@Axes}{psline}{linewidth=.0015,linestyle=dotted,dotsep=.004}
\newpsobject{PST@Solid}{psline}{linewidth=.0015,linestyle=solid}
\newpsobject{PST@Dashed}{psline}{linewidth=.0015,linestyle=dashed,dash=.01 .01}
\newpsobject{PST@Dotted}{psline}{linewidth=.0025,linestyle=dotted,dotsep=.008}
\newpsobject{PST@LongDash}{psline}{linewidth=.0015,linestyle=dashed,dash=.02 .01}
\newpsobject{PST@Diamond}{psdots}{linewidth=.001,linestyle=solid,dotstyle=square,dotangle=45}
\newpsobject{PST@Filldiamond}{psdots}{linewidth=.001,linestyle=solid,dotstyle=square*,dotangle=45}
\newpsobject{PST@Cross}{psdots}{linewidth=.001,linestyle=solid,dotstyle=+,dotangle=45}
\newpsobject{PST@Plus}{psdots}{linewidth=.001,linestyle=solid,dotstyle=+}
\newpsobject{PST@Square}{psdots}{linewidth=.001,linestyle=solid,dotstyle=square}
\newpsobject{PST@Circle}{psdots}{linewidth=.001,linestyle=solid,dotstyle=o}
\newpsobject{PST@Triangle}{psdots}{linewidth=.001,linestyle=solid,dotstyle=triangle}
\newpsobject{PST@Pentagon}{psdots}{linewidth=.001,linestyle=solid,dotstyle=pentagon}
\newpsobject{PST@Fillsquare}{psdots}{linewidth=.001,linestyle=solid,dotstyle=square*}
\newpsobject{PST@Fillcircle}{psdots}{linewidth=.001,linestyle=solid,dotstyle=*}
\newpsobject{PST@Filltriangle}{psdots}{linewidth=.001,linestyle=solid,dotstyle=triangle*}
\newpsobject{PST@Fillpentagon}{psdots}{linewidth=.001,linestyle=solid,dotstyle=pentagon*}
\newpsobject{PST@Arrow}{psline}{linewidth=.001,linestyle=solid}
\catcode`@=12

\fi
\psset{unit=5.0in,xunit=5.0in,yunit=3.0in}
\pspicture(0.000000,0.000000)(1.000000,1.000000)
\ifx\nofigs\undefined
\catcode`@=11

\PST@Border(0.1910,0.1260)
(0.2060,0.1260)

\rput[r](0.1750,0.1260){0.0000}
\PST@Border(0.1910,0.2102)
(0.2060,0.2102)

\rput[r](0.1750,0.2102){0.0002}
\PST@Border(0.1910,0.2944)
(0.2060,0.2944)

\rput[r](0.1750,0.2944){0.0004}
\PST@Border(0.1910,0.3786)
(0.2060,0.3786)

\rput[r](0.1750,0.3786){0.0006}
\PST@Border(0.1910,0.4628)
(0.2060,0.4628)

\rput[r](0.1750,0.4628){0.0008}
\PST@Border(0.1910,0.5470)
(0.2060,0.5470)

\rput[r](0.1750,0.5470){0.0010}
\PST@Border(0.1910,0.6312)
(0.2060,0.6312)

\rput[r](0.1750,0.6312){0.0012}
\PST@Border(0.1910,0.7154)
(0.2060,0.7154)

\rput[r](0.1750,0.7154){0.0014}
\PST@Border(0.1910,0.7996)
(0.2060,0.7996)

\rput[r](0.1750,0.7996){0.0016}
\PST@Border(0.1910,0.8838)
(0.2060,0.8838)

\rput[r](0.1750,0.8838){0.0018}
\PST@Border(0.1910,0.9680)
(0.2060,0.9680)

\rput[r](0.1750,0.9680){0.0020}
\PST@Border(0.1910,0.1260)
(0.1910,0.1460)

\rput(0.1910,0.0840){ 0}
\PST@Border(0.2869,0.1260)
(0.2869,0.1460)

\rput(0.2869,0.0840){ 5}
\PST@Border(0.3828,0.1260)
(0.3828,0.1460)

\rput(0.3828,0.0840){ 10}
\PST@Border(0.4786,0.1260)
(0.4786,0.1460)

\rput(0.4786,0.0840){ 15}
\PST@Border(0.5745,0.1260)
(0.5745,0.1460)

\rput(0.5745,0.0840){ 20}
\PST@Border(0.6704,0.1260)
(0.6704,0.1460)

\rput(0.6704,0.0840){ 25}
\PST@Border(0.7663,0.1260)
(0.7663,0.1460)

\rput(0.7663,0.0840){ 30}
\PST@Border(0.8621,0.1260)
(0.8621,0.1460)

\rput(0.8621,0.0840){ 35}
\PST@Border(0.9580,0.1260)
(0.9580,0.1460)

\rput(0.9580,0.0840){ 40}
\PST@Border(0.1910,0.9680)
(0.1910,0.1260)
(0.9580,0.1260)
(0.9580,0.9680)
(0.1910,0.9680)

\rput{L}(0.0420,0.5470){$\Delta x$}
\rput(0.5745,0.0210){$\Delta F$}
\rput[r](0.6550,0.9270){punti allungamento}
\PST@Circle(0.1910,0.1260)
\PST@Circle(0.2662,0.2439)
\PST@Circle(0.3413,0.3154)
\PST@Circle(0.4167,0.4039)
\PST@Circle(0.4919,0.4965)
\PST@Circle(0.5670,0.5891)
\PST@Circle(0.6422,0.6817)
\PST@Circle(0.7175,0.7701)
\PST@Circle(0.7927,0.8628)
\PST@Circle(0.8679,0.9427)
\PST@Circle(0.7105,0.9270)
\rput[r](0.6550,0.8850){punti accorciamento}
\PST@Cross(0.1910,0.1260)
\PST@Cross(0.2662,0.2439)
\PST@Cross(0.3413,0.3154)
\PST@Cross(0.4167,0.4081)
\PST@Cross(0.4919,0.4965)
\PST@Cross(0.5670,0.5891)
\PST@Cross(0.6422,0.6817)
\PST@Cross(0.7175,0.7785)
\PST@Cross(0.7927,0.8628)
\PST@Cross(0.8679,0.9427)
\PST@Cross(0.7105,0.8850)
\rput[r](0.6550,0.8430){interpolazione allungamento}
\PST@Dashed(0.6710,0.8430)
(0.7500,0.8430)

\PST@Dashed(0.1910,0.1374)
(0.1910,0.1374)
(0.1978,0.1456)
(0.2047,0.1538)
(0.2115,0.1620)
(0.2183,0.1702)
(0.2252,0.1784)
(0.2320,0.1866)
(0.2389,0.1948)
(0.2457,0.2030)
(0.2525,0.2112)
(0.2594,0.2194)
(0.2662,0.2276)
(0.2730,0.2358)
(0.2799,0.2439)
(0.2867,0.2521)
(0.2936,0.2603)
(0.3004,0.2685)
(0.3072,0.2767)
(0.3141,0.2849)
(0.3209,0.2931)
(0.3277,0.3013)
(0.3346,0.3095)
(0.3414,0.3177)
(0.3483,0.3259)
(0.3551,0.3341)
(0.3619,0.3423)
(0.3688,0.3505)
(0.3756,0.3587)
(0.3824,0.3669)
(0.3893,0.3751)
(0.3961,0.3833)
(0.4030,0.3915)
(0.4098,0.3997)
(0.4166,0.4079)
(0.4235,0.4161)
(0.4303,0.4243)
(0.4371,0.4325)
(0.4440,0.4407)
(0.4508,0.4489)
(0.4576,0.4571)
(0.4645,0.4653)
(0.4713,0.4735)
(0.4782,0.4817)
(0.4850,0.4899)
(0.4918,0.4981)
(0.4987,0.5063)
(0.5055,0.5145)
(0.5123,0.5227)
(0.5192,0.5309)
(0.5260,0.5391)
(0.5329,0.5473)
(0.5397,0.5555)
(0.5465,0.5637)
(0.5534,0.5719)
(0.5602,0.5801)
(0.5670,0.5883)
(0.5739,0.5965)
(0.5807,0.6047)
(0.5876,0.6129)
(0.5944,0.6211)
(0.6012,0.6293)
(0.6081,0.6375)
(0.6149,0.6457)
(0.6217,0.6539)
(0.6286,0.6621)
(0.6354,0.6703)
(0.6423,0.6785)
(0.6491,0.6867)
(0.6559,0.6949)
(0.6628,0.7031)
(0.6696,0.7113)
(0.6764,0.7195)
(0.6833,0.7277)
(0.6901,0.7359)
(0.6969,0.7441)
(0.7038,0.7523)
(0.7106,0.7605)
(0.7175,0.7687)
(0.7243,0.7769)
(0.7311,0.7851)
(0.7380,0.7933)
(0.7448,0.8015)
(0.7516,0.8097)
(0.7585,0.8179)
(0.7653,0.8261)
(0.7722,0.8343)
(0.7790,0.8425)
(0.7858,0.8507)
(0.7927,0.8589)
(0.7995,0.8671)
(0.8063,0.8753)
(0.8132,0.8835)
(0.8200,0.8917)
(0.8269,0.8999)
(0.8337,0.9081)
(0.8405,0.9163)
(0.8474,0.9245)
(0.8542,0.9327)
(0.8610,0.9409)
(0.8679,0.9491)

\rput[r](0.6550,0.8010){interpolazione accorciamento}
\PST@Dotted(0.6710,0.8010)
(0.7500,0.8010)

\PST@Dotted(0.1910,0.1378)
(0.1910,0.1378)
(0.1978,0.1460)
(0.2047,0.1543)
(0.2115,0.1625)
(0.2183,0.1707)
(0.2252,0.1789)
(0.2320,0.1871)
(0.2389,0.1953)
(0.2457,0.2035)
(0.2525,0.2118)
(0.2594,0.2200)
(0.2662,0.2282)
(0.2730,0.2364)
(0.2799,0.2446)
(0.2867,0.2528)
(0.2936,0.2610)
(0.3004,0.2693)
(0.3072,0.2775)
(0.3141,0.2857)
(0.3209,0.2939)
(0.3277,0.3021)
(0.3346,0.3103)
(0.3414,0.3186)
(0.3483,0.3268)
(0.3551,0.3350)
(0.3619,0.3432)
(0.3688,0.3514)
(0.3756,0.3596)
(0.3824,0.3678)
(0.3893,0.3761)
(0.3961,0.3843)
(0.4030,0.3925)
(0.4098,0.4007)
(0.4166,0.4089)
(0.4235,0.4171)
(0.4303,0.4254)
(0.4371,0.4336)
(0.4440,0.4418)
(0.4508,0.4500)
(0.4576,0.4582)
(0.4645,0.4664)
(0.4713,0.4746)
(0.4782,0.4829)
(0.4850,0.4911)
(0.4918,0.4993)
(0.4987,0.5075)
(0.5055,0.5157)
(0.5123,0.5239)
(0.5192,0.5322)
(0.5260,0.5404)
(0.5329,0.5486)
(0.5397,0.5568)
(0.5465,0.5650)
(0.5534,0.5732)
(0.5602,0.5814)
(0.5670,0.5897)
(0.5739,0.5979)
(0.5807,0.6061)
(0.5876,0.6143)
(0.5944,0.6225)
(0.6012,0.6307)
(0.6081,0.6389)
(0.6149,0.6472)
(0.6217,0.6554)
(0.6286,0.6636)
(0.6354,0.6718)
(0.6423,0.6800)
(0.6491,0.6882)
(0.6559,0.6965)
(0.6628,0.7047)
(0.6696,0.7129)
(0.6764,0.7211)
(0.6833,0.7293)
(0.6901,0.7375)
(0.6969,0.7457)
(0.7038,0.7540)
(0.7106,0.7622)
(0.7175,0.7704)
(0.7243,0.7786)
(0.7311,0.7868)
(0.7380,0.7950)
(0.7448,0.8033)
(0.7516,0.8115)
(0.7585,0.8197)
(0.7653,0.8279)
(0.7722,0.8361)
(0.7790,0.8443)
(0.7858,0.8525)
(0.7927,0.8608)
(0.7995,0.8690)
(0.8063,0.8772)
(0.8132,0.8854)
(0.8200,0.8936)
(0.8269,0.9018)
(0.8337,0.9101)
(0.8405,0.9183)
(0.8474,0.9265)
(0.8542,0.9347)
(0.8610,0.9429)
(0.8679,0.9511)

\PST@Border(0.1910,0.9680)
(0.1910,0.1260)
(0.9580,0.1260)
(0.9580,0.9680)
(0.1910,0.9680)

\catcode`@=12
\fi
\endpspicture

\end{figure}
\begin{figure}[p]\caption{Grafico relativo all'estensimetro n.8. In ascissa sono riportati i valori di $\Delta F~(\unit{N})$ e in ordinata la corrispondente variazione di lunghezza del filo $\Delta x~(\unit{m})$. Le barre di errore su entrambe le misure sono troppo piccole per poter essere rappresentate su questa scala.}\label{8}
% GNUPLOT: LaTeX picture using PSTRICKS macros
% Define new PST objects, if not already defined
\ifx\PSTloaded\undefined
\def\PSTloaded{t}
\psset{arrowsize=.01 3.2 1.4 .3}
\psset{dotsize=.125}
\catcode`@=11

\newpsobject{PST@Border}{psline}{linewidth=.0015,linestyle=solid}
\newpsobject{PST@Axes}{psline}{linewidth=.0015,linestyle=dotted,dotsep=.004}
\newpsobject{PST@Solid}{psline}{linewidth=.0015,linestyle=solid}
\newpsobject{PST@Dashed}{psline}{linewidth=.0015,linestyle=dashed,dash=.01 .01}
\newpsobject{PST@Dotted}{psline}{linewidth=.0025,linestyle=dotted,dotsep=.008}
\newpsobject{PST@LongDash}{psline}{linewidth=.0015,linestyle=dashed,dash=.02 .01}
\newpsobject{PST@Diamond}{psdots}{linewidth=.001,linestyle=solid,dotstyle=square,dotangle=45}
\newpsobject{PST@Filldiamond}{psdots}{linewidth=.001,linestyle=solid,dotstyle=square*,dotangle=45}
\newpsobject{PST@Cross}{psdots}{linewidth=.001,linestyle=solid,dotstyle=+,dotangle=45}
\newpsobject{PST@Plus}{psdots}{linewidth=.001,linestyle=solid,dotstyle=+}
\newpsobject{PST@Square}{psdots}{linewidth=.001,linestyle=solid,dotstyle=square}
\newpsobject{PST@Circle}{psdots}{linewidth=.001,linestyle=solid,dotstyle=o}
\newpsobject{PST@Triangle}{psdots}{linewidth=.001,linestyle=solid,dotstyle=triangle}
\newpsobject{PST@Pentagon}{psdots}{linewidth=.001,linestyle=solid,dotstyle=pentagon}
\newpsobject{PST@Fillsquare}{psdots}{linewidth=.001,linestyle=solid,dotstyle=square*}
\newpsobject{PST@Fillcircle}{psdots}{linewidth=.001,linestyle=solid,dotstyle=*}
\newpsobject{PST@Filltriangle}{psdots}{linewidth=.001,linestyle=solid,dotstyle=triangle*}
\newpsobject{PST@Fillpentagon}{psdots}{linewidth=.001,linestyle=solid,dotstyle=pentagon*}
\newpsobject{PST@Arrow}{psline}{linewidth=.001,linestyle=solid}
\catcode`@=12

\fi
\psset{unit=5.0in,xunit=5.0in,yunit=3.0in}
\pspicture(0.000000,0.000000)(1.000000,1.000000)
\ifx\nofigs\undefined
\catcode`@=11

\PST@Border(0.1910,0.1260)
(0.2060,0.1260)

\rput[r](0.1750,0.1260){0.0000}
\PST@Border(0.1910,0.2313)
(0.2060,0.2313)

\rput[r](0.1750,0.2313){0.0002}
\PST@Border(0.1910,0.3365)
(0.2060,0.3365)

\rput[r](0.1750,0.3365){0.0004}
\PST@Border(0.1910,0.4418)
(0.2060,0.4418)

\rput[r](0.1750,0.4418){0.0006}
\PST@Border(0.1910,0.5470)
(0.2060,0.5470)

\rput[r](0.1750,0.5470){0.0008}
\PST@Border(0.1910,0.6523)
(0.2060,0.6523)

\rput[r](0.1750,0.6523){0.0010}
\PST@Border(0.1910,0.7575)
(0.2060,0.7575)

\rput[r](0.1750,0.7575){0.0012}
\PST@Border(0.1910,0.8628)
(0.2060,0.8628)

\rput[r](0.1750,0.8628){0.0014}
\PST@Border(0.1910,0.9680)
(0.2060,0.9680)

\rput[r](0.1750,0.9680){0.0016}
\PST@Border(0.1910,0.1260)
(0.1910,0.1460)

\rput(0.1910,0.0840){ 0}
\PST@Border(0.2869,0.1260)
(0.2869,0.1460)

\rput(0.2869,0.0840){ 5}
\PST@Border(0.3828,0.1260)
(0.3828,0.1460)

\rput(0.3828,0.0840){ 10}
\PST@Border(0.4786,0.1260)
(0.4786,0.1460)

\rput(0.4786,0.0840){ 15}
\PST@Border(0.5745,0.1260)
(0.5745,0.1460)

\rput(0.5745,0.0840){ 20}
\PST@Border(0.6704,0.1260)
(0.6704,0.1460)

\rput(0.6704,0.0840){ 25}
\PST@Border(0.7663,0.1260)
(0.7663,0.1460)

\rput(0.7663,0.0840){ 30}
\PST@Border(0.8621,0.1260)
(0.8621,0.1460)

\rput(0.8621,0.0840){ 35}
\PST@Border(0.9580,0.1260)
(0.9580,0.1460)

\rput(0.9580,0.0840){ 40}
\PST@Border(0.1910,0.9680)
(0.1910,0.1260)
(0.9580,0.1260)
(0.9580,0.9680)
(0.1910,0.9680)

\rput{L}(0.0420,0.5470){$\Delta x$}
\rput(0.5745,0.0210){$\Delta F$}
\rput[r](0.6550,0.9270){punti allungamento}
\PST@Circle(0.1910,0.1260)
\PST@Circle(0.2662,0.2207)
\PST@Circle(0.3413,0.3155)
\PST@Circle(0.4167,0.4049)
\PST@Circle(0.4919,0.4996)
\PST@Circle(0.5670,0.5891)
\PST@Circle(0.6422,0.6838)
\PST@Circle(0.7175,0.7785)
\PST@Circle(0.7927,0.8680)
\PST@Circle(0.8679,0.9575)
\PST@Circle(0.7105,0.9270)
\rput[r](0.6550,0.8850){punti accorciamento}
\PST@Cross(0.1910,0.1260)
\PST@Cross(0.2662,0.2260)
\PST@Cross(0.3413,0.3207)
\PST@Cross(0.4167,0.4102)
\PST@Cross(0.4919,0.4996)
\PST@Cross(0.5670,0.5891)
\PST@Cross(0.6422,0.6838)
\PST@Cross(0.7175,0.7838)
\PST@Cross(0.7927,0.8733)
\PST@Cross(0.8679,0.9575)
\PST@Cross(0.7105,0.8850)
\rput[r](0.6550,0.8430){interpolazione allungamento}
\PST@Dashed(0.6710,0.8430)
(0.7500,0.8430)

\PST@Dashed(0.1910,0.1283)
(0.1910,0.1283)
(0.1978,0.1367)
(0.2047,0.1451)
(0.2115,0.1535)
(0.2183,0.1619)
(0.2252,0.1704)
(0.2320,0.1788)
(0.2389,0.1872)
(0.2457,0.1956)
(0.2525,0.2040)
(0.2594,0.2124)
(0.2662,0.2208)
(0.2730,0.2292)
(0.2799,0.2376)
(0.2867,0.2460)
(0.2936,0.2544)
(0.3004,0.2628)
(0.3072,0.2712)
(0.3141,0.2796)
(0.3209,0.2880)
(0.3277,0.2964)
(0.3346,0.3048)
(0.3414,0.3132)
(0.3483,0.3216)
(0.3551,0.3300)
(0.3619,0.3385)
(0.3688,0.3469)
(0.3756,0.3553)
(0.3824,0.3637)
(0.3893,0.3721)
(0.3961,0.3805)
(0.4030,0.3889)
(0.4098,0.3973)
(0.4166,0.4057)
(0.4235,0.4141)
(0.4303,0.4225)
(0.4371,0.4309)
(0.4440,0.4393)
(0.4508,0.4477)
(0.4576,0.4561)
(0.4645,0.4645)
(0.4713,0.4729)
(0.4782,0.4813)
(0.4850,0.4897)
(0.4918,0.4981)
(0.4987,0.5065)
(0.5055,0.5150)
(0.5123,0.5234)
(0.5192,0.5318)
(0.5260,0.5402)
(0.5329,0.5486)
(0.5397,0.5570)
(0.5465,0.5654)
(0.5534,0.5738)
(0.5602,0.5822)
(0.5670,0.5906)
(0.5739,0.5990)
(0.5807,0.6074)
(0.5876,0.6158)
(0.5944,0.6242)
(0.6012,0.6326)
(0.6081,0.6410)
(0.6149,0.6494)
(0.6217,0.6578)
(0.6286,0.6662)
(0.6354,0.6746)
(0.6423,0.6830)
(0.6491,0.6915)
(0.6559,0.6999)
(0.6628,0.7083)
(0.6696,0.7167)
(0.6764,0.7251)
(0.6833,0.7335)
(0.6901,0.7419)
(0.6969,0.7503)
(0.7038,0.7587)
(0.7106,0.7671)
(0.7175,0.7755)
(0.7243,0.7839)
(0.7311,0.7923)
(0.7380,0.8007)
(0.7448,0.8091)
(0.7516,0.8175)
(0.7585,0.8259)
(0.7653,0.8343)
(0.7722,0.8427)
(0.7790,0.8511)
(0.7858,0.8595)
(0.7927,0.8680)
(0.7995,0.8764)
(0.8063,0.8848)
(0.8132,0.8932)
(0.8200,0.9016)
(0.8269,0.9100)
(0.8337,0.9184)
(0.8405,0.9268)
(0.8474,0.9352)
(0.8542,0.9436)
(0.8610,0.9520)
(0.8679,0.9604)

\rput[r](0.6550,0.8010){interpolazione accorciamento}
\PST@Dotted(0.6710,0.8010)
(0.7500,0.8010)

\PST@Dotted(0.1910,0.1314)
(0.1910,0.1314)
(0.1978,0.1398)
(0.2047,0.1482)
(0.2115,0.1566)
(0.2183,0.1650)
(0.2252,0.1734)
(0.2320,0.1818)
(0.2389,0.1902)
(0.2457,0.1986)
(0.2525,0.2070)
(0.2594,0.2154)
(0.2662,0.2237)
(0.2730,0.2321)
(0.2799,0.2405)
(0.2867,0.2489)
(0.2936,0.2573)
(0.3004,0.2657)
(0.3072,0.2741)
(0.3141,0.2825)
(0.3209,0.2909)
(0.3277,0.2993)
(0.3346,0.3077)
(0.3414,0.3161)
(0.3483,0.3245)
(0.3551,0.3329)
(0.3619,0.3413)
(0.3688,0.3497)
(0.3756,0.3581)
(0.3824,0.3665)
(0.3893,0.3749)
(0.3961,0.3833)
(0.4030,0.3917)
(0.4098,0.4001)
(0.4166,0.4085)
(0.4235,0.4169)
(0.4303,0.4253)
(0.4371,0.4337)
(0.4440,0.4420)
(0.4508,0.4504)
(0.4576,0.4588)
(0.4645,0.4672)
(0.4713,0.4756)
(0.4782,0.4840)
(0.4850,0.4924)
(0.4918,0.5008)
(0.4987,0.5092)
(0.5055,0.5176)
(0.5123,0.5260)
(0.5192,0.5344)
(0.5260,0.5428)
(0.5329,0.5512)
(0.5397,0.5596)
(0.5465,0.5680)
(0.5534,0.5764)
(0.5602,0.5848)
(0.5670,0.5932)
(0.5739,0.6016)
(0.5807,0.6100)
(0.5876,0.6184)
(0.5944,0.6268)
(0.6012,0.6352)
(0.6081,0.6436)
(0.6149,0.6520)
(0.6217,0.6603)
(0.6286,0.6687)
(0.6354,0.6771)
(0.6423,0.6855)
(0.6491,0.6939)
(0.6559,0.7023)
(0.6628,0.7107)
(0.6696,0.7191)
(0.6764,0.7275)
(0.6833,0.7359)
(0.6901,0.7443)
(0.6969,0.7527)
(0.7038,0.7611)
(0.7106,0.7695)
(0.7175,0.7779)
(0.7243,0.7863)
(0.7311,0.7947)
(0.7380,0.8031)
(0.7448,0.8115)
(0.7516,0.8199)
(0.7585,0.8283)
(0.7653,0.8367)
(0.7722,0.8451)
(0.7790,0.8535)
(0.7858,0.8619)
(0.7927,0.8703)
(0.7995,0.8786)
(0.8063,0.8870)
(0.8132,0.8954)
(0.8200,0.9038)
(0.8269,0.9122)
(0.8337,0.9206)
(0.8405,0.9290)
(0.8474,0.9374)
(0.8542,0.9458)
(0.8610,0.9542)
(0.8679,0.9626)

\PST@Border(0.1910,0.9680)
(0.1910,0.1260)
(0.9580,0.1260)
(0.9580,0.9680)
(0.1910,0.9680)

\catcode`@=12
\fi
\endpspicture

\end{figure}
\begin{figure}[p]\caption{Grafico relativo all'estensimetro n.9. In ascissa sono riportati i valori di $\Delta F~(\unit{N})$ e in ordinata la corrispondente variazione di lunghezza del filo $\Delta x~(\unit{m})$. Le barre di errore su entrambe le misure sono troppo piccole per poter essere rappresentate su questa scala.}\label{9}
% GNUPLOT: LaTeX picture using PSTRICKS macros
% Define new PST objects, if not already defined
\ifx\PSTloaded\undefined
\def\PSTloaded{t}
\psset{arrowsize=.01 3.2 1.4 .3}
\psset{dotsize=.125}
\catcode`@=11

\newpsobject{PST@Border}{psline}{linewidth=.0015,linestyle=solid}
\newpsobject{PST@Axes}{psline}{linewidth=.0015,linestyle=dotted,dotsep=.004}
\newpsobject{PST@Solid}{psline}{linewidth=.0015,linestyle=solid}
\newpsobject{PST@Dashed}{psline}{linewidth=.0015,linestyle=dashed,dash=.01 .01}
\newpsobject{PST@Dotted}{psline}{linewidth=.0025,linestyle=dotted,dotsep=.008}
\newpsobject{PST@LongDash}{psline}{linewidth=.0015,linestyle=dashed,dash=.02 .01}
\newpsobject{PST@Diamond}{psdots}{linewidth=.001,linestyle=solid,dotstyle=square,dotangle=45}
\newpsobject{PST@Filldiamond}{psdots}{linewidth=.001,linestyle=solid,dotstyle=square*,dotangle=45}
\newpsobject{PST@Cross}{psdots}{linewidth=.001,linestyle=solid,dotstyle=+,dotangle=45}
\newpsobject{PST@Plus}{psdots}{linewidth=.001,linestyle=solid,dotstyle=+}
\newpsobject{PST@Square}{psdots}{linewidth=.001,linestyle=solid,dotstyle=square}
\newpsobject{PST@Circle}{psdots}{linewidth=.001,linestyle=solid,dotstyle=o}
\newpsobject{PST@Triangle}{psdots}{linewidth=.001,linestyle=solid,dotstyle=triangle}
\newpsobject{PST@Pentagon}{psdots}{linewidth=.001,linestyle=solid,dotstyle=pentagon}
\newpsobject{PST@Fillsquare}{psdots}{linewidth=.001,linestyle=solid,dotstyle=square*}
\newpsobject{PST@Fillcircle}{psdots}{linewidth=.001,linestyle=solid,dotstyle=*}
\newpsobject{PST@Filltriangle}{psdots}{linewidth=.001,linestyle=solid,dotstyle=triangle*}
\newpsobject{PST@Fillpentagon}{psdots}{linewidth=.001,linestyle=solid,dotstyle=pentagon*}
\newpsobject{PST@Arrow}{psline}{linewidth=.001,linestyle=solid}
\catcode`@=12

\fi
\psset{unit=5.0in,xunit=5.0in,yunit=3.0in}
\pspicture(0.000000,0.000000)(1.000000,1.000000)
\ifx\nofigs\undefined
\catcode`@=11

\PST@Border(0.1910,0.1260)
(0.2060,0.1260)

\rput[r](0.1750,0.1260){0.0000}
\PST@Border(0.1910,0.2313)
(0.2060,0.2313)

\rput[r](0.1750,0.2313){0.0002}
\PST@Border(0.1910,0.3365)
(0.2060,0.3365)

\rput[r](0.1750,0.3365){0.0004}
\PST@Border(0.1910,0.4418)
(0.2060,0.4418)

\rput[r](0.1750,0.4418){0.0006}
\PST@Border(0.1910,0.5470)
(0.2060,0.5470)

\rput[r](0.1750,0.5470){0.0008}
\PST@Border(0.1910,0.6523)
(0.2060,0.6523)

\rput[r](0.1750,0.6523){0.0010}
\PST@Border(0.1910,0.7575)
(0.2060,0.7575)

\rput[r](0.1750,0.7575){0.0012}
\PST@Border(0.1910,0.8628)
(0.2060,0.8628)

\rput[r](0.1750,0.8628){0.0014}
\PST@Border(0.1910,0.9680)
(0.2060,0.9680)

\rput[r](0.1750,0.9680){0.0016}
\PST@Border(0.1910,0.1260)
(0.1910,0.1460)

\rput(0.1910,0.0840){ 0}
\PST@Border(0.2869,0.1260)
(0.2869,0.1460)

\rput(0.2869,0.0840){ 5}
\PST@Border(0.3828,0.1260)
(0.3828,0.1460)

\rput(0.3828,0.0840){ 10}
\PST@Border(0.4786,0.1260)
(0.4786,0.1460)

\rput(0.4786,0.0840){ 15}
\PST@Border(0.5745,0.1260)
(0.5745,0.1460)

\rput(0.5745,0.0840){ 20}
\PST@Border(0.6704,0.1260)
(0.6704,0.1460)

\rput(0.6704,0.0840){ 25}
\PST@Border(0.7663,0.1260)
(0.7663,0.1460)

\rput(0.7663,0.0840){ 30}
\PST@Border(0.8621,0.1260)
(0.8621,0.1460)

\rput(0.8621,0.0840){ 35}
\PST@Border(0.9580,0.1260)
(0.9580,0.1460)

\rput(0.9580,0.0840){ 40}
\PST@Border(0.1910,0.9680)
(0.1910,0.1260)
(0.9580,0.1260)
(0.9580,0.9680)
(0.1910,0.9680)

\rput{L}(0.0420,0.5470){$\Delta x$}
\rput(0.5745,0.0210){$\Delta F$}
\rput[r](0.6550,0.9270){punti allungamento}
\PST@Circle(0.1910,0.1260)
\PST@Circle(0.2662,0.2207)
\PST@Circle(0.3413,0.3102)
\PST@Circle(0.4167,0.3996)
\PST@Circle(0.4919,0.4891)
\PST@Circle(0.5670,0.5733)
\PST@Circle(0.6422,0.6575)
\PST@Circle(0.7175,0.7154)
\PST@Circle(0.7927,0.8312)
\PST@Circle(0.8679,0.9206)
\PST@Circle(0.7105,0.9270)
\rput[r](0.6550,0.8850){punti accorciamento}
\PST@Cross(0.1910,0.1313)
\PST@Cross(0.2662,0.2207)
\PST@Cross(0.3413,0.3155)
\PST@Cross(0.4167,0.4049)
\PST@Cross(0.4919,0.4891)
\PST@Cross(0.5670,0.5786)
\PST@Cross(0.6422,0.6628)
\PST@Cross(0.7175,0.7470)
\PST@Cross(0.7927,0.8364)
\PST@Cross(0.8679,0.9206)
\PST@Cross(0.7105,0.8850)
\rput[r](0.6550,0.8430){interpolazione allungamento}
\PST@Dashed(0.6710,0.8430)
(0.7500,0.8430)

\PST@Dashed(0.1910,0.1342)
(0.1910,0.1342)
(0.1978,0.1420)
(0.2047,0.1499)
(0.2115,0.1578)
(0.2183,0.1657)
(0.2252,0.1736)
(0.2320,0.1815)
(0.2389,0.1893)
(0.2457,0.1972)
(0.2525,0.2051)
(0.2594,0.2130)
(0.2662,0.2209)
(0.2730,0.2288)
(0.2799,0.2366)
(0.2867,0.2445)
(0.2936,0.2524)
(0.3004,0.2603)
(0.3072,0.2682)
(0.3141,0.2761)
(0.3209,0.2839)
(0.3277,0.2918)
(0.3346,0.2997)
(0.3414,0.3076)
(0.3483,0.3155)
(0.3551,0.3234)
(0.3619,0.3312)
(0.3688,0.3391)
(0.3756,0.3470)
(0.3824,0.3549)
(0.3893,0.3628)
(0.3961,0.3707)
(0.4030,0.3785)
(0.4098,0.3864)
(0.4166,0.3943)
(0.4235,0.4022)
(0.4303,0.4101)
(0.4371,0.4180)
(0.4440,0.4258)
(0.4508,0.4337)
(0.4576,0.4416)
(0.4645,0.4495)
(0.4713,0.4574)
(0.4782,0.4652)
(0.4850,0.4731)
(0.4918,0.4810)
(0.4987,0.4889)
(0.5055,0.4968)
(0.5123,0.5047)
(0.5192,0.5125)
(0.5260,0.5204)
(0.5329,0.5283)
(0.5397,0.5362)
(0.5465,0.5441)
(0.5534,0.5520)
(0.5602,0.5598)
(0.5670,0.5677)
(0.5739,0.5756)
(0.5807,0.5835)
(0.5876,0.5914)
(0.5944,0.5993)
(0.6012,0.6071)
(0.6081,0.6150)
(0.6149,0.6229)
(0.6217,0.6308)
(0.6286,0.6387)
(0.6354,0.6466)
(0.6423,0.6544)
(0.6491,0.6623)
(0.6559,0.6702)
(0.6628,0.6781)
(0.6696,0.6860)
(0.6764,0.6939)
(0.6833,0.7017)
(0.6901,0.7096)
(0.6969,0.7175)
(0.7038,0.7254)
(0.7106,0.7333)
(0.7175,0.7412)
(0.7243,0.7490)
(0.7311,0.7569)
(0.7380,0.7648)
(0.7448,0.7727)
(0.7516,0.7806)
(0.7585,0.7884)
(0.7653,0.7963)
(0.7722,0.8042)
(0.7790,0.8121)
(0.7858,0.8200)
(0.7927,0.8279)
(0.7995,0.8357)
(0.8063,0.8436)
(0.8132,0.8515)
(0.8200,0.8594)
(0.8269,0.8673)
(0.8337,0.8752)
(0.8405,0.8830)
(0.8474,0.8909)
(0.8542,0.8988)
(0.8610,0.9067)
(0.8679,0.9146)

\rput[r](0.6550,0.8010){interpolazione accorciamento}
\PST@Dotted(0.6710,0.8010)
(0.7500,0.8010)

\PST@Dotted(0.1910,0.1370)
(0.1910,0.1370)
(0.1978,0.1450)
(0.2047,0.1529)
(0.2115,0.1609)
(0.2183,0.1688)
(0.2252,0.1768)
(0.2320,0.1848)
(0.2389,0.1927)
(0.2457,0.2007)
(0.2525,0.2086)
(0.2594,0.2166)
(0.2662,0.2245)
(0.2730,0.2325)
(0.2799,0.2404)
(0.2867,0.2484)
(0.2936,0.2563)
(0.3004,0.2643)
(0.3072,0.2722)
(0.3141,0.2802)
(0.3209,0.2881)
(0.3277,0.2961)
(0.3346,0.3040)
(0.3414,0.3120)
(0.3483,0.3199)
(0.3551,0.3279)
(0.3619,0.3358)
(0.3688,0.3438)
(0.3756,0.3518)
(0.3824,0.3597)
(0.3893,0.3677)
(0.3961,0.3756)
(0.4030,0.3836)
(0.4098,0.3915)
(0.4166,0.3995)
(0.4235,0.4074)
(0.4303,0.4154)
(0.4371,0.4233)
(0.4440,0.4313)
(0.4508,0.4392)
(0.4576,0.4472)
(0.4645,0.4551)
(0.4713,0.4631)
(0.4782,0.4710)
(0.4850,0.4790)
(0.4918,0.4869)
(0.4987,0.4949)
(0.5055,0.5029)
(0.5123,0.5108)
(0.5192,0.5188)
(0.5260,0.5267)
(0.5329,0.5347)
(0.5397,0.5426)
(0.5465,0.5506)
(0.5534,0.5585)
(0.5602,0.5665)
(0.5670,0.5744)
(0.5739,0.5824)
(0.5807,0.5903)
(0.5876,0.5983)
(0.5944,0.6062)
(0.6012,0.6142)
(0.6081,0.6221)
(0.6149,0.6301)
(0.6217,0.6380)
(0.6286,0.6460)
(0.6354,0.6540)
(0.6423,0.6619)
(0.6491,0.6699)
(0.6559,0.6778)
(0.6628,0.6858)
(0.6696,0.6937)
(0.6764,0.7017)
(0.6833,0.7096)
(0.6901,0.7176)
(0.6969,0.7255)
(0.7038,0.7335)
(0.7106,0.7414)
(0.7175,0.7494)
(0.7243,0.7573)
(0.7311,0.7653)
(0.7380,0.7732)
(0.7448,0.7812)
(0.7516,0.7891)
(0.7585,0.7971)
(0.7653,0.8050)
(0.7722,0.8130)
(0.7790,0.8210)
(0.7858,0.8289)
(0.7927,0.8369)
(0.7995,0.8448)
(0.8063,0.8528)
(0.8132,0.8607)
(0.8200,0.8687)
(0.8269,0.8766)
(0.8337,0.8846)
(0.8405,0.8925)
(0.8474,0.9005)
(0.8542,0.9084)
(0.8610,0.9164)
(0.8679,0.9243)

\PST@Border(0.1910,0.9680)
(0.1910,0.1260)
(0.9580,0.1260)
(0.9580,0.9680)
(0.1910,0.9680)

\catcode`@=12
\fi
\endpspicture

\end{figure}
\begin{figure}[p]\caption{Grafico relativo all'estensimetro n.14. In ascissa sono riportati i valori di $\Delta F~(\unit{N})$ e in ordinata la corrispondente variazione di lunghezza del filo $\Delta x~(\unit{m})$. Le barre di errore su entrambe le misure sono troppo piccole per poter essere rappresentate su questa scala.}\label{14}
% GNUPLOT: LaTeX picture using PSTRICKS macros
% Define new PST objects, if not already defined
\ifx\PSTloaded\undefined
\def\PSTloaded{t}
\psset{arrowsize=.01 3.2 1.4 .3}
\psset{dotsize=.125}
\catcode`@=11

\newpsobject{PST@Border}{psline}{linewidth=.0015,linestyle=solid}
\newpsobject{PST@Axes}{psline}{linewidth=.0015,linestyle=dotted,dotsep=.004}
\newpsobject{PST@Solid}{psline}{linewidth=.0015,linestyle=solid}
\newpsobject{PST@Dashed}{psline}{linewidth=.0015,linestyle=dashed,dash=.01 .01}
\newpsobject{PST@Dotted}{psline}{linewidth=.0025,linestyle=dotted,dotsep=.008}
\newpsobject{PST@LongDash}{psline}{linewidth=.0015,linestyle=dashed,dash=.02 .01}
\newpsobject{PST@Diamond}{psdots}{linewidth=.001,linestyle=solid,dotstyle=square,dotangle=45}
\newpsobject{PST@Filldiamond}{psdots}{linewidth=.001,linestyle=solid,dotstyle=square*,dotangle=45}
\newpsobject{PST@Cross}{psdots}{linewidth=.001,linestyle=solid,dotstyle=+,dotangle=45}
\newpsobject{PST@Plus}{psdots}{linewidth=.001,linestyle=solid,dotstyle=+}
\newpsobject{PST@Square}{psdots}{linewidth=.001,linestyle=solid,dotstyle=square}
\newpsobject{PST@Circle}{psdots}{linewidth=.001,linestyle=solid,dotstyle=o}
\newpsobject{PST@Triangle}{psdots}{linewidth=.001,linestyle=solid,dotstyle=triangle}
\newpsobject{PST@Pentagon}{psdots}{linewidth=.001,linestyle=solid,dotstyle=pentagon}
\newpsobject{PST@Fillsquare}{psdots}{linewidth=.001,linestyle=solid,dotstyle=square*}
\newpsobject{PST@Fillcircle}{psdots}{linewidth=.001,linestyle=solid,dotstyle=*}
\newpsobject{PST@Filltriangle}{psdots}{linewidth=.001,linestyle=solid,dotstyle=triangle*}
\newpsobject{PST@Fillpentagon}{psdots}{linewidth=.001,linestyle=solid,dotstyle=pentagon*}
\newpsobject{PST@Arrow}{psline}{linewidth=.001,linestyle=solid}
\catcode`@=12

\fi
\psset{unit=5.0in,xunit=5.0in,yunit=3.0in}
\pspicture(0.000000,0.000000)(1.000000,1.000000)
\ifx\nofigs\undefined
\catcode`@=11

\PST@Border(0.2070,0.1260)
(0.2220,0.1260)

\rput[r](0.1910,0.1260){-0.0005}
\PST@Border(0.2070,0.2663)
(0.2220,0.2663)

\rput[r](0.1910,0.2663){0.0000}
\PST@Border(0.2070,0.4067)
(0.2220,0.4067)

\rput[r](0.1910,0.4067){0.0005}
\PST@Border(0.2070,0.5470)
(0.2220,0.5470)

\rput[r](0.1910,0.5470){0.0010}
\PST@Border(0.2070,0.6873)
(0.2220,0.6873)

\rput[r](0.1910,0.6873){0.0015}
\PST@Border(0.2070,0.8277)
(0.2220,0.8277)

\rput[r](0.1910,0.8277){0.0020}
\PST@Border(0.2070,0.9680)
(0.2220,0.9680)

\rput[r](0.1910,0.9680){0.0025}
\PST@Border(0.2070,0.1260)
(0.2070,0.1460)

\rput(0.2070,0.0840){ 0}
\PST@Border(0.3009,0.1260)
(0.3009,0.1460)

\rput(0.3009,0.0840){ 5}
\PST@Border(0.3948,0.1260)
(0.3948,0.1460)

\rput(0.3948,0.0840){ 10}
\PST@Border(0.4886,0.1260)
(0.4886,0.1460)

\rput(0.4886,0.0840){ 15}
\PST@Border(0.5825,0.1260)
(0.5825,0.1460)

\rput(0.5825,0.0840){ 20}
\PST@Border(0.6764,0.1260)
(0.6764,0.1460)

\rput(0.6764,0.0840){ 25}
\PST@Border(0.7703,0.1260)
(0.7703,0.1460)

\rput(0.7703,0.0840){ 30}
\PST@Border(0.8641,0.1260)
(0.8641,0.1460)

\rput(0.8641,0.0840){ 35}
\PST@Border(0.9580,0.1260)
(0.9580,0.1460)

\rput(0.9580,0.0840){ 40}
\PST@Border(0.2070,0.9680)
(0.2070,0.1260)
(0.9580,0.1260)
(0.9580,0.9680)
(0.2070,0.9680)

\rput{L}(0.0420,0.5470){$\Delta x$}
\rput(0.5825,0.0210){$\Delta F$}
\rput[r](0.6710,0.9270){punti allungamento}
\PST@Circle(0.2070,0.2663)
\PST@Circle(0.2806,0.3365)
\PST@Circle(0.3542,0.4067)
\PST@Circle(0.4280,0.4796)
\PST@Circle(0.5016,0.5470)
\PST@Circle(0.5752,0.6200)
\PST@Circle(0.6488,0.6901)
\PST@Circle(0.7226,0.7603)
\PST@Circle(0.7962,0.8333)
\PST@Circle(0.8698,0.9006)
\PST@Circle(0.7265,0.9270)
\rput[r](0.6710,0.8850){punti accorciamento}
\PST@Cross(0.2070,0.2663)
\PST@Cross(0.2806,0.3365)
\PST@Cross(0.3542,0.4067)
\PST@Cross(0.4280,0.4796)
\PST@Cross(0.5016,0.5498)
\PST@Cross(0.5752,0.6200)
\PST@Cross(0.6488,0.6873)
\PST@Cross(0.7226,0.7575)
\PST@Cross(0.7962,0.8277)
\PST@Cross(0.8698,0.9006)
\PST@Cross(0.7265,0.8850)
\rput[r](0.6710,0.8430){interpolazione allungamento}
\PST@Dashed(0.6870,0.8430)
(0.7660,0.8430)

\PST@Dashed(0.2070,0.2661)
(0.2070,0.2661)
(0.2137,0.2725)
(0.2204,0.2790)
(0.2271,0.2854)
(0.2338,0.2918)
(0.2405,0.2982)
(0.2472,0.3046)
(0.2539,0.3111)
(0.2606,0.3175)
(0.2673,0.3239)
(0.2739,0.3303)
(0.2806,0.3368)
(0.2873,0.3432)
(0.2940,0.3496)
(0.3007,0.3560)
(0.3074,0.3625)
(0.3141,0.3689)
(0.3208,0.3753)
(0.3275,0.3817)
(0.3342,0.3881)
(0.3409,0.3946)
(0.3476,0.4010)
(0.3543,0.4074)
(0.3610,0.4138)
(0.3677,0.4203)
(0.3744,0.4267)
(0.3811,0.4331)
(0.3878,0.4395)
(0.3944,0.4460)
(0.4011,0.4524)
(0.4078,0.4588)
(0.4145,0.4652)
(0.4212,0.4716)
(0.4279,0.4781)
(0.4346,0.4845)
(0.4413,0.4909)
(0.4480,0.4973)
(0.4547,0.5038)
(0.4614,0.5102)
(0.4681,0.5166)
(0.4748,0.5230)
(0.4815,0.5295)
(0.4882,0.5359)
(0.4949,0.5423)
(0.5016,0.5487)
(0.5083,0.5551)
(0.5149,0.5616)
(0.5216,0.5680)
(0.5283,0.5744)
(0.5350,0.5808)
(0.5417,0.5873)
(0.5484,0.5937)
(0.5551,0.6001)
(0.5618,0.6065)
(0.5685,0.6130)
(0.5752,0.6194)
(0.5819,0.6258)
(0.5886,0.6322)
(0.5953,0.6386)
(0.6020,0.6451)
(0.6087,0.6515)
(0.6154,0.6579)
(0.6221,0.6643)
(0.6288,0.6708)
(0.6354,0.6772)
(0.6421,0.6836)
(0.6488,0.6900)
(0.6555,0.6965)
(0.6622,0.7029)
(0.6689,0.7093)
(0.6756,0.7157)
(0.6823,0.7221)
(0.6890,0.7286)
(0.6957,0.7350)
(0.7024,0.7414)
(0.7091,0.7478)
(0.7158,0.7543)
(0.7225,0.7607)
(0.7292,0.7671)
(0.7359,0.7735)
(0.7426,0.7800)
(0.7493,0.7864)
(0.7560,0.7928)
(0.7626,0.7992)
(0.7693,0.8056)
(0.7760,0.8121)
(0.7827,0.8185)
(0.7894,0.8249)
(0.7961,0.8313)
(0.8028,0.8378)
(0.8095,0.8442)
(0.8162,0.8506)
(0.8229,0.8570)
(0.8296,0.8635)
(0.8363,0.8699)
(0.8430,0.8763)
(0.8497,0.8827)
(0.8564,0.8891)
(0.8631,0.8956)
(0.8698,0.9020)

\rput[r](0.6710,0.8010){interpolazione accorciamento}
\PST@Dotted(0.6870,0.8010)
(0.7660,0.8010)

\PST@Dotted(0.2070,0.2670)
(0.2070,0.2670)
(0.2137,0.2734)
(0.2204,0.2798)
(0.2271,0.2862)
(0.2338,0.2926)
(0.2405,0.2990)
(0.2472,0.3053)
(0.2539,0.3117)
(0.2606,0.3181)
(0.2673,0.3245)
(0.2739,0.3309)
(0.2806,0.3373)
(0.2873,0.3437)
(0.2940,0.3501)
(0.3007,0.3564)
(0.3074,0.3628)
(0.3141,0.3692)
(0.3208,0.3756)
(0.3275,0.3820)
(0.3342,0.3884)
(0.3409,0.3948)
(0.3476,0.4012)
(0.3543,0.4075)
(0.3610,0.4139)
(0.3677,0.4203)
(0.3744,0.4267)
(0.3811,0.4331)
(0.3878,0.4395)
(0.3944,0.4459)
(0.4011,0.4523)
(0.4078,0.4586)
(0.4145,0.4650)
(0.4212,0.4714)
(0.4279,0.4778)
(0.4346,0.4842)
(0.4413,0.4906)
(0.4480,0.4970)
(0.4547,0.5034)
(0.4614,0.5097)
(0.4681,0.5161)
(0.4748,0.5225)
(0.4815,0.5289)
(0.4882,0.5353)
(0.4949,0.5417)
(0.5016,0.5481)
(0.5083,0.5545)
(0.5149,0.5608)
(0.5216,0.5672)
(0.5283,0.5736)
(0.5350,0.5800)
(0.5417,0.5864)
(0.5484,0.5928)
(0.5551,0.5992)
(0.5618,0.6056)
(0.5685,0.6119)
(0.5752,0.6183)
(0.5819,0.6247)
(0.5886,0.6311)
(0.5953,0.6375)
(0.6020,0.6439)
(0.6087,0.6503)
(0.6154,0.6567)
(0.6221,0.6631)
(0.6288,0.6694)
(0.6354,0.6758)
(0.6421,0.6822)
(0.6488,0.6886)
(0.6555,0.6950)
(0.6622,0.7014)
(0.6689,0.7078)
(0.6756,0.7142)
(0.6823,0.7205)
(0.6890,0.7269)
(0.6957,0.7333)
(0.7024,0.7397)
(0.7091,0.7461)
(0.7158,0.7525)
(0.7225,0.7589)
(0.7292,0.7653)
(0.7359,0.7716)
(0.7426,0.7780)
(0.7493,0.7844)
(0.7560,0.7908)
(0.7626,0.7972)
(0.7693,0.8036)
(0.7760,0.8100)
(0.7827,0.8164)
(0.7894,0.8227)
(0.7961,0.8291)
(0.8028,0.8355)
(0.8095,0.8419)
(0.8162,0.8483)
(0.8229,0.8547)
(0.8296,0.8611)
(0.8363,0.8675)
(0.8430,0.8738)
(0.8497,0.8802)
(0.8564,0.8866)
(0.8631,0.8930)
(0.8698,0.8994)

\PST@Border(0.2070,0.9680)
(0.2070,0.1260)
(0.9580,0.1260)
(0.9580,0.9680)
(0.2070,0.9680)

\catcode`@=12
\fi
\endpspicture

\end{figure}
\begin{figure}[p]\caption{Grafico relativo all'estensimetro n.16. In ascissa sono riportati i valori di $\Delta F~(\unit{N})$ e in ordinata la corrispondente variazione di lunghezza del filo $\Delta x~(\unit{m})$. Le barre di errore su entrambe le misure sono troppo piccole per poter essere rappresentate su questa scala.}\label{16}
% GNUPLOT: LaTeX picture using PSTRICKS macros
% Define new PST objects, if not already defined
\ifx\PSTloaded\undefined
\def\PSTloaded{t}
\psset{arrowsize=.01 3.2 1.4 .3}
\psset{dotsize=.125}
\catcode`@=11

\newpsobject{PST@Border}{psline}{linewidth=.0015,linestyle=solid}
\newpsobject{PST@Axes}{psline}{linewidth=.0015,linestyle=dotted,dotsep=.004}
\newpsobject{PST@Solid}{psline}{linewidth=.0015,linestyle=solid}
\newpsobject{PST@Dashed}{psline}{linewidth=.0015,linestyle=dashed,dash=.01 .01}
\newpsobject{PST@Dotted}{psline}{linewidth=.0025,linestyle=dotted,dotsep=.008}
\newpsobject{PST@LongDash}{psline}{linewidth=.0015,linestyle=dashed,dash=.02 .01}
\newpsobject{PST@Diamond}{psdots}{linewidth=.001,linestyle=solid,dotstyle=square,dotangle=45}
\newpsobject{PST@Filldiamond}{psdots}{linewidth=.001,linestyle=solid,dotstyle=square*,dotangle=45}
\newpsobject{PST@Cross}{psdots}{linewidth=.001,linestyle=solid,dotstyle=+,dotangle=45}
\newpsobject{PST@Plus}{psdots}{linewidth=.001,linestyle=solid,dotstyle=+}
\newpsobject{PST@Square}{psdots}{linewidth=.001,linestyle=solid,dotstyle=square}
\newpsobject{PST@Circle}{psdots}{linewidth=.001,linestyle=solid,dotstyle=o}
\newpsobject{PST@Triangle}{psdots}{linewidth=.001,linestyle=solid,dotstyle=triangle}
\newpsobject{PST@Pentagon}{psdots}{linewidth=.001,linestyle=solid,dotstyle=pentagon}
\newpsobject{PST@Fillsquare}{psdots}{linewidth=.001,linestyle=solid,dotstyle=square*}
\newpsobject{PST@Fillcircle}{psdots}{linewidth=.001,linestyle=solid,dotstyle=*}
\newpsobject{PST@Filltriangle}{psdots}{linewidth=.001,linestyle=solid,dotstyle=triangle*}
\newpsobject{PST@Fillpentagon}{psdots}{linewidth=.001,linestyle=solid,dotstyle=pentagon*}
\newpsobject{PST@Arrow}{psline}{linewidth=.001,linestyle=solid}
\catcode`@=12

\fi
\psset{unit=5.0in,xunit=5.0in,yunit=3.0in}
\pspicture(0.000000,0.000000)(1.000000,1.000000)
\ifx\nofigs\undefined
\catcode`@=11

\PST@Border(0.1910,0.1260)
(0.2060,0.1260)

\rput[r](0.1750,0.1260){0.0000}
\PST@Border(0.1910,0.2196)
(0.2060,0.2196)

\rput[r](0.1750,0.2196){0.0002}
\PST@Border(0.1910,0.3131)
(0.2060,0.3131)

\rput[r](0.1750,0.3131){0.0004}
\PST@Border(0.1910,0.4067)
(0.2060,0.4067)

\rput[r](0.1750,0.4067){0.0006}
\PST@Border(0.1910,0.5002)
(0.2060,0.5002)

\rput[r](0.1750,0.5002){0.0008}
\PST@Border(0.1910,0.5938)
(0.2060,0.5938)

\rput[r](0.1750,0.5938){0.0010}
\PST@Border(0.1910,0.6873)
(0.2060,0.6873)

\rput[r](0.1750,0.6873){0.0012}
\PST@Border(0.1910,0.7809)
(0.2060,0.7809)

\rput[r](0.1750,0.7809){0.0014}
\PST@Border(0.1910,0.8744)
(0.2060,0.8744)

\rput[r](0.1750,0.8744){0.0016}
\PST@Border(0.1910,0.9680)
(0.2060,0.9680)

\rput[r](0.1750,0.9680){0.0018}
\PST@Border(0.1910,0.1260)
(0.1910,0.1460)

\rput(0.1910,0.0840){ 0}
\PST@Border(0.2869,0.1260)
(0.2869,0.1460)

\rput(0.2869,0.0840){ 5}
\PST@Border(0.3828,0.1260)
(0.3828,0.1460)

\rput(0.3828,0.0840){ 10}
\PST@Border(0.4786,0.1260)
(0.4786,0.1460)

\rput(0.4786,0.0840){ 15}
\PST@Border(0.5745,0.1260)
(0.5745,0.1460)

\rput(0.5745,0.0840){ 20}
\PST@Border(0.6704,0.1260)
(0.6704,0.1460)

\rput(0.6704,0.0840){ 25}
\PST@Border(0.7663,0.1260)
(0.7663,0.1460)

\rput(0.7663,0.0840){ 30}
\PST@Border(0.8621,0.1260)
(0.8621,0.1460)

\rput(0.8621,0.0840){ 35}
\PST@Border(0.9580,0.1260)
(0.9580,0.1460)

\rput(0.9580,0.0840){ 40}
\PST@Border(0.1910,0.9680)
(0.1910,0.1260)
(0.9580,0.1260)
(0.9580,0.9680)
(0.1910,0.9680)

\rput{L}(0.0420,0.5470){$\Delta x$}
\rput(0.5745,0.0210){$\Delta F$}
\rput[r](0.6550,0.9270){punti allungamento}
\PST@Circle(0.1910,0.1260)
\PST@Circle(0.2662,0.2196)
\PST@Circle(0.3413,0.3084)
\PST@Circle(0.4167,0.3926)
\PST@Circle(0.4919,0.4815)
\PST@Circle(0.5670,0.5657)
\PST@Circle(0.6422,0.6499)
\PST@Circle(0.7175,0.7388)
\PST@Circle(0.7927,0.8230)
\PST@Circle(0.8679,0.9072)
\PST@Circle(0.7105,0.9270)
\rput[r](0.6550,0.8850){punti accorciamento}
\PST@Cross(0.1910,0.1260)
\PST@Cross(0.2662,0.2196)
\PST@Cross(0.3413,0.3084)
\PST@Cross(0.4167,0.3926)
\PST@Cross(0.4919,0.4768)
\PST@Cross(0.5670,0.5657)
\PST@Cross(0.6422,0.6499)
\PST@Cross(0.7175,0.7341)
\PST@Cross(0.7927,0.8230)
\PST@Cross(0.8679,0.9072)
\PST@Cross(0.7105,0.8850)
\rput[r](0.6550,0.8430){interpolazione allungamento}
\PST@Dashed(0.6710,0.8430)
(0.7500,0.8430)

\PST@Dashed(0.1910,0.1323)
(0.1910,0.1323)
(0.1978,0.1402)
(0.2047,0.1480)
(0.2115,0.1559)
(0.2183,0.1638)
(0.2252,0.1716)
(0.2320,0.1795)
(0.2389,0.1873)
(0.2457,0.1952)
(0.2525,0.2030)
(0.2594,0.2109)
(0.2662,0.2188)
(0.2730,0.2266)
(0.2799,0.2345)
(0.2867,0.2423)
(0.2936,0.2502)
(0.3004,0.2580)
(0.3072,0.2659)
(0.3141,0.2738)
(0.3209,0.2816)
(0.3277,0.2895)
(0.3346,0.2973)
(0.3414,0.3052)
(0.3483,0.3130)
(0.3551,0.3209)
(0.3619,0.3288)
(0.3688,0.3366)
(0.3756,0.3445)
(0.3824,0.3523)
(0.3893,0.3602)
(0.3961,0.3681)
(0.4030,0.3759)
(0.4098,0.3838)
(0.4166,0.3916)
(0.4235,0.3995)
(0.4303,0.4073)
(0.4371,0.4152)
(0.4440,0.4231)
(0.4508,0.4309)
(0.4576,0.4388)
(0.4645,0.4466)
(0.4713,0.4545)
(0.4782,0.4623)
(0.4850,0.4702)
(0.4918,0.4781)
(0.4987,0.4859)
(0.5055,0.4938)
(0.5123,0.5016)
(0.5192,0.5095)
(0.5260,0.5173)
(0.5329,0.5252)
(0.5397,0.5331)
(0.5465,0.5409)
(0.5534,0.5488)
(0.5602,0.5566)
(0.5670,0.5645)
(0.5739,0.5723)
(0.5807,0.5802)
(0.5876,0.5881)
(0.5944,0.5959)
(0.6012,0.6038)
(0.6081,0.6116)
(0.6149,0.6195)
(0.6217,0.6273)
(0.6286,0.6352)
(0.6354,0.6431)
(0.6423,0.6509)
(0.6491,0.6588)
(0.6559,0.6666)
(0.6628,0.6745)
(0.6696,0.6824)
(0.6764,0.6902)
(0.6833,0.6981)
(0.6901,0.7059)
(0.6969,0.7138)
(0.7038,0.7216)
(0.7106,0.7295)
(0.7175,0.7374)
(0.7243,0.7452)
(0.7311,0.7531)
(0.7380,0.7609)
(0.7448,0.7688)
(0.7516,0.7766)
(0.7585,0.7845)
(0.7653,0.7924)
(0.7722,0.8002)
(0.7790,0.8081)
(0.7858,0.8159)
(0.7927,0.8238)
(0.7995,0.8316)
(0.8063,0.8395)
(0.8132,0.8474)
(0.8200,0.8552)
(0.8269,0.8631)
(0.8337,0.8709)
(0.8405,0.8788)
(0.8474,0.8866)
(0.8542,0.8945)
(0.8610,0.9024)
(0.8679,0.9102)

\rput[r](0.6550,0.8010){interpolazione accorciamento}
\PST@Dotted(0.6710,0.8010)
(0.7500,0.8010)

\PST@Dotted(0.1910,0.1319)
(0.1910,0.1319)
(0.1978,0.1397)
(0.2047,0.1476)
(0.2115,0.1554)
(0.2183,0.1633)
(0.2252,0.1711)
(0.2320,0.1790)
(0.2389,0.1868)
(0.2457,0.1947)
(0.2525,0.2025)
(0.2594,0.2104)
(0.2662,0.2182)
(0.2730,0.2261)
(0.2799,0.2339)
(0.2867,0.2418)
(0.2936,0.2496)
(0.3004,0.2575)
(0.3072,0.2653)
(0.3141,0.2732)
(0.3209,0.2810)
(0.3277,0.2888)
(0.3346,0.2967)
(0.3414,0.3045)
(0.3483,0.3124)
(0.3551,0.3202)
(0.3619,0.3281)
(0.3688,0.3359)
(0.3756,0.3438)
(0.3824,0.3516)
(0.3893,0.3595)
(0.3961,0.3673)
(0.4030,0.3752)
(0.4098,0.3830)
(0.4166,0.3909)
(0.4235,0.3987)
(0.4303,0.4066)
(0.4371,0.4144)
(0.4440,0.4222)
(0.4508,0.4301)
(0.4576,0.4379)
(0.4645,0.4458)
(0.4713,0.4536)
(0.4782,0.4615)
(0.4850,0.4693)
(0.4918,0.4772)
(0.4987,0.4850)
(0.5055,0.4929)
(0.5123,0.5007)
(0.5192,0.5086)
(0.5260,0.5164)
(0.5329,0.5243)
(0.5397,0.5321)
(0.5465,0.5400)
(0.5534,0.5478)
(0.5602,0.5556)
(0.5670,0.5635)
(0.5739,0.5713)
(0.5807,0.5792)
(0.5876,0.5870)
(0.5944,0.5949)
(0.6012,0.6027)
(0.6081,0.6106)
(0.6149,0.6184)
(0.6217,0.6263)
(0.6286,0.6341)
(0.6354,0.6420)
(0.6423,0.6498)
(0.6491,0.6577)
(0.6559,0.6655)
(0.6628,0.6734)
(0.6696,0.6812)
(0.6764,0.6891)
(0.6833,0.6969)
(0.6901,0.7047)
(0.6969,0.7126)
(0.7038,0.7204)
(0.7106,0.7283)
(0.7175,0.7361)
(0.7243,0.7440)
(0.7311,0.7518)
(0.7380,0.7597)
(0.7448,0.7675)
(0.7516,0.7754)
(0.7585,0.7832)
(0.7653,0.7911)
(0.7722,0.7989)
(0.7790,0.8068)
(0.7858,0.8146)
(0.7927,0.8225)
(0.7995,0.8303)
(0.8063,0.8381)
(0.8132,0.8460)
(0.8200,0.8538)
(0.8269,0.8617)
(0.8337,0.8695)
(0.8405,0.8774)
(0.8474,0.8852)
(0.8542,0.8931)
(0.8610,0.9009)
(0.8679,0.9088)

\PST@Border(0.1910,0.9680)
(0.1910,0.1260)
(0.9580,0.1260)
(0.9580,0.9680)
(0.1910,0.9680)

\catcode`@=12
\fi
\endpspicture

\end{figure}
\begin{figure}[p]\caption{Grafico relativo all'estensimetro n.17. In ascissa sono riportati i valori di $\Delta F~(\unit{N})$ e in ordinata la corrispondente variazione di lunghezza del filo $\Delta x~(\unit{m})$. Le barre di errore su entrambe le misure sono troppo piccole per poter essere rappresentate su questa scala.}\label{17}
% GNUPLOT: LaTeX picture using PSTRICKS macros
% Define new PST objects, if not already defined
\ifx\PSTloaded\undefined
\def\PSTloaded{t}
\psset{arrowsize=.01 3.2 1.4 .3}
\psset{dotsize=.125}
\catcode`@=11

\newpsobject{PST@Border}{psline}{linewidth=.0015,linestyle=solid}
\newpsobject{PST@Axes}{psline}{linewidth=.0015,linestyle=dotted,dotsep=.004}
\newpsobject{PST@Solid}{psline}{linewidth=.0015,linestyle=solid}
\newpsobject{PST@Dashed}{psline}{linewidth=.0015,linestyle=dashed,dash=.01 .01}
\newpsobject{PST@Dotted}{psline}{linewidth=.0025,linestyle=dotted,dotsep=.008}
\newpsobject{PST@LongDash}{psline}{linewidth=.0015,linestyle=dashed,dash=.02 .01}
\newpsobject{PST@Diamond}{psdots}{linewidth=.001,linestyle=solid,dotstyle=square,dotangle=45}
\newpsobject{PST@Filldiamond}{psdots}{linewidth=.001,linestyle=solid,dotstyle=square*,dotangle=45}
\newpsobject{PST@Cross}{psdots}{linewidth=.001,linestyle=solid,dotstyle=+,dotangle=45}
\newpsobject{PST@Plus}{psdots}{linewidth=.001,linestyle=solid,dotstyle=+}
\newpsobject{PST@Square}{psdots}{linewidth=.001,linestyle=solid,dotstyle=square}
\newpsobject{PST@Circle}{psdots}{linewidth=.001,linestyle=solid,dotstyle=o}
\newpsobject{PST@Triangle}{psdots}{linewidth=.001,linestyle=solid,dotstyle=triangle}
\newpsobject{PST@Pentagon}{psdots}{linewidth=.001,linestyle=solid,dotstyle=pentagon}
\newpsobject{PST@Fillsquare}{psdots}{linewidth=.001,linestyle=solid,dotstyle=square*}
\newpsobject{PST@Fillcircle}{psdots}{linewidth=.001,linestyle=solid,dotstyle=*}
\newpsobject{PST@Filltriangle}{psdots}{linewidth=.001,linestyle=solid,dotstyle=triangle*}
\newpsobject{PST@Fillpentagon}{psdots}{linewidth=.001,linestyle=solid,dotstyle=pentagon*}
\newpsobject{PST@Arrow}{psline}{linewidth=.001,linestyle=solid}
\catcode`@=12

\fi
\psset{unit=5.0in,xunit=5.0in,yunit=3.0in}
\pspicture(0.000000,0.000000)(1.000000,1.000000)
\ifx\nofigs\undefined
\catcode`@=11

\PST@Border(0.1910,0.1260)
(0.2060,0.1260)

\rput[r](0.1750,0.1260){0.0000}
\PST@Border(0.1910,0.2313)
(0.2060,0.2313)

\rput[r](0.1750,0.2313){0.0002}
\PST@Border(0.1910,0.3365)
(0.2060,0.3365)

\rput[r](0.1750,0.3365){0.0004}
\PST@Border(0.1910,0.4418)
(0.2060,0.4418)

\rput[r](0.1750,0.4418){0.0006}
\PST@Border(0.1910,0.5470)
(0.2060,0.5470)

\rput[r](0.1750,0.5470){0.0008}
\PST@Border(0.1910,0.6523)
(0.2060,0.6523)

\rput[r](0.1750,0.6523){0.0010}
\PST@Border(0.1910,0.7575)
(0.2060,0.7575)

\rput[r](0.1750,0.7575){0.0012}
\PST@Border(0.1910,0.8628)
(0.2060,0.8628)

\rput[r](0.1750,0.8628){0.0014}
\PST@Border(0.1910,0.9680)
(0.2060,0.9680)

\rput[r](0.1750,0.9680){0.0016}
\PST@Border(0.1910,0.1260)
(0.1910,0.1460)

\rput(0.1910,0.0840){ 0}
\PST@Border(0.2869,0.1260)
(0.2869,0.1460)

\rput(0.2869,0.0840){ 5}
\PST@Border(0.3828,0.1260)
(0.3828,0.1460)

\rput(0.3828,0.0840){ 10}
\PST@Border(0.4786,0.1260)
(0.4786,0.1460)

\rput(0.4786,0.0840){ 15}
\PST@Border(0.5745,0.1260)
(0.5745,0.1460)

\rput(0.5745,0.0840){ 20}
\PST@Border(0.6704,0.1260)
(0.6704,0.1460)

\rput(0.6704,0.0840){ 25}
\PST@Border(0.7663,0.1260)
(0.7663,0.1460)

\rput(0.7663,0.0840){ 30}
\PST@Border(0.8621,0.1260)
(0.8621,0.1460)

\rput(0.8621,0.0840){ 35}
\PST@Border(0.9580,0.1260)
(0.9580,0.1460)

\rput(0.9580,0.0840){ 40}
\PST@Border(0.1910,0.9680)
(0.1910,0.1260)
(0.9580,0.1260)
(0.9580,0.9680)
(0.1910,0.9680)

\rput{L}(0.0420,0.5470){$\Delta x$}
\rput(0.5745,0.0210){$\Delta F$}
\rput[r](0.6550,0.9270){punti allungamento}
\PST@Circle(0.1910,0.1260)
\PST@Circle(0.2662,0.2155)
\PST@Circle(0.3413,0.2944)
\PST@Circle(0.4167,0.3786)
\PST@Circle(0.4919,0.4523)
\PST@Circle(0.5670,0.5470)
\PST@Circle(0.6422,0.6259)
\PST@Circle(0.7175,0.7049)
\PST@Circle(0.7927,0.7838)
\PST@Circle(0.8679,0.8680)
\PST@Circle(0.7105,0.9270)
\rput[r](0.6550,0.8850){punti accorciamento}
\PST@Cross(0.1910,0.1313)
\PST@Cross(0.2662,0.2102)
\PST@Cross(0.3413,0.2944)
\PST@Cross(0.4167,0.3786)
\PST@Cross(0.4919,0.4628)
\PST@Cross(0.5670,0.5470)
\PST@Cross(0.6422,0.6259)
\PST@Cross(0.7175,0.7049)
\PST@Cross(0.7927,0.7838)
\PST@Cross(0.8679,0.8680)
\PST@Cross(0.7105,0.8850)
\rput[r](0.6550,0.8430){interpolazione allungamento}
\PST@Dashed(0.6710,0.8430)
(0.7500,0.8430)

\PST@Dashed(0.1910,0.1302)
(0.1910,0.1302)
(0.1978,0.1377)
(0.2047,0.1452)
(0.2115,0.1526)
(0.2183,0.1601)
(0.2252,0.1676)
(0.2320,0.1750)
(0.2389,0.1825)
(0.2457,0.1899)
(0.2525,0.1974)
(0.2594,0.2049)
(0.2662,0.2123)
(0.2730,0.2198)
(0.2799,0.2273)
(0.2867,0.2347)
(0.2936,0.2422)
(0.3004,0.2496)
(0.3072,0.2571)
(0.3141,0.2646)
(0.3209,0.2720)
(0.3277,0.2795)
(0.3346,0.2870)
(0.3414,0.2944)
(0.3483,0.3019)
(0.3551,0.3093)
(0.3619,0.3168)
(0.3688,0.3243)
(0.3756,0.3317)
(0.3824,0.3392)
(0.3893,0.3467)
(0.3961,0.3541)
(0.4030,0.3616)
(0.4098,0.3690)
(0.4166,0.3765)
(0.4235,0.3840)
(0.4303,0.3914)
(0.4371,0.3989)
(0.4440,0.4064)
(0.4508,0.4138)
(0.4576,0.4213)
(0.4645,0.4287)
(0.4713,0.4362)
(0.4782,0.4437)
(0.4850,0.4511)
(0.4918,0.4586)
(0.4987,0.4661)
(0.5055,0.4735)
(0.5123,0.4810)
(0.5192,0.4884)
(0.5260,0.4959)
(0.5329,0.5034)
(0.5397,0.5108)
(0.5465,0.5183)
(0.5534,0.5258)
(0.5602,0.5332)
(0.5670,0.5407)
(0.5739,0.5481)
(0.5807,0.5556)
(0.5876,0.5631)
(0.5944,0.5705)
(0.6012,0.5780)
(0.6081,0.5855)
(0.6149,0.5929)
(0.6217,0.6004)
(0.6286,0.6078)
(0.6354,0.6153)
(0.6423,0.6228)
(0.6491,0.6302)
(0.6559,0.6377)
(0.6628,0.6452)
(0.6696,0.6526)
(0.6764,0.6601)
(0.6833,0.6675)
(0.6901,0.6750)
(0.6969,0.6825)
(0.7038,0.6899)
(0.7106,0.6974)
(0.7175,0.7049)
(0.7243,0.7123)
(0.7311,0.7198)
(0.7380,0.7272)
(0.7448,0.7347)
(0.7516,0.7422)
(0.7585,0.7496)
(0.7653,0.7571)
(0.7722,0.7646)
(0.7790,0.7720)
(0.7858,0.7795)
(0.7927,0.7869)
(0.7995,0.7944)
(0.8063,0.8019)
(0.8132,0.8093)
(0.8200,0.8168)
(0.8269,0.8243)
(0.8337,0.8317)
(0.8405,0.8392)
(0.8474,0.8466)
(0.8542,0.8541)
(0.8610,0.8616)
(0.8679,0.8690)

\rput[r](0.6550,0.8010){interpolazione accorciamento}
\PST@Dotted(0.6710,0.8010)
(0.7500,0.8010)

\PST@Dotted(0.1910,0.1319)
(0.1910,0.1319)
(0.1978,0.1393)
(0.2047,0.1468)
(0.2115,0.1542)
(0.2183,0.1617)
(0.2252,0.1691)
(0.2320,0.1766)
(0.2389,0.1840)
(0.2457,0.1915)
(0.2525,0.1989)
(0.2594,0.2064)
(0.2662,0.2138)
(0.2730,0.2213)
(0.2799,0.2287)
(0.2867,0.2362)
(0.2936,0.2436)
(0.3004,0.2511)
(0.3072,0.2585)
(0.3141,0.2660)
(0.3209,0.2734)
(0.3277,0.2809)
(0.3346,0.2883)
(0.3414,0.2958)
(0.3483,0.3032)
(0.3551,0.3107)
(0.3619,0.3181)
(0.3688,0.3256)
(0.3756,0.3330)
(0.3824,0.3405)
(0.3893,0.3479)
(0.3961,0.3554)
(0.4030,0.3628)
(0.4098,0.3703)
(0.4166,0.3777)
(0.4235,0.3852)
(0.4303,0.3927)
(0.4371,0.4001)
(0.4440,0.4076)
(0.4508,0.4150)
(0.4576,0.4225)
(0.4645,0.4299)
(0.4713,0.4374)
(0.4782,0.4448)
(0.4850,0.4523)
(0.4918,0.4597)
(0.4987,0.4672)
(0.5055,0.4746)
(0.5123,0.4821)
(0.5192,0.4895)
(0.5260,0.4970)
(0.5329,0.5044)
(0.5397,0.5119)
(0.5465,0.5193)
(0.5534,0.5268)
(0.5602,0.5342)
(0.5670,0.5417)
(0.5739,0.5491)
(0.5807,0.5566)
(0.5876,0.5640)
(0.5944,0.5715)
(0.6012,0.5789)
(0.6081,0.5864)
(0.6149,0.5938)
(0.6217,0.6013)
(0.6286,0.6087)
(0.6354,0.6162)
(0.6423,0.6236)
(0.6491,0.6311)
(0.6559,0.6385)
(0.6628,0.6460)
(0.6696,0.6534)
(0.6764,0.6609)
(0.6833,0.6683)
(0.6901,0.6758)
(0.6969,0.6832)
(0.7038,0.6907)
(0.7106,0.6981)
(0.7175,0.7056)
(0.7243,0.7130)
(0.7311,0.7205)
(0.7380,0.7279)
(0.7448,0.7354)
(0.7516,0.7428)
(0.7585,0.7503)
(0.7653,0.7577)
(0.7722,0.7652)
(0.7790,0.7727)
(0.7858,0.7801)
(0.7927,0.7876)
(0.7995,0.7950)
(0.8063,0.8025)
(0.8132,0.8099)
(0.8200,0.8174)
(0.8269,0.8248)
(0.8337,0.8323)
(0.8405,0.8397)
(0.8474,0.8472)
(0.8542,0.8546)
(0.8610,0.8621)
(0.8679,0.8695)

\PST@Border(0.1910,0.9680)
(0.1910,0.1260)
(0.9580,0.1260)
(0.9580,0.9680)
(0.1910,0.9680)

\catcode`@=12
\fi
\endpspicture

\end{figure}
\begin{figure}[p]\caption{Grafico relativo all'estensimetro n.18. In ascissa sono riportati i valori di $\Delta F~(\unit{N})$ e in ordinata la corrispondente variazione di lunghezza del filo $\Delta x~(\unit{m})$. Le barre di errore su entrambe le misure sono troppo piccole per poter essere rappresentate su questa scala.}\label{18}
% GNUPLOT: LaTeX picture using PSTRICKS macros
% Define new PST objects, if not already defined
\ifx\PSTloaded\undefined
\def\PSTloaded{t}
\psset{arrowsize=.01 3.2 1.4 .3}
\psset{dotsize=.125}
\catcode`@=11

\newpsobject{PST@Border}{psline}{linewidth=.0015,linestyle=solid}
\newpsobject{PST@Axes}{psline}{linewidth=.0015,linestyle=dotted,dotsep=.004}
\newpsobject{PST@Solid}{psline}{linewidth=.0015,linestyle=solid}
\newpsobject{PST@Dashed}{psline}{linewidth=.0015,linestyle=dashed,dash=.01 .01}
\newpsobject{PST@Dotted}{psline}{linewidth=.0025,linestyle=dotted,dotsep=.008}
\newpsobject{PST@LongDash}{psline}{linewidth=.0015,linestyle=dashed,dash=.02 .01}
\newpsobject{PST@Diamond}{psdots}{linewidth=.001,linestyle=solid,dotstyle=square,dotangle=45}
\newpsobject{PST@Filldiamond}{psdots}{linewidth=.001,linestyle=solid,dotstyle=square*,dotangle=45}
\newpsobject{PST@Cross}{psdots}{linewidth=.001,linestyle=solid,dotstyle=+,dotangle=45}
\newpsobject{PST@Plus}{psdots}{linewidth=.001,linestyle=solid,dotstyle=+}
\newpsobject{PST@Square}{psdots}{linewidth=.001,linestyle=solid,dotstyle=square}
\newpsobject{PST@Circle}{psdots}{linewidth=.001,linestyle=solid,dotstyle=o}
\newpsobject{PST@Triangle}{psdots}{linewidth=.001,linestyle=solid,dotstyle=triangle}
\newpsobject{PST@Pentagon}{psdots}{linewidth=.001,linestyle=solid,dotstyle=pentagon}
\newpsobject{PST@Fillsquare}{psdots}{linewidth=.001,linestyle=solid,dotstyle=square*}
\newpsobject{PST@Fillcircle}{psdots}{linewidth=.001,linestyle=solid,dotstyle=*}
\newpsobject{PST@Filltriangle}{psdots}{linewidth=.001,linestyle=solid,dotstyle=triangle*}
\newpsobject{PST@Fillpentagon}{psdots}{linewidth=.001,linestyle=solid,dotstyle=pentagon*}
\newpsobject{PST@Arrow}{psline}{linewidth=.001,linestyle=solid}
\catcode`@=12

\fi
\psset{unit=5.0in,xunit=5.0in,yunit=3.0in}
\pspicture(0.000000,0.000000)(1.000000,1.000000)
\ifx\nofigs\undefined
\catcode`@=11

\PST@Border(0.1910,0.1260)
(0.2060,0.1260)

\rput[r](0.1750,0.1260){0.0000}
\PST@Border(0.1910,0.2663)
(0.2060,0.2663)

\rput[r](0.1750,0.2663){0.0002}
\PST@Border(0.1910,0.4067)
(0.2060,0.4067)

\rput[r](0.1750,0.4067){0.0004}
\PST@Border(0.1910,0.5470)
(0.2060,0.5470)

\rput[r](0.1750,0.5470){0.0006}
\PST@Border(0.1910,0.6873)
(0.2060,0.6873)

\rput[r](0.1750,0.6873){0.0008}
\PST@Border(0.1910,0.8277)
(0.2060,0.8277)

\rput[r](0.1750,0.8277){0.0010}
\PST@Border(0.1910,0.9680)
(0.2060,0.9680)

\rput[r](0.1750,0.9680){0.0012}
\PST@Border(0.1910,0.1260)
(0.1910,0.1460)

\rput(0.1910,0.0840){ 0}
\PST@Border(0.2869,0.1260)
(0.2869,0.1460)

\rput(0.2869,0.0840){ 5}
\PST@Border(0.3828,0.1260)
(0.3828,0.1460)

\rput(0.3828,0.0840){ 10}
\PST@Border(0.4786,0.1260)
(0.4786,0.1460)

\rput(0.4786,0.0840){ 15}
\PST@Border(0.5745,0.1260)
(0.5745,0.1460)

\rput(0.5745,0.0840){ 20}
\PST@Border(0.6704,0.1260)
(0.6704,0.1460)

\rput(0.6704,0.0840){ 25}
\PST@Border(0.7663,0.1260)
(0.7663,0.1460)

\rput(0.7663,0.0840){ 30}
\PST@Border(0.8621,0.1260)
(0.8621,0.1460)

\rput(0.8621,0.0840){ 35}
\PST@Border(0.9580,0.1260)
(0.9580,0.1460)

\rput(0.9580,0.0840){ 40}
\PST@Border(0.1910,0.9680)
(0.1910,0.1260)
(0.9580,0.1260)
(0.9580,0.9680)
(0.1910,0.9680)

\rput{L}(0.0420,0.5470){$\Delta x$}
\rput(0.5745,0.0210){$\Delta F$}
\rput[r](0.6550,0.9270){punti allungamento}
\PST@Circle(0.1910,0.1260)
\PST@Circle(0.2662,0.2312)
\PST@Circle(0.3413,0.3225)
\PST@Circle(0.4167,0.4137)
\PST@Circle(0.4919,0.5049)
\PST@Circle(0.5670,0.5961)
\PST@Circle(0.6422,0.6873)
\PST@Circle(0.7175,0.7856)
\PST@Circle(0.7927,0.8698)
\PST@Circle(0.8679,0.9610)
\PST@Circle(0.7105,0.9270)
\rput[r](0.6550,0.8850){punti accorciamento}
\PST@Cross(0.1910,0.1260)
\PST@Cross(0.2662,0.2242)
\PST@Cross(0.3413,0.3225)
\PST@Cross(0.4167,0.4137)
\PST@Cross(0.4919,0.4979)
\PST@Cross(0.5670,0.5821)
\PST@Cross(0.6422,0.6803)
\PST@Cross(0.7175,0.7785)
\PST@Cross(0.7927,0.8627)
\PST@Cross(0.8679,0.9610)
\PST@Cross(0.7105,0.8850)
\rput[r](0.6550,0.8430){interpolazione allungamento}
\PST@Dashed(0.6710,0.8430)
(0.7500,0.8430)

\PST@Dashed(0.1910,0.1350)
(0.1910,0.1350)
(0.1978,0.1433)
(0.2047,0.1517)
(0.2115,0.1601)
(0.2183,0.1685)
(0.2252,0.1769)
(0.2320,0.1852)
(0.2389,0.1936)
(0.2457,0.2020)
(0.2525,0.2104)
(0.2594,0.2188)
(0.2662,0.2272)
(0.2730,0.2355)
(0.2799,0.2439)
(0.2867,0.2523)
(0.2936,0.2607)
(0.3004,0.2691)
(0.3072,0.2774)
(0.3141,0.2858)
(0.3209,0.2942)
(0.3277,0.3026)
(0.3346,0.3110)
(0.3414,0.3193)
(0.3483,0.3277)
(0.3551,0.3361)
(0.3619,0.3445)
(0.3688,0.3529)
(0.3756,0.3612)
(0.3824,0.3696)
(0.3893,0.3780)
(0.3961,0.3864)
(0.4030,0.3948)
(0.4098,0.4031)
(0.4166,0.4115)
(0.4235,0.4199)
(0.4303,0.4283)
(0.4371,0.4367)
(0.4440,0.4450)
(0.4508,0.4534)
(0.4576,0.4618)
(0.4645,0.4702)
(0.4713,0.4786)
(0.4782,0.4870)
(0.4850,0.4953)
(0.4918,0.5037)
(0.4987,0.5121)
(0.5055,0.5205)
(0.5123,0.5289)
(0.5192,0.5372)
(0.5260,0.5456)
(0.5329,0.5540)
(0.5397,0.5624)
(0.5465,0.5708)
(0.5534,0.5791)
(0.5602,0.5875)
(0.5670,0.5959)
(0.5739,0.6043)
(0.5807,0.6127)
(0.5876,0.6210)
(0.5944,0.6294)
(0.6012,0.6378)
(0.6081,0.6462)
(0.6149,0.6546)
(0.6217,0.6629)
(0.6286,0.6713)
(0.6354,0.6797)
(0.6423,0.6881)
(0.6491,0.6965)
(0.6559,0.7048)
(0.6628,0.7132)
(0.6696,0.7216)
(0.6764,0.7300)
(0.6833,0.7384)
(0.6901,0.7468)
(0.6969,0.7551)
(0.7038,0.7635)
(0.7106,0.7719)
(0.7175,0.7803)
(0.7243,0.7887)
(0.7311,0.7970)
(0.7380,0.8054)
(0.7448,0.8138)
(0.7516,0.8222)
(0.7585,0.8306)
(0.7653,0.8389)
(0.7722,0.8473)
(0.7790,0.8557)
(0.7858,0.8641)
(0.7927,0.8725)
(0.7995,0.8808)
(0.8063,0.8892)
(0.8132,0.8976)
(0.8200,0.9060)
(0.8269,0.9144)
(0.8337,0.9227)
(0.8405,0.9311)
(0.8474,0.9395)
(0.8542,0.9479)
(0.8610,0.9563)
(0.8679,0.9646)

\rput[r](0.6550,0.8010){interpolazione accorciamento}
\PST@Dotted(0.6710,0.8010)
(0.7500,0.8010)

\PST@Dotted(0.1910,0.1318)
(0.1910,0.1318)
(0.1978,0.1401)
(0.2047,0.1485)
(0.2115,0.1568)
(0.2183,0.1652)
(0.2252,0.1735)
(0.2320,0.1818)
(0.2389,0.1902)
(0.2457,0.1985)
(0.2525,0.2069)
(0.2594,0.2152)
(0.2662,0.2236)
(0.2730,0.2319)
(0.2799,0.2403)
(0.2867,0.2486)
(0.2936,0.2570)
(0.3004,0.2653)
(0.3072,0.2737)
(0.3141,0.2820)
(0.3209,0.2903)
(0.3277,0.2987)
(0.3346,0.3070)
(0.3414,0.3154)
(0.3483,0.3237)
(0.3551,0.3321)
(0.3619,0.3404)
(0.3688,0.3488)
(0.3756,0.3571)
(0.3824,0.3655)
(0.3893,0.3738)
(0.3961,0.3822)
(0.4030,0.3905)
(0.4098,0.3988)
(0.4166,0.4072)
(0.4235,0.4155)
(0.4303,0.4239)
(0.4371,0.4322)
(0.4440,0.4406)
(0.4508,0.4489)
(0.4576,0.4573)
(0.4645,0.4656)
(0.4713,0.4740)
(0.4782,0.4823)
(0.4850,0.4906)
(0.4918,0.4990)
(0.4987,0.5073)
(0.5055,0.5157)
(0.5123,0.5240)
(0.5192,0.5324)
(0.5260,0.5407)
(0.5329,0.5491)
(0.5397,0.5574)
(0.5465,0.5658)
(0.5534,0.5741)
(0.5602,0.5825)
(0.5670,0.5908)
(0.5739,0.5991)
(0.5807,0.6075)
(0.5876,0.6158)
(0.5944,0.6242)
(0.6012,0.6325)
(0.6081,0.6409)
(0.6149,0.6492)
(0.6217,0.6576)
(0.6286,0.6659)
(0.6354,0.6743)
(0.6423,0.6826)
(0.6491,0.6909)
(0.6559,0.6993)
(0.6628,0.7076)
(0.6696,0.7160)
(0.6764,0.7243)
(0.6833,0.7327)
(0.6901,0.7410)
(0.6969,0.7494)
(0.7038,0.7577)
(0.7106,0.7661)
(0.7175,0.7744)
(0.7243,0.7828)
(0.7311,0.7911)
(0.7380,0.7994)
(0.7448,0.8078)
(0.7516,0.8161)
(0.7585,0.8245)
(0.7653,0.8328)
(0.7722,0.8412)
(0.7790,0.8495)
(0.7858,0.8579)
(0.7927,0.8662)
(0.7995,0.8746)
(0.8063,0.8829)
(0.8132,0.8912)
(0.8200,0.8996)
(0.8269,0.9079)
(0.8337,0.9163)
(0.8405,0.9246)
(0.8474,0.9330)
(0.8542,0.9413)
(0.8610,0.9497)
(0.8679,0.9580)

\PST@Border(0.1910,0.9680)
(0.1910,0.1260)
(0.9580,0.1260)
(0.9580,0.9680)
(0.1910,0.9680)

\catcode`@=12
\fi
\endpspicture

\end{figure}
\begin{figure}[p]\caption{Dipendenza lineare di $K$ ($\cdot 10^6 \unitfrac m N$) dal reciproco della sezione del filo $\nicefrac 1 S$ ($\cdot 10^{-6} \unit{m^{-2}}$) per gli estensimetri di acciaio di uguale lunghezza, ovvero i n. 6, 7, 8 e 9.}\label{kes}
% GNUPLOT: LaTeX picture using PSTRICKS macros
% Define new PST objects, if not already defined
\ifx\PSTloaded\undefined
\def\PSTloaded{t}
\psset{arrowsize=.01 3.2 1.4 .3}
\psset{dotsize=.125}
\catcode`@=11

\newpsobject{PST@Border}{psline}{linewidth=.0015,linestyle=solid}
\newpsobject{PST@Axes}{psline}{linewidth=.0015,linestyle=dotted,dotsep=.004}
\newpsobject{PST@Solid}{psline}{linewidth=.0015,linestyle=solid}
\newpsobject{PST@Dashed}{psline}{linewidth=.0015,linestyle=dashed,dash=.01 .01}
\newpsobject{PST@Dotted}{psline}{linewidth=.0025,linestyle=dotted,dotsep=.008}
\newpsobject{PST@LongDash}{psline}{linewidth=.0015,linestyle=dashed,dash=.02 .01}
\newpsobject{PST@Diamond}{psdots}{linewidth=.001,linestyle=solid,dotstyle=square,dotangle=45}
\newpsobject{PST@Filldiamond}{psdots}{linewidth=.001,linestyle=solid,dotstyle=square*,dotangle=45}
\newpsobject{PST@Cross}{psdots}{linewidth=.001,linestyle=solid,dotstyle=+,dotangle=45}
\newpsobject{PST@Plus}{psdots}{linewidth=.001,linestyle=solid,dotstyle=+}
\newpsobject{PST@Square}{psdots}{linewidth=.001,linestyle=solid,dotstyle=square}
\newpsobject{PST@Circle}{psdots}{linewidth=.001,linestyle=solid,dotstyle=o}
\newpsobject{PST@Triangle}{psdots}{linewidth=.001,linestyle=solid,dotstyle=triangle}
\newpsobject{PST@Pentagon}{psdots}{linewidth=.001,linestyle=solid,dotstyle=pentagon}
\newpsobject{PST@Fillsquare}{psdots}{linewidth=.001,linestyle=solid,dotstyle=square*}
\newpsobject{PST@Fillcircle}{psdots}{linewidth=.001,linestyle=solid,dotstyle=*}
\newpsobject{PST@Filltriangle}{psdots}{linewidth=.001,linestyle=solid,dotstyle=triangle*}
\newpsobject{PST@Fillpentagon}{psdots}{linewidth=.001,linestyle=solid,dotstyle=pentagon*}
\newpsobject{PST@Arrow}{psline}{linewidth=.001,linestyle=solid}
\catcode`@=12

\fi
\psset{unit=5.0in,xunit=5.0in,yunit=3.0in}
\pspicture(0.000000,0.000000)(1.000000,1.000000)
\ifx\nofigs\undefined
\catcode`@=11

\PST@Border(0.1430,0.1260)
(0.1580,0.1260)

\rput[r](0.1270,0.1260){ 40}
\PST@Border(0.1430,0.2944)
(0.1580,0.2944)

\rput[r](0.1270,0.2944){ 45}
\PST@Border(0.1430,0.4628)
(0.1580,0.4628)

\rput[r](0.1270,0.4628){ 50}
\PST@Border(0.1430,0.6312)
(0.1580,0.6312)

\rput[r](0.1270,0.6312){ 55}
\PST@Border(0.1430,0.7996)
(0.1580,0.7996)

\rput[r](0.1270,0.7996){ 60}
\PST@Border(0.1430,0.9680)
(0.1580,0.9680)

\rput[r](0.1270,0.9680){ 65}
\PST@Border(0.1430,0.1260)
(0.1430,0.1460)

\rput(0.1430,0.0840){8.5}
\PST@Border(0.2161,0.1260)
(0.2161,0.1460)

\rput(0.2161,0.0840){9.0}
\PST@Border(0.2892,0.1260)
(0.2892,0.1460)

\rput(0.2892,0.0840){9.5}
\PST@Border(0.3623,0.1260)
(0.3623,0.1460)

\rput(0.3623,0.0840){10.0}
\PST@Border(0.4354,0.1260)
(0.4354,0.1460)

\rput(0.4354,0.0840){10.5}
\PST@Border(0.5085,0.1260)
(0.5085,0.1460)

\rput(0.5085,0.0840){11.0}
\PST@Border(0.5815,0.1260)
(0.5815,0.1460)

\rput(0.5815,0.0840){11.5}
\PST@Border(0.6546,0.1260)
(0.6546,0.1460)

\rput(0.6546,0.0840){12.0}
\PST@Border(0.7277,0.1260)
(0.7277,0.1460)

\rput(0.7277,0.0840){12.5}
\PST@Border(0.8008,0.1260)
(0.8008,0.1460)

\rput(0.8008,0.0840){13.0}
\PST@Border(0.8739,0.1260)
(0.8739,0.1460)

\rput(0.8739,0.0840){13.5}
\PST@Border(0.9470,0.1260)
(0.9470,0.1460)

\rput(0.9470,0.0840){14.0}
\PST@Border(0.1430,0.9680)
(0.1430,0.1260)
(0.9470,0.1260)
(0.9470,0.9680)
(0.1430,0.9680)

\rput{L}(0.0420,0.5470){$\bar{K}$}
\rput(0.5450,0.0210){$\nicefrac 1 S$}
\PST@Solid(0.1826,0.1967)
(0.1826,0.2102)

\PST@Solid(0.1751,0.1967)
(0.1901,0.1967)

\PST@Solid(0.1751,0.2102)
(0.1901,0.2102)

\PST@Solid(0.3690,0.2843)
(0.3690,0.2910)

\PST@Solid(0.3615,0.2843)
(0.3765,0.2843)

\PST@Solid(0.3615,0.2910)
(0.3765,0.2910)

\PST@Solid(0.9012,0.9208)
(0.9012,0.9343)

\PST@Solid(0.8937,0.9208)
(0.9087,0.9208)

\PST@Solid(0.8937,0.9343)
(0.9087,0.9343)

\PST@Solid(0.6096,0.6076)
(0.6096,0.6346)

\PST@Solid(0.6021,0.6076)
(0.6171,0.6076)

\PST@Solid(0.6021,0.6346)
(0.6171,0.6346)

\PST@Diamond(0.1826,0.2035)
\PST@Diamond(0.3690,0.2877)
\PST@Diamond(0.9012,0.9276)
\PST@Diamond(0.6096,0.6211)
\PST@Dashed(0.1826,0.1583)
(0.1826,0.1583)
(0.1899,0.1660)
(0.1971,0.1737)
(0.2044,0.1813)
(0.2117,0.1890)
(0.2189,0.1966)
(0.2262,0.2043)
(0.2334,0.2120)
(0.2407,0.2196)
(0.2479,0.2273)
(0.2552,0.2350)
(0.2625,0.2426)
(0.2697,0.2503)
(0.2770,0.2580)
(0.2842,0.2656)
(0.2915,0.2733)
(0.2988,0.2810)
(0.3060,0.2886)
(0.3133,0.2963)
(0.3205,0.3040)
(0.3278,0.3116)
(0.3351,0.3193)
(0.3423,0.3270)
(0.3496,0.3346)
(0.3568,0.3423)
(0.3641,0.3499)
(0.3713,0.3576)
(0.3786,0.3653)
(0.3859,0.3729)
(0.3931,0.3806)
(0.4004,0.3883)
(0.4076,0.3959)
(0.4149,0.4036)
(0.4222,0.4113)
(0.4294,0.4189)
(0.4367,0.4266)
(0.4439,0.4343)
(0.4512,0.4419)
(0.4585,0.4496)
(0.4657,0.4573)
(0.4730,0.4649)
(0.4802,0.4726)
(0.4875,0.4802)
(0.4947,0.4879)
(0.5020,0.4956)
(0.5093,0.5032)
(0.5165,0.5109)
(0.5238,0.5186)
(0.5310,0.5262)
(0.5383,0.5339)
(0.5456,0.5416)
(0.5528,0.5492)
(0.5601,0.5569)
(0.5673,0.5646)
(0.5746,0.5722)
(0.5819,0.5799)
(0.5891,0.5876)
(0.5964,0.5952)
(0.6036,0.6029)
(0.6109,0.6106)
(0.6181,0.6182)
(0.6254,0.6259)
(0.6327,0.6335)
(0.6399,0.6412)
(0.6472,0.6489)
(0.6544,0.6565)
(0.6617,0.6642)
(0.6690,0.6719)
(0.6762,0.6795)
(0.6835,0.6872)
(0.6907,0.6949)
(0.6980,0.7025)
(0.7053,0.7102)
(0.7125,0.7179)
(0.7198,0.7255)
(0.7270,0.7332)
(0.7343,0.7409)
(0.7415,0.7485)
(0.7488,0.7562)
(0.7561,0.7639)
(0.7633,0.7715)
(0.7706,0.7792)
(0.7778,0.7868)
(0.7851,0.7945)
(0.7924,0.8022)
(0.7996,0.8098)
(0.8069,0.8175)
(0.8141,0.8252)
(0.8214,0.8328)
(0.8287,0.8405)
(0.8359,0.8482)
(0.8432,0.8558)
(0.8504,0.8635)
(0.8577,0.8712)
(0.8650,0.8788)
(0.8722,0.8865)
(0.8795,0.8942)
(0.8867,0.9018)
(0.8940,0.9095)
(0.9012,0.9172)

\PST@Border(0.1430,0.9680)
(0.1430,0.1260)
(0.9470,0.1260)
(0.9470,0.9680)
(0.1430,0.9680)

\catcode`@=12
\fi
\endpspicture

\end{figure}
\begin{figure}[p]\caption{Dipendenza lineare di $K$ ($\cdot 10^6 \unitfrac m N$) dalla lunghezza del filo $\ell$ ($\unit{m}$) per gli estensimetri di acciaio di uguale sezione, ovvero i n. 14, 16, 17 e 18.}\label{kel}
% GNUPLOT: LaTeX picture using PSTRICKS macros
% Define new PST objects, if not already defined
\ifx\PSTloaded\undefined
\def\PSTloaded{t}
\psset{arrowsize=.01 3.2 1.4 .3}
\psset{dotsize=.125}
\catcode`@=11

\newpsobject{PST@Border}{psline}{linewidth=.0015,linestyle=solid}
\newpsobject{PST@Axes}{psline}{linewidth=.0015,linestyle=dotted,dotsep=.004}
\newpsobject{PST@Solid}{psline}{linewidth=.0015,linestyle=solid}
\newpsobject{PST@Dashed}{psline}{linewidth=.0015,linestyle=dashed,dash=.01 .01}
\newpsobject{PST@Dotted}{psline}{linewidth=.0025,linestyle=dotted,dotsep=.008}
\newpsobject{PST@LongDash}{psline}{linewidth=.0015,linestyle=dashed,dash=.02 .01}
\newpsobject{PST@Diamond}{psdots}{linewidth=.001,linestyle=solid,dotstyle=square,dotangle=45}
\newpsobject{PST@Filldiamond}{psdots}{linewidth=.001,linestyle=solid,dotstyle=square*,dotangle=45}
\newpsobject{PST@Cross}{psdots}{linewidth=.001,linestyle=solid,dotstyle=+,dotangle=45}
\newpsobject{PST@Plus}{psdots}{linewidth=.001,linestyle=solid,dotstyle=+}
\newpsobject{PST@Square}{psdots}{linewidth=.001,linestyle=solid,dotstyle=square}
\newpsobject{PST@Circle}{psdots}{linewidth=.001,linestyle=solid,dotstyle=o}
\newpsobject{PST@Triangle}{psdots}{linewidth=.001,linestyle=solid,dotstyle=triangle}
\newpsobject{PST@Pentagon}{psdots}{linewidth=.001,linestyle=solid,dotstyle=pentagon}
\newpsobject{PST@Fillsquare}{psdots}{linewidth=.001,linestyle=solid,dotstyle=square*}
\newpsobject{PST@Fillcircle}{psdots}{linewidth=.001,linestyle=solid,dotstyle=*}
\newpsobject{PST@Filltriangle}{psdots}{linewidth=.001,linestyle=solid,dotstyle=triangle*}
\newpsobject{PST@Fillpentagon}{psdots}{linewidth=.001,linestyle=solid,dotstyle=pentagon*}
\newpsobject{PST@Arrow}{psline}{linewidth=.001,linestyle=solid}
\catcode`@=12

\fi
\psset{unit=5.0in,xunit=5.0in,yunit=3.0in}
\pspicture(0.000000,0.000000)(1.000000,1.000000)
\ifx\nofigs\undefined
\catcode`@=11

\PST@Border(0.1270,0.1260)
(0.1420,0.1260)

\rput[r](0.1110,0.1260){20}
\PST@Border(0.1270,0.2025)
(0.1420,0.2025)

\rput[r](0.1110,0.2025){25}
\PST@Border(0.1270,0.2791)
(0.1420,0.2791)

\rput[r](0.1110,0.2791){30}
\PST@Border(0.1270,0.3556)
(0.1420,0.3556)

\rput[r](0.1110,0.3556){35}
\PST@Border(0.1270,0.4322)
(0.1420,0.4322)

\rput[r](0.1110,0.4322){40}
\PST@Border(0.1270,0.5087)
(0.1420,0.5087)

\rput[r](0.1110,0.5087){45}
\PST@Border(0.1270,0.5853)
(0.1420,0.5853)

\rput[r](0.1110,0.5853){50}
\PST@Border(0.1270,0.6618)
(0.1420,0.6618)

\rput[r](0.1110,0.6618){55}
\PST@Border(0.1270,0.7384)
(0.1420,0.7384)

\rput[r](0.1110,0.7384){60}
\PST@Border(0.1270,0.8149)
(0.1420,0.8149)

\rput[r](0.1110,0.8149){65}
\PST@Border(0.1270,0.8915)
(0.1420,0.8915)

\rput[r](0.1110,0.8915){70}
\PST@Border(0.1270,0.9680)
(0.1420,0.9680)

\rput[r](0.1110,0.9680){75}
\PST@Border(0.1270,0.1260)
(0.1270,0.1460)

\rput(0.1270,0.0840){0.3}
\PST@Border(0.2637,0.1260)
(0.2637,0.1460)

\rput(0.2637,0.0840){0.4}
\PST@Border(0.4003,0.1260)
(0.4003,0.1460)

\rput(0.4003,0.0840){0.5}
\PST@Border(0.5370,0.1260)
(0.5370,0.1460)

\rput(0.5370,0.0840){0.6}
\PST@Border(0.6737,0.1260)
(0.6737,0.1460)

\rput(0.6737,0.0840){0.7}
\PST@Border(0.8103,0.1260)
(0.8103,0.1460)

\rput(0.8103,0.0840){0.8}
\PST@Border(0.9470,0.1260)
(0.9470,0.1460)

\rput(0.9470,0.0840){0.9}
\PST@Border(0.1270,0.9680)
(0.1270,0.1260)
(0.9470,0.1260)
(0.9470,0.9680)
(0.1270,0.9680)

\rput{L}(0.0420,0.5470){$\bar{K}$}
\rput(0.5370,0.0210){$\ell$}
\PST@Solid(0.8103,0.7996)
(0.8103,0.8027)

\PST@Solid(0.8028,0.7996)
(0.8178,0.7996)

\PST@Solid(0.8028,0.8027)
(0.8178,0.8027)

\PST@Solid(0.4003,0.4261)
(0.4003,0.4291)

\PST@Solid(0.3928,0.4261)
(0.4078,0.4261)

\PST@Solid(0.3928,0.4291)
(0.4078,0.4291)

\PST@Solid(0.2637,0.3296)
(0.2637,0.3327)

\PST@Solid(0.2562,0.3296)
(0.2712,0.3296)

\PST@Solid(0.2562,0.3327)
(0.2712,0.3327)

\PST@Solid(0.5370,0.5393)
(0.5370,0.5424)

\PST@Solid(0.5295,0.5393)
(0.5445,0.5393)

\PST@Solid(0.5295,0.5424)
(0.5445,0.5424)

\PST@Diamond(0.8103,0.8011)
\PST@Diamond(0.4003,0.4276)
\PST@Diamond(0.2637,0.3311)
\PST@Diamond(0.5370,0.5409)
\PST@Dashed(0.1270,0.1991)
(0.1270,0.1991)
(0.1353,0.2063)
(0.1436,0.2135)
(0.1518,0.2206)
(0.1601,0.2278)
(0.1684,0.2350)
(0.1767,0.2422)
(0.1850,0.2494)
(0.1933,0.2566)
(0.2015,0.2638)
(0.2098,0.2710)
(0.2181,0.2781)
(0.2264,0.2853)
(0.2347,0.2925)
(0.2430,0.2997)
(0.2512,0.3069)
(0.2595,0.3141)
(0.2678,0.3213)
(0.2761,0.3284)
(0.2844,0.3356)
(0.2927,0.3428)
(0.3009,0.3500)
(0.3092,0.3572)
(0.3175,0.3644)
(0.3258,0.3716)
(0.3341,0.3788)
(0.3424,0.3859)
(0.3506,0.3931)
(0.3589,0.4003)
(0.3672,0.4075)
(0.3755,0.4147)
(0.3838,0.4219)
(0.3921,0.4291)
(0.4003,0.4362)
(0.4086,0.4434)
(0.4169,0.4506)
(0.4252,0.4578)
(0.4335,0.4650)
(0.4417,0.4722)
(0.4500,0.4794)
(0.4583,0.4866)
(0.4666,0.4937)
(0.4749,0.5009)
(0.4832,0.5081)
(0.4914,0.5153)
(0.4997,0.5225)
(0.5080,0.5297)
(0.5163,0.5369)
(0.5246,0.5440)
(0.5329,0.5512)
(0.5411,0.5584)
(0.5494,0.5656)
(0.5577,0.5728)
(0.5660,0.5800)
(0.5743,0.5872)
(0.5826,0.5944)
(0.5908,0.6015)
(0.5991,0.6087)
(0.6074,0.6159)
(0.6157,0.6231)
(0.6240,0.6303)
(0.6323,0.6375)
(0.6405,0.6447)
(0.6488,0.6518)
(0.6571,0.6590)
(0.6654,0.6662)
(0.6737,0.6734)
(0.6819,0.6806)
(0.6902,0.6878)
(0.6985,0.6950)
(0.7068,0.7022)
(0.7151,0.7093)
(0.7234,0.7165)
(0.7316,0.7237)
(0.7399,0.7309)
(0.7482,0.7381)
(0.7565,0.7453)
(0.7648,0.7525)
(0.7731,0.7596)
(0.7813,0.7668)
(0.7896,0.7740)
(0.7979,0.7812)
(0.8062,0.7884)
(0.8145,0.7956)
(0.8228,0.8028)
(0.8310,0.8100)
(0.8393,0.8171)
(0.8476,0.8243)
(0.8559,0.8315)
(0.8642,0.8387)
(0.8725,0.8459)
(0.8807,0.8531)
(0.8890,0.8603)
(0.8973,0.8674)
(0.9056,0.8746)
(0.9139,0.8818)
(0.9222,0.8890)
(0.9304,0.8962)
(0.9387,0.9034)
(0.9470,0.9106)

\PST@Border(0.1270,0.9680)
(0.1270,0.1260)
(0.9470,0.1260)
(0.9470,0.9680)
(0.1270,0.9680)

\catcode`@=12
\fi
\endpspicture

\end{figure}
\begin{figure}[p]\caption{Il prodotto $P = \bar{K}d^2$ è costante al variare di $d^2$ in ascissa ($\unit{m^2}$).}\label{P}
% GNUPLOT: LaTeX picture using PSTRICKS macros
% Define new PST objects, if not already defined
\ifx\PSTloaded\undefined
\def\PSTloaded{t}
\psset{arrowsize=.01 3.2 1.4 .3}
\psset{dotsize=.01}
\catcode`@=11

\newpsobject{PST@Border}{psline}{linewidth=.0015,linestyle=solid}
\newpsobject{PST@Axes}{psline}{linewidth=.0015,linestyle=dotted,dotsep=.004}
\newpsobject{PST@Solid}{psline}{linewidth=.0015,linestyle=solid}
\newpsobject{PST@Dashed}{psline}{linewidth=.0015,linestyle=dashed,dash=.01 .01}
\newpsobject{PST@Dotted}{psline}{linewidth=.0025,linestyle=dotted,dotsep=.008}
\newpsobject{PST@LongDash}{psline}{linewidth=.0015,linestyle=dashed,dash=.02 .01}
\newpsobject{PST@Diamond}{psdots}{linewidth=.001,linestyle=solid,dotstyle=square,dotangle=45}
\newpsobject{PST@Filldiamond}{psdots}{linewidth=.001,linestyle=solid,dotstyle=square*,dotangle=45}
\newpsobject{PST@Cross}{psdots}{linewidth=.001,linestyle=solid,dotstyle=+,dotangle=45}
\newpsobject{PST@Plus}{psdots}{linewidth=.001,linestyle=solid,dotstyle=+}
\newpsobject{PST@Square}{psdots}{linewidth=.001,linestyle=solid,dotstyle=square}
\newpsobject{PST@Circle}{psdots}{linewidth=.001,linestyle=solid,dotstyle=o}
\newpsobject{PST@Triangle}{psdots}{linewidth=.001,linestyle=solid,dotstyle=triangle}
\newpsobject{PST@Pentagon}{psdots}{linewidth=.001,linestyle=solid,dotstyle=pentagon}
\newpsobject{PST@Fillsquare}{psdots}{linewidth=.001,linestyle=solid,dotstyle=square*}
\newpsobject{PST@Fillcircle}{psdots}{linewidth=.001,linestyle=solid,dotstyle=*}
\newpsobject{PST@Filltriangle}{psdots}{linewidth=.001,linestyle=solid,dotstyle=triangle*}
\newpsobject{PST@Fillpentagon}{psdots}{linewidth=.001,linestyle=solid,dotstyle=pentagon*}
\newpsobject{PST@Arrow}{psline}{linewidth=.001,linestyle=solid}
\catcode`@=12

\fi
\psset{unit=5.0in,xunit=5.0in,yunit=3.0in}
\pspicture(0.000000,0.000000)(1.000000,1.000000)
\ifx\nofigs\undefined
\catcode`@=11

\PST@Border(0.1430,0.1260)
(0.1580,0.1260)

\rput[r](0.1270,0.1260){5.0}
\PST@Border(0.1430,0.2102)
(0.1580,0.2102)

\rput[r](0.1270,0.2102){5.2}
\PST@Border(0.1430,0.2944)
(0.1580,0.2944)

\rput[r](0.1270,0.2944){5.4}
\PST@Border(0.1430,0.3786)
(0.1580,0.3786)

\rput[r](0.1270,0.3786){5.6}
\PST@Border(0.1430,0.4628)
(0.1580,0.4628)

\rput[r](0.1270,0.4628){5.8}
\PST@Border(0.1430,0.5470)
(0.1580,0.5470)

\rput[r](0.1270,0.5470){6.0}
\PST@Border(0.1430,0.6312)
(0.1580,0.6312)

\rput[r](0.1270,0.6312){6.2}
\PST@Border(0.1430,0.7154)
(0.1580,0.7154)

\rput[r](0.1270,0.7154){6.4}
\PST@Border(0.1430,0.7996)
(0.1580,0.7996)

\rput[r](0.1270,0.7996){6.6}
\PST@Border(0.1430,0.8838)
(0.1580,0.8838)

\rput[r](0.1270,0.8838){6.8}
\PST@Border(0.1430,0.9680)
(0.1580,0.9680)

\rput[r](0.1270,0.9680){7.0}
\PST@Border(0.1430,0.1260)
(0.1430,0.1460)

\rput(0.1430,0.0840){0.09}
\PST@Border(0.2770,0.1260)
(0.2770,0.1460)

\rput(0.2770,0.0840){0.10}
\PST@Border(0.4110,0.1260)
(0.4110,0.1460)

\rput(0.4110,0.0840){0.11}
\PST@Border(0.5450,0.1260)
(0.5450,0.1460)

\rput(0.5450,0.0840){0.12}
\PST@Border(0.6790,0.1260)
(0.6790,0.1460)

\rput(0.6790,0.0840){0.13}
\PST@Border(0.8130,0.1260)
(0.8130,0.1460)

\rput(0.8130,0.0840){0.14}
\PST@Border(0.9470,0.1260)
(0.9470,0.1460)

\rput(0.9470,0.0840){0.15}
\PST@Border(0.1430,0.9680)
(0.1430,0.1260)
(0.9470,0.1260)
(0.9470,0.9680)
(0.1430,0.9680)

\rput{L}(0.0420,0.5470){$P$}
\rput(0.5450,0.0210){$l$}
\PST@Solid(0.8800,0.5681)
(0.8800,0.6523)

\PST@Solid(0.8725,0.5681)
(0.8875,0.5681)

\PST@Solid(0.8725,0.6523)
(0.8875,0.6523)

\PST@Solid(0.6388,0.3702)
(0.6388,0.4544)

\PST@Solid(0.6313,0.3702)
(0.6463,0.3702)

\PST@Solid(0.6313,0.4544)
(0.6463,0.4544)

\PST@Solid(0.1832,0.4754)
(0.1832,0.5596)

\PST@Solid(0.1757,0.4754)
(0.1907,0.4754)

\PST@Solid(0.1757,0.5596)
(0.1907,0.5596)

\PST@Solid(0.3976,0.4839)
(0.3976,0.5680)

\PST@Solid(0.3901,0.4839)
(0.4051,0.4839)

\PST@Solid(0.3901,0.5680)
(0.4051,0.5680)

\PST@Diamond(0.8800,0.6102)
\PST@Diamond(0.6388,0.4123)
\PST@Diamond(0.1832,0.5175)
\PST@Diamond(0.3976,0.5260)
\PST@Dashed(0.1430,0.4886)
(0.1430,0.4886)
(0.1511,0.4892)
(0.1592,0.4898)
(0.1674,0.4904)
(0.1755,0.4910)
(0.1836,0.4916)
(0.1917,0.4922)
(0.1998,0.4928)
(0.2080,0.4934)
(0.2161,0.4940)
(0.2242,0.4946)
(0.2323,0.4952)
(0.2405,0.4957)
(0.2486,0.4963)
(0.2567,0.4969)
(0.2648,0.4975)
(0.2729,0.4981)
(0.2811,0.4987)
(0.2892,0.4993)
(0.2973,0.4999)
(0.3054,0.5005)
(0.3135,0.5011)
(0.3217,0.5017)
(0.3298,0.5023)
(0.3379,0.5028)
(0.3460,0.5034)
(0.3542,0.5040)
(0.3623,0.5046)
(0.3704,0.5052)
(0.3785,0.5058)
(0.3866,0.5064)
(0.3948,0.5070)
(0.4029,0.5076)
(0.4110,0.5082)
(0.4191,0.5088)
(0.4272,0.5094)
(0.4354,0.5100)
(0.4435,0.5105)
(0.4516,0.5111)
(0.4597,0.5117)
(0.4678,0.5123)
(0.4760,0.5129)
(0.4841,0.5135)
(0.4922,0.5141)
(0.5003,0.5147)
(0.5085,0.5153)
(0.5166,0.5159)
(0.5247,0.5165)
(0.5328,0.5171)
(0.5409,0.5176)
(0.5491,0.5182)
(0.5572,0.5188)
(0.5653,0.5194)
(0.5734,0.5200)
(0.5815,0.5206)
(0.5897,0.5212)
(0.5978,0.5218)
(0.6059,0.5224)
(0.6140,0.5230)
(0.6222,0.5236)
(0.6303,0.5242)
(0.6384,0.5248)
(0.6465,0.5253)
(0.6546,0.5259)
(0.6628,0.5265)
(0.6709,0.5271)
(0.6790,0.5277)
(0.6871,0.5283)
(0.6952,0.5289)
(0.7034,0.5295)
(0.7115,0.5301)
(0.7196,0.5307)
(0.7277,0.5313)
(0.7358,0.5319)
(0.7440,0.5324)
(0.7521,0.5330)
(0.7602,0.5336)
(0.7683,0.5342)
(0.7765,0.5348)
(0.7846,0.5354)
(0.7927,0.5360)
(0.8008,0.5366)
(0.8089,0.5372)
(0.8171,0.5378)
(0.8252,0.5384)
(0.8333,0.5390)
(0.8414,0.5396)
(0.8495,0.5401)
(0.8577,0.5407)
(0.8658,0.5413)
(0.8739,0.5419)
(0.8820,0.5425)
(0.8902,0.5431)
(0.8983,0.5437)
(0.9064,0.5443)
(0.9145,0.5449)
(0.9226,0.5455)
(0.9308,0.5461)
(0.9389,0.5467)
(0.9470,0.5472)

\PST@Border(0.1430,0.9680)
(0.1430,0.1260)
(0.9470,0.1260)
(0.9470,0.9680)
(0.1430,0.9680)

\catcode`@=12
\fi
\endpspicture

\end{figure}
\begin{figure}[p]\caption{Il rapporto $R = \nicefrac{\bar{K}}{\ell}$ è costante al variare di $\ell$ in ascissa (\unit{m}).}\label{R}
% GNUPLOT: LaTeX picture using PSTRICKS macros
% Define new PST objects, if not already defined
\ifx\PSTloaded\undefined
\def\PSTloaded{t}
\psset{arrowsize=.01 3.2 1.4 .3}
\psset{dotsize=.01}
\catcode`@=11

\newpsobject{PST@Border}{psline}{linewidth=.0015,linestyle=solid}
\newpsobject{PST@Axes}{psline}{linewidth=.0015,linestyle=dotted,dotsep=.004}
\newpsobject{PST@Solid}{psline}{linewidth=.0015,linestyle=solid}
\newpsobject{PST@Dashed}{psline}{linewidth=.0015,linestyle=dashed,dash=.01 .01}
\newpsobject{PST@Dotted}{psline}{linewidth=.0025,linestyle=dotted,dotsep=.008}
\newpsobject{PST@LongDash}{psline}{linewidth=.0015,linestyle=dashed,dash=.02 .01}
\newpsobject{PST@Diamond}{psdots}{linewidth=.001,linestyle=solid,dotstyle=square,dotangle=45}
\newpsobject{PST@Filldiamond}{psdots}{linewidth=.001,linestyle=solid,dotstyle=square*,dotangle=45}
\newpsobject{PST@Cross}{psdots}{linewidth=.001,linestyle=solid,dotstyle=+,dotangle=45}
\newpsobject{PST@Plus}{psdots}{linewidth=.001,linestyle=solid,dotstyle=+}
\newpsobject{PST@Square}{psdots}{linewidth=.001,linestyle=solid,dotstyle=square}
\newpsobject{PST@Circle}{psdots}{linewidth=.001,linestyle=solid,dotstyle=o}
\newpsobject{PST@Triangle}{psdots}{linewidth=.001,linestyle=solid,dotstyle=triangle}
\newpsobject{PST@Pentagon}{psdots}{linewidth=.001,linestyle=solid,dotstyle=pentagon}
\newpsobject{PST@Fillsquare}{psdots}{linewidth=.001,linestyle=solid,dotstyle=square*}
\newpsobject{PST@Fillcircle}{psdots}{linewidth=.001,linestyle=solid,dotstyle=*}
\newpsobject{PST@Filltriangle}{psdots}{linewidth=.001,linestyle=solid,dotstyle=triangle*}
\newpsobject{PST@Fillpentagon}{psdots}{linewidth=.001,linestyle=solid,dotstyle=pentagon*}
\newpsobject{PST@Arrow}{psline}{linewidth=.001,linestyle=solid}
\catcode`@=12

\fi
\psset{unit=5.0in,xunit=5.0in,yunit=3.0in}
\pspicture(0.000000,0.000000)(1.000000,1.000000)
\ifx\nofigs\undefined
\catcode`@=11

\PST@Border(0.1590,0.1260)
(0.1740,0.1260)

\rput[r](0.1430,0.1260){70.0}
\PST@Border(0.1590,0.3365)
(0.1740,0.3365)

\rput[r](0.1430,0.3365){75.0}
\PST@Border(0.1590,0.5470)
(0.1740,0.5470)

\rput[r](0.1430,0.5470){80.0}
\PST@Border(0.1590,0.7575)
(0.1740,0.7575)

\rput[r](0.1430,0.7575){85.0}
\PST@Border(0.1590,0.9680)
(0.1740,0.9680)

\rput[r](0.1430,0.9680){90.0}
\PST@Border(0.2378,0.1260)
(0.2378,0.1460)

\rput(0.2378,0.0840){ 0.4}
\PST@Border(0.3954,0.1260)
(0.3954,0.1460)

\rput(0.3954,0.0840){ 0.5}
\PST@Border(0.5530,0.1260)
(0.5530,0.1460)

\rput(0.5530,0.0840){ 0.6}
\PST@Border(0.7106,0.1260)
(0.7106,0.1460)

\rput(0.7106,0.0840){ 0.7}
\PST@Border(0.8682,0.1260)
(0.8682,0.1460)

\rput(0.8682,0.0840){ 0.8}
\PST@Border(0.1590,0.9680)
(0.1590,0.1260)
(0.9470,0.1260)
(0.9470,0.9680)
(0.1590,0.9680)

\rput{L}(0.0420,0.5470){$R$}
\rput(0.5530,0.0210){$l$}
\PST@Solid(0.8682,0.5344)
(0.8682,0.5681)

\PST@Solid(0.8607,0.5344)
(0.8757,0.5344)

\PST@Solid(0.8607,0.5681)
(0.8757,0.5681)

\PST@Solid(0.3954,0.4965)
(0.3954,0.5554)

\PST@Solid(0.3879,0.4965)
(0.4029,0.4965)

\PST@Solid(0.3879,0.5554)
(0.4029,0.5554)

\PST@Solid(0.2378,0.6607)
(0.2378,0.7365)

\PST@Solid(0.2303,0.6607)
(0.2453,0.6607)

\PST@Solid(0.2303,0.7365)
(0.2453,0.7365)

\PST@Solid(0.5530,0.4628)
(0.5530,0.5049)

\PST@Solid(0.5455,0.4628)
(0.5605,0.4628)

\PST@Solid(0.5455,0.5049)
(0.5605,0.5049)

\PST@Diamond(0.8682,0.5512)
\PST@Diamond(0.3954,0.5260)
\PST@Diamond(0.2378,0.6986)
\PST@Diamond(0.5530,0.4839)
\PST@Dashed(0.1590,0.6307)
(0.1590,0.6307)
(0.1670,0.6292)
(0.1749,0.6277)
(0.1829,0.6262)
(0.1908,0.6248)
(0.1988,0.6233)
(0.2068,0.6218)
(0.2147,0.6203)
(0.2227,0.6188)
(0.2306,0.6174)
(0.2386,0.6159)
(0.2466,0.6144)
(0.2545,0.6129)
(0.2625,0.6115)
(0.2704,0.6100)
(0.2784,0.6085)
(0.2864,0.6070)
(0.2943,0.6056)
(0.3023,0.6041)
(0.3102,0.6026)
(0.3182,0.6011)
(0.3262,0.5997)
(0.3341,0.5982)
(0.3421,0.5967)
(0.3500,0.5952)
(0.3580,0.5938)
(0.3659,0.5923)
(0.3739,0.5908)
(0.3819,0.5893)
(0.3898,0.5878)
(0.3978,0.5864)
(0.4057,0.5849)
(0.4137,0.5834)
(0.4217,0.5819)
(0.4296,0.5805)
(0.4376,0.5790)
(0.4455,0.5775)
(0.4535,0.5760)
(0.4615,0.5746)
(0.4694,0.5731)
(0.4774,0.5716)
(0.4853,0.5701)
(0.4933,0.5687)
(0.5013,0.5672)
(0.5092,0.5657)
(0.5172,0.5642)
(0.5251,0.5628)
(0.5331,0.5613)
(0.5411,0.5598)
(0.5490,0.5583)
(0.5570,0.5568)
(0.5649,0.5554)
(0.5729,0.5539)
(0.5809,0.5524)
(0.5888,0.5509)
(0.5968,0.5495)
(0.6047,0.5480)
(0.6127,0.5465)
(0.6207,0.5450)
(0.6286,0.5436)
(0.6366,0.5421)
(0.6445,0.5406)
(0.6525,0.5391)
(0.6605,0.5377)
(0.6684,0.5362)
(0.6764,0.5347)
(0.6843,0.5332)
(0.6923,0.5318)
(0.7003,0.5303)
(0.7082,0.5288)
(0.7162,0.5273)
(0.7241,0.5258)
(0.7321,0.5244)
(0.7401,0.5229)
(0.7480,0.5214)
(0.7560,0.5199)
(0.7639,0.5185)
(0.7719,0.5170)
(0.7798,0.5155)
(0.7878,0.5140)
(0.7958,0.5126)
(0.8037,0.5111)
(0.8117,0.5096)
(0.8196,0.5081)
(0.8276,0.5067)
(0.8356,0.5052)
(0.8435,0.5037)
(0.8515,0.5022)
(0.8594,0.5008)
(0.8674,0.4993)
(0.8754,0.4978)
(0.8833,0.4963)
(0.8913,0.4948)
(0.8992,0.4934)
(0.9072,0.4919)
(0.9152,0.4904)
(0.9231,0.4889)
(0.9311,0.4875)
(0.9390,0.4860)
(0.9470,0.4845)

\PST@Border(0.1590,0.9680)
(0.1590,0.1260)
(0.9470,0.1260)
(0.9470,0.9680)
(0.1590,0.9680)

\catcode`@=12
\fi
\endpspicture

\end{figure}
\begin{table}[hp]\caption{Tempi (\unit{s}) rilevati per il passaggio della sfera sui vari traguardi (\unit{cm}). Cinque sfere di diametro \unit[1.5]{mm}.}
\centering \small
\begin{tabular}{r*5c}
5 &22.0633 &21.9768 &21.9622 &22.0995 &21.8152\\
10 &44.2588 &43.9159 &44.0570 &43.9366 &43.4889\\
15 &66.3605 &66.2565 &66.0354 &65.8532 &65.2545\\
20 &88.5749 &88.2890 &87.9168 &87.8935 &86.9652\\
25 &110.9141 &110.3737 &109.9505 &109.6918 &108.8561\\
30 &133.1851 &132.3357 &132.1190 &131.4671 &130.5232\\
35 &155.3989 &154.5919 &154.1245 &153.4904 &152.3414\\
40 &177.6639 &176.5493 &176.0020 &175.4550 &174.2094\\
45 &199.7706 &198.6755 &198.0071 &197.4141 &195.9174\\
50 &221.9873 &221.0572 &220.0171 &219.4166 &217.6295
\end{tabular}
\end{table}
\begin{table}[hp]\caption{Tempi (\unit{s}) rilevati per il passaggio della sfera sui vari traguardi (\unit{cm}). Cinque sfere di diametro \unit[2/32]{''}.}
\centering \small
\begin{tabular}{r*5c}
5 &19.3441 &19.1003 &19.1785 &19.0366 &19.2417\\
10 &38.5618 &38.4871 &38.4383 &38.2734 &38.2671\\
15 &58.0700 &57.8006 &58.1617 &57.4505 &58.1468\\
20 &77.4698 &77.2979 &77.4469 &76.6469 &76.5524\\
25 &96.6133 &96.4791 &96.9129 &95.8414 &95.7619\\
30 &116.0165 &115.6841 &116.2029 &114.9287 &115.2599\\
35 &135.3374 &135.2231 &135.5667 &134.2188 &134.9346\\
40 &154.7546 &154.5461 &154.8837 &153.4309 &153.3084\\
45 &173.8791 &173.8185 &174.3125 &172.4481 &172.4526\\
50 &193.1626 &193.0940 &193.5777 &191.8238 &191.6567
\end{tabular}
\end{table}
\begin{table}[hp]\caption{Tempi (\unit{s}) rilevati per il passaggio della sfera sui vari traguardi (\unit{cm}). Cinque sfere di diametro \unit[2]{mm}.}
\centering \small
\begin{tabular}{r*5c}
5 &12.0735 &12.0563 &12.0795 &12.0336 &11.9043\\
10 &24.2311 &24.1561 &24.1574 &24.1013 &23.9867\\
15 &36.2779 &36.5074 &36.2528 &36.1132 &35.9502\\
20 &48.5869 &48.2169 &48.3942 &48.1237 &48.0338\\
25 &60.7042 &60.2633 &60.3286 &60.2714 &60.0204\\
30 &72.6668 &72.4050 &72.2811 &72.3240 &72.1130\\
35 &84.8895 &84.5837 &84.4658 &84.3816 &84.1202\\
40 &96.9701 &96.7455 &96.2148 &96.3929 &96.1234\\
45 &109.2424 &108.7270 &108.4763 &108.5751 &108.1043\\
50 &121.2001 &120.8799 &120.5727 &120.6063 &120.0331
\end{tabular}
\end{table}
\begin{table}[hp]\caption{Tempi (\unit{s}) rilevati per il passaggio della sfera sui vari traguardi (\unit{cm}). Cinque sfere di diametro \unit[3/32]{''}.}
\centering \small
\begin{tabular}{r*5c}
5 &8.3659 &8.2929 &8.5119 &8.1874 &8.2944\\
10 &16.9555 &16.8747 &17.0141 &16.5488 &16.6416\\
15 &25.3888 &25.4030 &25.5628 &24.9299 &24.9519\\
20 &34.0958 &33.8457 &34.0994 &33.3936 &33.4111\\
25 &42.5096 &42.4109 &42.5701 &41.8353 &41.6701\\
30 &51.0556 &51.0103 &51.1092 &50.2093 &50.0831\\
35 &59.6284 &59.4389 &59.7534 &58.5955 &58.6774\\
40 &68.1454 &67.9336 &68.1824 &66.9764 &66.9438\\
45 &76.7282 &76.3866 &76.8270 &75.4457 &75.2839\\
50 &85.3041 &84.9327 &85.3860 &83.8293 &83.7272
\end{tabular}
\end{table}
\begin{table}[hp]\caption{Tempi (\unit{s}) rilevati per il passaggio della sfera sui vari traguardi (\unit{cm}). Cinque sfere di diametro \unit[4/32]{''}.}
\centering \small
\begin{tabular}{r*5c}
10 &9.1740 &9.2749 &9.3356 &9.4129 &9.2203\\
20 &18.7734 &18.6987 &18.7404 &18.8126 &18.5922\\
30 &28.3608 &28.1444 &28.1084 &28.3104 &28.0037\\
40 &37.8503 &37.6021 &37.6675 &37.6935 &37.3833\\
50 &47.3757 &47.0648 &47.0555 &47.2037 &46.7653
\end{tabular}
\end{table}
\begin{table}[hp]\caption{Tempi (\unit{s}) rilevati per il passaggio della sfera sui vari traguardi (\unit{cm}). Cinque sfere di diametro \unit[5/32]{''}.}
\centering \small
\begin{tabular}{r*5c}
10 &5.8941 &5.9632 &5.9886 &5.9362 &5.8888\\
20 &11.9636 &11.9059 &12.0151 &11.9613 &11.6978\\
30 &17.9970 &17.9487 &18.0514 &18.0878 &17.7542\\
40 &24.0087 &23.8471 &24.0976 &23.8800 &23.7861\\
50 &29.9813 &29.8997 &30.0141 &29.7883 &29.7493
\end{tabular}
\end{table}
\begin{table}[hp]\caption{Tempi (\unit{s}) rilevati per il passaggio delle sfere da $D_7$ a $D_{10}$ sul traguardo dei \unit[50]{cm}.}
\centering \small
\begin{tabular}{r*5c}
$D_7$ &20.4155 &20.4998 &20.3931 &20.4337 &20.5051\\
$D_8$ &15.0105 &14.5210 &14.7475 &14.7935 &14.8191\\
$D_9$ &10.8372 &10.8669 &10.8057 &11.0179 &10.7077\\
$D_{10}$ &8.0948 &8.0813 &7.9078 &8.1931 &8.1917
\end{tabular}
\end{table}
\subsection*{Formule}
\begin{description}
 \item[Propagazione dell'errore sul modulo di Young]
\begin{equation*}
 \sigma_Y = Y\sqrt{\left(\dfrac{\sigma_\ell}{\ell}\right)^2 + \left(\dfrac{\sigma_K}{K}\right)^2 + 4\left(\dfrac{\sigma_d}{d}\right)^2}
\end{equation*}
 \item[Propagazione dell'errore su $P$]
\begin{equation*}
 \sigma_P = P\sqrt{\left(\dfrac{\sigma_{\bar{K}}}{\bar{K}}\right)^2 + 4\left(\dfrac{\sigma_d}{d}\right)^2}
\end{equation*}
 \item[Propagazione dell'errore su $R$]
\begin{equation*}
 \sigma_R = R\sqrt{\left(\dfrac{\sigma_\ell}{\ell}\right)^2 + \left(\dfrac{\sigma_{\bar{K}}}{\bar{K}}\right)^2}
\end{equation*}
 \item[Media pesata]
\begin{equation*}
 \bar{x}=\left(\sum_i \dfrac{x_i}{\sigma_{x_i}^2} \right)\left(\sum_i \dfrac{1}{\sigma_{x_i}^2} \right)^{-1} \qquad \sigma_{\bar{x}} = \left(\sum_i \dfrac{1}{\sigma_{x_i}^2} \right)^{-\nicefrac 1 2}
\end{equation*}
\item[Interpolazione lineare $y=kx+y_0$]
\begin{align*}
y_0 &= \dfrac{1}{\Delta}\left[ \left(\sum_{i=1}^Nx_i^2\right) \left(\sum_{i=1}^Ny_i\right)-\left(\sum_{i=1}^Nx_i\right)\left(\sum_{i=1}^Nx_iy_i\right)\right]\\[3pt]
k &= \dfrac{1}{\Delta}\left[N \left(\sum_{i=1}^Nx_iy_i\right)-\left(\sum_{i=1}^Nx_i\right)\left(\sum_{i=1}^Ny_i\right)\right]\\[3pt]
\Delta &= N\sum_{i=1}^Nx_i^2 - \left(\sum_{i=1}^Nx_i\right)^2\\[6pt]
\sigma_{y_0}^2 &= \dfrac{\sigma_y^2}{\Delta}\sum_{i=1}^Nx_i^2\\[3pt]
\sigma_k^2 &= \dfrac{N\sigma_y^2}{\Delta}
\end{align*}

\end{description}
\end{document}
